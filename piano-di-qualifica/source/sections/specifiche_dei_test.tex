\section{Specifica dei test}

Sono stati determinati, come descritto anche nel documento \textit{Norme di Progetto 1.0.0}, 4 tipi di test ovvero: \textit{Test di unità, Test di Integrazione, Test di sistema, Test di accettazione} e \textit{Test di regressione}. Si adotterà quindi il \textbf{Modello a V} che permette lo sviluppo dei test in parallelo alle attività di analisi e progettazione. Così facendo i test sviluppati saranno in grado di verificare sia la correttezza del programma software sia l'implementazione. Sara disponibile inoltre una tabella degli esiti dei test per una facile consultazione.

Per i test valgono le seguenti sigle:

    \begin{itemize}
        \item \textbf{I}: per indicare che il è effettivamente implementato.
        \item \textbf{NI}: per indicare che il test è ancora non implementato.
    \end{itemize}
    
Mentre per lo stato valgono le seguenti sigle:

    \begin{itemize}
        \item \textbf{S}: il test ha esito soddisfacente.
        \item \textbf{NS}: il test ha esito non soddisfacente.
    \end{itemize}
    
    
    \subsection{Test di unità}
    \subsection{Test di Integrazione}
 
    \subsection{Test di Sistema}
    
    Per rispettare i requisiti identificati nel documento \textit{Analisi dei Requisiti 1.0.0} e per garantire il funzionamento del prodotto si eseguono i seguenti Test di sistema:\\
    
    \subsubsection{Test di sistema per i requisiti funzionali}
    \hphantom{}
    %startTable
    \def\testspecification{
    {   TSOF1,
        Verifica della possibiltà di scelta nell'inserimento dei dati dell'utente,
        NI,
        RF01
    },
    {   TSOF1.1,
        Verifica dell'inserimento dati tramite file csv,
        NI,
        RFO1.1
    },
    {   TSOF1.2,
        Verifica dell'inserimento dati tramite query al database,
        NI,
        RFO1.2
    },
    {   TSOF1.3,
        Verifica dell'apparimento del messaggio d'errore in caso di errore nell'inserimento dei dati,
        NI,
        RFO1.3
    },
    {   TSOF1.4,
        Verifica della corretta conversione in JSON dei dati,
        NI,
        RFO1.4
    },
     {   TSOF2,
        Verifica della possibilità di scelta nella visualizzazione dei dati,
        NI,
        RFO2
    },
    {   TSOF2.1,
        Nella scelta per la visualizzazione deve esserci: Scatter Plot Matrix,
        NI,
        RFO2.1
    },
    {   TSOF2.2,
        Nella scelta per la visualizzazione deve esserci: Heatmap ,
        NI,
        RFO2.2
    },
    {   TSOF2.2.1,
        Verifica della possibiltà di sfumatura nel colore dell'Heapmap,
        NI,
        RFO2.2.1
    },
    {   TSFF2.3,
        Nella scelta per la visualizzazione deve esserci:Correlation Heatmap,
        NI,
        RFO2.3
    },
    {   TSOF2.4,
        Nella scelta per la visualizzazione deve esserci: Force Field,
        NI,
        RFO2.4
    },
    {   TSOF2.5,
        Nella scelta per la visualizzazione deve esserci: Linear Projection,
        NI,
        RFO2.5
    },
    {   TSOF2.6,
        Nella scelta per la visualizzazione deve esserci: Parallel Coordinates,
        NI,
        RFO2.6
    },
    {   TSOF3,
        Verifica della possibile scelta delle lables presenti nel dataset,
        NI,
        RFO3
    },
    {   TSOF3.1,
        Verifica delle possibilità di visualizzazione delle lables presenti nel dataset,
        NI,
        RFO3.1
    },
    {   TSOF4,
        Verifica della possibilità di scartare features e di inserire features scartate,
        NI,
        RFO4
    },
    {   TSOF5,
        Verifica della modifica delle impostazioni della visualizzazione Heatmap,
        NI,
        RFO5
    },
    {   TSOF5.1,
        Verifica della modifica del range dei dati all'interno della visualizzazione Heatmap,
        NI,
        RFO5.1
    },
    {   TSOF5.2,
        Verifica della possibile normalizzazione del database in modo globale e per riga,
        NI,
        RFO5.2 RFO5.2.1 RFO5.2.2
    },
    {   TSOF5.3,
        Dopo la selezione dell'Heatmap si verifica la possibilità di ordinare il dataset in ordine alfabetico e in cluster,
        NI,
        RFO5.3 RFO5.3.1 RFO5.3.2
    },
    {   TSOF5.4,
        Si Verifica la possibilità di assegnazione del colre al range nella visualizzazione Heatmap,
        NI,
        RFO5.4
    },
    {   TSOF6,
        Si verifica la possibilità di scelta nell'Heatmap di una metrica a distanza,
        NI,
        RFO6
    },
    {   TSFF6.1,
        Si verifica la possibilità di scelta di una funzione di distanza nelle visualizzazioni Heatmap e nel Force Field,
        NI,
        RFF6.1
    },
    {   TSOF6.1.1,
        Si verifica la possibilità di scelta della distanza Eclidea nelle visualizzioni Heatmap e Force Field,
        NI,
        RFF6.1.1 RFF6.1.2
    },
    {   TSOF7,
        Si verifica la possibilità di scelta di un algoritmo per la riduzione dei componenti nella visualizzaione Linear Projection,
        NI,
        RFO7
    },
    {   TSFF7.1,
        Si verifica la presenza nelle scelte dei seguenti algoritmi per la riduzione dei componenti: PCA UMAP e t-SNE,
        NI,
        RFF7.1 RFF7.2 RFF7.3
    },
    {   TSOF8.1,
        Si verifica che il sistema elabori i dati con le impostazioni di default e con le impostazioni dell'utente,
        NI,
        RFO8.1 RFO8.2
    },
    {   TSOF9,
        Si verifica che la visualizzazione creata dal sistema sia visibile a schermo,
        NI,
        RFO9},
    {   TSOF9.2,
        Si verifica che la visualizzazione creata dal sistema sia salvabile in formato PNG,
        NI,
        RFO9.2
    },
    {   TSOF10,
        Si verifica che un messaggio di errore compari a schermo se i files sono caricati scorrettamente,
        NI,
        RFO10
    },
    }
    \newcommand*\testdescription{}
\foreach \x [count=\nj] in \testspecification
{
    \foreach \y [count=\ni] in \x
    {
        \ifnum\ni=4
            \xappto\testdescription{\y}
            \gappto\testdescription{\\}
            \gappto\testdescription{\hline}
        \else
            \xappto\testdescription{\y&}
        \fi
    }
}

% Impostazioni della tabella
\tabulinesep = 2mm % padding
\taburowcolors [1] 2{pari .. dispari} % colori delle righe
\begin{longtabu} to \textwidth {| X[0.1,c m] | X[0.2,c m] | X[0.1,c m] | X[0.1,c m]|} % larghezza delle colonne
\hline
\rowcolor{header} % colore dell'header

\textbf{ID test} & \textbf{Descrizione} & \textbf{Esito} & \textbf{ID requisito} \\
\hline
\testdescription

\end{longtabu}

\undef\testdescription{}

    %endTable
    
    \subsubsection{Test di sistema per i requisiti di qualità}
        %startTable
    \def\testspecification{
    {   TSOQ1,
        Si verifica che i dati ricevuti da fonti esterne non siano modificati o errati,
        NI,
        RQO1
    },
    {   TSOQ2,
        Si verifica che il numero di errori rilevati nel codice rispetti il valore specifiato nella metruca: Densità di errori,
        NI,
        RQO2
    },
    {   TSOQ3,
        Si verifica che i messaggi di errore e avviso siano chiari da comprendere per l'utente con l'uso della metrica: Qualità della messaggistica,
        NI,
        RQO3
    },
    {   TSOQ4,
        Si verifica che il numero di click per svolgere ogni task deve essere minore del valore sufficiente data dalla mterica: Numero di click,
        NI,
        RQO4
    },
    {   TSOQ5,
        Si verifica che la profondità della struttura dell'applicativo deve essere minore del valore sufficiente dato dalla metrica: Site depth,
        NI,
        RQO5
    },
    {   TSOQ6,
        Si verifica che la durata di ogni task deve essere minore del valore sufficiente specificato dalla metrica: Response time,
        NI,
        RQO6
    },
    {   TSOQ7,
        Si verifica che la quantità dei possibili percorsi di branchink rispetti la metrica: Complessità ciclomatica,
        NI,
        RQO7
    },
    {   TSOQ8,
        Si verifica che la percentuale di test indipendenti rispetti la metrica: Indipendenza dei test,
        NI,
        RQO8
    },
    {   TSOQ9,
        Si verifica che l'indice di comprensione del codice ovvero il rapporto tra codice e commento rispetti il vaolore sufficiente dato dalla metrica: Facilità di comprensione,
        NI,
        RQO9
    },
    {   TSOQ10,
        Si verifica che il numero di procedure che chiamono un'altra procedura sia minore del valore sufficiente dato dalla metrica: Structural Fan-In,
        NI,
        RQO10
    },
    {   TSOQ11,
        Si verifica che il numero di procedure che necessita ogni altra procedura sia inferiore al valore sufficiente dato dalla metrica: Structural Fan-Out,
        NI,
        RQO11
    },
      }
    \newcommand*\testdescription{}
\foreach \x [count=\nj] in \testspecification
{
    \foreach \y [count=\ni] in \x
    {
        \ifnum\ni=4
            \xappto\testdescription{\y}
            \gappto\testdescription{\\}
            \gappto\testdescription{\hline}
        \else
            \xappto\testdescription{\y&}
        \fi
    }
}

% Impostazioni della tabella
\tabulinesep = 2mm % padding
\taburowcolors [1] 2{pari .. dispari} % colori delle righe
\begin{longtabu} to \textwidth {| X[0.1,c m] | X[0.2,c m] | X[0.1,c m] | X[0.1,c m]|} % larghezza delle colonne
\hline
\rowcolor{header} % colore dell'header

\textbf{ID test} & \textbf{Descrizione} & \textbf{Esito} & \textbf{ID requisito} \\
\hline
\testdescription

\end{longtabu}

\undef\testdescription{}

    %endTable  
        
       \subsubsection{Test di sistema per i requisiti di vincolo}
        %startTable
    \def\testspecification{
    {   TSOV1,
        Si verifica che il codice sorgente dell'applicazione sia opn source,
        NI,
        RVO1
    },
    {   TSDV",
        Si verifica che l'applicazione sia sviluppata in Javascript con l'utilizzo della libreria d3.js,
        NI,
        RVD2 RVD2.1
    },
    {   TSDV2.2,
        Si verifica che il beckend dell'applicazione sia sviluppato con node.js con l'utilizzo del framework express,
        NI,
        RVD2.2
    },
    {   TSDV2.3,
        Si verifica che il frontend dell'applicazione sia sviluppato con React con l'utilizzo del framework Ant Design,
        NI,
        RVD2.3
    },
    {   TSOV3,
        Si verifica che i dati siano convertibili in JSON,
        NI,
        RVO3
    },
    {   TSOV4,
        Si verifica che nella visualizzazione Scatter Plot Matrix si possano visualizzare al massimo 5 features,
        NI,
        RVO4
    },
    {   TSDV5,
        Si verifica che la libreria per PCA isa ml-pca,
        NI,
        RVD5
    },
    {   TSDV6,
        Si verifica che la libreria per Umap sia tsne-js,
        NI,
        RVD6
    },
    {   TSDV7,
        Si verifica che la libreria per t-SNE sia tsne-js,
        NI,
        RVD7
    },
    {   TSDV8,
        Si verifica che la libreria per le distanze sia ml-distance,
        NI,
        RVD8
    },
    {   TSDV9,
        Si verifica che la libreria per la matrice di correlazione sia jeezy,
        NI,
        RVD9
    },
      }
    \newcommand*\testdescription{}
\foreach \x [count=\nj] in \testspecification
{
    \foreach \y [count=\ni] in \x
    {
        \ifnum\ni=4
            \xappto\testdescription{\y}
            \gappto\testdescription{\\}
            \gappto\testdescription{\hline}
        \else
            \xappto\testdescription{\y&}
        \fi
    }
}

% Impostazioni della tabella
\tabulinesep = 2mm % padding
\taburowcolors [1] 2{pari .. dispari} % colori delle righe
\begin{longtabu} to \textwidth {| X[0.1,c m] | X[0.2,c m] | X[0.1,c m] | X[0.1,c m]|} % larghezza delle colonne
\hline
\rowcolor{header} % colore dell'header

\textbf{ID test} & \textbf{Descrizione} & \textbf{Esito} & \textbf{ID requisito} \\
\hline
\testdescription

\end{longtabu}

\undef\testdescription{}

    %endTable
    
    