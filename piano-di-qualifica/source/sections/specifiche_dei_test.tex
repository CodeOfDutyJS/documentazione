\section{Specifica dei test}

Sono stati determinati, come descritto anche nel documento \textit{Norme di Progetto 1.0.0}, 4 tipi di test ovvero: \textit{Test di unità, Test di Integrazione, Test di sistema, Test di accettazione} e \textit{Test di regressione}. Si adotterà quindi il \textbf{Modello a V} che permette lo sviluppo dei test in parallelo alle attività di analisi e progettazione. Così facendo i test sviluppati saranno in grado di verificare sia la correttezza del programma software sia l'implementazione. Sara disponibile inoltre una tabella degli esiti dei test per una facile consultazione.

Per i test valgono le seguenti sigle:

    \begin{itemize}
        \item \textbf{I}: per indicare che il è effettivamente implementato.
        \item \textbf{NI}: per indicare che il test è ancora non implementato.
    \end{itemize}
    
Mentre per lo stato valgono le seguenti sigle:

    \begin{itemize}
        \item \textbf{S}: il test ha esito soddisfacente.
        \item \textbf{NS}: il test ha esito non soddisfacente.
    \end{itemize}
    
    
    \subsection{Test di unità}
    \subsection{Test di Integrazione}
    
    \subsection{Test di Sistema}
    
    Per rispettare i requisiti identificati nel documento \textit{Analisi dei Requisiti 1.0.0} e per garantire il funzionamento del prodotto si eseguono i seguenti Test di sistema:
    
    \hphantom{}
    %startTable
    \def\testspecification{
    {   RF01 ,
        TSOF1,
        Verifica della possibiltà di scelta nell'inserimento dei dati dell'utente,
        NI},
    {   RFO1.1,
        TSOF1.1,
        Verifica dell'inserimento dati tramite file csv,
        NI},
    {   RFO1.2,
        TSOF1.2,
        Verifica dell'inserimento dati tramite query al database,
        NI},
    {   RFO1.3,
        TSOF1.3,
        Verifica dell'apparimento del messaggio d'errore in caso di errore nell'inserimento dei dati,
        NI},
    {   RFO1.4,
        TSOF1.4,
        Verifica della corretta conversione in JSON dei dati,
        NI},
    {   RFO2,
        TSOF2,
        Verifica della possibilità di scelta nella visualizzazione dei dati,
        NI},
    }
    \newcommand*\testdescription{}
\foreach \x [count=\nj] in \testspecification
{
    \foreach \y [count=\ni] in \x
    {
        \ifnum\ni=4
            \xappto\testdescription{\y}
            \gappto\testdescription{\\}
            \gappto\testdescription{\hline}
        \else
            \xappto\testdescription{\y&}
        \fi\fi
    }
}

% Impostazioni della tabella
\tabulinesep = 2mm % padding
\taburowcolors [1] 2{pari .. dispari} % colori delle righe
\begin{longtabu} to \textwidth {| X[0.1,c m] | X[0.1,c m] | X[0.2,c m] | X[0.1,c m]|} % larghezza delle colonne
\hline
\rowcolor{header} % colore dell'header

\textbf{} & \textbf{ID test} & \textbf{Descrizione} & \textbf{Esito} & \textbf{ID requisito}\\
\hline
\testdescription

\end{longtabu}

\undef\testdescription{}

    %endTable
    
    
    \def\testspecification{    
    {   RFO2.1,
        TSOF2.1,
        Nella scelta per la visualizzazione deve esserci: Scatter Plot Matrix,
        NI},
    {   RFO2.2,
        TSOF2.2,
        Nella scelta per la visualizzazione deve esserci: Heatmap ,
        NI},
    {   RFO2.2.1,
        TSOF2.2.1,
        Verifica della possibiltà di sfumatura nel colore dell'Heapmap,
        NI},
    {   RFO2.3,
        TSFF2.3,
        Nella scelta per la visualizzazione deve esserci:Correlation Heatmap,
        NI},
    {   RFO2.4,
        TSOF2.4,
        Nella scelta per la visualizzazione deve esserci: Force Field,
        NI},
    }
    \newcommand*\testdescription{}
\foreach \x [count=\nj] in \testspecification
{
    \foreach \y [count=\ni] in \x
    {
        \ifnum\ni=4
            \xappto\testdescription{\y}
            \gappto\testdescription{\\}
            \gappto\testdescription{\hline}
        \else
            \xappto\testdescription{\y&}
        \fi\fi
    }
}

% Impostazioni della tabella
\tabulinesep = 2mm % padding
\taburowcolors [1] 2{pari .. dispari} % colori delle righe
\begin{longtabu} to \textwidth {| X[0.1,c m] | X[0.1,c m] | X[0.2,c m] | X[0.1,c m]|} % larghezza delle colonne
\hline
\rowcolor{header} % colore dell'header

\textbf{} & \textbf{ID test} & \textbf{Descrizione} & \textbf{Esito} & \textbf{ID requisito}\\
\hline
\testdescription

\end{longtabu}

\undef\testdescription{}

    %endTable
    
    \subsection{Test di accettazione}
    \subsection{Test di regressione}