\section{Specifica dei test}

Sono stati determinati, come descritto anche nel documento \NdP , 5 tipi di test ovvero: 
\textit{Test di unità, Test di Integrazione, Test di sistema, Test di accettazione} e \textit{Test di regressione}. 
Si adotterà quindi il \textbf{Modello a V} che permette lo sviluppo dei test in parallelo alle attività di analisi e progettazione. 
Così facendo i test sviluppati saranno in grado di verificare sia la correttezza del programma software sia l'implementazione. 
Sara disponibile inoltre una tabella degli esiti dei test per una facile consultazione.
Per i test valgono le seguenti sigle:

    \begin{itemize}
        \item \textbf{S}: il test è stato implementato e ha esito soddisfacente.
        \item \textbf{NS}: il test è stato implementato e non ha esito soddisfacente.
        \item \textbf{NI}: per indicare che il test è ancora non implementato.
    \end{itemize}
 
\subsection{Test di Sistema}
\subsubsection{Test di sistema per i requisiti funzionali}
    \hphantom{}

    %startTable
    \def\testspecification{
    {   
        TSOF1,
        Si verifica che l'utente possa caricare i dati per la visualizzazione tramite file CSV,
        S,
        RFO1 RFO1.1
    },
    {   
        TSOF1.1,
        Si verifica che l'utente possa inserire i dati tramite query al database,
        S,
        RFO1.2
    },
    {   
        TSOF1.2,
        Si verifica che appaia un messaggio d'errore in caso di errore nell'inserimento dei dati,
        NI,
        RFO1.3
    },
    {   
        TSOF1.3,
        Si verifica che i dati caricati provenienti da qualsiasi fonte siano convertiti in JSON,
        S,
        RFO1.4
    },
    {   
        TSOF2,
        Si verifica la possibilità di scelta nella visualizzazione dei dati. In particolare devono essere presenti le seguenti visualizzazioni:
        \unexpanded{
        \begin{itemize}
            \item Scatter Plot Matrix
            \item Heatmap
            \item Force Field
            \item Linear Projection
        \end{itemize}},
        S,
        RFO2 RFO2.1 RFO2.2 RFO2.4 RFO2.5
    },
    {   
        TSFF2.1,
        Si verifica che nella scelta per la visualizzazione deve esserci: Correlation Heatmap,
        S,
        RFF2.3
    },
    {   
        TSFF2.2,
        Si verifica che nella scelta per la visualizzazione deve esserci: Parallel Coordinates,
        NI,
        RFF2.6
    },
    {   
        TSOF3,
        Si verifica che l'utente possa manipolare i dati nel dataset,
        NI,
        RFO3
    },
    {   
        TSOF3.1,
        SI verifica che l'utente possa selezionare le variabili target tra le feauteres del dataset,
        S,
        RFO3.1
    },
    {   
        TSOF3.2,
        SI verifica che l'utente possa selezionare le features a cui è interessato,
        S,
        RFO3.2
    },
    {   
        TSOF3.3,
        SI verifica che l'utente possa normalizzare i dati presenti nel dataset nei seguenti modi:
        \unexpanded{
        \begin{itemize}
            \item Globalmente
            \item Per riga
            \item Per colonna
        \end{itemize}},
        NI,
        RFO3.3 RFO3.3.1 RFO 3.3.2 RFO 3.3.3
    },
    {   
        TSFF3.4,
        SI verifica che l'utente possa selezionare le righe del dataset a cui è interessato,
        NI,
        RFF3.4
    },
    {   
        TSOF4,
        Si verifica che l'utente possa modificare le impostazioni che influenzano la visualizzazione,
        NI,
        RFO4
    },
    {   
        TSOF4.1,
        Si verifica che l'utente possa assegnare una classe di visualizzazione ad ogni variabile target selezionata,
        NI,
        RFO4.1
    },
    {   
        TSOF4.2,
        Si verifica che l'utente possa modificare il range dei dati da considerare nella Heatmap,
        NI,
        RFO4.2
    },
    {   
        TSOF4.3,
        Si verifica che l'utente possa assegnare un colore al range di un Heatmap,
        NI,
        RFO4.3
    },
    {   
        TSOF4.4,
        Si verifica che l'utente che ha selezionato la visualizzazione Heatmap possa ordinare in ordine alfabetico e in cluster il dataset,
        S,
        RFO4.4 RFO4.4.1 RFO4.4.2
    },
    {   
        TSOF5,
        Si verifica che l'utente possa calcolare la matrice di distanza se ha scelto la visualizzazione Heatmap o Forcefield,
        S,
        RFO5
    },    
    {   
        TSOF5.1,
        Si verifica che siano disponibili i seguenti tipi di distanze per calcolare la matrice di distanza:
        \unexpanded{
            \begin{itemize}
                \item euclidea;
                \item Manhattan.
            \end{itemize}
        }
        E che siano disponibili per Forcefield e Heatmap,
        S,
        RFO5.1 RFO5.1.1 RFO5.1.2
    },
    {   
        TSOF6,
        Si verifica che quando viene scelta la visualizzazione Linear Projection sia calcolata la riduzione delle componenti,
        S,
        RFO6
    },
    {   
        TSFF6.1,
        Si verifica che l'utente possa scegliere tra i seguenti algoritmi per la riduzione dei componenti:
        \unexpanded{
        \begin{itemize}
            \item PCA;
            \item UMAP;
            \item t-SNE.
        \end{itemize}},
        NI,
        RFO6.1 RFF6.1.1 RFF6.1.2 6.1.3
    },
    {   
        TSOF7,
        Si verifica che la visualizzazione creata dal sistema sia salvabile in formato PNG,
        NI,
        RFO7
    },
    {   
        TSOF8,
        Si verifica che un messaggio di errore compaia a schermo se i files sono caricati scorrettamente,
        NI,
        RFO8
    },
    }
    \newcommand*\testdescription{}
\foreach \x [count=\nj] in \testspecification
{
    \foreach \y [count=\ni] in \x
    {
        \ifnum\ni=4
            \xappto\testdescription{\y}
            \gappto\testdescription{\\}
            \gappto\testdescription{\hline}
        \else
            \xappto\testdescription{\y&}
        \fi
    }
}

% Impostazioni della tabella
\tabulinesep = 2mm % padding
\taburowcolors [1] 2{pari .. dispari} % colori delle righe
\begin{longtabu} to \textwidth {| X[0.1,c m] | X[0.2,c m] | X[0.1,c m] | X[0.1,c m]|} % larghezza delle colonne
\hline
\rowcolor{header} % colore dell'header

\textbf{ID test} & \textbf{Descrizione} & \textbf{Esito} & \textbf{ID requisito} \\
\hline
\testdescription

\end{longtabu}

\undef\testdescription{}

    %endTable
    
\subsubsection{Test di sistema per i requisiti di qualità}
    %startTable
    \def\testspecification{
    {   
        TSOQ1,
        Si verifica che sia stato prodotto un manuale d'uso per l'utente,
        S,
        RQO1
    },
    {   
        TSOQ2,
        Si verifica che sia stato prodotto un manuale manutentore,
        S,
        RQO2
    },
    {   
        TSOQ3,
        Si verifica che il codice sia stato pubblicato su un repository pubblico,
        S,
        RQO3
    },
    {   
        TSOQ4,
        Si verifica che il codice segua le norme di stile specificate nel documento Norme di Progetto,
        NI,
        RQO4
    },
    {   
        TSOQ5,
        Si verifica che nella codifica si eviti l'utilizzo di chiamate ricorsive dove è possibile,
        S,
        RQO5
    },
      }
    \newcommand*\testdescription{}
\foreach \x [count=\nj] in \testspecification
{
    \foreach \y [count=\ni] in \x
    {
        \ifnum\ni=4
            \xappto\testdescription{\y}
            \gappto\testdescription{\\}
            \gappto\testdescription{\hline}
        \else
            \xappto\testdescription{\y&}
        \fi
    }
}

% Impostazioni della tabella
\tabulinesep = 2mm % padding
\taburowcolors [1] 2{pari .. dispari} % colori delle righe
\begin{longtabu} to \textwidth {| X[0.1,c m] | X[0.2,c m] | X[0.1,c m] | X[0.1,c m]|} % larghezza delle colonne
\hline
\rowcolor{header} % colore dell'header

\textbf{ID test} & \textbf{Descrizione} & \textbf{Esito} & \textbf{ID requisito} \\
\hline
\testdescription

\end{longtabu}

\undef\testdescription{}

    %endTable  
        
\subsubsection{Test di sistema per i requisiti di vincolo}
    %startTable
    \def\testspecification{
    {   
        TSOV1,
        Si verifica che il codice sorgente dell'applicazione sia open source,
        NI,
        RVO1
    },
    {   
        TSOV2,
        Si verifica che l'applicazione sia sviluppata in Javascript con l'utilizzo della libreria d3.js,
        S,
        RVD2 RVD2.1
    },
    {   
        TSDV2.1,
        Si verifica che il backend dell'applicazione sia sviluppato con node.js con l'utilizzo del framework Express,
        S,
        RVD2.2
    },
    {   
        TSDV2.2,
        Si verifica che il frontend dell'applicazione sia sviluppato con React con l'utilizzo del framework Ant Design,
        S,
        RVD2.3
    },
    {   
        TSOV3,
        Si verifica che i dati siano convertibili in JSON,
        S,
        RVO3
    },
    {   
        TSOV4,
        Si verifica che nella visualizzazione Scatter Plot Matrix si possano visualizzare al massimo 5 features,
        S,
        RVO4
    },
    {   
        TSDV5,
        Si verifica che la libreria per PCA sia ml-pca,
        S,
        RVD5
    },
    {   
        TSDV6,
        Si verifica che la libreria per Umap sia tsne-js,
        NI,
        RVD6
    },
    {   
        TSDV7,
        Si verifica che la libreria per t-SNE sia tsne-js,
        NI,
        RVD7
    },
    {   
        TSDV8,
        Si verifica che la libreria per le distanze sia ml-distance,
        S,
        RVD8
    },
    {   
        TSDV9,
        Si verifica che la libreria per la matrice di correlazione sia jeezy,
        NI,
        RVD9
    },
      }
    \newcommand*\testdescription{}
\foreach \x [count=\nj] in \testspecification
{
    \foreach \y [count=\ni] in \x
    {
        \ifnum\ni=4
            \xappto\testdescription{\y}
            \gappto\testdescription{\\}
            \gappto\testdescription{\hline}
        \else
            \xappto\testdescription{\y&}
        \fi
    }
}

% Impostazioni della tabella
\tabulinesep = 2mm % padding
\taburowcolors [1] 2{pari .. dispari} % colori delle righe
\begin{longtabu} to \textwidth {| X[0.1,c m] | X[0.2,c m] | X[0.1,c m] | X[0.1,c m]|} % larghezza delle colonne
\hline
\rowcolor{header} % colore dell'header

\textbf{ID test} & \textbf{Descrizione} & \textbf{Esito} & \textbf{ID requisito} \\
\hline
\testdescription

\end{longtabu}

\undef\testdescription{}

    %endTable
     
\subsection{Test di accettazione}
    \subsubsection{Test di accettazione per i requisiti funzionali}
        \hphantom{}
        %startTable
        \def\testspecification{
        {   
            TAOF1,
            Si verifica che l'utente possa caricare i dati per la visualizzazione tramite file CSV,
            S,
            RFO1 RFO1.1
        },
        {   
            TAOF1.1,
            Si verifica che l'utente possa inserire i dati tramite query al database,
            S,
            RFO1.2
        },
        {   
            TAOF1.2,
            Si verifica che appaia un messaggio d'errore in caso di errore nell'inserimento dei dati,
            NI,
            RFO1.3
        },
        {   
            TAOF1.3,
            Si verifica che i dati caricati provenienti da qualsiasi fonte siano convertiti in JSON,
            S,
            RFO1.4
        },
        {   
            TAOF2,
            Si verifica la possibilità di scelta nella visualizzazione dei dati. In particolare devono essere presenti le seguenti visualizzazioni:
            \unexpanded{
            \begin{itemize}
                \item Scatter Plot Matrix
                \item Heatmap
                \item Force Field
                \item Linear Projection
            \end{itemize}},
            S,
            RFO2 RFO2.1 RFO2.2 RFO2.4 RFO2.5
        },
        {   
            TAOF3,
            Si verifica che l'utente possa manipolare i dati nel dataset,
            NI,
            RFO3
        },
        {   
            TAOF3.1,
            SI verifica che l'utente possa selezionare le variabili target tra le feauteres del dataset,
            S,
            RFO3.1
        },
        {   
            TAOF3.2,
            SI verifica che l'utente possa selezionare le features a cui è interessato,
            S,
            RFO3.2
        },
        {   
            TAOF3.3,
            SI verifica che l'utente possa normalizzare i dati presenti nel dataset nei seguenti modi:
            \unexpanded{
            \begin{itemize}
                \item Globalmente
                \item Per riga
                \item Per colonna
            \end{itemize}},
            NI,
            RFO3.3 RFO3.3.1 RFO3.3.2 RFO3.3.3
        },
        {   
            TAOF4,
            Si verifica che l'utente possa modificare le impostazioni che influenzano la visualizzazione,
            NI,
            RFO4
        },
        {   
            TAOF4.1,
            Si verifica che l'utente possa assegnare una classe di visualizzazione ad ogni variabile target selezionata,
            NI,
            RFO4.1
        },
        {   
            TAOF4.2,
            Si verifica che l'utente possa modificare il range dei dati da considerare nella Heatmap,
            NI,
            RFO4.2
        },
        {   
            TAOF4.3,
            Si verifica che l'utente possa assegnare un colore al range di un Heatmap,
            NI,
            RFO4.3
        },
        {   
            TAOF4.4,
            Si verifica che l'utente che ha selezionato la visualizzazione Heatmap possa ordinare in ordine alfabetico e in cluster il dataset,
            S,
            RFO4.4 RFO4.4.1 RFO4.4.2
        },
        {   
            TAOF5,
            Si verifica che l'utente possa calcolare la matrice di distanza se ha scelto la visualizzazione Heatmap o Forcefield,
            S,
            RFO5
        },    
        {   
            TAOF5.1,
            Si verifica che siano disponibili i seguenti tipi di distanze per calcolare la matrice di distanza:
            \unexpanded{
                \begin{itemize}
                    \item euclidea;
                    \item Manhattan.
                \end{itemize}
            }
            E che siano disponibili per Forcefield e Heatmap,
            S,
            RFO5.1 RFO5.1.1 RFO5.1.2
        },
        {   
            TAOF6,
            Si verifica che quando viene scelta la visualizzazione Linear Projection sia calcolata la riduzione delle componenti,
            S,
            RFO6
        },
        {   
            TAFF6.1,
            Si verifica che l'utente possa scegliere tra i seguenti algoritmi per la riduzione dei componenti:
            \unexpanded{
            \begin{itemize}
                \item PCA;
                \item UMAP;
                \item t-SNE.
            \end{itemize}}
            Due o più di questi algoritmi devono essere implementati,
            NI,
            RFO6.1 RFF6.1.1 RFF6.1.2 6.1.3
        },
        {   
            TAOF7,
            Si verifica che la visualizzazione creata dal sistema sia salvabile in formato PNG,
            NI,
            RFO7
        },
        {   
            TAOF8,
            Si verifica che un messaggio di errore compaia a schermo se i files sono caricati scorrettamente,
            NI,
            RFO8
        },
        }
        \newcommand*\testdescription{}
\foreach \x [count=\nj] in \testspecification
{
    \foreach \y [count=\ni] in \x
    {
        \ifnum\ni=4
            \xappto\testdescription{\y}
            \gappto\testdescription{\\}
            \gappto\testdescription{\hline}
        \else
            \xappto\testdescription{\y&}
        \fi
    }
}

% Impostazioni della tabella
\tabulinesep = 2mm % padding
\taburowcolors [1] 2{pari .. dispari} % colori delle righe
\begin{longtabu} to \textwidth {| X[0.1,c m] | X[0.2,c m] | X[0.1,c m] | X[0.1,c m]|} % larghezza delle colonne
\hline
\rowcolor{header} % colore dell'header

\textbf{ID test} & \textbf{Descrizione} & \textbf{Esito} & \textbf{ID requisito} \\
\hline
\testdescription

\end{longtabu}

\undef\testdescription{}

        %endTable
    \subsubsection{Test di accettazione per i requisiti di qualità}
        %startTable
        \def\testspecification{
            {   
                TAOQ1,
                Si verifica che sia stato prodotto un manuale d'uso per l'utente,
                S,
                RQO1
            },
            {   
                TAOQ2,
                Si verifica che sia stato prodotto un manuale manutentore,
                S,
                RQO2
            },
            {   
                TAOQ3,
                Si verifica che il codice sia stato pubblicato su un repository pubblico,
                S,
                RQO3
            },
        }
        \newcommand*\testdescription{}
\foreach \x [count=\nj] in \testspecification
{
    \foreach \y [count=\ni] in \x
    {
        \ifnum\ni=4
            \xappto\testdescription{\y}
            \gappto\testdescription{\\}
            \gappto\testdescription{\hline}
        \else
            \xappto\testdescription{\y&}
        \fi
    }
}

% Impostazioni della tabella
\tabulinesep = 2mm % padding
\taburowcolors [1] 2{pari .. dispari} % colori delle righe
\begin{longtabu} to \textwidth {| X[0.1,c m] | X[0.2,c m] | X[0.1,c m] | X[0.1,c m]|} % larghezza delle colonne
\hline
\rowcolor{header} % colore dell'header

\textbf{ID test} & \textbf{Descrizione} & \textbf{Esito} & \textbf{ID requisito} \\
\hline
\testdescription

\end{longtabu}

\undef\testdescription{}

        %endTable  
            
    \subsubsection{Test di accettazione per i requisiti di vincolo}
        %startTable
        \def\testspecification{
            {   
                TAOV1,
                Si verifica che il codice sorgente dell'applicazione sia open source,
                NI,
                RVO1
            },
            {   
                TAOV2,
                Si verifica che l'applicazione sia sviluppata in Javascript con l'utilizzo della libreria d3.js,
                S,
                RVD2 RVD2.1
            },
            {   
                TADV2.1,
                Si verifica che il backend dell'applicazione sia sviluppato con node.js con l'utilizzo del framework Express,
                S,
                RVD2.2
            },
            {   
                TADV2.2,
                Si verifica che il frontend dell'applicazione sia sviluppato con React con l'utilizzo del framework Ant Design,
                S,
                RVD2.3
            },
        }
        \newcommand*\testdescription{}
\foreach \x [count=\nj] in \testspecification
{
    \foreach \y [count=\ni] in \x
    {
        \ifnum\ni=4
            \xappto\testdescription{\y}
            \gappto\testdescription{\\}
            \gappto\testdescription{\hline}
        \else
            \xappto\testdescription{\y&}
        \fi
    }
}

% Impostazioni della tabella
\tabulinesep = 2mm % padding
\taburowcolors [1] 2{pari .. dispari} % colori delle righe
\begin{longtabu} to \textwidth {| X[0.1,c m] | X[0.2,c m] | X[0.1,c m] | X[0.1,c m]|} % larghezza delle colonne
\hline
\rowcolor{header} % colore dell'header

\textbf{ID test} & \textbf{Descrizione} & \textbf{Esito} & \textbf{ID requisito} \\
\hline
\testdescription

\end{longtabu}

\undef\testdescription{}

        %endTable
         
    