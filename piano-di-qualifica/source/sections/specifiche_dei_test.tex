
\section{Specifica dei test}

Sono stati determinati, come descritto anche nel documento \textit{Norme di Progetto 1.0.0}, 4 tipi di test ovvero: \textit{Test di unità, Test di Integrazione, Test di sistema, Test di accettazione} e \textit{Test di regressione}. Si adotterà quindi il \textbf{Modello a V}\glo{} che permette lo sviluppo dei test in parallelo alle attività di analisi e progettazione. Così facendo i test sviluppati saranno in grado di verificare sia la correttezza del programma software sia l'implementazione. Sara disponibile inoltre una tabella degli esiti dei test per una facile consultazione.

Per i test valgono le seguenti sigle:

    \begin{itemize}
        \item \textbf{I}: per indicare che il è effettivamente implementato.
        \item \textbf{NI}: per indicare che il test è ancora non implementato.
    \end{itemize}
    
Mentre per lo stato valgono le seguenti sigle:

    \begin{itemize}
        \item \textbf{S}: il test ha esito soddisfacente.
        \item \textbf{NS}: il test ha esito non soddisfacente.
    \end{itemize}
    

 
\subsection{Test di Sistema}

    Per rispettare i requisiti identificati nel documento \textit{Analisi dei Requisiti 1.0.0} e per garantire il funzionamento del prodotto si eseguono i seguenti Test di sistema:\\
    
\subsubsection{Test di sistema per i requisiti funzionali}
    \hphantom{}

    %startTable
    \def\testspecification{
    {   
        TSOF1,
        Si verifica che l'utente possa di caricare i dati per la visualizzazione tramite file CSV,
        NI,
        RF01 RFO1.1
    },
    {   
        TSOF1.2,
        Si verifica che l'utente possa inserire i dati tramite query al database,
        NI,
        RFO1.2
    },
    {   
        TSOF1.3,
        Si verifica che appaia un messaggio d'errore in caso di errore nell'inserimento dei dati,
        NI,
        RFO1.3
    },
    {   
        TSOF1.4,
        Si verifica che la conversione in JSON dei dati avvenga in modo corretto,
        NI,
        RFO1.4
    },
     {   
        TSOF2,
        Si verifica la possibilità di scelta nella visualizzazione dei dati. In particolare devono essere presenti le seguenti visualizzazioni:
        \unexpanded{
        \begin{itemize}
            \item Scatter Plot Matrix
            \item Heatmap
            \item Correlation Heatmap
            \item Force Field
            \item Linear Projection
        \end{itemize}},
        NI,
        RFO2 RFO2.1 RFO2.2 RFO2.3 RFO2.4 RFO2.5
    },
    {   TSOF2.2.1,
        Si verifica che l'utente possa scegliere il colore della sfumatura nella visualizzazione Heatmap,
        NI,
        RFO2.2.1
    },
    {   
        TSFF2.6,
        Si verifica che nella scelta per la visualizzazione deve esserci: Parallel Coordinates,
        NI,
        RFF2.6
    },
    {   
        TSOF3,
        Si verifica che l'utente possa scegliere quali labels far vedere nella visualizzazione,
        NI,
        RFO3
    },
    {   
        TSOF3.1,
        Si verifica che l'utente possa scegliere come visualizzare le labels presenti nel dataset,
        NI,
        RFO3.1
    },
    {   
        TSOF4,
        Si verifica la possibilità di scartare features e di inserire features scartate,
        NI,
        RFO4
    },
    {   
        TSOF5,
        Si verifica che l'utente possa modificare il range dei dati all'interno della visualizzazione Heatmap,
        NI,
        RFO5 RFO5.1
    },
    {   
        TSOF5.2,
        Si verifica la possibilità di normalizzazione del database in modo globale{,} per riga o per colonna,
        NI,
        RFO5.2 RFO5.2.1 RFO5.2.2
    },
    {   
        TSOF5.3,
        Dopo la selezione dell'Heatmap si verifica la possibilità di ordinare il dataset in ordine alfabetico e in cluster,
        NI,
        RFO5.3 RFO5.3.1 RFO5.3.2
    },
    {   
        TSOF5.4,
        Si verifica la possibilità di assegnazione del colore al range nella visualizzazione Heatmap,
        NI,
        RFO5.4
    },
    {   
        TSOF6,
        Si verifica la possibilità di scelta nell'Heatmap di una matrice a distanza per il calcolo della distanza,
        NI,
        RFO6
    },
    {   
        TSFF6.1,
        Si verifica la possibilità di scelta di una funzione di distanza nelle visualizzazioni Heatmap e nel Force Field. In particolare la distanza Euclidea,
        NI,
        RFF6.1 RFF6.1.1
    },
    {   
        TSFF6.2,
        Si verifica la possibilità di scelta di una funzione di distanza nelle visualizzazioni Heatmap e nel Force Field. In particolare la distanza di Manhattan,
        NI,
        RFF6.1.2
    },
    {   
        TSOF7,
        Si verifica la possibilità di scelta di un algoritmo per la riduzione dei componenti nella visualizzazione Linear Projection,
        NI,
        RFO7
    },
    {   
        TSFF7.1,
        Si verifica che l'utente possa scegliere tra i seguenti algoritmi per la riduzione dei componenti:
        \unexpanded{
        \begin{itemize}
            \item PCA
            \item UMAP
            \item t-SNE
        \end{itemize}},
        NI,
        RFF7.1 RFF7.2 RFF7.3
    },
    {   
        TSOF8.1,
        Si verifica che il sistema elabori i dati con impostazioni di default,
        NI,
        RFO8.1
    },
    {   
        TSOF8.2,
        Si verifica che il sistema elabori i dati con impostazioni di personalizzate dell'utente,
        NI,
        RFO8.1
    },
    {   
        TSOF9,
        Si verifica che la visualizzazione creata dal sistema sia visibile a schermo,
        NI,
        RFO9},
    {   
        TSOF9.2,
        Si verifica che la visualizzazione creata dal sistema sia salvabile in formato PNG,
        NI,
        RFO9.2
    },
    {   
        TSOF10,
        Si verifica che un messaggio di errore compaia a schermo se i files sono caricati scorrettamente,
        NI,
        RFO10
    },
    }
    \newcommand*\testdescription{}
\foreach \x [count=\nj] in \testspecification
{
    \foreach \y [count=\ni] in \x
    {
        \ifnum\ni=4
            \xappto\testdescription{\y}
            \gappto\testdescription{\\}
            \gappto\testdescription{\hline}
        \else
            \xappto\testdescription{\y&}
        \fi
    }
}

% Impostazioni della tabella
\tabulinesep = 2mm % padding
\taburowcolors [1] 2{pari .. dispari} % colori delle righe
\begin{longtabu} to \textwidth {| X[0.1,c m] | X[0.2,c m] | X[0.1,c m] | X[0.1,c m]|} % larghezza delle colonne
\hline
\rowcolor{header} % colore dell'header

\textbf{ID test} & \textbf{Descrizione} & \textbf{Esito} & \textbf{ID requisito} \\
\hline
\testdescription

\end{longtabu}

\undef\testdescription{}

    %endTable
    
\subsubsection{Test di sistema per i requisiti di qualità}
    %startTable
    \def\testspecification{
    {   
        TSOQ1,
        Si verifica che i dati ricevuti da fonti esterne non siano modificati o errati,
        NI,
        RQO1
    },
    {   
        TSOQ2,
        Si verifica che il numero di errori rilevati nel codice rispetti il valore specificato nella metrica: Densità di errori,
        NI,
        RQO2
    },
    {   
        TSOQ3,
        Si verifica che i messaggi di errore e avviso siano chiari da comprendere per l'utente con l'uso della metrica: Qualità della messaggistica,
        NI,
        RQO3
    },
    {   
        TSOQ4,
        Si verifica che il numero di click per svolgere ogni task sia minore del valore sufficiente data dalla metrica: Numero di click,
        NI,
        RQO4
    },
    {   
        TSOQ5,
        Si verifica che la profondità della struttura dell'applicativo sia minore del valore sufficiente dato dalla metrica: Site depth,
        NI,
        RQO5
    },
    {   
        TSOQ6,
        Si verifica che la durata di ogni task sia minore del valore sufficiente specificato dalla metrica: Response time,
        NI,
        RQO6
    },
    {   
        TSOQ7,
        Si verifica che la quantità dei possibili percorsi di branching rispetti la metrica: Complessità ciclomatica,
        NI,
        RQO7
    },
    {   
        TSOQ8,
        Si verifica che la percentuale di test indipendenti rispetti la metrica: Indipendenza dei test,
        NI,
        RQO8
    },
    {   
        TSOQ9,
        Si verifica che l'indice di comprensione del codice ovvero{,} il rapporto tra codice e commento{,} rispetti il valore sufficiente dato dalla metrica: Facilità di comprensione,
        NI,
        RQO9
    },
    {   
        TSOQ10,
        Si verifica che il numero di procedure che chiamano un'altra procedura sia minore del valore sufficiente dato dalla metrica: Structural Fan-In,
        NI,
        RQO10
    },
    {   
        TSOQ11,
        Si verifica che il numero di procedure che necessita ogni altra procedura sia inferiore al valore sufficiente dato dalla metrica: Structural Fan-Out,
        NI,
        RQO11
    },
      }
    \newcommand*\testdescription{}
\foreach \x [count=\nj] in \testspecification
{
    \foreach \y [count=\ni] in \x
    {
        \ifnum\ni=4
            \xappto\testdescription{\y}
            \gappto\testdescription{\\}
            \gappto\testdescription{\hline}
        \else
            \xappto\testdescription{\y&}
        \fi
    }
}

% Impostazioni della tabella
\tabulinesep = 2mm % padding
\taburowcolors [1] 2{pari .. dispari} % colori delle righe
\begin{longtabu} to \textwidth {| X[0.1,c m] | X[0.2,c m] | X[0.1,c m] | X[0.1,c m]|} % larghezza delle colonne
\hline
\rowcolor{header} % colore dell'header

\textbf{ID test} & \textbf{Descrizione} & \textbf{Esito} & \textbf{ID requisito} \\
\hline
\testdescription

\end{longtabu}

\undef\testdescription{}

    %endTable  
        
\subsubsection{Test di sistema per i requisiti di vincolo}
    %startTable
    \def\testspecification{
    {   
        TSOV1,
        Si verifica che il codice sorgente dell'applicazione sia open source,
        NI,
        RVO1
    },
    {   
        TSDV,
        Si verifica che l'applicazione sia sviluppata in Javascript con l'utilizzo della libreria d3.js,
        NI,
        RVD2 RVD2.1
    },
    {   
        TSDV2.2,
        Si verifica che il backend dell'applicazione sia sviluppato con node.js con l'utilizzo del framework Express,
        NI,
        RVD2.2
    },
    {   
        TSDV2.3,
        Si verifica che il frontend dell'applicazione sia sviluppato con React con l'utilizzo del framework Ant Design,
        NI,
        RVD2.3
    },
    {   
        TSOV3,
        Si verifica che i dati siano convertibili in JSON,
        NI,
        RVO3
    },
    {   
        TSOV4,
        Si verifica che nella visualizzazione Scatter Plot Matrix si possano visualizzare al massimo 5 features,
        NI,
        RVO4
    },
    {   
        TSDV5,
        Si verifica che la libreria per PCA sia ml-pca,
        NI,
        RVD5
    },
    {   
        TSDV6,
        Si verifica che la libreria per Umap sia tsne-js,
        NI,
        RVD6
    },
    {   
        TSDV7,
        Si verifica che la libreria per t-SNE sia tsne-js,
        NI,
        RVD7
    },
    {   
        TSDV8,
        Si verifica che la libreria per le distanze sia ml-distance,
        NI,
        RVD8
    },
    {   
        TSDV9,
        Si verifica che la libreria per la matrice di correlazione sia jeezy,
        NI,
        RVD9
    },
      }
    \newcommand*\testdescription{}
\foreach \x [count=\nj] in \testspecification
{
    \foreach \y [count=\ni] in \x
    {
        \ifnum\ni=4
            \xappto\testdescription{\y}
            \gappto\testdescription{\\}
            \gappto\testdescription{\hline}
        \else
            \xappto\testdescription{\y&}
        \fi
    }
}

% Impostazioni della tabella
\tabulinesep = 2mm % padding
\taburowcolors [1] 2{pari .. dispari} % colori delle righe
\begin{longtabu} to \textwidth {| X[0.1,c m] | X[0.2,c m] | X[0.1,c m] | X[0.1,c m]|} % larghezza delle colonne
\hline
\rowcolor{header} % colore dell'header

\textbf{ID test} & \textbf{Descrizione} & \textbf{Esito} & \textbf{ID requisito} \\
\hline
\testdescription

\end{longtabu}

\undef\testdescription{}

    %endTable
    
\subsection{Test di unità}
    I test di Unità hanno l'obbiettivo di verificare il funzionamento di ogni unità che compone l'applicazione. Questi test verrano definiti in una fase successiva, in particolare nella fase in cui verrà richiesta l'implementazione.
    
\subsection{Test di Integrazione}
    I test d'integrazione hanno l'obbiettivo di assicurare il funzionamento tra le varie unità, in modo da integrarsi tra loro. Questi test verrano definiti in una fase successiva, in particolare nella fase in cui verrà richiesta l'implementazione.
    
\subsection{Test di accettazione}
    I test di accettazione vengono svolti insieme al proponente per assicurare il gradimento dell'utente. Questi test verrano definiti in una fase successiva, in particolare nella fase in cui verrà richiesta l'implementazione.
    
\subsection{Test di regressione}
    I test di regressione vengono apportati ogni volta che viene apportata una modifica a un'unità del software. Questi test verrano definiti in una fase successiva, in particolare nella fase in cui verrà richiesta l'implementazione.

