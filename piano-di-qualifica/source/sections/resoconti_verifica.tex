\newpage
\section{Resoconti di verifica ed esiti delle revisioni}

In questa sezione vengono mostrati i valori delle metriche calcolate a termine del periodo di progettazione di dettaglio e codifica. Saranno mostrati sia i valori che rientrano nel range prestabilito sia quelli che non rientrano, nel secondo caso verranno segnalati come problemi nelle conclusioni che saranno divise per fasi.

\subsection{Metriche dei processi primari}
\subsubsection{percentuale di requisiti soddisfatti}

    \begin{figure}[H]
        \centering
        \includegraphics[width=15 cm]{source/sections/images/num-requisiti.png}
        \caption{Grafico dei requisiti}
    \end{figure}

    \begin{figure}[H]
        \centering
        \includegraphics[width=15 cm]{source/sections/images/percentuale-requisiti.png}
        \caption{percentuale dei requisiti soddisfatti}
    \end{figure}

\subsubsection{Metriche di pianificazione}
    Le metriche di pianificazione che mostrano il rispettare dei costi e dei tempi sono stati calcolati in questa versione del documento in 4 gruppi denotati dalle seguenti sigle:
    \begin{itemize}
        \item \textbf{An}: Periodo di analisi, ovvero dal 26-11-2020 al 10-01-2021
        \item \textbf{CC}: periodo di consolidamento dei requisiti, ovvero dal 12-01-2021 al 18-01-2021
        \item \textbf{PA}: Progettazione architetturale, ovvero dal 19-01-2021 al 01-03-2021, che rappresenta la data progettata per la consegna
        \item \textbf{PAS}: Progettazione architetturale con consegna "a sportell0", ovvero dal 02-03-2021 al 09-03-2021, che rappresenta la data spostata con consegna il 10 marzo
        \item \textbf{PDC}:Progettazione di dettaglio e codifica
    \end{itemize}
    
    
    \begin{longtabu} to \textwidth {| X[0.1,c m] | X[0.1,c m]| X[0.1,c m]| X[0.1,c m]| X[0.1,c m]| X[0.1,c m] |}
        \hline
        \rowcolor{header}
        \textbf{Fase} &
        \textbf{EV} &
        \textbf{PV} &
        \textbf{AC} &
        \textbf{SV} &
        \textbf{CV} \\
        \hline
        An & - & - & - & - & -  \\ 
        \hline
        CR & 734 & 734 & 650 & 0 & 84 \\
        \hline
        PA & 2588 & 3624 & 3439 & -1035 & -841\\
        \hline
        PAS & 1035 & 0 & 966 & 0 & 69 \\
        \hline
        PDC & 5935 & 5935 & 5935 & 0 & 69 \\
        \hline 
        \end{longtabu}


        \begin{figure}[H]
            \centering
            \includegraphics[width=13 cm]{source/sections/images/Earned_value.png}
            \caption{Grafico dei valori dell'Earned value}
        \end{figure}

        \begin{figure}[H]
            \centering
            \includegraphics[width=13 cm]{source/sections/images/planned_value.png}
            \caption{Grafico dei valori del Planned value}
        \end{figure}


        \begin{figure}[H]
            \centering
            \includegraphics[width=13 cm]{source/sections/images/actual_cost.png}
            \caption{Grafico dei valori dell'Actual cost}
        \end{figure}


        \begin{figure}[H]
            \centering
            \includegraphics[width=13 cm]{source/sections/images/schedule_variance.png}
            \caption{Grafico dei valori dello Schedule variance}
        \end{figure}


        \begin{figure}[H]
            \centering
            \includegraphics[width=10 cm]{source/sections/images/cost_variance.png}
            \caption{Grafico dei valori del Cost variance}
        \end{figure}

\newpage
\subsection{Metriche dei processi di supporto}
\subsubsection{Valori dell'indice di Gulpease}

Per ogni documento stilato è stato calcolato l'indice di Gulpease\glo{}. I valori sono dati dalla seguente tabella ed il rispettivo grafico con valore sufficiente >40 e valore ottimo >80.

\hphantom{}
\tabulinesep = 2mm % padding
\taburowcolors [1] 2{pari .. dispari} % colori delle righe

\begin{longtabu} to \textwidth {| X[0.2,c m]  | X[0.1,c m] | X[0.1,c m]| X[0.1,c m] | X[0.1,c m] |}
\hline
\rowcolor{header}
\textbf{Data del calcolo} &  
\textbf{Norme di Progetto} & 
\textbf{Analisi dei Requisiti} & 
\textbf{Piano di Qualifica} & 
\textbf{Piano di Progetto} \\
\hline

\multirow[c]{2}{*}{2020-12-03} & v0.1.0 & v0.2.4 & v0.2.1 & v0.1.0 \\
\cline{2-5} 
& 65 & 61 & 80 & 91 \\ 
\hline
\multirow[c]{2}{*}{2021-01-10} & v1.0.0 & v1.0.0 & v1.0.0 & v1.0.0 \\ 
\cline{2-5} 
 & 68 & 69 & 70 & 79 \\ 
\hline
\multirow[c]{2}{*}{2021-02-05}  & v1.0.1 & v1.0.5 & v1.0.2 & v1.0.2 \\ 
\cline{2-5} 
 & 75 & 74 & 64 & 83 \\ 
\hline
\multirow[c]{2}{*}{2021-02-20}  & v1.1.3 & v1.1.0 & v1.1.2 & v1.0.3 \\ 
\cline{2-5} 
 & 82 & 68 & 72 & 69 \\ 
\hline
\multirow[c]{2}{*}{2021-03-09} & v2.0.0 & v2.0.0 & v2.0.0 & v2.0.0 \\ 
\cline{2-5} 
 & 74 & 70 & 70 & 74 \\ 
 \hline
 \multirow[c]{2}{*}{2021-03-26}  & v2.0.2 & v2.1.0 & v2.0.1 & v2.0.3 \\ 
 \cline{2-5} 
  & 73 & 69 & 70 & 73 \\ 
 \hline
 \multirow[c]{2}{*}{2021-03-4}  & v2.2.0 & v2.1.3 & v2.2.1 & v2.2.3 \\ 
 \cline{2-5} 
  & 70 & 65 & 71 & 70 \\ 
 \hline
 \multirow[c]{2}{*}{2021-04-20}  & v3.0.0 & v3.0.0 & v3.0.0 & v3.0.0 \\ 
 \cline{2-5} 
  & 68 & 64 & 72 & 68 \\ 
 \hline
 \multirow[c]{2}{*}{2021-05-03}  & v3.0.0 & v3.0.0 & v3.1.0 & v3.0.3 \\ 
 \cline{2-5} 
  & - & - & - & - \\ 
 \hline
 \multirow[c]{2}{*}{2021-05-14}  & v3.0.0 & v3.0.0 & v3.2.2 & v3.0.4 \\ 
 \cline{2-5} 
  & - & - & - & - \\ 
 \hline
 \multirow[c]{2}{*}{2021-05-20}  & v3.0.0 & v3.0.0 & v4.0.0 & v4.0.0 \\ 
 \cline{2-5} 
  & - & - & - & - \\ 
 \hline
\end{longtabu}


\begin{figure}[H]
    \centering
    \includegraphics[width=13 cm]{source/sections/images/IdG_NdP.png}
    \caption{Indice di Gulpease - Norme di Progetto}
\end{figure}

\begin{figure}[H]
    \centering
    \includegraphics[width=13 cm]{source/sections/images/IdG_AR.png}
    \caption{Indice di Gulpease - Analisi dei Requisiti}
\end{figure}
\begin{figure}[H]
    \centering
    \includegraphics[width=13 cm]{source/sections/images/IdG_PdQ.png}
    \caption{Indice di Gulpease - Piano di Qualifica}
\end{figure}
\begin{figure}[H]
    \centering
    \includegraphics[width=13 cm]{source/sections/images/IdG_PdP.png}
    \caption{Indice di Gulpease - Piano di Progetto}
\end{figure}

\subsubsection{Errori ortografici}

Per quanto riguarda gli errori ortografici, oltre alla revisione fatta dai membri del gruppo, si è utilizzato anche lo spellchecker di Overleaf.

\newpage
\subsubsection{Percentuale di metriche soddisfatte}
    Per le metriche: Percentuale dei requisiti soddisfatti, complessità ciclomatica, sfin e sfout, alle quali sono stati determinati valori sufficienti e ottimi per ogni unità di calcolo,
    verranno considerate ottime se la maggior parte delle unità presenteranno valori ottimi. L'equivalente
    sarà considerato per i valori sufficienti e insufficienti. 


    \begin{longtabu} to \textwidth {| X[0.2,c m] | X[0.1,c m] | X[0.1,c m] |}
        \hline
        \rowcolor{header}
        \textbf{Metrica} &
        \textbf{Valore} &
        \textbf{Esito}\\
        \hline
        \hyperlink{subsubsection.5.1.1}{percentuale requisiti soddisfatti}& 100\% & ottimo \\ 
        \hline
        \hyperlink{subsubsection.5.1.2}{Earned Value} & 3371 & ottimo  \\ 
        \hline
        \hyperlink{subsubsection.5.1.2}{Planned Value} & 3371 & ottimo  \\
        \hline
        \hyperlink{subsubsection.5.1.2}{Actual cost} & 3371 & ottimo  \\
        \hline
        \hyperlink{subsubsection.5.1.2}{Schedule variance Value} & 0 & ottimo  \\
        \hline
        \hyperlink{subsubsection.5.1.2}{Cost variance Value} & 0 & ottimo  \\
        \hline
        \hyperlink{subsubsection.5.2.1}{Gulpease index} & 0 & ottimo  \\
        \hline
        \hyperlink{subsubsection.5.2.2}{Errori ortografici} & 0 & ottimo  \\
        \hline
        \hyperlink{subsubsection.5.2.4}{Code coverage} & 76,4\% & sufficiente \\
        \hline
        \hyperlink{subsubsection.5.2.5}{Numero di test superati} & 100\% & ottimo \\
        \hline
        \hyperlink{subsubsection.5.3.1}{Correttezza dello scambio dei dati} & 0 & ottimo \\
        \hline
        \hyperlink{subsubsection.5.3.2}{Numero di click} & 2 & ottimo \\
        \hline
        \hyperlink{subsubsection.5.3.3}{Site depth} & 1 & ottimo \\
        \hline
        \hyperlink{subsubsection.5.3.3}Response time & 54,8 & ottimo \\
        \hline
        \hyperlink{subsubsection.5.3.5}{Complessità ciclomatica} & 97,82\% & ottimo \\
        \hline
        \hyperlink{subsubsection.5.3.6}{Facilità di comprensione} & 0,1 & sufficiente \\
        \hline
        \hyperlink{subsubsection.5.3.7}{sfin} & 51,11\% & sufficiente \\
        \hline
        \hyperlink{subsubsection.5.3.7}{sfout} & 12,78\% & sufficiente \\
        \hline
        
        \end{longtabu}

        \begin{figure}[H]
            \centering
            \includegraphics[width=12 cm]{source/sections/images/graf_metriche.png}
            \caption{Grafico dei valori delle metriche soddisfatte}
        \end{figure}

        \begin{figure}[H]
            \centering
            \includegraphics[width=14 cm]{source/sections/images/percentuale-metriche-soddisfatte.png}
            \caption{Grafico delle percentuali delle metriche soddisfatte}
        \end{figure}
\newpage
\subsubsection{Code coverage}
 In seguito i valori del Code coverage per calcolati dal 20 Marzo al 20 Maggio.
 Nella prima immagine troviamo i dati calcolati per la parte Client, al quale va sommato il valore della Code coverage della parte server.
 In definitiva il valore completo è 76,8\%.
    \begin{figure}[H]
        \centering
        \includegraphics[width=16 cm]{source/sections/images/CodeCoverage.png}
        \caption{Grafico dei valori del Code coverage - GitHub}
    \end{figure}


\subsubsection{Numero di test superati}
Nel seguente grafico possiamo vedere che la totalità dei test superati ed implementati è pari al 100\% del numero di test.
    \begin{figure}[H]
        \centering
        \includegraphics[width=10 cm]{source/sections/images/num-test.png}
        \caption{Grafico numero di test}
    \end{figure}


\newpage
\subsection{Metriche sulla qualità di prodotto}
\subsubsection{Numero di click}

\begin{longtabu} to \textwidth {| X[0.2,c m] | X[0.1,c m]| X[0.1,c m]| }
    \hline
    \rowcolor{header}
    \textbf{Task} &
    \textbf{Numero di click} &
    \textbf{Esito}\\
    \hline
    caricamento dati da csv & 2 & ottimo \\ 
    \hline
    caricamento dati da server & 3 & ottimo \\
    \hline
    visualizzare i dati & 3 + numero di features & sufficiente \\
    \hline 
    \end{longtabu}


\subsubsection{Site depth}
    L'applicazione web utilizza React e quindi fa in modo che la profondità del sito sia sempre pari ad 1, in quanto tutte le visualizzazioni vengono mostrate quando selezionate e calcolate sulla stessa pagina una alla volta

    \begin{figure}[H]
        \centering
        \includegraphics[width=10 cm]{source/sections/images/response-time.png}
        \caption{Grafico del response time}
    \end{figure}
    
\subsubsection{Response Time}
    Il response time è stato calcolato su file csv di 150 record e i risultati in ms sono i seguenti:

    \begin{figure}[H]
        \centering
        \includegraphics[width=10 cm]{source/sections/images/response-time.png}
        \caption{Grafico del response time}
    \end{figure}
    
\subsubsection{Complessità Ciclomatica}
    Nella seguente tabell e grafico si può vedere i valori della complessità cicolmatica divisi per valori:

        Come si può notare dal segeuente graico il numero di funzioni al di sotto della soglia per considerarsi ottime è pari al 99\%



    \begin{figure}[H]
        \centering
        \includegraphics[width=10 cm]{source/sections/images/tabella_CC.JPG}
        \caption{Grafico della percentuale dei test}
    \end{figure}

    \begin{figure}[H]
        \centering
        \includegraphics[width=10 cm]{source/sections/images/CC.png}
        \caption{Grafico della percentuale dei test}
    \end{figure}

    \subsubsection{Facilità di comprensione}

        \begin{figure}[H]
            \centering
            \includegraphics[width=10 cm]{source/sections/images/facilitaDelCodice.png}
            \caption{Grafico della facilità di comprensione}
        \end{figure}

    In seguito i grafici che mostrano l'andamento del numero di righe riguradnati, nel primo grafico il codice, e nel secondo
    le righe di commento e righe vuote

    \begin{figure}[H]
        \centering
        \includegraphics[width=10 cm]{source/sections/images/Valori-delle-righe.png}
        \caption{Grafico del numero di righe dei commenti e righe vuote}
    \end{figure}

    \begin{figure}[H]
        \centering
        \includegraphics[width=10 cm]{source/sections/images/numCodice.png}
        \caption{Grafico del numero di righe di codice}
    \end{figure}

    \begin{figure}[H]
        \centering
        \includegraphics[width=10 cm]{source/sections/images/facilita_comprensione.png}
        \caption{Grafico della facilità di comprensione}
    \end{figure}

    I forti cambiamenti nell'andamento delle funzioni sono dati da un aggiornamento del progetto fatto in data
    9 Aprile, nel quale si è implementato un comando yarn che si preoccpa di creare la maggiorparte del
    codice CSS, che ha fatto scendere in modo drastico il numero di righe di codice del progetto effettivo. Maggiori chiarimenti nella sezione conclusiva.

\subsubsection{Stractural FAN-in and FAN-out  [sfin and sfout]}
    E' stato calcolato lo Stractural FAN-in e FAN-out di tutti i 44 file del progetto. I risultati sono esposti nella seguente tabella:

    \begin{longtabu} to \textwidth {| X[0.1,c m] | X[0.1,c m] | X[0.1,c m] | X[0.1,c m] | X[0.1,c m] |}
        \hline
        \rowcolor{header}
        \textbf{Esito} &
        \textbf{IN} &
        \textbf{percentiale sul totale} &
        \textbf{OUT} &
        \textbf{percentuale sul totale} \\
        \hline
        Ottimo & 11 & 25\% & 5 & 11,36\% \\ 
        \hline
        Sufficiente & 26 & 59,09\% & 32 & 72,72\% \\ 
        \hline
        Insufficiente & 7 & 15,90\% & 7 & 15,90\% \\ 
        \hline

        \end{longtabu}

        \begin{figure}[H]
            \centering
            \includegraphics[width=13 cm]{source/sections/images/SfinSfout.png}
            \caption{Grafico delo stractural FAN-in e FAN-out}
        \end{figure}

\newpage
\subsection{Conclusioni}

\subsubsection{Analisi}
Durante il periodo di analisi tutta la documentazione da presentare in ingresso alla revisione dei requisiti è stata sottoposta ad un'analisi meticolosa della struttura del documento, della chiarezza e degli errori ortografici. La verifica di ogni documento è stata svolta da 2 componenti del gruppo per assicurare il minor numero di errori possibili.
\newline
In conclusione dai valori raggiunti dal grafico e dalla tabella soprastanti, si evince un discreto lavoro di redattori e verificatori. In particolare sono stati utili il Piano di Qualifica e le Norme di Progetto per avere un punto di riferimento, sia agli analisti nella scrittura dei documenti, sia ai verificatori per controllare con metriche e con parametri oggettivi.

\subsubsection{Consolidamento dei requisiti}
Durante questo periodo si sono svolte le attività di consolidamento in anticipazione alla Revisione dei Requisiti.
In Conclusione i valori delle metriche in questo periodo sono tutte soddisfacenti ed alcune ottime, ciò indica il rispetto delle tempistiche ed una buona pianificazione da parte dei redattori del Piano di Progetto

\subsubsection{Programmazione architetturale}
Durante questo periodo si è individuata una soluzione architetturale del progetto e si è redatto il Proof of concept.
Gli scopi della programmazione architetturale non sono stati rispettati come si può vedere dalla Schedule variance con valore negativo, che indica un ritardo temporale da parte del gruppo.
Si è così deciso di sfruttare la consegna a sportello e recuperare la progettazione, recuperando così la perdita precedente. Si può vedere che la schedule variance del PAS è 0 che indica il rispetto delle tempistiche e l'Earned value della PAS corrisponde alle perdite del PA.
I valori del Cost variance leggermente positivi indicano una riuscita del rispetto dei costi e delle tempistiche ma senza un grande margine.
In Conclusione il gruppo è riuscito a recuperare grazie alla consegna a sportello, e non ha dovuto riconsegnare nella consegna successiva che avrebbe portato un ritardo e una perdita nei costi decisamente maggiore.

\newpage
\subsubsection{Progettazione di dettaglio e codifica}

Durante questo periodo il progetto didattico ha preso forma. In particolare il gruppo ha posto l'accento sulla concretizzazione della product baseline, quindi dei design pattern, classi e attività necessarie alla codifica.
Infatti l'esito delle metriche inerenti a questi scopi è ottima il che conferma che una buona pianificazione aumenta la qualità del prodotto finale.
In particolare le metriche di pianificazione mostrano che non ci sono stati ritardi rispetto alla pianificazione e che si è rispettato il costo pianificato.
Le metriche della qualità di prodotto sono anch'esse tutte ottime, il che indica una buona direzione dell'andamento del progetto.

Una fonte di rischio di bassa qualità potrebbe essere il valore insufficiente di alcune metriche. Le metriche che hanno avuto un'esito insufficiente ovvero la percentuale dei requisiti soddisfatti, il code coverage e la facilità di comprensione del codice, saranno migliorate
nella fase finale di validazione e collaudo mediante una maggiore esecuzione di test e facendo leva sull'uso del ciclo di Deming, nel quale
vengono individuati i problemi, pianificate le soluzioni, verificare la soluzione e infine applicarla.
In particolare  verranno applicate le seguenti soluzioni:

\newpage
\begin{itemize}
    \item \textbf{Facilità di comprensione del codice}: inserire commenti nelle parti di codice del progetto dove la complessità ciclomatica è superiore a 2
    \item \textbf{Code coverage}: aumentarne il valore nella fase di validazione e collaudo secondo la pianificazione della prossima fase  ponendo l'accento sui file con valori di FAN-in e FAN-out insufficienti
    \item \textbf{percentuale requisiti soddisfatti}: soddisfare il 26\% di requisiti rimanenti implementandoli nel progetto
\end{itemize} 

\subsubsection{Verifica e collaudo}

Durante questo ultimo periodo il progetto didattico è stato completato e portato a termine.
L'accento è stato posto sul completamento dei test di integrazione ed unità e sulla  verifica dei documenti, secondo il modus operandi finora applicato.
In particolare il gruppo si è posto l'obbiettivo di raggiungere un valore del code coverage del 75\%, valore molto alto, ma che è stato soddisfatto in pieno.
Inoltre il gruppo ha provveduto a completare gli obbiettivi posti dalle conclusione della fase di progettazione e collaudo.
Obbiettivi che sono stati soddisfatti secondo le metriche calcolate e mostrate in questo documento.\\
 \\
Per quanto riguarda le fonti di rischio rilevate nella fase precedente i risultati sono stati più che positivi.
La percentuale dei requisiti soddisfatti ha esito ottimo, mentre la facilià di comprensione e la code coverage hanno esito soddisfacente.
 Altre metriche importanti che delineano la buona riuscita del progetto sono le metriche di pianificazione dei tempi e dei costi che sono risultate ottime.\\
 In conclusione il progetto didattico è stato utile al gruppo nel crearsi un'idea più chiara di come pianificare un intero progetto e come strutturare le fase e soprattutto i processi che ne compongono.
 