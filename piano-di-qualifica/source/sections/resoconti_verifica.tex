\section{Resoconti di verifica ed esiti delle revisioni}

In questa sezione vengono mostrati i valori delle matriche calcolate a termine del periodo di revisione dei requisiti. Saranno mostrati i sia i valori che rientrato nel range prestabilito sia quelli che non rientrano e nel secondo caso verranno mostrate come problemi nella parte finale della sezione.

\subsection{Riassunto delle attività di verifica}
Durante il periodo di Analisi tutta la documentazione da presentare in ingresso alla revisione dei requisiti è stata sottoposta ad un'analisi meticolosa della struttura del documento, della chiarezza e degli errori ortografici. La verifica di ogni documento è stata svolta da 2 componenti del gruppo per assicurare il minor numero di errori possibili.

\subsection{Risultati delle verifiche tramite analisi}

\subsubsection{Valori dell'indice di Gulpease}

Per ogni documento stilato è stato calcolato l'indice di Gulpease\glo{} in due momenti: al 2021/12/03 e al GG/MM/AAA alla versione 1.0.0 che è stata consegnata per la revisione dei requisiti. I valori sono dati dalla seguente tabella ed il rispettivo grafico con valore sufficiente >40 e valore ottimo >80.

\hphantom{}
\renewcommand*{\arraystretch}{1.4}
\setlength{\tabcolsep}{10pt} % Default value: 6pt

\begin{longtable}[c]{|c|c|c|c|c|c|}
\hline
\rowcolor[HTML]{FFCC67}
Data del calcolo  & \begin{tabular}[c|]{@{}c@{}}Studio di \\ fattibilità\end{tabular} & \begin{tabular}[c]{@{}c@{}}Norme di \\ progetto\end{tabular} & \begin{tabular}[c]{@{}c@{}}Analisi dei\\ requisiti\end{tabular} & \begin{tabular}[c]{@{}c@{}}Piano di \\ qualifica\end{tabular} & \begin{tabular}[c]{@{}c@{}}Piano di \\ progetto\end{tabular} \\ \hline

\endfirsthead
%
\endhead
%
\multirow{2}{}{2020/12/03} & v0.0.8 & v2 & v3 & v4 & v5 \\ \cline{2-6} 
                            & 95 & 65  & 61   & 80   & 91  \\ \hline
\multirow{2}{}{data}      & v1.0.0  & v1.0.0  & v1.0.0 & v1.0.0  &v1.0.0   \\ \cline{2-6} 
                            & x1   & x2 & x3  & x4 & x5 \\ \hline
\end{longtable}


\newpage
\begin{figure}[htp]
    \centering
    \includegraphics[width=15cm]{GraficoGulpease}
    \caption{Grafico dell'indice di Gulpease}
    \label{fig:img-valori-gulpease}
\end{figure}


\subsubsection{Errori ortografici}

Per quanto riguarda gli errori ortografici, oltre alla revisione fatta dai membri del gruppo, si è utilizzato anche lo spellchecker di Overleaf. Non sono presenti quindi errori di ortografia nei documnenti con versione v1.0.0


\subsection{Conclusioni}

In conclusione dai valori raggiunti dal grafico e dalla tabella soprastanti, si evince un discreto lavoro di redattori e verificatori.
In particolare sono stati utili il Piano di qualifica e le Norme di progetto per avere un punto di riferimento. Sia agli analisti nella scrittura dei documenti, sia ai verificatori per controllare con metriche e con parametri oggettivi.