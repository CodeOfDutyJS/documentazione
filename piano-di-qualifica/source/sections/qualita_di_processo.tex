\section{Qualità di processo}
Per ottenere la più alta qualità delle attività e processi, e per rientrare nelle tempistiche stipulate nel Piano di progetto, verrà analizzato ed applicato l' ISO/IEC/IEEE 12207:1995.

Sono stati scelti quindi i processi ritenuti più adeguati ed attinenti al progetto i quail sono stati divisi in due categorie: primari e di supporto. Verrano di seguito riportate i processi appartenenti a tali categorie, e i loro obiettivi e metriche in modo da ottenere una qualità quantificabile.
Una maggiore informazione e descrizione delle metriche si trovano nel documento: \textit{Norme di progetto}.

    \subsection{Processi primari}
    
    
        \subsubsection{Fornitura}
        Il processo di fornitura contiene le attività e le task del fornitore, come l'analisi e la determinazione di procedure e risorse necessarie per lo svolgimento del progetto.
        
            \paragraph{Strategie e Obiettivi}
            \begin{itemize}
                \item \textbf{Analisi}: il fornitore conduce una revisione dei requisiti e definisce una proposta in risposta alla richiesta del proponente. Definisce quindi i requisiti classificandoli in obbligatori, desiderabili e opzionali
                \item \textbf{Approvazione e consegna}: il prodotto deve essere in primo luogo approvato e in seguit consegnato secondo le specifiche di contratto e secondo le metriche descritte in questo documento.
                \item \textbf{Qualità}: ottenere e mantenere la qualità dei processi e del prodotto mediante le loro metriche basate anche sulle specifiche richieste dal proponente.
            \end{itemize}
            
            
            \paragraph{Metriche}
            Metriche
            \def\productquality{
            {   Percentuale dei requisiti soddisfatti,
                $ \frac{ReqSoddisfatti}{ReqTotali}$, 
                $ 100 \% $,
                $ 100 \% $
            },
        }
 \newcommand*\metricstable{}
\foreach \x [count=\nj] in \productquality
{
    \foreach \y [count=\ni] in \x
    {
        \ifnum\ni=4
            \xappto\metricstable{\y}
            \gappto\metricstable{\\}
            \gappto\metricstable{\hline}
        \else\ifnum\ni=5
            \xappto\metricstable{\noexpand\multicolumn{4}{| l |}{NOTE: \y}}
            \gappto\metricstable{\\}
            \gappto\metricstable{\hline}
        \else
            \xappto\metricstable{\y&}
        \fi\fi
    }
}

% Impostazioni della tabella
\tabulinesep = 2mm % padding
\taburowcolors [1] 2{pari .. dispari} % colori delle righe
\begin{longtabu} to \textwidth {| X[0.7,c m] | X[0.7,c m] | X[0.4,c m] | X[0.4,c m]|} % larghezza delle colonne
\hline
\rowcolor{header} % colore dell'header

\textbf{Nome} & \textbf{Formula} & \textbf{Valore sufficiente} & \textbf{Valore ottimo}\\
\hline
\metricstable

\end{longtabu}

\undef\metricstable{}


            
            
          
            
        \subsubsection{Pianificazione}
        L'attività di pianificazione ingloba il monitoramento delle risorse quali i tempi a disposizione i costi e la divisione dei ruoli e la loro distribuzione.La scelta di mettere la Pianificazione come processo primario è data dalla sua importanza nel progetto. Inoltre viene esposta maggiormente all'interno del documento Piano di progetto.
            \paragraph{Strategie e Obiettivi}
                \begin{itemize}
                    \item \textbf{Costi}: analizzare e quantificare costi e tempistiche
                    \item \textbf{Metriche}: per fare in modo che i  tali costi non si discostino più di un margine prescelto devono essere usate le metriche scelte.
                    \item \textbf{Calendarizzazione}: per assicurare che le tempistiche dello svolgimento del progetto siano adeguate bisogna seguire le direttive della calendarizzazione del progetto
                    \item \textbf{Aggiornamenti}: mantenere aggiornato il gruppo ed i verbali durante tutti il progetto in modo da quantificare il cambiamento delle risorse
                \end{itemize}
                
           \paragraph{Metriche}
           alcunee metriche per la pianificazione sono usate per calcolarne delle altre e la loro formula è semplicemente data dal volre intero che hanno
            \def\productquality{
            {   Budget at Completion [BG],
                numero intero, 
                $ \pm 5 \% $ del preventivo,
                pari al preventivo
            },
            {   Earned value [EV],
                BAC - $\%$ di lavoro completato, 
                $ >0$,
                $ >0$
            },
            {   Planned value [PV],
                valore pianificato nel momento del calcolo, 
                $ >0$,
                $ >0$
            },
            {  Acual cost [AC],
               numero intero, 
                0 < AC $\leq$ budget totale,
                0 < AC $\leq$ PV
            },
            {   Schedule variance [SV],
               SV = EV - PV, 
                $ >0$,
                $ 0$
            },
            {    Cost variance [CV],
               CV = EV - AC, 
                $ >0$,
                $ \geq 0$
            },
        }
 \newcommand*\metricstable{}
\foreach \x [count=\nj] in \productquality
{
    \foreach \y [count=\ni] in \x
    {
        \ifnum\ni=4
            \xappto\metricstable{\y}
            \gappto\metricstable{\\}
            \gappto\metricstable{\hline}
        \else\ifnum\ni=5
            \xappto\metricstable{\noexpand\multicolumn{4}{| l |}{NOTE: \y}}
            \gappto\metricstable{\\}
            \gappto\metricstable{\hline}
        \else
            \xappto\metricstable{\y&}
        \fi\fi
    }
}

% Impostazioni della tabella
\tabulinesep = 2mm % padding
\taburowcolors [1] 2{pari .. dispari} % colori delle righe
\begin{longtabu} to \textwidth {| X[0.7,c m] | X[0.7,c m] | X[0.4,c m] | X[0.4,c m]|} % larghezza delle colonne
\hline
\rowcolor{header} % colore dell'header

\textbf{Nome} & \textbf{Formula} & \textbf{Valore sufficiente} & \textbf{Valore ottimo}\\
\hline
\metricstable

\end{longtabu}

\undef\metricstable{}



        
    \subsubsection{Sviluppo}
    Il processo contiene le attività di design, scrittura del codice e accettazione del codice software.
    
        \paragraph{Strategie e obiettivi}
        \begin{itemize}
            \item \textbf{Architettura}: si stabilisce un primo livello di architettura nel quale si identificano gli elementi portanti della struttura del prodotto trasformando quindi i requisiti software in un'architettura che ne decrive il Top-level.
            \item \textbf{Design}: lo sviluppatore deve sviluppare un design dettagliato per ogni unità software.
            \item \textbf{Integrazione di sistema}: Il piano di integrazione deve integrare tutti gli elementi software.
        \end{itemize}
        
        \paragraph{Metriche}
        Le metriche di questo processo non sono ancora state stabilite, in quanto è stato ritenuto prematuro definire matriche oggettive in questa fase del progetto.Inoltre il documento nella durata del progetto verrà modificato e raffinato per migliorarne l'usabilità e la qualità.
        
        
        
        \subsection{Processi di supporto}
            \subsubsection{Documentazione}
            In questo processo vengono descritte gli obiettivi e processi per una stesura di qualità della documentazione, parte molto importante del progetto. Vengono quindi trattate misure per facilitare la lettura dei documenti.
            
            \paragraph{Strategie e Obiettivi}
            \begin{itemize}
                \item \textbf{Facilità di lettura}: mantenere il testo ad un livello di difficoltà di lettura comprensibile e stimato dalle metriche in modo da non affaticare la lettura dei documenti ed aumentare la produttività
                \item \textbf{Numero di parole}: il numero di parole complessivo non deve essere troppio ampio. Lo scopo è mantenere solo il materiale utile al gruppo all'interno dei documenti
                \item \textbf{Ortografia}: il testo non deve contenere errori ortografici o grammaticali
                \item \textbf{Aggiornamento}: i documenti devono seguire le metriche indicate e devono essere aggiornati se si riconoscono criteri più utili al progetto.
            \end{itemize}
            
            \paragraph{Metriche}
            Metriche
            
             \def\productquality{
                            {   Gunning's fog index,
                                $0.4*(\frac{ Parole}{Frasi} + 100* \frac{Complesse}{ Frasi})$, 
                                $ \geq 16$,
                                $ \geq 12 $
                            },
                            {   Gulpease index,
                                $89 + (300*Frasi - 10*\frac{Lettere}{Parole}$, 
                                $40 < IG \leq 100$,
                                $80 < IG \leq 100$
                            },
                            {   Correttezza ortografica,
                                numero totale di errori, 
                                0,
                                0
                            },
                        }
\newcommand*\metricstable{}
\foreach \x [count=\nj] in \productquality
{
    \foreach \y [count=\ni] in \x
    {
        \ifnum\ni=4
            \xappto\metricstable{\y}
            \gappto\metricstable{\\}
            \gappto\metricstable{\hline}
        \else\ifnum\ni=5
            \xappto\metricstable{\noexpand\multicolumn{4}{| l |}{NOTE: \y}}
            \gappto\metricstable{\\}
            \gappto\metricstable{\hline}
        \else
            \xappto\metricstable{\y&}
        \fi\fi
    }
}

% Impostazioni della tabella
\tabulinesep = 2mm % padding
\taburowcolors [1] 2{pari .. dispari} % colori delle righe
\begin{longtabu} to \textwidth {| X[0.7,c m] | X[0.7,c m] | X[0.4,c m] | X[0.4,c m]|} % larghezza delle colonne
\hline
\rowcolor{header} % colore dell'header

\textbf{Nome} & \textbf{Formula} & \textbf{Valore sufficiente} & \textbf{Valore ottimo}\\
\hline
\metricstable

\end{longtabu}

\undef\metricstable{}


\textit{Parole} = numero totale di parole del documento

\textit{Frasi} = numero totale di parole del documento

\textit{Complesse} = numero di parole ritenute complesse nel documento

\textit{Lettere} = numero totale di lettere del documento
            
            \subsubsection{Quality assurance}
            Questo processo è orientato alla quantificazione della qualità. Assicura quindi un grado di qualità minimo e orienta verso un grado di qualità ottimo totale del progetto.
            
            \paragraph{Strategie e obiettivi}
            \begin{itemize}
                \item \textbf{Metriche}: tenere conto sempre delle metriche di ogni processo. Che implica quindi la loro misurazione e il loro monitoraggio
                \item \textbf{Garanziaa della qualità}: garantire sempre un grado di soddisfacimento sufficiente delle metriche.
            \end{itemize}
            
            \paragraph{Metriche}
            metriche
            \def\productquality{
                {   Percentuale di metriche soddisfatte,
                    $\frac{Soddisfatte}{Totali}$,
                    $ \geq 60 \%$,
                    $ \geq 80 \% $
                },
            }
\newcommand*\metricstable{}
\foreach \x [count=\nj] in \productquality
{
    \foreach \y [count=\ni] in \x
    {
        \ifnum\ni=4
            \xappto\metricstable{\y}
            \gappto\metricstable{\\}
            \gappto\metricstable{\hline}
        \else\ifnum\ni=5
            \xappto\metricstable{\noexpand\multicolumn{4}{| l |}{NOTE: \y}}
            \gappto\metricstable{\\}
            \gappto\metricstable{\hline}
        \else
            \xappto\metricstable{\y&}
        \fi\fi
    }
}

% Impostazioni della tabella
\tabulinesep = 2mm % padding
\taburowcolors [1] 2{pari .. dispari} % colori delle righe
\begin{longtabu} to \textwidth {| X[0.7,c m] | X[0.7,c m] | X[0.4,c m] | X[0.4,c m]|} % larghezza delle colonne
\hline
\rowcolor{header} % colore dell'header

\textbf{Nome} & \textbf{Formula} & \textbf{Valore sufficiente} & \textbf{Valore ottimo}\\
\hline
\metricstable

\end{longtabu}

\undef\metricstable{}


\textit{Soddisfatte} = numero di metriche con valore soddifacente

\textit{Totali} = numero totale di metriche
 
            \subsubsection{Verification process}
            Il processo di verifica è il metodo per determinare se il prodotto software ha i requisiti e le condizioni imposte, e nella ricerca e correzione di anomalie. Quindi si controllano costo e performance del prodotto. Inoltre il processo di verifica deve essere introdotto il prima possibile e deve essere automatizzato.
            
            \paragraph{Strategie e oboiettivi}
            \begin{itemize}
                \item \textbf{Anomalie}: le anomalie vanno individuate e corrette
                \item \textbf{Strumenti}: vanno applicate gli strumenti per una corretta verifica e le metriche per la quantizzazione della qualità.
            \end{itemize}
            
            \paragraph{Metriche}
            Metriche per il processo di verifica:
            
            \def\productquality{
                {   Code coverage,
                    $\frac{Testate}{Totali}$,
                    $ 75 \%$,
                    $ 100 \% $
                },
                {   Numero di test superati,
                    numero intero,
                    = numero totale di test,
                    = numero totale di test
                },
            }
\newcommand*\metricstable{}
\foreach \x [count=\nj] in \productquality
{
    \foreach \y [count=\ni] in \x
    {
        \ifnum\ni=4
            \xappto\metricstable{\y}
            \gappto\metricstable{\\}
            \gappto\metricstable{\hline}
        \else\ifnum\ni=5
            \xappto\metricstable{\noexpand\multicolumn{4}{| l |}{NOTE: \y}}
            \gappto\metricstable{\\}
            \gappto\metricstable{\hline}
        \else
            \xappto\metricstable{\y&}
        \fi\fi
    }
}

% Impostazioni della tabella
\tabulinesep = 2mm % padding
\taburowcolors [1] 2{pari .. dispari} % colori delle righe
\begin{longtabu} to \textwidth {| X[0.7,c m] | X[0.7,c m] | X[0.4,c m] | X[0.4,c m]|} % larghezza delle colonne
\hline
\rowcolor{header} % colore dell'header

\textbf{Nome} & \textbf{Formula} & \textbf{Valore sufficiente} & \textbf{Valore ottimo}\\
\hline
\metricstable

\end{longtabu}

\undef\metricstable{}


\textit{Testate} = numero totale di linee di codice testate

\textit{Totali} = numero totale di linee di codice
            