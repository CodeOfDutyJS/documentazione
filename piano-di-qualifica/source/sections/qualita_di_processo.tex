\section{Qualità di processo}
Per ottenere la più alta qualità delle attività e processi, e per rientrare nelle tempistiche stipulate nel \PdP , sono istanziati alcuni dei processi, ritenuti più adeguati ed attinenti al progetto, descritti dall'ISO/IEC/IEEE 12207:1995.

Verranno di seguito riportati i processi, insieme ai loro obiettivi e metriche in modo da ottenere una qualità quantificabile. Maggiori informazioni e la descrizione delle metriche si trovano nel documento: \NdP .

    \subsection{Processi primari}
    
    
        \subsubsection{Fornitura}
        Il processo di fornitura contiene le attività e le task del fornitore, come l'analisi e la determinazione di procedure e risorse necessarie per lo svolgimento del progetto.
        
            \paragraph{Strategie e Obiettivi}
            \begin{itemize}
                \item \textbf{Analisi}: Il fornitore conduce una revisione dei requisiti e definisce una proposta in risposta alla richiesta del proponente. Definisce quindi i requisiti classificandoli in obbligatori, desiderabili e opzionali;
                \item \textbf{Approvazione e consegna}: il prodotto deve essere in primo luogo approvato e in seguito consegnato secondo le specifiche di contratto e secondo le metriche descritte in questo documento;
                \item \textbf{Qualità}: ottenere e mantenere la qualità dei processi e del prodotto mediante le loro metriche basate anche sulle specifiche richieste dal proponente.
            \end{itemize}
            
            
            \paragraph{Metriche}
            
            \hphantom{}
            %startTable
            \def\productquality{
            {   Percentuale dei requisiti soddisfatti,
                $ \frac{ReqSoddisfatti}{ReqTotali}$, 
                $ 100 \% $,
                $ 100 \% $
            },
        }
        \newcommand*\metricstable{}
\foreach \x [count=\nj] in \productquality
{
    \foreach \y [count=\ni] in \x
    {
        \ifnum\ni=5
            \xappto\metricstable{\y}
            \gappto\metricstable{\\}
            \gappto\metricstable{\hline}
        \else\ifnum\ni=6
            \xappto\metricstable{\noexpand\multicolumn{5}{| l |}{NOTE: \y}}
            \gappto\metricstable{\\}
            \gappto\metricstable{\hline}
        \else
            \xappto\metricstable{\y&}
        \fi\fi
    }
}

% Impostazioni della tabella
\tabulinesep = 2mm % padding
\taburowcolors [1] 2{pari .. dispari} % colori delle righe
\begin{longtabu} to \textwidth {| X[0.5, c m] | X[0.6,c m] | X[0.9,c m] | X[0.8,c m] | X[1.2,c m]|} % larghezza delle colonne
\hline
\rowcolor{header} % colore dell'header

\textbf{Codice} & \textbf{Nome} & \textbf{Formula} & \textbf{Valore sufficiente} & \textbf{Valore ottimo}\\
\hline
\metricstable

\end{longtabu}

\undef\metricstable{}

        %endTable
            
        \newpage    
        \subsubsection{Pianificazione}
        L'attività di pianificazione ingloba il monitoraggio delle risorse: i tempi a disposizione, i costi, la divisione dei ruoli e la loro distribuzione.
            \paragraph{Strategie e Obiettivi}
                \begin{itemize}
                    \item \textbf{Costi}: Analizzare e quantificare costi e tempistiche;
                    \item \textbf{Metriche}: per fare in modo che i costi non si discostino più di un margine prescelto devono essere usate le metriche scelte;
                    \item \textbf{Calendarizzazione}: per assicurare che le tempistiche dello svolgimento del progetto siano adeguate bisogna seguire le direttive della calendarizzazione del progetto;
                    \item \textbf{Aggiornamenti}: mantenere aggiornato il gruppo ed i verbali durante tutti il progetto in modo da quantificare il cambiamento delle risorse.
                \end{itemize}
                
           \paragraph{Metriche}
            
            \hphantom{}
           %startTable
            \def\productquality{
            {   Budget at Completion [BG],
                numero intero, 
                $ \pm 5 \% $ del preventivo,
                pari al preventivo
            },
            {   Earned value [EV],
                BAC - $\%$ di lavoro completato, 
                $ >0$,
                $ >0$
            },
            {   Planned value [PV],
                valore pianificato nel momento del calcolo, 
                $ >0$,
                $ >0$
            },
            {  Acual cost [AC],
               numero intero, 
                0 < AC $\leq$ budget totale,
                0 < AC $\leq$ PV
            },
            {   Schedule variance [SV],
               SV = EV - PV, 
                $ >0$,
                $ 0$
            },
            {    Cost variance [CV],
               CV = EV - AC, 
                $ >0$,
                $ \geq 0$
            },
        }
        \newcommand*\metricstable{}
\foreach \x [count=\nj] in \productquality
{
    \foreach \y [count=\ni] in \x
    {
        \ifnum\ni=5
            \xappto\metricstable{\y}
            \gappto\metricstable{\\}
            \gappto\metricstable{\hline}
        \else\ifnum\ni=6
            \xappto\metricstable{\noexpand\multicolumn{5}{| l |}{NOTE: \y}}
            \gappto\metricstable{\\}
            \gappto\metricstable{\hline}
        \else
            \xappto\metricstable{\y&}
        \fi\fi
    }
}

% Impostazioni della tabella
\tabulinesep = 2mm % padding
\taburowcolors [1] 2{pari .. dispari} % colori delle righe
\begin{longtabu} to \textwidth {| X[0.5, c m] | X[0.6,c m] | X[0.9,c m] | X[0.8,c m] | X[1.2,c m]|} % larghezza delle colonne
\hline
\rowcolor{header} % colore dell'header

\textbf{Codice} & \textbf{Nome} & \textbf{Formula} & \textbf{Valore sufficiente} & \textbf{Valore ottimo}\\
\hline
\metricstable

\end{longtabu}

\undef\metricstable{}

        %endTable

    \newpage  
    
    \subsubsection{Sviluppo}
    Il processo contiene le attività di design, scrittura del codice e accettazione del codice software.
    
        \paragraph{Strategie e obiettivi}
        \begin{itemize}
            \item \textbf{Architettura}: Si stabilisce un primo livello di architettura nel quale si identificano gli elementi portanti della struttura del prodotto trasformando quindi i requisiti software in un'architettura che ne descrive il Top-level;
            \item \textbf{Design}: lo sviluppatore deve sviluppare un design dettagliato per ogni unità software;
            \item \textbf{Integrazione di sistema}: Il piano di integrazione deve integrare tutti gli elementi software.
        \end{itemize}
        
        \paragraph{Metriche}
        Le metriche di questo processo non sono ancora state stabilite, in quanto è stato ritenuto prematuro definirle in questa fase del progetto.Inoltre il documento nella durata del progetto verrà modificato e raffinato per migliorarne l'usabilità e la qualità.
        
        
        
        \subsection{Processi di supporto}
            \subsubsection{Documentazione}
            In questo processo vengono descritti gli obiettivi e processi per una stesura di qualità della documentazione, parte molto importante del progetto. Vengono quindi trattate misure per facilitare la lettura dei documenti.
            
            \paragraph{Strategie e Obiettivi}
            \begin{itemize}
                \item \textbf{Facilità di lettura}: Mantenere il testo ad un livello di difficoltà di lettura comprensibile è stimato dalle metriche in modo da non rendere la lettura dei documenti faticosa;
                \item \textbf{Numero di parole}: il numero di parole complessivo non deve essere troppo ampio. Lo scopo è mantenere solo il materiale utile al gruppo all'interno dei documenti;
                \item \textbf{Ortografia}: il testo non deve contenere errori ortografici o grammaticali;
                \item \textbf{Aggiornamento}: i documenti devono seguire le metriche indicate e devono essere aggiornati se si riconoscono criteri più utili al progetto.
            \end{itemize}
    \newpage
            \paragraph{Metriche}
            
            \hphantom{}
            %startTable
         \def\productquality{
                           {   Gulpease index,
                                $89 + (300*Frasi - 10*\frac{Lettere}{Parole}$, 
                                $40 < IG \leq 100$,
                                $80 < IG \leq 100$
                                %\textit{Parole} = numero totale di parole del documento
                                %\textit{Frasi} = numero totale di parole del documento
                                %\textit{Complesse} = numero di parole ritenute complesse nel documento
                                %\textit{Lettere} = numero totale di lettere del documento
                            },
                            {   Correttezza ortografica,
                                numero totale di errori, 
                                0,
                                0
                            },
                        }
                    \newcommand*\metricstable{}
\foreach \x [count=\nj] in \productquality
{
    \foreach \y [count=\ni] in \x
    {
        \ifnum\ni=5
            \xappto\metricstable{\y}
            \gappto\metricstable{\\}
            \gappto\metricstable{\hline}
        \else\ifnum\ni=6
            \xappto\metricstable{\noexpand\multicolumn{5}{| l |}{NOTE: \y}}
            \gappto\metricstable{\\}
            \gappto\metricstable{\hline}
        \else
            \xappto\metricstable{\y&}
        \fi\fi
    }
}

% Impostazioni della tabella
\tabulinesep = 2mm % padding
\taburowcolors [1] 2{pari .. dispari} % colori delle righe
\begin{longtabu} to \textwidth {| X[0.5, c m] | X[0.6,c m] | X[0.9,c m] | X[0.8,c m] | X[1.2,c m]|} % larghezza delle colonne
\hline
\rowcolor{header} % colore dell'header

\textbf{Codice} & \textbf{Nome} & \textbf{Formula} & \textbf{Valore sufficiente} & \textbf{Valore ottimo}\\
\hline
\metricstable

\end{longtabu}

\undef\metricstable{}

                    %endTable
                    
                    
            \subsubsection{Gestione della Qualità}
            Questo processo è orientato alla quantificazione della qualità. Assicura quindi un grado di qualità minimo e orienta verso un grado di qualità ottimo totale del progetto.
            
            \paragraph{Strategie e obiettivi}
            \begin{itemize}
                \item \textbf{Metriche}: Tenere sempre conto delle metriche di ogni processo. Che implica la loro misurazione e il loro monitoraggio;
                \item \textbf{Garanzia della qualità}: garantire sempre un grado di soddisfacimento sufficiente delle metriche.
            \end{itemize}
            
            \paragraph{Metriche}
            \hphantom{}
            %startTable
            \def\productquality{
                {   Percentuale di metriche soddisfatte,
                    $\frac{Soddisfatte}{Totali}$,
                    $ \geq 60 \%$,
                    $ \geq 80 \% $
                },
            }
            \newcommand*\metricstable{}
\foreach \x [count=\nj] in \productquality
{
    \foreach \y [count=\ni] in \x
    {
        \ifnum\ni=5
            \xappto\metricstable{\y}
            \gappto\metricstable{\\}
            \gappto\metricstable{\hline}
        \else\ifnum\ni=6
            \xappto\metricstable{\noexpand\multicolumn{5}{| l |}{NOTE: \y}}
            \gappto\metricstable{\\}
            \gappto\metricstable{\hline}
        \else
            \xappto\metricstable{\y&}
        \fi\fi
    }
}

% Impostazioni della tabella
\tabulinesep = 2mm % padding
\taburowcolors [1] 2{pari .. dispari} % colori delle righe
\begin{longtabu} to \textwidth {| X[0.5, c m] | X[0.6,c m] | X[0.9,c m] | X[0.8,c m] | X[1.2,c m]|} % larghezza delle colonne
\hline
\rowcolor{header} % colore dell'header

\textbf{Codice} & \textbf{Nome} & \textbf{Formula} & \textbf{Valore sufficiente} & \textbf{Valore ottimo}\\
\hline
\metricstable

\end{longtabu}

\undef\metricstable{}

            %endTable
            
\textit{Soddisfatte} = numero di metriche con valore soddisfacente

\textit{Totali} = numero totale di metriche
 
            \newpage
            \subsubsection{Verifica}
            Il processo di verifica è il metodo per determinare se il prodotto software ha i requisiti e le condizioni imposte, e per ricercare e correggere anomalie. Quindi si controllano costo e performance del prodotto. Inoltre il processo di verifica deve essere introdotto il prima possibile e deve essere automatizzato.
            
            \paragraph{Strategie e obiettivi}
            \begin{itemize}
                \item \textbf{Anomalie}: Le anomalie vanno individuate e corrette;
                \item \textbf{Strumenti}: vanno applicati gli strumenti per una corretta verifica e le metriche per la quantificazione della qualità.
            \end{itemize}
            
            \paragraph{Metriche}
            
            \hphantom{}
            %startTable
            \def\productquality{
                {   Code coverage,
                    $\frac{Testate}{Totali}$,
                    $ 75 \%$,
                    $ 100 \% $
                    %\textit{Testate} = numero totale di linee di codice testate
                    %\textit{Totali} = numero totale di linee di codice
                },
                {   Numero di test superati,
                    numero intero,
                    = numero totale di test,
                    = numero totale di test
                },
            }
            \newcommand*\metricstable{}
\foreach \x [count=\nj] in \productquality
{
    \foreach \y [count=\ni] in \x
    {
        \ifnum\ni=5
            \xappto\metricstable{\y}
            \gappto\metricstable{\\}
            \gappto\metricstable{\hline}
        \else\ifnum\ni=6
            \xappto\metricstable{\noexpand\multicolumn{5}{| l |}{NOTE: \y}}
            \gappto\metricstable{\\}
            \gappto\metricstable{\hline}
        \else
            \xappto\metricstable{\y&}
        \fi\fi
    }
}

% Impostazioni della tabella
\tabulinesep = 2mm % padding
\taburowcolors [1] 2{pari .. dispari} % colori delle righe
\begin{longtabu} to \textwidth {| X[0.5, c m] | X[0.6,c m] | X[0.9,c m] | X[0.8,c m] | X[1.2,c m]|} % larghezza delle colonne
\hline
\rowcolor{header} % colore dell'header

\textbf{Codice} & \textbf{Nome} & \textbf{Formula} & \textbf{Valore sufficiente} & \textbf{Valore ottimo}\\
\hline
\metricstable

\end{longtabu}

\undef\metricstable{}

             %endTable
         
         
         
