\section{Qualità di Prodotto}
In questa sezione sono indicati valori sufficienti e ottimi delle metriche individuate e descritte nelle Norme di Progetto.
In particolare una misurazione di qualità per essere considerata soddisfatta deve essere superiore al valore sufficiente indicato. 
Il valore ottimo individua un obiettivo di qualità da raggiungere o mantenere per l'intera durata dell'attività di progetto.
\subsection{ISO 9126}
Per la Qualità di Prodotto il team ha deciso di seguire i principi di ISO 9126.
Lo Standard definisce 6 caratteristiche di qualità che deve possedere un prodotto: funzionalità, affidabilità, efficienza, usabilità, manutenibilità e portabilità. 
Non sono state individuate metriche di portabilità in quanto è stata considerata una caratteristica non rilevante nel prodotto sviluppato. 
ISO 9126 oltre a 6 caratteristiche individua 3 tipologie di misurazioni: della qualità interna, esterna e qualità in uso. 
In questa sezione non è tenuto conto della distinzione e non sono presenti metriche di qualità in uso, che richiedono necessariamente l'utilizzo del prodotto da parte di un utente.

\subsection{Funzionalità}
\def\productquality{
    {   , 
        Correttezza dello scambio dei dati,
        $N_{err}$, 
        -,
        0
    },
}
\newcommand*\metricstable{}
\foreach \x [count=\nj] in \productquality
{
    \foreach \y [count=\ni] in \x
    {
        \ifnum\ni=4
            \xappto\metricstable{\y}
            \gappto\metricstable{\\}
            \gappto\metricstable{\hline}
        \else\ifnum\ni=5
            \xappto\metricstable{\noexpand\multicolumn{4}{| l |}{NOTE: \y}}
            \gappto\metricstable{\\}
            \gappto\metricstable{\hline}
        \else
            \xappto\metricstable{\y&}
        \fi\fi
    }
}

% Impostazioni della tabella
\tabulinesep = 2mm % padding
\taburowcolors [1] 2{pari .. dispari} % colori delle righe
\begin{longtabu} to \textwidth {| X[0.7,c m] | X[0.7,c m] | X[0.4,c m] | X[0.4,c m]|} % larghezza delle colonne
\hline
\rowcolor{header} % colore dell'header

\textbf{Nome} & \textbf{Formula} & \textbf{Valore sufficiente} & \textbf{Valore ottimo}\\
\hline
\metricstable

\end{longtabu}

\undef\metricstable{}


\subsection{Affidabilità}
\def\productquality{
    {   , 
        Densità di errori,
        $N_{err}/SLOC$, 
        -,
        -,
        è stato ritenuto prematuro definire valori sufficienti e ottimi per la metrica
    },
}
\newcommand*\metricstable{}
\foreach \x [count=\nj] in \productquality
{
    \foreach \y [count=\ni] in \x
    {
        \ifnum\ni=4
            \xappto\metricstable{\y}
            \gappto\metricstable{\\}
            \gappto\metricstable{\hline}
        \else\ifnum\ni=5
            \xappto\metricstable{\noexpand\multicolumn{4}{| l |}{NOTE: \y}}
            \gappto\metricstable{\\}
            \gappto\metricstable{\hline}
        \else
            \xappto\metricstable{\y&}
        \fi\fi
    }
}

% Impostazioni della tabella
\tabulinesep = 2mm % padding
\taburowcolors [1] 2{pari .. dispari} % colori delle righe
\begin{longtabu} to \textwidth {| X[0.7,c m] | X[0.7,c m] | X[0.4,c m] | X[0.4,c m]|} % larghezza delle colonne
\hline
\rowcolor{header} % colore dell'header

\textbf{Nome} & \textbf{Formula} & \textbf{Valore sufficiente} & \textbf{Valore ottimo}\\
\hline
\metricstable

\end{longtabu}

\undef\metricstable{}


\subsection{Efficienza}


\subsection{Usabilità}


\subsection{Manutenibilità}
