\section{Qualità di Prodotto}
In questa sezione sono indicati valori sufficienti e ottimi delle metriche individuate e descritte nelle Norme di Progetto.
In particolare una misurazione di qualità per essere considerata soddisfatta deve essere superiore al valore sufficiente indicato. 
Il valore ottimo individua un obiettivo di qualità da raggiungere o mantenere per l'intera durata dell'attività di progetto.
\subsection{ISO 9126}
Per la Qualità di Prodotto il team ha deciso di seguire i principi di ISO 9126.
Lo Standard definisce 6 caratteristiche di qualità che deve possedere un prodotto: funzionalità, affidabilità, efficienza, usabilità, manutenibilità e portabilità. 
Non sono state individuate metriche di portabilità in quanto è stata considerata una caratteristica non rilevante nel prodotto sviluppato. 
ISO 9126 oltre a 6 caratteristiche individua 3 tipologie di misurazioni: della qualità interna, esterna e qualità in uso. 
In questa sezione non è tenuto conto della distinzione e non sono presenti metriche di qualità in uso, che richiedono necessariamente l'utilizzo del prodotto da parte di un utente.

\subsection{Funzionalità}
La funzionalità rappresenta la capacità del software di fornire le funzioni, espresse ed implicite, necessarie per operare in determinate condizioni, cioè in un determinato contesto.
\def\productquality{
    {
        Correttezza dello scambio dei dati,
        $D_{err} = \#\ di\ errori$, 
        -,
        $D_{err} = 0$,
        un valore sufficiente sarà deciso in fasi successive del progetto
    },
}
\newcommand*\metricstable{}
\foreach \x [count=\nj] in \productquality
{
    \foreach \y [count=\ni] in \x
    {
        \ifnum\ni=5
            \xappto\metricstable{\y}
            \gappto\metricstable{\\}
            \gappto\metricstable{\hline}
        \else\ifnum\ni=6
            \xappto\metricstable{\noexpand\multicolumn{5}{| l |}{NOTE: \y}}
            \gappto\metricstable{\\}
            \gappto\metricstable{\hline}
        \else
            \xappto\metricstable{\y&}
        \fi\fi
    }
}

% Impostazioni della tabella
\tabulinesep = 2mm % padding
\taburowcolors [1] 2{pari .. dispari} % colori delle righe
\begin{longtabu} to \textwidth {| X[0.5, c m] | X[0.6,c m] | X[0.9,c m] | X[0.8,c m] | X[1.2,c m]|} % larghezza delle colonne
\hline
\rowcolor{header} % colore dell'header

\textbf{Codice} & \textbf{Nome} & \textbf{Formula} & \textbf{Valore sufficiente} & \textbf{Valore ottimo}\\
\hline
\metricstable

\end{longtabu}

\undef\metricstable{}


\subsection{Affidabilità}
Rappresenta la capacità di un prodotto software di mantenere il livello di prestazione quando viene utilizzato in condizioni specificate. Possibili limitazioni all'affidabilità del software possono essere causate da errori di requisiti, nella progettazione, nel codice. Le evidenze di tali errori possono essere rilevate a seconda delle condizioni in cui il prodotto è utilizzato oppure alle opzioni scelte, piuttosto che al momento in cui è utilizzato.
\def\productquality{
    {
        Densità di errori,
        $E_{density} = A_{err}/B_{tests}$, 
        -,
        -,
        è stato ritenuto prematuro definire valori sufficienti e ottimi per la metrica
    },
}
\newcommand*\metricstable{}
\foreach \x [count=\nj] in \productquality
{
    \foreach \y [count=\ni] in \x
    {
        \ifnum\ni=5
            \xappto\metricstable{\y}
            \gappto\metricstable{\\}
            \gappto\metricstable{\hline}
        \else\ifnum\ni=6
            \xappto\metricstable{\noexpand\multicolumn{5}{| l |}{NOTE: \y}}
            \gappto\metricstable{\\}
            \gappto\metricstable{\hline}
        \else
            \xappto\metricstable{\y&}
        \fi\fi
    }
}

% Impostazioni della tabella
\tabulinesep = 2mm % padding
\taburowcolors [1] 2{pari .. dispari} % colori delle righe
\begin{longtabu} to \textwidth {| X[0.5, c m] | X[0.6,c m] | X[0.9,c m] | X[0.8,c m] | X[1.2,c m]|} % larghezza delle colonne
\hline
\rowcolor{header} % colore dell'header

\textbf{Codice} & \textbf{Nome} & \textbf{Formula} & \textbf{Valore sufficiente} & \textbf{Valore ottimo}\\
\hline
\metricstable

\end{longtabu}

\undef\metricstable{}


\subsection{Usabilità}
Rappresenta la capacità di un prodotto software di essere comprensibile, di poter essere studiato, di risultare attraente da parte di un utente sotto determinate condizioni d'uso.
\def\productquality{
    {
        Qualità della messaggistica,
        $Q_{mex} = A_{clear}/B_{tot}$, 
        $Q_{mex} \geq 0.85$,
        $Q_{mex} = 1$
    },
    {
        Numero di click,
        $C_{click}(t) = \#\ di\ click$,
        $C_{click} \leq 6$,
        $C_{click} \leq 4$,
        il task misurato è descritto nelle Norme di Progetto
    },
    {
        Site depth,
        $S_{depth} = td(G)$,
        $S_{depth} \leq 6$,
        $S_{depth} \leq 4$
    },
}
\newcommand*\metricstable{}
\foreach \x [count=\nj] in \productquality
{
    \foreach \y [count=\ni] in \x
    {
        \ifnum\ni=5
            \xappto\metricstable{\y}
            \gappto\metricstable{\\}
            \gappto\metricstable{\hline}
        \else\ifnum\ni=6
            \xappto\metricstable{\noexpand\multicolumn{5}{| l |}{NOTE: \y}}
            \gappto\metricstable{\\}
            \gappto\metricstable{\hline}
        \else
            \xappto\metricstable{\y&}
        \fi\fi
    }
}

% Impostazioni della tabella
\tabulinesep = 2mm % padding
\taburowcolors [1] 2{pari .. dispari} % colori delle righe
\begin{longtabu} to \textwidth {| X[0.5, c m] | X[0.6,c m] | X[0.9,c m] | X[0.8,c m] | X[1.2,c m]|} % larghezza delle colonne
\hline
\rowcolor{header} % colore dell'header

\textbf{Codice} & \textbf{Nome} & \textbf{Formula} & \textbf{Valore sufficiente} & \textbf{Valore ottimo}\\
\hline
\metricstable

\end{longtabu}

\undef\metricstable{}


\subsection{Efficienza}
La capacità di realizzare le funzioni richieste nel minor tempo possibile ed utilizzando nel miglior modo le risorse necessarie, quando opera in determinate condizioni. 
\def\productquality{
    {
        Response time,
        $T_{response} = B_{end}-A_{start}$, 
        -,
        -,
        il task a cui fa riferimento la metrica è indicato nelle norme di progetto
    },
}
\newcommand*\metricstable{}
\foreach \x [count=\nj] in \productquality
{
    \foreach \y [count=\ni] in \x
    {
        \ifnum\ni=5
            \xappto\metricstable{\y}
            \gappto\metricstable{\\}
            \gappto\metricstable{\hline}
        \else\ifnum\ni=6
            \xappto\metricstable{\noexpand\multicolumn{5}{| l |}{NOTE: \y}}
            \gappto\metricstable{\\}
            \gappto\metricstable{\hline}
        \else
            \xappto\metricstable{\y&}
        \fi\fi
    }
}

% Impostazioni della tabella
\tabulinesep = 2mm % padding
\taburowcolors [1] 2{pari .. dispari} % colori delle righe
\begin{longtabu} to \textwidth {| X[0.5, c m] | X[0.6,c m] | X[0.9,c m] | X[0.8,c m] | X[1.2,c m]|} % larghezza delle colonne
\hline
\rowcolor{header} % colore dell'header

\textbf{Codice} & \textbf{Nome} & \textbf{Formula} & \textbf{Valore sufficiente} & \textbf{Valore ottimo}\\
\hline
\metricstable

\end{longtabu}

\undef\metricstable{}




\subsection{Manutenibilità}
Capacità di un prodotto software di essere modificato. Le modifiche possono in cludere correzioni o adattamenti del software a modifiche negli ambienti, nei requisiti e nelle specifiche funzionali.
\def\productquality{
    {
        Complessità ciclomatica,
        $v(G) = e - n + 2p$, 
        -,
        -,
        è stato ritenuto prematuro definire valori sufficienti e ottimi per la metrica
    },
    {
        Indipendenza dei test,
        $I_{test} = A_{ind}/B_{tests}$,
        $I_{test} \geq 0.90$,
        $I_{test} = 1$
    },
    {
        Facilità di comprensione,
        $F_{compr} = A_{comm}/SLOC$,
        -,
        -,
        non sono stati forniti valori sufficienti e ottimi in quanto ritenuto prematuro
    },
    {
        Sfin,
        $sfin(u) =  \sum u_{caller}$,
        $sfin(u) \geq 2$,
        $sfin(u) \geq 4$
    },
    {
        Sfout, 
        $sfout(u) = \sum u_{callee}$,
        $sfout(u) \leq 0$,
        $sfout(u) = 0$
    },
}
\newcommand*\metricstable{}
\foreach \x [count=\nj] in \productquality
{
    \foreach \y [count=\ni] in \x
    {
        \ifnum\ni=5
            \xappto\metricstable{\y}
            \gappto\metricstable{\\}
            \gappto\metricstable{\hline}
        \else\ifnum\ni=6
            \xappto\metricstable{\noexpand\multicolumn{5}{| l |}{NOTE: \y}}
            \gappto\metricstable{\\}
            \gappto\metricstable{\hline}
        \else
            \xappto\metricstable{\y&}
        \fi\fi
    }
}

% Impostazioni della tabella
\tabulinesep = 2mm % padding
\taburowcolors [1] 2{pari .. dispari} % colori delle righe
\begin{longtabu} to \textwidth {| X[0.5, c m] | X[0.6,c m] | X[0.9,c m] | X[0.8,c m] | X[1.2,c m]|} % larghezza delle colonne
\hline
\rowcolor{header} % colore dell'header

\textbf{Codice} & \textbf{Nome} & \textbf{Formula} & \textbf{Valore sufficiente} & \textbf{Valore ottimo}\\
\hline
\metricstable

\end{longtabu}

\undef\metricstable{}
