\section{Introduzione}

    \subsection{Premessa}
    Questo documento, in uso per l'intera durata del progetto, è soggetto a futuri cambiamenti dovuti al procedere del lavoro e al raffinamento dello stesso. Il cambiamento può avvenire sui processi, sulla qualità di prodotto e sulle metriche che sono legate ad essi. Il documento è quindi prodotto in maniera incrementale per garantire la massima qualità di quest'ultimo e di conseguenza del progetto.
    
    \subsection{Scopo del documento}
    Lo scopo del documento è il raggiungimento e mantenimento di requisiti di qualità nelle attività, processi e nella qualità di prodotto tramite metriche oggettive. Per raggiungere questo obiettivo viene svolta una verifica continua sugli argomenti descritti dal documento. Permettendo, quindi, una correzione della qualità in modo veloce ed efficiente.
    
    \subsection{Scopo del prodotto}
    Il prodotto consiste nello sviluppo di una piattaforma web, \textit{HD viz}, che ha lo scopo di fornire all'utente la possibilità di visualizzare dati multimediali con più di 15 dimensioni. Inoltre per la creazione dei grafici verrà usata la libreria D3.js. Mentre i dati verranno estratti da un database SQL o NoSQL.
    
    \subsection{Glossario}
    Il Glossario raccoglie i termini d'interesse relativi al progetto e viene fornito per facilitare la consultazione del documento. I termini di contenuti nel glossario sono marcati con un pedice \glo{}.

    \subsection{Riferimenti}
        \subsubsection{Riferimento normativi}
            \begin{itemize}
                \item \textbf{Capitolato d'appalto C4 - HD Viz: visualizzazione di dati multidimensionali}:
                \href{https://www.math.unipd.it/~tullio/IS-1/2020/Progetto/C4.pdf}{https://www.math.unipd.it/~tullio/IS-1/2020/Progetto/C4.pdf}
                \item \textbf{Norme di Progetto 1.0}
            \end{itemize}
            
        \subsubsection{Riferimenti informativi}
            \begin{itemize}
                \item \textbf{ISO/IEC 12207}:
                \href{https://www.math.unipd.it/~tullio/IS-1/2009/Approfondimenti/ISO_12207-1995.pdf}{PDF ISO/IEC 12207}
                \item \textbf{Indice di Gulpease}:
                \href{https://it.wikipedia.org/wiki/Indice\_Gulpease}{https://it.wikipedia.org/wiki/Indice\_Gulpease}
                \item \textbf{Metriche correlate alla pianificazione}:
                \href{https://www.smartsheet.com/hacking-pmp-how-calculate-schedule-variance}{https://www.smartsheet.com/hacking-pmp-how-calculate-schedule-variance}
                \item \textbf{Complessità ciclomatica}: \href{https://it.wikipedia.org/wiki/Complessit\%C3\%A0\_ciclomatica}{https://it.wikipedia.org/wiki/Complessit\%C3\%A0\_ciclomatica}
                \item \textbf{Sfin e Sfout}: \href{https://www.math.unipd.it/~tullio/IS-1/2004/Approfondimenti/Fan-in\_Fan-out.html}{https://www.math.unipd.it/~tullio/IS-1/2004/Approfondimenti/Fan-in\_Fan-out.html}
                \item \textbf{Diverse metriche e riferimenti a ISO 9126}: \href{http://www.colonese.it/00-Manuali\_Pubblicatii/07-ISO-IEC9126\_v2.pdf}{http://www.colonese.it/00-Manuali\_Pubblicatii/07-ISO-IEC9126\_v2.pdf}
            \end{itemize}
