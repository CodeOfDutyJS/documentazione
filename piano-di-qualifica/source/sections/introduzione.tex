\section{Introduzione}

    \subsection{Premessa}
    Questo documento,essendo pensato per l'uso per la durata intera del progetto, è soggetto a futuri cambiamenti dati dal procedere del lavoro ed al raffinamento dello stesso. Quindi il Piano di Qualifica potrà venir ridefinito in futuro per il raggiungimento del suo mantenimento. Il cambiamento può avvenire sia sui processi, sulla qualità di prodotto e sia sulle metriche che sono legate ad essi. Quindi il documento verrà prodotto in maniera incrementale per garantire la massima qualità del documento e di conseguenza del progetto.
    
    \subsection{Scopo del documento}
    Lo scopo del documento è mantenere un qualità nelle attività,processi e nella qualità di prodotto obbiettivi e metriche oggettive. Per raggiungere questo obiettivo viene svolta una verifica continua sugli argomenti descritti dal documento. Ottenendo, quindi, una correzione della qualità in modo veloce ed efficiente.
    
    \subsection{Scopo del prodotto}
    Il prodotto consiste nello sviluppo di una piattaforma web, \textit{HD viz}, che ha lo scopo di fornire all'utente la possibilità di visualizzare dati multimediali con più di 15 dimensioni. Inoltre per la creazione dei grafici verrà usata la libreria D3.js. Mentre i dati verranno estratti da un database SQL o NoSQL.
    
    \subsection{Glossario}
    Il Glossario che raccoglie i termini d'interesse relativi al progetto viene fornito con gli stessi per facilitarne la consultazione. I termini di contenuti nel glossario sono marcati con un pedice \glo{}.

    \subsection{Riferimenti}
        \subsubsection{Riferimento normativi}
            \begin{itemize}
                \item \textbf{Capitolato d'appalto C4 - HD Viz: visualizzazione di dati multidimensionali}:
                \href{https://www.math.unipd.it/~tullio/IS-1/2020/Progetto/C4.pdf}{https://www.math.unipd.it/~tullio/IS-1/2020/Progetto/C4.pdf}
            \end{itemize}
            
        \subsubsection{Riferimenti informativi}
            \begin{itemize}
                \item \textbf{ISO/IEC 12207}:
                \href{https://www.math.unipd.it/~tullio/IS-1/2009/Approfondimenti/ISO_12207-1995.pdf}{PDF ISO/IEC 12207}
                \item \textbf{Indice di Gulpease}:
                \href{https://it.wikipedia.org/wiki/Indice_Gulpease}{https://it.wikipedia.org/wiki/Indice_Gulpease}
                \item \textbf{Metriche correlate alla pianificazione}:
                \href{https://www.smartsheet.com/hacking-pmp-how-calculate-schedule-variance}{https://www.smartsheet.com/hacking-pmp-how-calculate-schedule-variance}
            \end{itemize}