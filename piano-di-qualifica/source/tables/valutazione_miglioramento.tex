\newcommand*\improvementeval{}
\foreach \x [count=\nj] in \problems
{
    \foreach \y [count=\ni] in \x
    {
        \ifnum\ni=1
            \xappto\improvementeval{\noexpand\textbf{\y}&}
        \else\ifnum\ni=3
            \xappto\improvementeval{\y}
            \gappto\improvementeval{\\}
            \gappto\improvementeval{\hline}
        \else
            \xappto\improvementeval{\y&}
        \fi\fi
    }
}

% Impostazioni della tabella
\tabulinesep = 2mm % padding
\taburowcolors [1] 2{pari .. dispari} % colori delle righe
\begin{longtabu} to \textwidth {| X[0.3,c m] | X[0.3,l m] | X[0.3,l m] |} % larghezza delle colonne
\hline
\rowcolor{header} % colore dell'header

\textbf{Problema} & \textbf{Descrizione} & \textbf{Soluzione} \\
\hline
\improvementeval

\end{longtabu}

\undef\improvementeval{}
