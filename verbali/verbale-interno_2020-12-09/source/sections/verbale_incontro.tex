\section{Verbale Incontro}
    % Descrizione verbosa incontro
	Il primo punto discusso è come stendere la documentazione, la scelta ricade su LaTeX. Per stendere la documentazione viene scelto un approccio incrementale, l'inesperienza del gruppo impedisce l'utilizzo	di altre modalità. Sono discusse le modalità di utilizzo dello strumento di versionamento Github, in particolare dove dove caricare tutti i file generati dall'attività	di progetto. È stato scelto di creare un'organizzazione su github e due repository separate	una contentente l'intera documentazione, una dedicata al il codice sorgente.
	\\ \\ 
	\noindent Per i termini particolari viene costruito un glossario, le parole a glossario sono contrassegnate da una G a pedice.
    Successivamente sono stati normati alcuni punti riguardanti la il processo di documentazione. Sono stati formati 3 sottogruppi ognuno si occupa di definire l'indice dei contenuti di uno tra i seguenti documenti: 
    \begin{itemize}
        \item Analisi dei requisiti;
        \item Norme di progetto;
        \item Piano di Qualifica.
    \end{itemize}	
    
    \noindent Al termine della riuniune è stato deciso il nome del gruppo, "Code of Duty" ed è creata l'organizzazione su Github.