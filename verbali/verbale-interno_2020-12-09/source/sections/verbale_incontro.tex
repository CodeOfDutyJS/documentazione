\section{Verbale Incontro}
    % Descrizione verbosa incontro
	Il primo punto discusso è stato come stendere la documentazione, la scelta è ricaduta su LaTeX. Per redarre la documentazione è stato scelto un approccio incrementale, l'inesperienza del gruppo impedisce l'utilizzo	di altre modalità. Sono state discusse le modalità di utilizzo dello strumento di versionamento Github, in particolare dove dove caricare tutti i file generati dall'attività di progetto. È stato scelto di creare un'organizzazione su github e due repository separate una contentente l'intera documentazione, una dedicata al il codice sorgente.
	\\ \\ 
	\noindent Per i termini particolari viene costruito un glossario, le parole a glossario sono contrassegnate da una G a pedice.
    Successivamente sono stati normati alcuni punti riguardanti la il processo di documentazione. Sono stati formati 3 sottogruppi ognuno si occupa di definire l'indice dei contenuti di uno tra i seguenti documenti: 
    \begin{itemize}
        \item \textit{Analisi dei Requisiti};
        \item \textit{Norme di Progetto};
        \item \textit{Piano di Qualifica}.
    \end{itemize}	
    
    \noindent Al termine della riuniune è stato deciso il nome del gruppo, "Code of Duty" ed è stata creata l'organizzazione su Github.