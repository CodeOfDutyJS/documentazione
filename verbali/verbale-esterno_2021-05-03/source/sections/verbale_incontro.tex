\section{Verbale incontro}
All'ordine del giorno sono stati discussi i seguenti topic:
\begin{itemize}
    \item progressi su \hd ;
    \item tecnica di campionamento.
\end{itemize}
Il proponente risulta soddisfatto dei progressi mostrati e delle visualizzazioni aggiunte all'applicazione web. Riguardo al campionamento suggerisce il campionamento stratificato, \url{https://it.wikipedia.org/wiki/Campionamento_stratificato}, tecnica che permetterebbe di avere una rappresentanza adeguata dell'intera popolazione (i record di un dataset). Tuttavia accetta anche una soluzione che campioni il dataset in modo randomico oppure una soluzione che elimini un numero di record in modo da mantenere bilanciata la popolazione rispetto ad una caratteristica del dataset, in questo ultimo caso si rimarca che la popolazione è rappresentata correttamente solo per quella particolare feature.
\\ \\ 
\noindent
Riguardo il manuale utente, Piccoli approva la soluzione adottata tramite Mardown e suggerisce una soluzione che permetterebbe una maggiore personalizzazione: Markdeep.
\noindent \textbf{Conclusioni:}
\cod\ non ha ancora implementato una forma di campionamento, dopo aver effettuato uno studio di fattibilità si deciderà se implementare il campionamento stratificato oppure una forma di campionamento randomica.