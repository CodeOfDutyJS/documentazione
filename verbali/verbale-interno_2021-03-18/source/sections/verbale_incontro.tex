\section{Verbale Incontro}
L'obiettivo dell'incontro era individuare gli eventuali design pattern di \hd\ e iniziare a disegnare il diagramma delle classi. In particolare sono stati individuati 2 pattern.
\textbf{Model-view-viewmodel}: le motivazioni che hanno spinto il team a scegliere questo pattern architetturale sono 2:
\begin{itemize}
    \item le tecnologie utilizzate, in particolare MobX e React, sembrano spingere verso un pattern di tipo MVC.
    \item MVVM sembra un pattern particolarmente adatto ad un'applicazione web;
\end{itemize}
\noindent \textbf{Template}: \hd\ offre diversi tipi di visualizzazione e i dati prima di essere visualizzati devono essere preparati in modi diversi fra loro, il pattern Template method promette estensibilità e che le classi derivate abbiano la stessa interfaccia.
\\ \\
\noindent Per quanto riguarda il lato server dell'applicazione è stato scelto di non rappresentarlo tramite diagramma delle classi, in quanto è uno script, non è stato utilizzato il paradigma ad oggetti, tantomeno classi. Saranno invece utilizzati diagrammi di sequenza.
\\ \\
\noindent\textbf{Conclusioni:}
I design pattern individuati sono:
\begin{itemize}
    \item Model-view-viewmodel;
    \item Template.
\end{itemize}