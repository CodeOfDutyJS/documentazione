\section{Verbale Incontro}
    Durante l'incontro sono state discusse e mostrate le correzioni ai casi d'uso segnalati dal docente.

    \noindent Gli analisti hanno deciso di introdurre una nuova terminologia, il concetto di \textbf{target} e \textbf{feature}:
    \begin{itemize}
        \item feature: è una variabile, in un database corrisponde al titolo di una colonna, in particolare ogni feature è un elemento misurabile;
        \item target: è una feature di cui si vuole effettuare un'analisi approfondita.
    \end{itemize}

    \noindent È stato discusso un punto emerso durante l'incontro con la proponente Zucchetti cioè se permettere l'esecuzione di query all'interno dell'applicazione web stessa: la decisione è stata rimandata a fasi successive del progetto, ai fini del POC le query saranno inserite in un file di configurazione che conterrà anche il collegamento al/ai database.

    \noindent Al termine dell'incontro è stata inviata un'email al docente e committente R. Cardin, per discutere delle modifiche ai casi d'uso.

    
