\section{Verbale Incontro}
    % Descrizione verbosa incontro
    Il primo argomento trattato sono le modalità della divisione del lavoro: metodo centralizzato o Kanban. Il gruppo ha deciso di utilizzare una strategia ibrida: ogni membro si occupa di aggiornare un documento excel contenente un diagramma di Gantt, preparato da un membro del gruppo in precedenza. Il documento contiene la divisione del lavoro e delle scadenze più precise rispetto ad una milestone. L'assegnazione e divisione dei compiti sono discussi tramite incontri oppure attraverso Telegram. In ogni caso la divisione dei compiti a grana grossa è decisa collettivamente.
    \\ \\
    \noindent Il gruppo si divide in sottogruppi, ogni sottogruppo è responsabile, per l'intero periodo che precede la RR, della stesura di un solo documento.
    \\ \\
    \noindent Sono state esposte e discusse le strutture dei documenti Piano di Qualifica e Analisi dei Requisiti. Riguardo l'Analisi dei Requisiti: la struttura scelta si scosta dallo standard IEEE 830-1998, è stata definita una struttura che dà un ruolo più rilevante ai casi d'uso. Il Piano di Qualifica, in particolare la sezione dedicata alla Qualità di Prodotto, seguirà lo standard ISO 9126. 
    \\ \\
    \noindent Organizzata l'attività dei verificatori per la documentazione: i documenti inizialmente sono verificati dai reddatori, un reddatore in ogni caso non deve verificare quanto ha scritto. Al termine della stesura del documento le attività di verifica e approvazione sono assegnate a membri del team che non sono stati reddatori del documento.
    \\ \\
    \noindent Il team si è assegnato una milestone: terminare la stesura dei documenti entro il 2020-12-31.
