\section{Verbale Incontro}
Durante l'incontro con la proponente Zucchetti e in particolare il referente Gregorio Piccoli è stato discusso il seguente:
\begin{itemize}
    \item come impostare la connessione a un database e se è necessario che \hd\ sia distribuita insieme a un database;
    \item database e sicurezza;
    \item sviluppo visualizzazione proiezione lineare multiasse.
\end{itemize}
Per quanto riguarda la connessione a un database è emerso che è naturale che \hd\ possa essere collegata a diversi tipi di database: il referente Gregorio Piccoli suggerisce inoltre che per acquisire le credenziali necessarie al collegamento si potrebbe utilizzare un file di configurazione (uno per ogni database).
Successivamente sono state discusse diverse problematiche di sicurezza: 
\begin{itemize}
    \item all'utente non devono essere visibili le credenziali di accesso del database, se non durante l'inserimento di queste;
    \item è scorretto permettere l'interrogazione del database tramite query direttamente da \hd , ma potrebbe ad esempio essere disponibile un'interfaccia che permetta operazioni limitate;
    \item se la connessione a un database avviene tramite interfaccia grafica e non tramite un file di configurazione, è necessario impedire gli attacchi brute force, ad esempio inserendo un CAPTCHA dopo il terzo tentativo di accesso errato.
\end{itemize}
Riguardo alla proiezione lineare multiasse: il referente suggerisce di implementarla manualmente invece di affidarsi a librerie di terze parti, essendo un tipo di visualizzazione particolarmente semplice.
\\ \\ \noindent
\textbf{Conclusioni:}
In una breve discussione successiva all'incontro è emerso che, per limitare i problemi di sicurezza \hd\ sarà pensata e sviluppata per essere distribuita in una sola rete nella sua interezza, ciò significa che database e server dovranno risiedere nella stessa rete.