\section{Descrizione generale}

\subsection{Obiettivi del prodotto}
    Lo scopo del progetto è quello di creare un'applicazione web che sia in grado di supportare la fase esplorativa dell'analisi dei dati attraverso diverse visualizzazioni che possono mettere in evidenza pattern nei dati.

\subsection{Funzioni del prodotto}
    L'applicazione deve permettere agli utenti di caricare dei dati da diverse origini e creare una visualizzazione adatta per l'analisi dei dati.
    \\
    In particolare:
    \begin{itemize}
        \item L'utente può:
        \begin{itemize}
            \item caricare i propri dati da file csv o importarli da un database
            \item creare una delle diverse visualizzazioni:
            \begin{itemize}
                \item Scatter Plot Matrix;
                \item Heatmap;
                \item Force Field;
                \item Proiezione lineare;
                \item Correlation Heatmap;
                \item Parallel Coordinates;
            \end{itemize}
            \item salvare la visualizzazione creata dall'applicazione.
        \end{itemize}
    \end{itemize}
    
\subsection{Caratteristiche degli utenti}
L'applicazione \emph{HD Viz} è rivolta agli utenti che vogliono supportare la fase esplorativa dell'analisi dei dati attraverso delle diverse visualizzazioni. Agli utenti si richiede che abbiano già dei dati in formato csv oppure che abbiano la possibilità di effettuare delle query in un database.


\subsection{Architettura del progetto}
\emph{HD Viz} sarà una PWA che comunicherà con la parte server soltanto per il recupero dei dati attraverso query a database.
\\
L'applicazione in questo modo potrà essere eseguita anche in assenza di connessione, conservando tutte le funzioni ad eccezione del recupero dati da un database.

\subsection{Vincoli generali}
L'utente per utilizzare l'applicazione deve disporre dei dati per le visualizzazioni, o un file csv o un database a cui può accedere.
