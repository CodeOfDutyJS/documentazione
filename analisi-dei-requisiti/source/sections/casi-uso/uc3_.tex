\subsection{UC3 - Selezione Manuale delle Etichette}
    
   % \includegraphics{}
    \begin{itemize}
    \item \textbf{Attore}: Utente
    \item \textbf{Descrizione}: Tra le dimensioni (colonne della matrice) del dataset caricato possono essere presenti delle etichette, cioè campi testuali che descrivono la categoria a cui il dato appartiene, l'applicazione offre un meccanismo tramite cui l'utente può selezionare queste tipo di variabili e separarle dalle variabili numeriche che invece compongono le coordinate del dato n-dimensionale.
    \item \textbf{Precondizione}:
    \begin{itemize}
        \item Eseguito l'upload del dataset come matrice N x M.
        \item Selezionato una tra i grafici Scatter Plot, Linear Projection, Force Field per la visualizzazione.
    \end{itemize}
    \item \textbf{Postcondizione}: Le variabili selezionate verranno considerate dal sistema come etichette e non faranno parte del dataset durante l'elaborazione dei dati.
    \item \textbf{Scenario Principale}: 
    \begin{enumerate}
        \item L'utente visualizza le variabili del dataset, che corrispondono alle colonne della matrice, e seleziona le variabili del dato che ritiene essere delle etichette, separandole quindi da quelle che verranno elaborate per la visualizzazione del grafico
    \end{enumerate}  
    \end{itemize}
    
    \subsection{UC4 - Assegnazione delle Caratteristiche di Visualizzazione alle Etichette}
    \includegraphics{}
    \begin{itemize}
    \item \textbf{Attore}: Utente
    \item \textbf{Descrizione}: L'utente decide di assegnare dei diversi modi per distinguere un punto nel grafico, assegnando una classe di visualizzazione (colore, forma) ad ogni sua etichetta, in questo modo risulta più semplice per l'analista vedere dei pattern nella visualizzazione del grafico.
    \item \textbf{Precondizione}:
    \begin{itemize}
        \item Eseguito l'upload del dataset come matrice N x M
        \item Selezionato una tra i grafici Scatter Plot, Linear Projection, Force Field per la visualizzazione.
        \item Nel dataset, almeno una variabile è stata selezionata come etichetta.
    \end{itemize}
    \item \textbf{Postcondizione}: Al variare del valore della etichetta X, i punti visualizzati assumono un diverso attributo della classe assegnata a quest'ultima.
    \item \textbf{Scenario Principale}: 
    \begin{enumerate}
        \item L'utente visualizza tutte le variabili che ha precedentemente indicato come etichette, e ad ognuna di esse associa una particolare classe di visualizzazione. 
        \item L'utente decide che non è interessato a distinguere i punti di una determinata etichetta, e quindi non associa alcuna classe a quella etichetta.
    \end{enumerate}  
    \end{itemize}
    
    \subsection{UC5 - Selezione Manuale delle Variabili}
 %   \includegraphics{}
    \begin{itemize}
    \item \textbf{Attore}: Utente
    \item \textbf{Descrizione}: L'utente ha la possibilità di scartare le variabili di un dato che non gli interessa visualizzare.
    \item \textbf{Precondizione}: 
     \begin{itemize}
        \item Eseguito l'upload del dataset come matrice N x M.
        \item Selezionato grafico Scatter Plot oppure Linear Projection per la visualizzazione.
    \end{itemize}
    \item \textbf{Postcondizione}: La matrice contiene M-i colonne, dove i è il numero di variabili che l'utente ha scelto di non visualizzare.
    \item \textbf{Scenario Principale}: 
    \begin{enumerate}
        \item L'utente visualizza le variabili del dataset, che corrispondono alle colonne della matrice, quelle che vengono selezionate non verranno visualizzate nel grafico
    \end{enumerate}  
    \end{itemize}
    

    \subsection{UC6 - Selezione Automatica delle Features}
  %  \includegraphics{}
    \begin{itemize}
    \item \textbf{Attore}: Utente
    \item \textbf{Descrizione}: Vengono suggerite le features a cui l'algoritmo di classificazione Random Forest attribuisce un punteggio più alto, queste features sono quelle che più incidono nella classificazione dei punti nel target scelto dall'utente.
    \item \textbf{Precondizione}: 
    \begin{itemize}
        \item Eseguito l'upload del dataset come matrice N x M.
        \item Selezionato grafico Scatter Plot oppure Linear Projection per la visualizzazione.
        \item Inserito numero di features da selezionare automaticamente.
    \end{itemize}
    \item \textbf{Postcondizione}: Viene visualizzato il miglior sottoinsieme di features, contenente il numero di variabili selezionato dall'utente, calcolato dall'algoritmo.
    \item \textbf{Scenario Principale}: 
    \begin{enumerate}
        \item L'utente seleziona l'opzione di selezione automatica delle features e inserisce il numero di elementi che il sottoinsieme calcolato dovrà contenere.
    \end{enumerate}
     \item \textbf{Inclusioni}:
        \begin{enumerate}
                \begin{enumerate}
                    \item inserimento del numero di PCs da calcolare (\hyperref[uc6.1]{UC6.1})
                \end{enumerate}
        \end{enumerate} 
    \end{itemize}
    
    \subsubsection{UC6.1 - Selezione Automatica delle Features - Inserimento numero di features}
    \label{uc6.1}
    \begin{itemize}
    \item \textbf{Attore}: Utente
    \item \textbf{Descrizione}: L'utente specifica da quante features sarà composto il sottoinsieme generato dall'algoritmo Random Forest  
    \item \textbf{Precondizione}: 
    \begin{itemize}
        \item Eseguito l'upload del dataset come matrice N x M.
        \item Selezionato grafico Scatter Plot oppure Linear Projection per la visualizzazione.
        \item Selezionata Selezione Automatica delle Features
    \end{itemize}  
    \item \textbf{Postcondizione}: Il numero di features da calcolare è stato inserito.
    \item \textbf{Scenario Principale}: 
    \begin{enumerate}
        \item L'utente inserisce un numero compreso tra 1 e il numero di features.
    \end{enumerate}  
    \end{itemize}
  
    
        \subsection{UC7 - Assegnare Colore al Range in una Heatmap}
  %  \includegraphics{}
    \begin{itemize}
    \item \textbf{Attore}: Utente
    \item \textbf{Descrizione}: L'utente seleziona il colore con cui vuole evidenziare i valori di una Heatmap, la sfumatura più bassa del colore selezionato verrà associata ai valori più bassi e la sfumatura più alta ai valori più alti.
    \item \textbf{Precondizione}: 
    \begin{itemize}
        \item Eseguito l'upload del dataset come matrice N x M.
        \item Selezionato grafico Heatmap per la visualizzazione
    \end{itemize}  
    \item \textbf{Postcondizione}: La Heatmap conterrà nelle sue caselle le sfumature del colore selezionato.
    \item \textbf{Scenario Principale}: 
    \begin{enumerate}
        \item L'utente seleziona il colore con cui vuole visualizzare la Heatmap.
    \end{enumerate}  
    \end{itemize}
    
     \subsection{UC8 - Modificare il Range di una Heatmap}
  %  \includegraphics{}
    \begin{itemize}
    \item \textbf{Attore}: Utente
    \item \textbf{Descrizione}: L'utente decide di modificare il range (range di default=[min, max] dove min=valore minimo presente nel dataset e max=valore massimo presente nel dataset), su cui ogni valore assume una diversa sfumatura di colore, cioè se viene scelto un range [min, max-i], tutti i valori >=max-i presenti nel dataset verranno visualizzati con lo stesso colore.
    \item \textbf{Precondizione}: 
    \begin{itemize}
        \item Eseguito l'upload del dataset come matrice N x M.
        \item Selezionato grafico Heatmap per la visualizzazione
    \end{itemize}  
    \item \textbf{Postcondizione}: Ricolorazione della Heatmap in base al range selezionato: il valore più scuro sarà associato al valore più alto del nuovo range e il valore più chiaro al valore più basso.
    \item \textbf{Scenario Principale}: 
    \begin{enumerate}
        \item L'utente modifica i valori di minimo e massimo su cui il sistema andrà a eseguire la sfumatura del colore
    \end{enumerate}  
    \end{itemize}
    
    \subsection{UC9 - Normalizzazione del dataset}
 %   \includegraphics{}
    \begin{itemize}
    \item \textbf{Attore}: Utente
    \item \textbf{Descrizione}: Un dataset può contenere dati non normalizzati, cioè la media è diversa da 0 o la varianza diversa da 1, normalizzare il dataset può risultare utile per una migliore visualizzazione dei dati.
    \item \textbf{Precondizione}: 
    \begin{itemize}
        \item Eseguito l'upload del dataset come matrice N x M.
        \item Selezionato grafico Heatmap per la visualizzazione
    \end{itemize}  
    \item \textbf{Postcondizione}: 
    \item \textbf{Scenario Principale}: Il dataset contenente dati non normalizzati viene normalizzato a seconda del tipo di normalizzazione scelto.
    \begin{enumerate}
        \item L'utente seleziona l'opzione di normalizzazione e specifica quale desidera effettuare.
    \end{enumerate}  
    \item \textbf{Generalizzazioni}: 
     \begin{enumerate}
            \item L'utente sceglie su quale subset di dati effettuare la normalizzazione
                \begin{enumerate}
                    \item Normalizzazione Globale (\hyperref[uc9.1]{UC9.1})
                    \item Normalizzazione per Colonna (\hyperref[uc9.2]{UC9.2})
                    \item Normalizzazione per Riga (\hyperref[uc9.3]{UC9.3})
                \end{enumerate}
        \end{enumerate}  
    \end{itemize}
    
    \subsubsection{UC9.1 - Normalizzazione Globale del dataset}
    \label{uc9.1}
    \begin{itemize}
    \item \textbf{Attore}: Utente
    \item \textbf{Descrizione}: Un dataset può contenere dati non normalizzati, cioè la media è diversa da 0 o la varianza diversa da 1, normalizzare il dataset può risultare utile per una migliore visualizzazione dei dati.
    In questo caso la normalizzazione viene fatta su tutto il dataset.
    \item \textbf{Precondizione}: 
    \begin{itemize}
        \item Eseguito l'upload del dataset come matrice N x M.
        \item Selezionato grafico Heatmap per la visualizzazione
    \end{itemize}  
    \item \textbf{Postcondizione}:  Ogni valore \({x_i}_j\) del dataset viene sostituito da \( {y_i}_j = \frac{{x_i}_j - \mu}{\sigma}\) dove \(\mu\) è la media dei valori presenti nel data set e \(\sigma\) la loro varianza.
    \item \textbf{Scenario Principale}: 
    \begin{enumerate}
        \item L'utente seleziona l'opzione di normalizzazione e specifica che desidera una Normalizzazione Globale.
    \end{enumerate}  
    \end{itemize}
    
    \subsubsection{UC9.2 - Normalizzazione per Colonna del dataset}
    \label{uc9.2}
    \begin{itemize}
    \item \textbf{Attore}: Utente
    \item \textbf{Descrizione}: Un dataset può contenere dati non normalizzati, cioè la media è diversa da 0 o la varianza diversa da 1, normalizzare il dataset può risultare utile per una migliore visualizzazione dei dati.
    In questo caso la normalizzazione viene fatta colonna per colonna.
    \item \textbf{Precondizione}: 
    \begin{itemize}
        \item Eseguito l'upload del dataset come matrice N x M.
        \item Selezionato grafico Heatmap per la visualizzazione
    \end{itemize}  
    \item \textbf{Postcondizione}: Ogni valore \({x_i}_j\) del dataset viene sostituito da \( {y_i}_j = \frac{{x_i}_j - \mu}{\sigma}\) dove \(\mu\) è la media dei valori presenti nella colonna j e \(\sigma\) la loro varianza sotto radice.
    \item \textbf{Scenario Principale}: 
    \begin{enumerate}
        \item L'utente seleziona l'opzione di normalizzazione e specifica che desidera una Normalizzazione per Colonna. 
    \end{enumerate}  
    \end{itemize}
    
    \subsubsection{UC9.3 - Normalizzazione per Riga del dataset}
    \label{uc9.3}
    \begin{itemize}
    \item \textbf{Attore}: Utente
    \item \textbf{Descrizione}: Un dataset può contenere dati non normalizzati, cioè la media è diversa da 0 o la varianza diversa da 1, normalizzare il dataset può risultare utile per una migliore visualizzazione dei dati.
    In questo caso la normalizzazione viene fatta riga per riga.
    \item \textbf{Precondizione}: 
    \begin{itemize}
        \item Eseguito l'upload del dataset come matrice N x M.
        \item Selezionato grafico Heatmap per la visualizzazione
    \end{itemize}  
    \item \textbf{Postcondizione}:  Ogni valore \({x_i}_j\) del dataset viene sostituito da \( {y_i}_j = \frac{{x_i}_j - \mu}{\sigma}\) dove \(\mu\) è la media dei valori presenti nella riga i e \(\sigma\) la loro varianza sotto radice.
    \item \textbf{Scenario Principale}: 
    \begin{enumerate}
        \item L'utente seleziona l'opzione di normalizzazione e specifica che desidera una Normalizzazione per Riga. 
    \end{enumerate}  
    \end{itemize}
    
    \subsection{UC10 - Ordinamento delle righe di una Heatmap}
  %  \includegraphics{}
    \begin{itemize}
    \item \textbf{Attore}: Utente
    \item \textbf{Descrizione}: L'utente seleziona in che modo organizzare la visualizzazione delle righe di una heatmap, scegliendo tra l'ordine alfabetico o raggruppamento in clusters.
    \item \textbf{Precondizione}: 
    \begin{itemize}
        \item Eseguito l'upload del dataset come matrice N x M.
        \item Selezionato grafico Heatmap per la visualizzazione
    \end{itemize}  
    \item \textbf{Postcondizione}: Riordinazione della Heatmap in base all'ordinamento selezionato.
    \item \textbf{Scenario Principale}: 
    \begin{enumerate}
        \item L'utente sceglie quale ordinamento desidera effettuare.
    \end{enumerate}  
    \item \textbf{Generalizzazioni}: 
     \begin{enumerate}
            \item L'utente sceglie quale tipo di ordinamento
                \begin{enumerate}
                    \item Ordine Alfabetico (\hyperref[uc10.1]{UC10.1})
                    \item Raggruppamento in Clusters (\hyperref[uc10.2]{UC10.2})
                    \end{enumerate}
        \end{enumerate} 
    \end{itemize}
    
    \subsubsection{UC10.1 - Ordinamento Heatmap in Ordine Alfabetico}
    \label{uc10.1}
    \includegraphics{}
    \begin{itemize}
    \item \textbf{Attore}: Utente
    \item \textbf{Descrizione}: Ordinamento delle righe della heatmap in ordine alfabetico.
    \item \textbf{Precondizione}: 
    \begin{itemize}
        \item Eseguito l'upload del dataset come matrice N x M.
        \item Selezionato grafico Heatmap per la visualizzazione
        \item Selezionata una etichetta di categoria su cui effettuare l'ordinamento.
    \end{itemize}  
    \item \textbf{Postcondizione}: Riordinazione delle righe della Heatmap in ordine alfabetico.
    \item \textbf{Scenario Principale}: 
    \begin{enumerate}
        \item L'utente sceglie di ordinare le righe in ordine alfabetico.
    \end{enumerate}  
    \end{itemize}
    
    \subsubsection{UC10.2 - Ordinamento Heatmap per Clusters}
    \label{uc10.2}
 %   \includegraphics{}
    \begin{itemize}
    \item \textbf{Attore}: Utente
    \item \textbf{Descrizione}: Le righe vengono raggruppate secondo un algoritmo di cluster gerarchico, la distanza calcolata  tra le righe, utilizzata dall'algoritmo, è quella euclidea.
    \item \textbf{Precondizione}: 
    \begin{itemize}
        \item Eseguito l'upload del dataset come matrice N x M.
        \item Selezionato grafico Heatmap per la visualizzazione
    \end{itemize}  
    \item \textbf{Postcondizione}: Le righe sono raggruppate secondo l'algoritmo di Cluster Gerarchico ed è visualizzato il dendrogramma sviluppato dall'algoritmo.
    \item \textbf{Scenario Principale}: 
    \begin{enumerate}
        \item L'utente sceglie di ordinare le righe tramite clustering.
    \end{enumerate}  
    \end{itemize}
    
    \subsection{UC11 - PCA Principal Components Analysis}
    % \includegraphics{}
    \begin{itemize}
    \item \textbf{Attore}: Utente
    \item \textbf{Descrizione}: Il PCA è una tecnica di riduzione dimensionale, cioè a partire da n features ne calcola k (k fissato) in modo che queste nuove k feature meglio approssimano le n features iniziali.
    \item \textbf{Precondizione}: 
    \begin{itemize}
        \item Eseguito l'upload del dataset come matrice N x M.
        \item Inserito numero di Principal Components (PCs) da visualizzare
    \end{itemize}  
    \item \textbf{Postcondizione}: Il dataset contiene k nuove colonne (dove k è il numero di PCs che l'utente ha scelto di calcolare), che sono le k proiezioni calcolate da PCA. 
    \item \textbf{Scenario Principale}: 
    \begin{enumerate}
        \item L'utente sceglie di applicare l'algoritmo PCA sul dataset.
    \end{enumerate}  
    \item \textbf{Inclusioni}:
        \begin{enumerate}
                \begin{enumerate}
                    \item inserimento del numero di PCs da calcolare (\hyperref[uc11.1]{UC11.1})
                \end{enumerate}
        \end{enumerate} 
    \end{itemize}
    
    \subsubsection{UC11.1 - PCA - Inserimento numero di PCs da calcolare}
    \label{uc11.1}
    \begin{itemize}
    \item \textbf{Attore}: Utente
    \item \textbf{Descrizione}: L'utente specifica quante nuove features il PCA deve calcolare.
    \item \textbf{Precondizione}: 
    \begin{itemize}
        \item Eseguito l'upload del dataset come matrice N x M.
        \item Selezionato PCA
    \end{itemize}  
    \item \textbf{Postcondizione}: Il numero k di features da calcolare è stato inserito.
    \item \textbf{Scenario Principale}: 
    \begin{enumerate}
        \item L'utente inserisce un numero k compreso tra 1 e il numero di features.
    \end{enumerate}  
    \end{itemize}
    
    \subsection{UC12 - Calcolo Matrice di Distanza}
    % \includegraphics{}
    \begin{itemize}
    \item \textbf{Attore}: Utente
    \item \textbf{Descrizione}: Data una matrice N x M viene calcolata la distanza tra ogni riga e viene restituita all'utente una matrice di distanza N x N dove \({x_i}_j\) è la distanza tra la riga i e la riga j della matrice di partenza.
    \item \textbf{Precondizione}: 
    \begin{itemize}
        \item Eseguito l'upload del dataset come matrice N x M.
        \item Selezionato grafico Heatmap o Force Field per la visualizzazione.
    \end{itemize}  
    \item \textbf{Postcondizione}: Calcolata matrice di distanza N x N dove \({x_i}_j\) è la distanza tra la riga i e la riga j della matrice di partenza.
    \item \textbf{Scenario Principale}: 
    \begin{enumerate}
        \item L'utente decide di calcolare la matrice di distanza corrispondente al dataset caricato, utilizzando un algoritmo di distanza a scelta tra Euclidea e Manhattan.
    \end{enumerate}  
    \item \textbf{Inclusioni}:
        \begin{enumerate}
                \begin{enumerate}
                    \item Scelta dell'algoritmo di distanza da utilizzare (\hyperref[uc6]{UC6})
                \end{enumerate}
        \end{enumerate} 
    \end{itemize}
    
  
    
    
    
    
    
    
    
    