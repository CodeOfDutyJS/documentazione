\subsection{UC7 - Selezione algoritmo di riduzione delle componenti}
    \label{uc7}
    
    \begin{figure}[htbp]
        \centering
        \includegraphics[width=0.9\textwidth]{source/sections/casi-uso/diagrams/uc7.pdf}
        \caption{UC7 - Selezione algoritmo di riduzione delle componenti}
        \label{fig:uc7}
    \end{figure}
    
    \begin{itemize}
    \item \textbf{Attore}: Utente
    \item \textbf{Descrizione}: L'utente sceglie l'algoritmo per la riduzione delle componenti per l'elaborazione dati
    \item \textbf{Precondizione}: 
    \begin{itemize}
        \item Eseguito l'upload del dataset come matrice $N\times M$ (\hyperref[uc1]{UC1}).
        \item Selezionato Linear Projection come visualizzazione \hyperref[uc2.5]{UC2.5}).
    \end{itemize}  
    \item \textbf{Postcondizione}: L'utente ha scelto l'algoritmo per la riduzione delle componenti
    \item \textbf{Scenario Principale}: 
    \begin{enumerate}
        \item L'utente seleziona l'algoritmo per la riduzione delle componenti tra quelli disponibili
    \end{enumerate}
    \item \textbf{Generalizzazioni}:
        \begin{enumerate}
            \item L'utente seleziona una delle seguenti distanze
                \begin{enumerate}
                    \item PCA (\hyperref[uc7.1]{UC7.1})
                    \item UMAP (\hyperref[uc7.2]{UC7.2})
                    \item t-SNE (\hyperref[uc7.3]{UC7.3})
                \end{enumerate}
        \end{enumerate}  
    \end{itemize}
    
    \subsubsection{UC7.1 - PCA}
    \label{uc7.1}
    \begin{itemize}
    \item \textbf{Attore}: Utente
    \item \textbf{Descrizione}: Il PCA è una tecnica di riduzione dimensionale, cioè a partire da n features ne calcola k (k fissato) in modo che queste nuove k feature meglio approssimano le n features iniziali.
    \item \textbf{Precondizione}: 
    \begin{itemize}
        \item Eseguito l'upload del dataset come matrice $N\times M$ (\hyperref[uc1]{UC1}).
        \item Selezionato Linear Projection come visualizzazione (\hyperref[uc2.5]{UC2.5}).
    \end{itemize}  
    \item \textbf{Postcondizione}: Il dataset contiene k nuove colonne (dove k è il numero di PCs che l'utente ha scelto di calcolare), che sono le k proiezioni calcolate da PCA. 
    \item \textbf{Scenario Principale}: 
    \begin{enumerate}
        \item L'utente sceglie di applicare l'algoritmo PCA sul dataset.
    \end{enumerate}  
    \item \textbf{Inclusioni}:
        \begin{enumerate}
                \item \begin{enumerate}
                    \item Inserimento del numero di PCs da calcolare (\hyperref[uc7.1.1]{UC7.1.1})
                \end{enumerate}
        \end{enumerate} 
    \end{itemize}
    
    \paragraph{UC7.1.1 - PCA - Inserimento numero di PCs da calcolare}
    \label{uc7.1.1}
    \begin{itemize}
    \item \textbf{Attore}: Utente
    \item \textbf{Descrizione}: L'utente specifica quante nuove features il PCA deve calcolare.
    \item \textbf{Precondizione}: 
    \begin{itemize}
        \item Eseguito l'upload del dataset come matrice $N\times M$ (\hyperref[uc1]{UC1}).
        \item Selezionato Linear Projection come visualizzazione (\hyperref[uc2.5]{UC2.5}).
        \item Selezionato PCA (\hyperref[uc7.1]{UC7.1})
    \end{itemize}  
    \item \textbf{Postcondizione}: Il numero k di features da calcolare è stato inserito.
    \item \textbf{Scenario Principale}: 
    \begin{enumerate}
        \item L'utente inserisce un numero k compreso tra 1 e il numero di features.
    \end{enumerate}  
    \end{itemize}
    
    \subsubsection{UC7.2 - UMAP}
    \label{uc7.2}
    \begin{itemize}
    \item \textbf{Attore}: Utente
    \item \textbf{Descrizione}: UMAP è una tecnica di riduzione dimensionale, cioè a partire da n features ne calcola k (k fissato) in modo che queste nuove k features meglio approssimano le n features iniziali.
    \item \textbf{Precondizione}: 
    \begin{itemize}
        \item Eseguito l'upload del dataset come matrice $N\times M$ (\hyperref[uc1]{UC1}).
        \item Selezionato Linear Projection come visualizzazione (\hyperref[uc2.5]{UC2.5}).
    \end{itemize}  
    \item \textbf{Postcondizione}: L'utente seleziona UMAP come algoritmo per la riduzione dimensionale
    % [[Sono calcolate due nuove features, cioè le coordinate in cui UMAP ha mappato i punti]]
    \item \textbf{Scenario Principale}: 
    \begin{enumerate}
        \item L'utente sceglie di applicare l'algoritmo UMAP sul dataset.
    \end{enumerate}
    \end{itemize}
    
    \subsubsection{UC7.3 - t-SNE}
    \label{uc7.3}
    \begin{itemize}
    \item \textbf{Attore}: Utente
    \item \textbf{Descrizione}: t-SNE è una tecnica di riduzione dimensionale, cioè a partire da n features ne calcola k (k fissato) in modo che queste nuove k features meglio approssimano le n features iniziali.
    \item \textbf{Precondizione}: 
    \begin{itemize}
        \item Eseguito l'upload del dataset come matrice $N\times M$ (\hyperref[uc1]{UC1}).
        \item Selezionato Linear Projection come visualizzazione (\hyperref[uc2.5]{UC2.5}).
    \end{itemize}  
    \item \textbf{Postcondizione}: L'utente seleziona t-SNE come algoritmo per la riduzione dimensionale 
    % [[Sono calcolate due nuove features, cioè le coordinate in cui il t-SNE ha mappato i punti]]
    \item \textbf{Scenario Principale}: 
    \begin{enumerate}
        \item L'utente sceglie di applicare l'algoritmo t-SNE sul dataset.
    \end{enumerate}
    \end{itemize}
    
    