\subsection{UC8 - Elaborazione dati}
    \label{uc8}
    \begin{itemize}
    \item \textbf{Attore}: Utente
    \item \textbf{Descrizione}: Vengono elaborati i dati secondo le impostazioni
    \item \textbf{Precondizione}: 
    \begin{itemize}
        \item Eseguito l'upload del dataset come matrice $N\times M$ (\hyperref[uc1]{UC1}).
        \item Selezionato un tipo di visualizzazione (\hyperref[uc2]{UC2}).
    \end{itemize}  
    \item \textbf{Postcondizione}: L'applicazione dopo che l'utente ha caricato il suo dataset e selezionato il tipo di visualizzazione, ha elaborato i dati secondo le impostazioni
    \item \textbf{Scenario Principale}: 
    \begin{enumerate}
        \item L'utente carica il suo dataset (\hyperref[uc1]{UC1})
        \item L'utente seleziona il tipo di visualizzazione tra quelle disponibili (\hyperref[uc2]{UC2})
        \item L'applicazione elabora i dati con impostazioni di default (\hyperref[uc8.1]{UC8.1})
        \item L'utente cambia le impostazioni (\hyperref[uc3]{UC3} o \hyperref[uc4]{UC4} \hyperref[uc5]{UC5} o \hyperref[uc6]{UC6} o \hyperref[uc7]{UC7})
        \item L'applicazione rielabora i dati (\hyperref[uc8.2]{UC8.2})
    \end{enumerate}
    \end{itemize}
    
    % Nel caso di visualizzazioni che richiedono delle impostazioni obbligatorie? (esempio calcolare matrice di distanza prima di visualizzare il grafico force-field)
    \subsubsection{UC8.1 - Avvio elaborazione dati con impostazioni di default}
    \label{uc8.1}
    \begin{itemize}
    \item \textbf{Attore}: Utente
    \item \textbf{Descrizione}: L'utente avvia l'elaborazione dati secondo le impostazioni di deafult
    \item \textbf{Precondizione}: 
    \begin{itemize}
        \item Eseguito l'upload del dataset come matrice $N\times M$ (\hyperref[uc1]{UC1}).
        \item Selezionato un tipo di visualizzazione (\hyperref[uc2]{UC2}).
    \end{itemize}  
    \item \textbf{Postcondizione}: L'applicazione dopo che l'utente dopo ha caricato il suo dataset e selezionato il tipo di visualizzazione, ha avviato l'elaborazione dei dati secondo le impostazioni di default
    \item \textbf{Scenario Principale}: 
    \begin{enumerate}
        \item L'utente carica il suo dataset (\hyperref[uc1]{UC1})
        \item L'utente seleziona il tipo di visualizzazione tra quelle disponibili (\hyperref[uc2]{UC2})
        \item L'applicazione elabora i dati
    \end{enumerate}
    \end{itemize}
    
    \subsubsection{UC8.2 - Avvio elaborazione dati con impostazioni utente}
    \label{uc8.2}
    \begin{itemize}
    \item \textbf{Attore}: Utente
    \item \textbf{Descrizione}:  L'utente avvia l'elaborazione dati secondo le impostazioni personalizzate
    \item \textbf{Precondizione}: 
    \begin{itemize}
        \item Eseguito l'upload del dataset come matrice $N\times M$ (\hyperref[uc1]{UC1}).
        \item Selezionato un tipo di visualizzazione (\hyperref[uc2]{UC2}).
        \item L'utente ha cambiato le impostazioni (\hyperref[uc3]{UC3} o \hyperref[uc4]{UC4} \hyperref[uc5]{UC5} o \hyperref[uc6]{UC6} o \hyperref[uc7]{UC7})
    \end{itemize}  
    \item \textbf{Postcondizione}: L'applicazione dopo che l'utente ha caricato il suo dataset e selezionato il tipo di visualizzazione, ha avviato l'elaborazione dei dati secondo le impostazioni personalizzate inserite dall'utente
    \item \textbf{Scenario Principale}: 
    \begin{enumerate}
        \item L'utente carica il suo dataset (\hyperref[uc1]{UC1})
        \item L'utente seleziona il tipo di visualizzazione tra quelle disponibili (\hyperref[uc2]{UC2})
        \item L'applicazione elabora i dati (\hyperref[uc8.1]{UC8.1})
        \item L'utente cambia le impostazioni (\hyperref[uc3]{UC3} o \hyperref[uc4]{UC4} \hyperref[uc5]{UC5} o \hyperref[uc6]{UC6} o \hyperref[uc7]{UC7})
        \item L'applicazione rielabora i dati
    \end{enumerate}
    \end{itemize}