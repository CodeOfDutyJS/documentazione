\subsection{UC4 - Selezione manuale delle Features}
\label{uc4}
 %   \includegraphics{}
    \begin{itemize}
    \item \textbf{Attore}: Utente
    \item \textbf{Descrizione}: L'utente può scartare le features di un dato che non è interessato a visualizzare.
    \item \textbf{Precondizione}: 
     \begin{itemize}
        \item Eseguito l'upload del dataset come matrice $N\times M$ (\hyperref[uc1]{UC1}).
        \item Selezionato un tipo di visualizzazione (\hyperref[uc2]{UC2}).
    \end{itemize}
    \item \textbf{Postcondizione}: La matrice contiene $M-i$ colonne, dove $i$ è il numero di features che l'utente ha scartato.
    \item \textbf{Scenario Principale}: 
    \begin{enumerate}
        \item L'utente visualizza le features del dataset, che corrispondono alle colonne della matrice
        \item L'utente scarta le features a cui non è interessato
    \end{enumerate}
    \item \textbf{Generalizzazioni}:
        \begin{enumerate}
            \item Selezione manuale delle features per Scatter Plot Matrix (\hyperref[uc4.1]{UC4.1})
        \end{enumerate}
    \end{itemize}
    
    \subsubsection{UC4.1 - Selezione manuale delle Features per Scatter Plot Matrix}
    \label{uc4.1}
    \begin{itemize}
    \item \textbf{Attore}: Utente
    \item \textbf{Descrizione}: L'utente può selezionare al massimo 5 features da visualizzare.
    \item \textbf{Precondizione}: 
     \begin{itemize}
        \item Eseguito l'upload del dataset come matrice $N\times M$ (\hyperref[uc1]{UC1}).
        \item Selezionato Scatter Plot Matrix come visualizzazione (\hyperref[uc2.1]{UC2.1}).
    \end{itemize}
    \item \textbf{Postcondizione}: La matrice contiene al massimo 5 colonne cioè le features che l'utente ha selezionato.
    \item \textbf{Scenario Principale}: 
    \begin{enumerate}
        \item L'utente visualizza le features del dataset, che corrispondono alle colonne della matrice
        \item L'utente sceglie le features che è intenzionato a visualizzare
    \end{enumerate}  
    \end{itemize}