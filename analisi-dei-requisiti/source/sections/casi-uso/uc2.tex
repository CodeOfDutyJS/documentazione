\subsection{UC2 - Visualizzazione dei dati}
\label{uc2}

    \begin{figure}[htbp]
        \centering
        \includegraphics[width=0.45\textwidth]{source/sections/casi-uso/diagrams/uc2.pdf}
        \caption{UC2 - Visualizzazione dei dati}
        \label{fig:uc2}
    \end{figure}

    % \includegraphics{}
    \begin{itemize}
    \item \textbf{Attore}: utente;
    \item \textbf{Descrizione}: l'utente all'interno dell'applicazione visualizza i dati
    \item \textbf{Precondizione}:
    \begin{itemize}
        \item il sistema è funzionante e raggiungibile;
        \item eseguito l'upload del dataset come matrice $N\times M$ (\hyperref[uc1]{UC1});
    \end{itemize}
    \item \textbf{Postcondizione}: l'utente visualizza i dati tramite uno dei grafici disponibili
    \item \textbf{Scenario Principale}: 
        \begin{enumerate}
            \item l'utente visualizza il grafico costruito a partire dai dati caricati.
        \end{enumerate}  
    \item \textbf{Generalizzazioni}:
        \begin{enumerate}
            \item l'utente visualizza i dati tramite una delle seguenti visualizzazioni:
                \begin{enumerate}
                    \item Scatter Plot Matrix (\hyperref[uc2.1]{UC2.1});
                    \item Heatmap (\hyperref[uc2.2]{UC2.2});
                    \item Correlation Heatmap (\hyperref[uc2.3]{UC2.3});
                    \item Force Field (\hyperref[uc2.4]{UC2.4});
                    \item Linear Projection (\hyperref[uc2.5]{UC2.5});
                    \item Parallel Coordinates (\hyperref[uc2.6]{UC2.6}).
                \end{enumerate}
        \end{enumerate}  
    \end{itemize}
    
    %%%
    \subsubsection{UC2.1 - Scatter Plot Matrix}
    \label{uc2.1}
    
    \begin{itemize}
    \item \textbf{Attore}: utente;
    \item \textbf{Descrizione}: l'utente visualizza i dati tramite \emph{Scatter Plot Matrix};
    \item \textbf{Precondizione}:
    \begin{itemize}
        \item il sistema è funzionante e raggiungibile;
        \item eseguito l'upload del dataset come matrice $N\times M$ (\hyperref[uc1]{UC1});
        \item le features del dataset da visualizzare sono al massimo 5
    \end{itemize}
    \item \textbf{Postcondizione}: l'utente visualizza i dati caricati tramite \emph{Scatter Plot Matrix};
    \item \textbf{Scenario Principale}: 
        \begin{enumerate}
            \item l'utente visualizza la Scatter Plot Matrix costruita a partire dai dati caricati.
        \end{enumerate}
    \end{itemize}
    
    %%%
    \subsubsection{UC2.2 - Heatmap}
    \label{uc2.2}
    
    \begin{itemize}
    \item \textbf{Attore}: utente;
    \item \textbf{Descrizione}: l'utente visualizza i dati tramite \emph{Heatmap};
    \item \textbf{Precondizione}:
    \begin{itemize}
        \item il sistema è funzionante e raggiungibile;
        \item eseguito l'upload del dataset come matrice $N\times M$ (\hyperref[uc1]{UC1});
    \end{itemize}
    \item \textbf{Postcondizione}: l'utente visualizza i dati caricati tramite \emph{Heatmap};
    \item \textbf{Scenario Principale}: 
        \begin{enumerate}
            \item \item l'utente visualizza la Heatmap costruita a partire dai dati caricati.
        \end{enumerate}
    \end{itemize}
    
    %%%
    \subsubsection{UC2.3 - Correlation Heatmap}
    \label{uc2.3}
    
    \begin{itemize}
    \item \textbf{Attore}: utente;
    \item \textbf{Descrizione}: l'utente visualizza i dati tramite \emph{Correlation Heatmap};
    \item \textbf{Precondizione}:
    \begin{itemize}
        \item il sistema è funzionante e raggiungibile;
        \item eseguito l'upload del dataset come matrice $N\times M$ (\hyperref[uc1]{UC1});
    \end{itemize}
    \item \textbf{Postcondizione}: l'utente visualizza i dati caricati tramite \emph{Correlation Heatmap};
    \item \textbf{Scenario Principale}: 
        \begin{enumerate}
            \item l'utente visualizza la Correlation Heatmap costruita a partire dai dati caricati.
        \end{enumerate}
    \end{itemize}
    
    %%%
    \subsubsection{UC2.4 - Force Field}
    \label{uc2.4}
    
    \begin{itemize}
    \item \textbf{Attore}: utente
    \item \textbf{Descrizione}: l'utente visualizza i dati tramite \emph{Force Field};
    \item \textbf{Precondizione}:
    \begin{itemize}
        \item il sistema è funzionante e raggiungibile;
       \item eseguito l'upload del dataset come matrice $N\times M$ (\hyperref[uc1]{UC1});
    \end{itemize}
    \item \textbf{Postcondizione}: l'utente visualizza i dati caricati tramite \emph{Force Field};
    \item \textbf{Scenario Principale}: 
        \begin{enumerate}
            \item l'utente visualizza il Force Field costruito a partire dai dati caricati.
        \end{enumerate}
    \end{itemize}
    
    %%%
    \subsubsection{UC2.5 - Linear Projection}
    \label{uc2.5}
    
    \begin{itemize}
    \item \textbf{Attore}: utente;
    \item \textbf{Descrizione}: l'utente visualizza i dati tramite \emph{Linear Projection};
    \item \textbf{Precondizione}:
    \begin{itemize}
        \item il sistema è funzionante e raggiungibile;
        \item eseguito l'upload del dataset come matrice $N\times M$ (\hyperref[uc1]{UC1});
    \end{itemize}
    \item \textbf{Postcondizione}: l'utente visualizza i dati caricati tramite \emph{Linear Projection};
    \item \textbf{Scenario Principale}: 
        \begin{enumerate}
            \item l'utente visualizza la \emph{Linear Projection} costruita a partire dai dati caricati.
        \end{enumerate}
    \end{itemize}
    
    %%%
    \subsubsection{UC2.6 - Parallel Coordinates}
    \label{uc2.6}
    
    \begin{itemize}
    \item \textbf{Attore}: utente;
    \item \textbf{Descrizione}:  l'utente visualizza i dati caricati tramite \emph{Parallel Coordinates};
    \item \textbf{Precondizione}:
    \begin{itemize}
        \item il sistema è funzionante e raggiungibile;
        \item eseguito l'upload del dataset come matrice $N\times M$ (\hyperref[uc1]{UC1});
    \end{itemize}
    \item \textbf{Postcondizione}: l'utente visualizza i dati caricati tramite \emph{Parallel Coordinates};
    \item \textbf{Scenario Principale}: 
        \begin{enumerate}
            \item l'utente visualizza la \emph{Parallel Coordinates} costruita a partire dai dati caricati.
        \end{enumerate}
    \end{itemize}
    
  
