\subsection{UC6 - Calcolo riduzione delle componenti}
    \label{uc6}
    
    \begin{figure}[htbp]
        \centering
        \includegraphics[width=0.9\textwidth]{source/sections/casi-uso/diagrams/uc6.pdf}
        \caption{UC6 - Calcolo riduzione delle componenti}
        \label{fig:uc6}
    \end{figure}
    
    \begin{itemize}
    \item \textbf{Attore}: utente;
    \item \textbf{Descrizione}: l'utente vuole calcolare la riduzione delle componenti per l'elaborazione dati;
    \item \textbf{Precondizione}: 
    \begin{itemize}
        \item eseguito l'upload del dataset come matrice $N\times M$ (\hyperref[uc1.1]{UC1.1});
        \item selezionato Linear Projection come visualizzazione \hyperref[uc2.1.5]{UC2.1.5}).
    \end{itemize}  
    \item \textbf{Postcondizione}: l'utente ottiene la riduzione delle componenti;
    \item \textbf{Scenario Principale}: 
    \begin{enumerate}
        \item l'utente vuole calcolare la riduzione delle componenti;
        \item l'utente seleziona l'algoritmo per la riduzione delle componenti tra quelli disponibili.
    \end{enumerate}
    \end{itemize}
    
    \subsubsection{UC6.1 - Selezione algoritmo di riduzione delle componenti}
    \label{uc6.1}
    
    \begin{itemize}
    \item \textbf{Attore}: utente;
    \item \textbf{Descrizione}: l'utente sceglie l'algoritmo per la riduzione delle componenti per l'elaborazione dati;
    \item \textbf{Precondizione}: 
    \begin{itemize}
        \item eseguito l'upload del dataset come matrice $N\times M$ (\hyperref[uc1.1]{UC1.1});
        \item selezionato Linear Projection come visualizzazione \hyperref[uc2.1.5]{UC2.1.5}).
    \end{itemize}  
    \item \textbf{Postcondizione}: l'utente ha scelto l'algoritmo per la riduzione delle componenti;
    \item \textbf{Scenario Principale}: 
    \begin{enumerate}
        \item l'utente seleziona l'algoritmo per la riduzione delle componenti tra quelli disponibili.
    \end{enumerate}
    \item \textbf{Generalizzazioni}:
        \begin{enumerate}
            \item l'utente seleziona uno dei seguenti algoritmi di riduzione delle componenti:
                \begin{enumerate}
                    \item PCA (\hyperref[uc6.1.1]{UC6.1.1});
                    \item UMAP (\hyperref[uc6.1.2]{UC6.1.2});
                    \item t-SNE (\hyperref[uc6.1.3]{UC6.1.3}).
                \end{enumerate}
        \end{enumerate}  
    \end{itemize}
    
    \paragraph{UC6.1.1 - PCA}
    \label{uc6.1.1}
        \begin{figure}[htbp]
        \centering
        \includegraphics[width=0.45\textwidth]{source/sections/casi-uso/diagrams/uc6_1_1.pdf}
        \caption{UC6.1.1 - PCA}
        \label{fig:uc6.1.1}
    \end{figure}
    \begin{itemize}
    \item \textbf{Attore}: utente;
    \item \textbf{Descrizione}: il PCA è una tecnica di riduzione dimensionale, cioè a partire da $n$ features ne calcola $k$ ($k$ fissato), queste nuove k feature approssimano meglio le $n$ features iniziali;
    \item \textbf{Precondizione}: 
    \begin{itemize}
        \item eseguito l'upload del dataset come matrice $N\times M$ (\hyperref[uc1.1]{UC1.1});
        \item selezionato Linear Projection come visualizzazione (\hyperref[uc2.1.5]{UC2.1.5}).
    \end{itemize}  
    \item \textbf{Postcondizione}: il dataset contiene k nuove colonne (dove $k$ è il numero di PCs che l'utente ha scelto di calcolare), che sono le $k$ proiezioni calcolate da PCA;
    \item \textbf{Scenario Principale}: 
    \begin{enumerate}
        \item l'utente sceglie di applicare l'algoritmo PCA sul dataset.
    \end{enumerate}  
    \item \textbf{Inclusioni}:
        \begin{enumerate}
            \item impostazioni PCA (\hyperref[uc8]{UC8}).
        \end{enumerate} 
    \end{itemize}
    
    \subparagraph{UC6.1.1.1 - Numero di PCs da calcolare}
    \label{uc6.1.1.1}
    \begin{itemize}
    \item \textbf{Attore}: utente;
    \item \textbf{Descrizione}: l'utente specifica quante nuove features il PCA deve calcolare;
    \item \textbf{Precondizione}: 
    \begin{itemize}
        \item eseguito l'upload del dataset come matrice $N\times M$ (\hyperref[uc1.1]{UC1.1});
        \item selezionato Linear Projection come visualizzazione (\hyperref[uc2.1.5]{UC2.1.5});
        \item selezionato PCA (\hyperref[uc6.1.1]{UC6.1.1}).
    \end{itemize}  
    \item \textbf{Postcondizione}: il numero k di features da calcolare è stato inserito;
    \item \textbf{Scenario Principale}: 
    \begin{enumerate}
        \item l'utente inserisce un numero k compreso tra 1 e il numero di features.
    \end{enumerate}  
    \end{itemize}
    
    
    \paragraph{UC6.1.2 - UMAP}
    \label{uc6.1.2}
    \begin{itemize}
    \item \textbf{Attore}: utente;
    \item \textbf{Descrizione}: UMAP è una tecnica di riduzione dimensionale, cioè a partire da $n$ features ne calcola $k$ ($k$ fissato), queste nuove k feature approssimano meglio le $n$ features iniziali;
    \item \textbf{Precondizione}: 
    \begin{itemize}
        \item eseguito l'upload del dataset come matrice $N\times M$ (\hyperref[uc1.1]{UC1.1});
        \item selezionato Linear Projection come visualizzazione (\hyperref[uc2.1.5]{UC2.1.5}).
    \end{itemize}  
    \item \textbf{Postcondizione}: l'utente seleziona UMAP come algoritmo per la riduzione dimensionale;
    \item \textbf{Scenario Principale}: 
    \begin{enumerate}
        \item l'utente sceglie di applicare l'algoritmo UMAP sul dataset.
    \end{enumerate}
    \end{itemize}
    
    \paragraph{UC6.1.3 - t-SNE}
    \label{uc6.1.3}
    \begin{itemize}
    \item \textbf{Attore}: utente;
    \item \textbf{Descrizione}: t-SNE è una tecnica di riduzione dimensionale, cioè a partire da $n$ features ne calcola $k$ ($k$ fissato), queste nuove k feature approssimano meglio le $n$ features iniziali;
    \item \textbf{Precondizione}: 
    \begin{itemize}
        \item eseguito l'upload del dataset come matrice $N\times M$ (\hyperref[uc1.1]{UC1.1});
        \item selezionato Linear Projection come visualizzazione (\hyperref[uc2.1.5]{UC2.1.5}).
    \end{itemize}  
    \item \textbf{Postcondizione}: l'utente seleziona t-SNE come algoritmo per la riduzione dimensionale;
    \item \textbf{Scenario Principale}: 
    \begin{enumerate}
        \item l'utente sceglie di applicare l'algoritmo t-SNE sul dataset.
    \end{enumerate}
    \end{itemize}
    

