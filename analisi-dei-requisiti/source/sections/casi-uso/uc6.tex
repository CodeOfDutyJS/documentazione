    \subsection{UC6 - Calcolo Matrice di Distanza}
    % \includegraphics{}
    \begin{itemize}
    \item \textbf{Attore}: Utente
    \item \textbf{Descrizione}: Data una matrice $N \times M$ viene calcolata la distanza tra ogni riga e viene restituita all'utente una matrice di distanza N x N dove ${x_i}_j$ è la distanza tra la riga i e la riga j della matrice di partenza.
    \item \textbf{Precondizione}: 
    \begin{itemize}
        \item Eseguito l'upload del dataset come matrice $N\times M$ (\hyperref[uc1]{UC1}).
        \item Selezionato Heatmap o Force Field come visualizzazione (\hyperref[uc2.2]{UC2.2} o \hyperref[uc2.4]{UC2.4}).
    \end{itemize}  
    \item \textbf{Postcondizione}: Calcolata matrice di distanza $N \times N$ dove ${x_i}_j$ è la distanza tra la riga $i$ e la riga $j$ della matrice di partenza.
    \item \textbf{Scenario Principale}: 
    \begin{enumerate}
        \item L'utente seleziona il calcolo della matrice di distanza corrispondente al dataset caricato
        \item L'utente seleziona la distanza (\hyperref[uc6.1]{UC6.1})
    \end{enumerate}  
    \item \textbf{Inclusioni}:
        \begin{enumerate}
                \item \begin{enumerate}
                    \item Scelta dell'algoritmo di distanza da utilizzare (\hyperref[uc6.1]{UC6.1})
                \end{enumerate}
        \end{enumerate} 
    \end{itemize}
    
    \subsubsection{UC6.1 - Selezione distanza}
    \label{uc6.1}
    \begin{itemize}
    \item \textbf{Attore}: Utente
    \item \textbf{Descrizione}: L'utente sceglie la distanza da utilizzare durante l'elaborazione dati
    \item \textbf{Precondizione}: 
    \begin{itemize}
        \item Eseguito l'upload del dataset come matrice $N\times M$ (\hyperref[uc1]{UC1}).
        \item Selezionato Heatmap o Force Field come visualizzazione (\hyperref[uc2.2]{UC2.2} o \hyperref[uc2.4]{UC2.4}).
    \end{itemize}  
    \item \textbf{Postcondizione}: L'utente ha scelto la distanza da utilizzare
    \item \textbf{Scenario Principale}: 
    \begin{enumerate}
        \item L'utente seleziona la distanza tra quelle disponibili
    \end{enumerate}
    \item \textbf{Generalizzazioni}:
        \begin{enumerate}
            \item L'utente seleziona una delle seguenti distanze
                \begin{enumerate}
                    \item Euclidea (\hyperref[uc6.1.1]{UC6.1.1})
                    \item Manhattan (\hyperref[uc6.1.2]{UC6.1.2})
                \end{enumerate}
        \end{enumerate}  
    \end{itemize}
    
    \paragraph{UC6.1.1 - Distanza Euclidea}
    \label{uc6.1.1}
    \begin{itemize}
    \item \textbf{Attore}: Utente
    \item \textbf{Descrizione}: L'utente sceglie la distanza \emph{Euclidea}
    \item \textbf{Precondizione}: 
    \begin{itemize}
        \item Eseguito l'upload del dataset come matrice $N\times M$ (\hyperref[uc1]{UC1}).
        \item Selezionato Heatmap o Force Field come visualizzazione (\hyperref[uc2.2]{UC2.2} o \hyperref[uc2.4]{UC2.4}).
    \end{itemize}  
    \item \textbf{Postcondizione}: L'utente ha scelto la distanza euclidea
    \item \textbf{Scenario Principale}: 
    \begin{enumerate}
        \item L'utente ha scelto la distanza euclidea
    \end{enumerate}
    \end{itemize}
    
    \paragraph{UC6.1.2 - Distanza di Manhattan}
    \label{uc6.1.2}
    \begin{itemize}
    \item \textbf{Attore}: Utente
    \item \textbf{Descrizione}: L'utente sceglie la distanza di \emph{Manhattan}
    \item \textbf{Precondizione}: 
    \begin{itemize}
        \item Eseguito l'upload del dataset come matrice $N\times M$ (\hyperref[uc1]{UC1}).
        \item Selezionato Heatmap o Force Field come visualizzazione (\hyperref[uc2.2]{UC2.2} o \hyperref[uc2.4]{UC2.4}).
    \end{itemize}  
    \item \textbf{Postcondizione}: L'utente ha scelto la distanza di Manhattan
    \item \textbf{Scenario Principale}: 
    \begin{enumerate}
        \item L'utente ha scelto la distanza di Manhattan
    \end{enumerate}
    \end{itemize}