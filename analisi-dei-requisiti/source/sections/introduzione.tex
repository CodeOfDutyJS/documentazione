\section{Introduzione}
    \subsection{Scopo del documento}
    Questo documento ha lo scopo di fornire una descrizione esauriente di tutti i casi d'uso e dei requisiti associati. Le informazioni contenute nel documento sono state il frutto dell'analisi e comprensione del capitolato \emph{HD Viz} proposto da \emph{Zucchetti S.p.A} e di alcuni incontri con il proponente.
    \subsection{Scopo del Prodotto}
    Lo scopo del capitolato \emph{HD Viz} è lo sviluppo di un'applicazione web che permetta la visualizzazione di dati con molte dimensioni a supporto della fase esplorativa dell'analisi dei dati. La parte di visualizzazione verrà affidata alla libreria JavaScript \textbf{d3.js}. Il gruppo \emph{Code of Duty} propone lo sviluppo di una PWA in grado di visualizzare dati provenienti da origini differenti. L'applicazione sarà in grado di funzionare anche offline.
    \subsection{Riferimenti}
    \subsubsection{Normativi}
    \begin{itemize}
        \item \textbf{Capitolato d'appalto C4} - HD Viz: Visualizzazione di dati multidimensionali:\\\url{https://www.math.unipd.it/~tullio/IS-1/2020/Progetto/C4.pdf};
        \item \textbf{Norme di Progetto}: \emph{Norme di progetto v1.0.0}.
    \end{itemize}
    \subsubsection{Informativi}
    \begin{itemize}
        \item \textbf{Capitolato d'appalto C4} - HD Viz: Visualizzazione di dati multidimensionali:\\\url{https://www.math.unipd.it/~tullio/IS-1/2020/Progetto/C4.pdf};
        \item \textbf{Materiale didattico}: \\\url{https://www.math.unipd.it/~rcardin/swea/2021/Diagrammi%20Use%20Case_4x4.pdf}
        \\\url{https://www.math.unipd.it/~tullio/IS-1/2020/Dispense/L07.pdf};
        \item \textbf{Libreria d3.js}:
        \\\url{https://github.com/d3/d3/wiki}.
        % \item \textbf{Libreria esterna consigliata per calcolo delle distanze}
        % \\\url{https://github.com/mljs/distance}
    \end{itemize}