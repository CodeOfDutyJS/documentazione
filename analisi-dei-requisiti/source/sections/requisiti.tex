\section{Requisiti}

    \subsection{Introduzione}
    La struttura dei requisiti è descritta nelle \emph{Norme di progetto}. Di seguito saranno elencati tutti i requisiti individuati dal gruppo. 
    
    \subsection{Requisiti funzionali}

\def\obb{Obbligatorio}

\def\requisitif{
    {RFO1, L'utente deve poter caricare i propri dati per la visualizzazione, \obb,Capitolato UC1},
    {RFO1.1, L'utente deve poter caricare i propri dati per la visualizzazione tramite file csv, \obb,Capitolato UC1.2},
    {RFO1.2, L'utente deve poter caricare i propri dati per la visualizzazione tramite query ad un database, \obb,Capitolato UC1.3},
    {RFO1.3, Il sistema deve visualizzare un messaggio di errore quando i dati caricati non sono corretti o è fallito un caricamento, \obb,Interno UC10},
    {RFO1.4, I dati caricati dovranno essere convertiti in JSON per uniformare l'elaborazione e la visualizzazione dei dati provenienti da fonti diverse, \obb, Interno},
    {RFO2, L'utente deve poter scegliere il tipo di visualizzazione, \obb,Capitolato UC2},
    {RFO2.1, L'utente deve poter scegliere Scatter Plot Matrix come tipo di visualizzazione, \obb,Capitolato UC2.1},
    {RFO2.2, L'utente deve poter scegliere Heatmap come tipo di visualizzazione, \obb,Capitolato UC2.2},
    {RFO2.2.1, L'utente deve poter scegliere il colore della sfumatura dell'Heatmap, \obb, UC5.4},
    {RFF2.3, L'utente deve poter scegliere Correlation Heatmap come tipo di visualizzazione, Facoltativo,Capitolato Interno UC2.3},
    {RFO2.4, L'utente deve poter scegliere Force Field come tipo di visualizzazione, \obb,Capitolato UC2.4},
    {RFO2.5, L'utente deve poter scegliere Linear Projection come tipo di visualizzazione, \obb,Capitolato UC2.5},
    {RFF2.6, L'utente deve poter scegliere Parallel Coordinates come tipo di visualizzazione, Facoltativo,Capitolato Interno UC2.6},
    {RFO3, L'utente deve poter scegliere le labels presenti nel suo dataset, \obb,Interno UC3},
    {RFO3.1, L'utente deve poter scegliere come visualizzare le labels presenti nel suo dataset all'interno della visualizzazione, \obb, Interno UC3.1},
    {RFO4, L'utente deve poter scartare le features di cui non è interessato o riaggiungere quelle scartate, \obb, Interno UC4},
    {RFO5, L'utente{,} se ha selezionato l'Heatmap{,} deve poter modificare le impostazioni che influenzano la visualizzazione, \obb,Interno UC5},
    {RFO5.1, L'utente se ha selezionato l'Heatmap deve poter modificare il range dei dati da considerare, \obb, Interno UC5.1},
    {RFO5.2, L'utente{,} se ha selezionato l'Heatmap{,} deve poter normalizzare il dataset, \obb,Interno UC5.2},
    {RFO5.2.1, L'utente deve poter normalizzare il dataset in modo globale, \obb, Interno UC5.2.1},
    {RFO5.2.2, L'utente deve poter normalizzare il dataset per colonna, \obb, Interno UC5.2.2},
    {RFO5.2.3, L'utente deve poter normalizzare il dataset per riga, \obb, Interno UC5.2.3},
    {RFO5.3, L'utente{,} se ha selezionato l'Heatmap{,} deve poter ordinare il dataset, \obb,Interno UC5.2},
    {RFO5.3.1, L'utente{,} se ha selezionato l'Heatmap{,} deve poter ordinare il dataset in ordine alfabetico, \obb,Interno UC5.3.1},
    {RFO5.3.2, L'utente{,} se ha selezionato l'Heatmap{,} deve poter ordinare il dataset in cluster, \obb,Capitolato UC5.3.2},
    {RFO5.4, L'utente se ha selezionato l'Heatmap deve poter assegnare un colore al range , \obb, Interno UC5.4},
    {RFO6, L'utente{,} se ha selezionato l'Heatmap{,} deve poter scegliere se utilizzare una matrice di distanza, \obb,Capitolato Interno UC6},
    {RFF6.1, L'utente{,} se ha selezionato l'Heatmap o il Force Field{,} deve poter scegliere quale funzione di distanza utilizzare, Facoltativo,Capitolato UC6.1},
    {RFF6.1.1, L'utente{,} se ha selezionato l'Heatmap o il Force Field{,} deve poter scegliere la distanza euclidea, Facoltativo,Interno UC6.1.1},
    {RFF6.1.2, L'utente{,} se ha selezionato l'Heatmap o il Force Field{,} deve poter scegliere la distanza di Manhattan, Facoltativo,Interno UC6.1.2},
    {RFO7, L'utente{,} se ha selezionato la Linear Projection{,} deve poter scegliere l'algoritmo per la riduzione delle componenti per l'elaborazione dati, \obb,Capitolato UC7},
    {RFF7.1, L'utente{,} se ha selezionato la Linear Projection{,} deve poter scegliere PCA come algoritmo per la riduzione delle componenti, Facoltativo,Interno UC7.1},
    {RFF7.2, L'utente{,} se ha selezionato la Linear Projection{,} deve poter scegliere UMAP come algoritmo per la riduzione delle componenti, Facoltativo,Capitolato UC7.2},
    {RFF7.3, L'utente{,} se ha selezionato la Linear Projection{,} deve poter scegliere t-SNE come algoritmo per la riduzione delle componenti, Facoltativo,Capitolato UC7.3},
    {RFO8.1, Il sistema deve elaborare i dati con le impostazioni di default, \obb,Interno UC8.1},
    {RFO8.2, Il sistema deve elaborare i dati con le impostazioni personalizzate dall'utente, \obb,Interno UC8.2},
    {RFO9, L'utente deve poter visualizzare la visualizzazione creata dal sistema, \obb,Capitolato UC9{,} UC9.1},
    {RFO9.2, L'utente deve salvare la visualizzazione come file PNG, \obb, Interno UC9.2},
    {RFO10, L'utente visualizza un messaggio di errore se carica i files scorrettamente, \obb, Interno UC10},
}






    % {RFO1.2, L'utente deve poter caricare i propri dati per la visualizzazione tramite query ad un database, \obb, \hyperref[uc1.3]{UC1.3}},
    % {RFO1.3, Il sistema deve visualizzare un messaggio di errore quando i dati caricati non sono corretti o è fallito un caricamento, \obb, \hyperref[uc10]{UC10}},
    % {RFO1.4, I dati caricati dovranno essere convertiti in JSON per uniformare l'elaborazione e la visualizzazione dei dati provenienti da fonti diverse, \obb, Decisione interna},
    % {RFO2, L'utente deve poter scegliere il tipo di visualizzazione, \obb, \hyperref[uc2]{UC2}},
    % {RFO2.1, L'utente deve poter scegliere Scatter Plot Matrix come tipo di visualizzazione, \obb, \hyperref[uc2.1]{UC2.1}},
    % {RFO2.2, L'utente deve poter scegliere Heatmap come tipo di visualizzazione, \obb, \hyperref[uc2.2]{UC2.2}},
    % {RFO2.2.1, L'utente deve poter scegliere il colore della sfumatura dell'Heatmap, \obb, \hyperref[uc2.7]{UC2.7}},
    % {RFF2.3, L'utente deve poter scegliere Correlation Heatmap come tipo di visualizzazione, Facoltativo, \hyperref[uc2.3]{UC2.3}},
    % {RFO2.4, L'utente deve poter scegliere Force Field come tipo di visualizzazione, \obb, \hyperref[uc2.4]{UC2.4}},
    % {RFO2.5, L'utente deve poter scegliere Linear Projection come tipo di visualizzazione, \obb, \hyperref[uc2.5]{UC2.5}},
    % {RFF2.6, L'utente deve poter scegliere Parallel Coordinates come tipo di visualizzazione, Facoltativo, \hyperref[uc2.6]{UC2.6}},
    % {RFO3, L'utente deve poter scegliere le etichette presenti nel suo dataset, \obb, \hyperref[uc3]{UC3}},
    % {RFO3.1, L'utente deve poter scegliere come visualizzare le etichette presenti nel suo dataset all'interno della visualizzazione, \obb, \hyperref[uc3.1]{UC3.1}},



%%%
%%%
%%%
\newcommand*\requisitiftable{}
\foreach \x [count=\nj] in \requisitif
{
    \foreach \y [count=\ni] in \x
    {
        \ifnum\ni=4
            \xappto\requisitiftable{\y}
            \gappto\requisitiftable{\\}
            \gappto\requisitiftable{\hline}
        \else
            \xappto\requisitiftable{\y & }
        \fi
    }
}

%\subsection{Requisiti funzionali}

% Impostazioni della tabella
\tabulinesep = 2mm % padding
\taburowcolors [1] 2{pari .. dispari} % colori delle righe
\addcontentsline{lot}{table}{Requisiti funzionali}
\begin{longtabu} to \textwidth {| X[0.2 l m] | X[0.4 l m] |  X[0.2 l m] | X[0.2 l m] |} % larghezza delle colonne
\hline
\rowcolor{header} % colore dell'header
    
\textbf{Requisito} & \textbf{Descrizione} & \textbf{Classificazione} & \textbf{Fonte} \\
\hline
\requisitiftable

\end{longtabu}
    \subsection{Requisiti di qualità}

\def\obb{Obbligatorio}
\def\pdq{Piano di Qualifica}

%startTable
\def\requisitiq{
    {RQO1, Deve essere prodotto un manuale d'uso per l'utente, \obb, Capitolato},
    {RQO2, Deve essere prodotto un manuale manutentore, \obb, Capitolato},
    {RQD3, Il codice deve essere pubblicato su un repository pubblico, Desiderabile, Capitolato},
    {RQO4, Il codice deve seguire le norme di stile specificate nel documento Norme di Progetto, \obb, Interno{,} Norme di Progetto 1.0.0},
    {RQO5, Nella codifica{,} deve essere evitato l'utilizzo di chiamate ricorsive se non in casi strettamente necessari, \obb, Interno{,} Norme di Progetto 1.0.0},
}
%endTable





%%%
%%%
%%%
\newcommand*\requisitiqtable{}
\foreach \x [count=\nj] in \requisitiq
{
    \foreach \y [count=\ni] in \x
    {
        \ifnum\ni=4
            \xappto\requisitiqtable{\y}
            \gappto\requisitiqtable{\\}
            \gappto\requisitiqtable{\hline}
        \else
            \xappto\requisitiqtable{\y & }
        \fi
    }
}

%\subsection{Requisiti funzionali}

% Impostazioni della tabella
\tabulinesep = 2mm % padding
\taburowcolors [1] 2{pari .. dispari} % colori delle righe
\addcontentsline{lot}{table}{Requisiti di qualità}
\begin{longtabu} to \textwidth {| X[0.2 l m] | X[0.4 l m] |  X[0.2 l m] | X[0.2 l m] |} % larghezza delle colonne
\hline
\rowcolor{header} % colore dell'header
    
\textbf{Requisito} & \textbf{Descrizione} & \textbf{Classificazione} & \textbf{Fonte} \\
\hline
\requisitiqtable

\end{longtabu}
    \subsection{Requisiti di vincolo}

\def\obb{Obbligatorio}

%startTable
\def\requisitiv{
    {RVO1, Il codice sorgente dell'applicazione essere open source, \obb, Capitolato},    
    {RVD2, L'applicazione deve essere sviluppata in JavaScript, \obb, Capitolato},
    {RVD2.1, Le visualizzazioni devono essere sviluppate con la libreria \noexpand\href{https://d3js.org/}{\noexpand\emph{d3.js}}, \obb, Capitolato},
    {RVD2.2, Il backend deve essere sviluppato con node.js e utilizzare il framework \noexpand\href{https://expressjs.com/}{\noexpand\emph{express}}, Desiderabile, Interno},
    {RVD2.3, Il frontend deve essere sviluppato con React e utilizzare il framework \noexpand\href{https://ant.design/}{\noexpand\emph{Ant Design}}, Desiderabile, Interno},
    {RVD2.4, Lo sviluppo dell’applicazione deve implementare test di unità e di intergrazione, Desiderabile, Interno},
    {RVO3, I dati caricati dovranno essere convertiti in JSON per uniformare l'elaborazione e la visualizzazione dei dati provenienti da fonti diverse, \obb, Interno},
    {RVO4, L'utente se ha selezionato Scatter Plot Matrix come visualizzazione può scegliere al massimo 5 features, \obb, Capitolato UC4.1},
    {RVD5, La libreria per PCA deve essere \noexpand\href{https://github.com/mljs/pca}{\noexpand\emph{ml-pca}}, Desiderabile, Interno},
    {RVD6, La libreria per UMAP deve essere \noexpand\href{https://github.com/PAIR-code/umap-js}{\noexpand\emph{umap-js}}, Desiderabile, Interno},
    {RVD7, La libreria per t-SNE deve essere \noexpand\href{https://github.com/scienceai/tsne-js}{\noexpand\emph{tsne-js}}, Desiderabile, Interno},
    {RVD8, La libreria per le distanze deve essere \noexpand\href{https://github.com/mljs/distance}{\noexpand\emph{ml-distance}}, Desiderabile, Interno},
    {RVD9, La libreria per la matrice di correlazione deve essere \noexpand\href{https://github.com/HarryStevens/jeezy}{\noexpand\emph{jeezy}}, Desiderabile, Interno},
}
%endTable




%%%
%%%
%%%
\newcommand*\requisitivtable{}
\foreach \x [count=\nj] in \requisitiv
{

    \foreach \y [count=\ni] in \x
    {
        \ifnum\ni=4
            \xappto\requisitivtable{\y}
            \gappto\requisitivtable{\\}
            \gappto\requisitivtable{\hline}
        \else
            \xappto\requisitivtable{\y & }
        \fi
    }
}
%\subsection{Requisiti funzionali}

% Impostazioni della tabella
\tabulinesep = 2mm % padding
\taburowcolors [1] 2{pari .. dispari} % colori delle righe
\addcontentsline{lot}{table}{Requisiti di vincolo}
\begin{longtabu} to \textwidth {| X[0.2 l m] | X[0.4 l m] |  X[0.2 l m] | X[0.2 l m] |} % larghezza delle colonne
\hline
\rowcolor{header} % colore dell'header
    
\textbf{Requisito} & \textbf{Descrizione} & \textbf{Classificazione} & \textbf{Fonte} \\
\hline
\requisitivtable

\end{longtabu}
    \subsection{Requisiti prestazionali}
Non sono stati individuati requisiti prestazionali in quanto le librerie e gli aspetti prettamente tecnologici saranno analizzati nel dettaglio durante le prossime revisioni, dopo la stesura della \emph{Technology Baseline}.
    
    \subsection{Tracciamento}
    \subsubsection{Fonte - Requisiti}

\paragraph{Capitolato}
\quad
\begin{multicols}{3}
    \begin{itemize}
        \item RFO1
        \item RFO1.1
        \item RFO1.2
        \item RFO2
        \item RFO2.1
        \item RFO2.2
        \item RFF2.3
        \item RFO2.4
        \item RFO2.5
        \item RFF2.6
        \item RFO5.3.2
        \item RFO6
        \item RFF6.1
        \item RFO7
        \item RFF7.2
        \item RFF7.3
        \item RFO9
        \item RVO1
        \item RVD2
        \item RVD2.1
        \item RVO4
    \end{itemize}
\end{multicols}

\paragraph{Interno}
\quad
\begin{multicols}{3}
    \begin{itemize}
        \item RFO1.3
        \item RFO1.4
        \item RFF2.3
        \item RFF2.6
        \item RFO3
        \item RFO3.1
        \item RFO4
        \item RFO5
        \item RFO5.1
        \item RFO5.2
        \item RFO5.2.1
        \item RFO5.2.2
        \item RFO5.2.3
        \item RFO5.3
        \item RFO5.3.1
        \item RFO5.4
        \item RFO6
        \item RFF6.1.1
        \item RFF6.1.2
        \item RFF7.1
        \item RFO8.1
        \item RFO8.2
        \item RFO9.2
        \item RFO10
        \item RQO1
        \item RQO2
        \item RQO3
        \item RQO4
        \item RQO5
        \item RQO6
        \item RQO7
        \item RQO8
        \item RQO9
        \item RQO10
        \item RQO11
        \item RVD2.2
        \item RVD2.3
        \item RVD2.4
        \item RVO3
        \item RVD5
        \item RVD6
        \item RVD7
        \item RVD8
        \item RVD9
    \end{itemize}
\end{multicols}

\paragraph{UC1}
\quad
\begin{multicols}{3}
    \begin{itemize}
        \item RFO1
        \item RFO1.1
        \item RFO1.2
    \end{itemize}
\end{multicols}

\paragraph{UC2}
\quad
\begin{multicols}{3}
    \begin{itemize}
        \item RFO2
        \item RFO2.1
        \item RFO2.2
        \item RFF2.3
        \item RFO2.4
        \item RFO2.5
        \item RFF2.6
    \end{itemize}
\end{multicols}

\paragraph{UC3}
\quad
\begin{multicols}{3}
    \begin{itemize}
        \item RFO3
        \item RFO3.1
    \end{itemize}
\end{multicols}

\paragraph{UC4}
\quad
\begin{multicols}{3}
    \begin{itemize}
        \item RFO4
        \item RVO4
    \end{itemize}
\end{multicols}

\paragraph{UC5}
\quad
\begin{multicols}{3}
    \begin{itemize}
        \item RFO2.2.1
        \item RFO5.1
        \item RFO5.2
        \item RFO5.2.2
        \item RFO5.2.3
        \item RFO5.3
        \item RFO5.3.1
        \item RFO5.3.2
        \item RFO5.4
    \end{itemize}
\end{multicols}

\paragraph{UC6}
\quad
\begin{multicols}{3}
    \begin{itemize}
        \item RFO6
        \item RFF6.1
        \item RFF6.1.1
        \item RFF6.1.2
    \end{itemize}
\end{multicols}

\paragraph{UC7}
\quad
\begin{multicols}{3}
    \begin{itemize}
        \item RFO7
        \item RFF7.1
        \item RFF7.2
        \item RFF7.3
    \end{itemize}
\end{multicols}

\paragraph{UC8}
\quad
\begin{multicols}{3}
    \begin{itemize}
        \item RFO8.1
        \item RFO8.2
    \end{itemize}
\end{multicols}

\paragraph{UC9}
\quad
\begin{multicols}{3}
    \begin{itemize}
        \item RFO9
        \item RFO9.2
    \end{itemize}
\end{multicols}

\paragraph{UC10}
\quad
\begin{multicols}{3}
    \begin{itemize}
        \item RFO10
    \end{itemize}
\end{multicols}
    
    \subsubsection{Riepilogo requisiti}
    
    \tabulinesep = 2mm % padding
    \taburowcolors [1] 2{pari .. dispari} % colori delle righe
    \addcontentsline{lot}{table}{Riepilogo requisiti}
    \begin{longtabu} to \textwidth {| X | X | X | X | X |} % larghezza delle colonne
    \hline
    \rowcolor{header} % colore dell'header
        
    \textbf{Tipologia} & \textbf{Obbligatorio} & \textbf{Facoltativo} & \textbf{Desiderabile} & \textbf{Totale} \\
    \hline
    Funzionale & 32 & 15 & 0 & 47\\
    \hline
    Di qualità & 4 & 0 & 1 & 5\\
    \hline
    Di vincolo & 10 & 0 & 8 & 18\\
    \hline
    
    \end{longtabu}






