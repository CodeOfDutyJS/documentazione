\subsection{Requisiti funzionali}

\def\obb{Obbligatorio}

%startTable
\def\requisitif{
    {RFO1, L'utente deve poter caricare i propri dati per la visualizzazione, \obb,Capitolato \noexpand\hyperref[uc1]{UC1}},
    {RFO1.1, L'utente deve poter caricare i propri dati per la visualizzazione tramite file csv, \obb,Capitolato \noexpand\hyperref[uc1.2]{UC1.2}},
    {RFO1.2, L'utente deve poter caricare i propri dati per la visualizzazione tramite query ad un database, \obb,Capitolato \noexpand\hyperref[uc1.3]{UC1.3}},
    {RFO1.3, Il sistema deve visualizzare un messaggio di errore quando i dati caricati non sono corretti o è fallito un caricamento, \obb,Interno \noexpand\hyperref[uc10]{UC10}},
    {RFO1.4, I dati caricati dovranno essere convertiti in JSON per uniformare l'elaborazione e la visualizzazione dei dati provenienti da fonti diverse, \obb, Interno},
    {RFO2, L'utente deve poter scegliere il tipo di visualizzazione, \obb,Capitolato \noexpand\hyperref[uc2]{UC2}},
    {RFO2.1, L'utente deve poter scegliere Scatter Plot Matrix come tipo di visualizzazione, \obb,Capitolato \noexpand\hyperref[uc2.1]{UC2.1}},
    {RFO2.2, L'utente deve poter scegliere Heatmap come tipo di visualizzazione, \obb,Capitolato \noexpand\hyperref[uc2.2]{UC2.2}},
    {RFO2.2.1, L'utente deve poter scegliere il colore della sfumatura dell'Heatmap, \obb, \noexpand\hyperref[uc5.4]{UC5.4}},
    {RFF2.3, L'utente deve poter scegliere Correlation Heatmap come tipo di visualizzazione, Facoltativo,Capitolato Interno \noexpand\hyperref[uc2.3]{UC2.3}},
    {RFO2.4, L'utente deve poter scegliere Force Field come tipo di visualizzazione, \obb,Capitolato \noexpand\hyperref[uc2.4]{UC2.4}},
    {RFO2.5, L'utente deve poter scegliere Linear Projection come tipo di visualizzazione, \obb,Capitolato \noexpand\hyperref[uc2.5]{UC2.5}},
    {RFF2.6, L'utente deve poter scegliere Parallel Coordinates come tipo di visualizzazione, Facoltativo,Capitolato Interno \noexpand\hyperref[uc2.6]{UC2.6}},
    {RFO3, L'utente deve poter scegliere le labels presenti nel suo dataset, \obb,Interno \noexpand\hyperref[uc3]{UC3}},
    {RFO3.1, L'utente deve poter scegliere come visualizzare le labels presenti nel suo dataset all'interno della visualizzazione, \obb, Interno \noexpand\hyperref[uc3.1]{UC3.1}},
    {RFO4, L'utente deve poter scartare le features di cui non è interessato o riaggiungere quelle scartate, \obb, Interno \noexpand\hyperref[uc4]{UC4}},
    {RFO5, L'utente{,} se ha selezionato l'Heatmap{,} deve poter modificare le impostazioni che influenzano la visualizzazione, \obb,Interno \noexpand\hyperref[uc5]{UC5}},
    {RFO5.1, L'utente se ha selezionato l'Heatmap deve poter modificare il range dei dati da considerare, \obb, Interno \noexpand\hyperref[uc5.1]{UC5.1}},
    {RFO5.2, L'utente{,} se ha selezionato l'Heatmap{,} deve poter normalizzare il dataset, \obb,Interno \noexpand\hyperref[uc5.2]{UC5.2}},
    {RFO5.2.1, L'utente deve poter normalizzare il dataset in modo globale, \obb, Interno \noexpand\hyperref[uc5.2.1]{UC5.2.1}},
    {RFO5.2.2, L'utente deve poter normalizzare il dataset per colonna, \obb, Interno \noexpand\hyperref[uc5.2.2]{UC5.2.2}},
    {RFO5.2.3, L'utente deve poter normalizzare il dataset per riga, \obb, Interno \noexpand\hyperref[uc5.2.3]{UC5.2.3}},
    {RFO5.3, L'utente{,} se ha selezionato l'Heatmap{,} deve poter ordinare il dataset, \obb,Interno \noexpand\hyperref[uc5.3]{UC5.3}},
    {RFO5.3.1, L'utente{,} se ha selezionato l'Heatmap{,} deve poter ordinare il dataset in ordine alfabetico, \obb,Interno \noexpand\hyperref[uc5.3.1]{UC5.3.1}},
    {RFO5.3.2, L'utente{,} se ha selezionato l'Heatmap{,} deve poter ordinare il dataset in cluster, \obb,Capitolato \noexpand\hyperref[uc5.3.2]{UC5.3.2}},
    {RFO5.4, L'utente se ha selezionato l'Heatmap deve poter assegnare un colore al range , \obb, Interno \noexpand\hyperref[uc5.4]{UC5.4}},
    {RFO6, L'utente{,} se ha selezionato l'Heatmap{,} deve poter scegliere se utilizzare una matrice di distanza, \obb,Capitolato Interno \noexpand\hyperref[uc6]{UC6}},
    {RFF6.1, L'utente{,} se ha selezionato l'Heatmap o il Force Field{,} deve poter scegliere quale funzione di distanza utilizzare, Facoltativo,Capitolato \noexpand\hyperref[uc6.1]{UC6.1}},
    {RFF6.1.1, L'utente{,} se ha selezionato l'Heatmap o il Force Field{,} deve poter scegliere la distanza euclidea, Facoltativo,Interno \noexpand\hyperref[uc6.1.1]{UC6.1.1}},
    {RFF6.1.2, L'utente{,} se ha selezionato l'Heatmap o il Force Field{,} deve poter scegliere la distanza di Manhattan, Facoltativo,Interno \noexpand\hyperref[uc6.1.2]{UC6.1.2}},
    {RFO7, L'utente{,} se ha selezionato la Linear Projection{,} deve poter scegliere l'algoritmo per la riduzione delle componenti per l'elaborazione dati, \obb,Capitolato \noexpand\hyperref[uc7]{UC7}},
    {RFF7.1, L'utente{,} se ha selezionato la Linear Projection{,} deve poter scegliere PCA come algoritmo per la riduzione delle componenti, Facoltativo,Interno \noexpand\hyperref[uc7.1]{UC7.1}},
    {RFF7.2, L'utente{,} se ha selezionato la Linear Projection{,} deve poter scegliere UMAP come algoritmo per la riduzione delle componenti, Facoltativo,Capitolato \noexpand\hyperref[uc7.2]{UC7.2}},
    {RFF7.3, L'utente{,} se ha selezionato la Linear Projection{,} deve poter scegliere t-SNE come algoritmo per la riduzione delle componenti, Facoltativo,Capitolato \noexpand\hyperref[uc7.3]{UC7.3}},
    {RFO8.1, Il sistema deve elaborare i dati con le impostazioni di default, \obb,Interno \noexpand\hyperref[uc8.1]{UC8.1}},
    {RFO8.2, Il sistema deve elaborare i dati con le impostazioni personalizzate dall'utente, \obb,Interno \noexpand\hyperref[uc8.2]{UC8.2}},
    {RFO9, L'utente deve poter visualizzare la visualizzazione creata dal sistema, \obb,Capitolato \noexpand\hyperref[uc9]{UC9}{,} \noexpand\hyperref[uc9.1]{UC9.1}},
    {RFO9.2, L'utente deve salvare la visualizzazione come file PNG, \obb, Interno \noexpand\hyperref[uc9.2]{UC9.2}},
    {RFO10, L'utente visualizza un messaggio di errore se carica i files scorrettamente, \obb, Interno \noexpand\hyperref[uc10]{UC10}},
}
%endTable


%%%
%%%
%%%
\newcommand*\requisitiftable{}
\foreach \x [count=\nj] in \requisitif
{
    \foreach \y [count=\ni] in \x
    {
        \ifnum\ni=4
            \xappto\requisitiftable{\y}
            \gappto\requisitiftable{\\}
            \gappto\requisitiftable{\hline}
        \else
            \xappto\requisitiftable{\y & }
        \fi
    }
}

%\subsection{Requisiti funzionali}

% Impostazioni della tabella
\tabulinesep = 2mm % padding
\taburowcolors [1] 2{pari .. dispari} % colori delle righe
\addcontentsline{lot}{table}{Requisiti funzionali}
\begin{longtabu} to \textwidth {| X[0.2 l m] | X[0.4 l m] |  X[0.2 l m] | X[0.2 l m] |} % larghezza delle colonne
\hline
\rowcolor{header} % colore dell'header
    
\textbf{Requisito} & \textbf{Descrizione} & \textbf{Classificazione} & \textbf{Fonte} \\
\hline
\requisitiftable

\end{longtabu}