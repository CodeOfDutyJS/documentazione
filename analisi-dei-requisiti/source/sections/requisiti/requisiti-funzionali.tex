\subsection{Requisiti funzionali}

\def\obb{Obbligatorio}

%startTable
\def\requisitif{
    {RFO1, L'utente deve poter caricare i propri dati per la visualizzazione, \obb,Capitolato \noexpand\hyperref[uc1]{UC1}},
    {RFO1.1, L'utente deve poter caricare i propri dati per la visualizzazione tramite file csv, \obb,Capitolato \noexpand\hyperref[uc1.1]{UC1.1}},
    {RFO1.2, L'utente deve poter caricare i propri dati per la visualizzazione tramite una tabella di un database, \obb,Capitolato \noexpand\hyperref[uc1.2]{UC1.2}},
    {RFO1.3, Il sistema deve visualizzare un messaggio di errore quando i dati caricati non sono corretti o è fallito un caricamento, \obb,Interno \noexpand\hyperref[uc10]{UC10}},
    {RFO1.4, I dati caricati dovranno essere convertiti in JSON per uniformare l'elaborazione e la visualizzazione dei dati provenienti da fonti diverse, \obb, Interno},
    {RFO2, L'utente deve poter scegliere il tipo di visualizzazione, \obb,Capitolato \noexpand\hyperref[uc2]{UC2}},
    {RFO2.1, L'utente deve poter scegliere Scatter Plot Matrix come tipo di visualizzazione, \obb,Capitolato \noexpand\hyperref[uc2.1]{UC2.1}},
    {RFO2.2, L'utente deve poter scegliere Heatmap come tipo di visualizzazione, \obb,Capitolato \noexpand\hyperref[uc2.2]{UC2.2}},
    {RFF2.3, L'utente deve poter scegliere Correlation Heatmap come tipo di visualizzazione, Facoltativo,Capitolato Interno \noexpand\hyperref[uc2.3]{UC2.3}},
    {RFO2.4, L'utente deve poter scegliere Force Field come tipo di visualizzazione, \obb,Capitolato \noexpand\hyperref[uc2.4]{UC2.4}},
    {RFO2.5, L'utente deve poter scegliere Linear Projection come tipo di visualizzazione, \obb,Capitolato \noexpand\hyperref[uc2.5]{UC2.5}},
    {RFF2.6, L'utente deve poter scegliere Parallel Coordinates come tipo di visualizzazione, Facoltativo,Capitolato Interno \noexpand\hyperref[uc2.6]{UC2.6}},
    {RFO3, L'utente deve poter manipolare i dati nel dataset, \obb,Interno \noexpand\hyperref[uc3]{UC3}},
    {RFO3.1, L'utente deve poter selezionare tra le features del dataset quelle da considerare come variabili target, \obb, Interno \noexpand\hyperref[uc3.1]{UC3.1}},
    {RFO3.2, L'utente deve poter selezionare le features a cui è interessato, \obb, Interno \noexpand\hyperref[uc3.2]{UC3.2}},
    {RFO3.3, L'utente deve poter normalizzare i dati presenti nel dataset, \obb, Interno \noexpand\hyperref[uc3.3]{UC3.3}},
    {RFO3.3.1, L'utente deve poter normalizzare globalmente i dati presenti nel dataset, \obb, Interno \noexpand\hyperref[uc3.3.1]{UC3.3.1}},
    {RFO3.3.2, L'utente deve poter normalizzare per riga i dati presenti nel dataset, \obb, Interno \noexpand\hyperref[uc3.3.2]{UC3.3.2}},
    {RFO3.3.3, L'utente deve poter normalizzare per colonna i dati presenti nel dataset, \obb, Interno \noexpand\hyperref[uc3.3.3]{UC3.3.3}},
    {RFF3.4, L'utente deve poter selezionare le righe del dataset a cui è interessato, Facoltativo, Interno \noexpand\hyperref[uc3.4]{UC3.4}},
    {RFO4, L'utente{,} deve poter modificare le impostazioni che influenzano la visualizzazione, \obb,Interno \noexpand\hyperref[uc4]{UC4}},
    {RFO4.1, L'utente{,} deve poter assegnare una classe di visualizzazione ad ogni variabile target selezionata,\obb,Interno \noexpand\hyperref[uc4.1]{UC4.1}},
    {RFO4.2, L'utente deve poter modificare il range dei dati da considerare nella Heatmap, \obb, Interno \noexpand\hyperref[uc4.2]{UC4.2}},
    {RFO4.3, L'utente deve poter assegnare un colore al range di una Heatmap, \obb, Interno \noexpand\hyperref[uc4.3]{UC4.3}},
     {RFO4.4, L'utente{,} se ha selezionato l'Heatmap{,} deve poter ordinare il dataset, \obb,Interno \noexpand\hyperref[uc4.4]{UC4.4}},
    {RFO4.4.1, L'utente{,} se ha selezionato l'Heatmap{,} deve poter ordinare il dataset in ordine alfabetico, \obb,Interno \noexpand\hyperref[uc4.4.1]{UC4.4.1}},
    {RFO5.3.2, L'utente{,} se ha selezionato l'Heatmap{,} deve poter ordinare il dataset in cluster, \obb,Capitolato \noexpand\hyperref[uc4.4.2]{UC4.4.2}},
    {RFO5, L'utente{,} deve poter calcolare la matrice di distanza se ha selezionato Heatmap o Forcefield, \obb,Interno \noexpand\hyperref[uc5]{UC5}},
    {RFO6, L'utente{,} se ha deciso di calcolare la matrice di distanza{,} deve poter scegliere quale funzione di distanza utilizzare, Obbligatorio,Capitolato \noexpand\hyperref[uc6]{UC6}},
    {RFO6.1, L'utente{,} se ha deciso di calcolare la matrice di distanza{,} deve poter scegliere la distanza euclidea, Obbligatorio,Interno \noexpand\hyperref[uc6.1]{UC6.1}},
    {RFO6.2, L'utente{,} se ha deciso di calcolare la matrice di distanza{,} deve poter scegliere la distanza di Manhattan, Obbligatorio,Interno \noexpand\hyperref[uc6.2]{UC6.2}},
    {RFO7, L'utente{,} se ha selezionato la Linear Projection{,} deve poter scegliere l'algoritmo per la riduzione delle componenti per l'elaborazione dati, \obb,Capitolato \noexpand\hyperref[uc7]{UC7}},
    {RFF7.1, L'utente{,} se ha selezionato la Linear Projection{,} deve poter scegliere PCA come algoritmo per la riduzione delle componenti, Facoltativo,Interno \noexpand\hyperref[uc7.1]{UC7.1}},
    {RFF7.2, L'utente{,} se ha selezionato la Linear Projection{,} deve poter scegliere UMAP come algoritmo per la riduzione delle componenti, Facoltativo,Capitolato \noexpand\hyperref[uc7.2]{UC7.2}},
    {RFF7.3, L'utente{,} se ha selezionato la Linear Projection{,} deve poter scegliere t-SNE come algoritmo per la riduzione delle componenti, Facoltativo,Capitolato \noexpand\hyperref[uc7.3]{UC7.3}},
    {RFF8, L'utente{,} se ha selezionato l'algoritmo PCA {,} deve poterne specificare i parametri, Facoltativo,Interno \noexpand\hyperref[uc8]{UC8}},
    {RFF8.1, L'utente{,} se ha selezionato l'algoritmo PCA {,} deve poter specificare il numero di PCs da calcolare, Facoltativo,Interno \noexpand\hyperref[uc8.1]{UC8.1}},
    {RFO9, L'utente deve salvare la visualizzazione come file PNG, \obb, Interno \noexpand\hyperref[uc9]{UC9}},
    {RFO10, L'utente visualizza un messaggio di errore se carica i files scorrettamente, \obb, Interno \noexpand\hyperref[uc10]{UC10}},
}
%endTable


%%%
%%%
%%%
\newcommand*\requisitiftable{}
\foreach \x [count=\nj] in \requisitif
{
    \foreach \y [count=\ni] in \x
    {
        \ifnum\ni=4
            \xappto\requisitiftable{\y}
            \gappto\requisitiftable{\\}
            \gappto\requisitiftable{\hline}
        \else
            \xappto\requisitiftable{\y & }
        \fi
    }
}

%\subsection{Requisiti funzionali}

% Impostazioni della tabella
\tabulinesep = 2mm % padding
\taburowcolors [1] 2{pari .. dispari} % colori delle righe
\addcontentsline{lot}{table}{Requisiti funzionali}
\begin{longtabu} to \textwidth {| X[0.2 l m] | X[0.4 l m] |  X[0.2 l m] | X[0.2 l m] |} % larghezza delle colonne
\hline
\rowcolor{header} % colore dell'header
    
\textbf{Requisito} & \textbf{Descrizione} & \textbf{Classificazione} & \textbf{Fonte} \\
\hline
\requisitiftable

\end{longtabu}
