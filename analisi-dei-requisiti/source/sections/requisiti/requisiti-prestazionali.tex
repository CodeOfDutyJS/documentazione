\subsection{Requisiti prestazionali}

\def\obb{Obbligatorio}

\def\requisitip{
    {ciao, fc, \obb, UC1},
    {Requisito, Descrizione, Classificazione, Fonte},
}





%%%
%%%
%%%
\newcommand*\requisitiptable{}
\foreach \x [count=\nj] in \requisitip
{
    \foreach \y [count=\ni] in \x
    {
        \ifnum\ni=4
            \xappto\requisitiptable{\y}
            \gappto\requisitiptable{\\}
            \gappto\requisitiptable{\hline}
        \else
            \xappto\requisitiptable{\y & }
        \fi
    }
}

%\subsection{Requisiti funzionali}

% Impostazioni della tabella
\tabulinesep = 2mm % padding
\taburowcolors [1] 2{pari .. dispari} % colori delle righe
\addcontentsline{lot}{table}{Requisiti prestazionali}
\begin{longtabu} to \textwidth {| X[0.2 l m] | X[0.4 l m] |  X[0.2 l m] | X[0.2 l m] |} % larghezza delle colonne
\hline
\rowcolor{header} % colore dell'header
    
\textbf{Requisito} & \textbf{Descrizione} & \textbf{Classificazione} & \textbf{Fonte} \\
\hline
\requisitiptable

\end{longtabu}