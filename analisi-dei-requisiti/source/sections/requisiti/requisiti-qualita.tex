\subsection{Requisiti di qualità}

\def\obb{Obbligatorio}

\def\requisitiq{
    {RQO1, L'utente deve poter scegliere come visualizzare le etichette presenti nel suo dataset all'interno della visualizzazione, \obb, Interno UC3.1},
    {RQO2, L'utente deve poter scartare le variabili a cui non è interessato o riaggiungere quelle scartate, \obb, Interno UC4},
    {RQO3, L'utente se ha selezionato l'Heatmap deve poter modificare il range, \obb, Interno UC5.1},
    {RQO4, L'utente deve poter normalizzare il dataset, \obb, Interno UC5.2},
    {RQO4.1, L'utente deve poter normalizzare il dataset in modo globale, \obb, Interno UC5.2.1},
    {RQO4.2, L'utente deve poter normalizzare il dataset per colonna, \obb, Interno UC5.2.2},
    {RQO4.3, L'utente deve poter normalizzare il dataset per riga, \obb, Interno UC5.2.3},
    {RQO5, L'utente se ha selezionato l'Heatmap deve poter ordinare le righe , \obb, Interno UC5.3},
    {RQO5.1, L'utente se ha selezionato l'Heatmap deve poter ordinare le righe in ordine alfabetico, \obb, Interno UC5.3.1},
    {RQO5.2, L'utente se ha selezionato l'Heatmap deve poter ordinare le righe per cluster gerarchico, \obb, Capitolato UC5.3.2},
    {RQO6, L'utente se ha selezionato l'Heatmap deve poter assegnare un colore al range , \obb, Interno UC5.4},
    {RQO7, L'utente deve salvare la visualizzazione come file PNG, \obb, Interno UC9.2},
    {RQO8, L'utente visualizza un messaggio di errore se carica i files scorrettamente, \obb, Interno UC10},
}





%%%
%%%
%%%
\newcommand*\requisitiqtable{}
\foreach \x [count=\nj] in \requisitiq
{
    \foreach \y [count=\ni] in \x
    {
        \ifnum\ni=4
            \xappto\requisitiqtable{\y}
            \gappto\requisitiqtable{\\}
            \gappto\requisitiqtable{\hline}
        \else
            \xappto\requisitiqtable{\y & }
        \fi
    }
}

%\subsection{Requisiti funzionali}

% Impostazioni della tabella
\tabulinesep = 2mm % padding
\taburowcolors [1] 2{pari .. dispari} % colori delle righe
\addcontentsline{lot}{table}{Requisiti di qualità}
\begin{longtabu} to \textwidth {| X[0.2 l m] | X[0.4 l m] |  X[0.2 l m] | X[0.2 l m] |} % larghezza delle colonne
\hline
\rowcolor{header} % colore dell'header
    
\textbf{Requisito} & \textbf{Descrizione} & \textbf{Classificazione} & \textbf{Fonte} \\
\hline
\requisitiqtable

\end{longtabu}