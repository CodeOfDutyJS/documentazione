\subsection{Requisiti di qualità}

\def\obb{Obbligatorio}

\def\requisitiq{
    {RQO1, L'utente deve poter scegliere come visualizzare le labels presenti nel suo dataset all'interno della visualizzazione, \obb, Interno UC3.1},
}





%%%
%%%
%%%
\newcommand*\requisitiqtable{}
\foreach \x [count=\nj] in \requisitiq
{
    \foreach \y [count=\ni] in \x
    {
        \ifnum\ni=4
            \xappto\requisitiqtable{\y}
            \gappto\requisitiqtable{\\}
            \gappto\requisitiqtable{\hline}
        \else
            \xappto\requisitiqtable{\y & }
        \fi
    }
}

%\subsection{Requisiti funzionali}

% Impostazioni della tabella
\tabulinesep = 2mm % padding
\taburowcolors [1] 2{pari .. dispari} % colori delle righe
\addcontentsline{lot}{table}{Requisiti di qualità}
\begin{longtabu} to \textwidth {| X[0.2 l m] | X[0.4 l m] |  X[0.2 l m] | X[0.2 l m] |} % larghezza delle colonne
\hline
\rowcolor{header} % colore dell'header
    
\textbf{Requisito} & \textbf{Descrizione} & \textbf{Classificazione} & \textbf{Fonte} \\
\hline
\requisitiqtable

\end{longtabu}