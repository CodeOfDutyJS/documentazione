\subsection{Requisiti di qualità}

\def\obb{Obbligatorio}
\def\pdq{Piano di Qualifica}

%startTable
\def\requisitiq{
    {RQO1, Deve essere prodotto un manuale d'uso per l'utente, \obb, Capitolato},
    {RQO2, Deve essere prodotto un manuale manutentore, \obb, Capitolato},
    {RQD3, Il codice deve essere pubblicato su un repository pubblico, Desiderabile, Capitolato},
    {RQO4, Il codice deve seguire le norme di stile specificate nel documento Norme di Progetto, \obb, Interno{,} \noexpand\NdP},
    {RQO5, Nella codifica{,} deve essere evitato l'utilizzo di chiamate ricorsive se non in casi strettamente necessari, \obb, Interno{,} \noexpand\NdP},
}
%endTable





%%%
%%%
%%%
\newcommand*\requisitiqtable{}
\foreach \x [count=\nj] in \requisitiq
{
    \foreach \y [count=\ni] in \x
    {
        \ifnum\ni=4
            \xappto\requisitiqtable{\y}
            \gappto\requisitiqtable{\\}
            \gappto\requisitiqtable{\hline}
        \else
            \xappto\requisitiqtable{\y & }
        \fi
    }
}

%\subsection{Requisiti funzionali}

% Impostazioni della tabella
\tabulinesep = 2mm % padding
\taburowcolors [1] 2{pari .. dispari} % colori delle righe
\addcontentsline{lot}{table}{Requisiti di qualità}
\begin{longtabu} to \textwidth {| X[0.2 l m] | X[0.4 l m] |  X[0.2 l m] | X[0.2 l m] |} % larghezza delle colonne
\hline
\rowcolor{header} % colore dell'header
    
\textbf{Requisito} & \textbf{Descrizione} & \textbf{Classificazione} & \textbf{Fonte} \\
\hline
\requisitiqtable

\end{longtabu}
