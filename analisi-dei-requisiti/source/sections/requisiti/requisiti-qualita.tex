\subsection{Requisiti di qualità}

\def\obb{Obbligatorio}
\def\pdq{Piano di Qualifica}

%startTable
\def\requisitiq{
    {RQO1, Correttezza dello scambio dei dati{,} i dati ricevuti da fonti esterne non sono modificati o errati, \obb, Interno{,} \pdq{} 0.1.1},
    {RQO2, Densità di errori{,} il numero di errori rilevati sul codice tramite i test di unità deve rispettare il valore sufficiente specificato, \obb, Interno{,} \pdq{} 0.1.2},
    {RQO3, Qualità della messaggistica{,} i messaggi di errore e di avviso all'utente devono essere chiari, \obb, Interno{,} \pdq{} 0.1.3},
    {RQO4, Numero di click{,} il numero di click per svolgere un determinato task deve essere minore del valore sufficiente specificato, \obb, Interno{,} \pdq{} 0.1.4},
    {RQO5, Site depth{,} la profondità dell’albero che rappresenta la struttura dell’applicativo deve essere minore del valore sufficiente specificato, \obb, Interno{,} \pdq{} 0.1.5},
    {RQO6, Response time{,} la durata di uno specifico task in secondi deve essere minore del valore sufficiente specificato, \obb, Interno{,} \pdq{} 0.1.6},
    {RQO7, Complessità ciclomatica{,} la quantità dei possibili percorsi di branching di una porzione di codice deve essere minore del valore sufficiente specificato, \obb, Interno{,} \pdq{} 0.1.7},
    {RQO8, Indipendenza dei test{,} la percentuale di test indipendenti sul totale dei test deve essere maggiore del valore sufficiente specificato, \obb, Interno{,} \pdq{} 0.1.8},
    {RQO9, Facilità di comprensione{,} l'indice di comprensione del codice è il rapporto tra il numero di righe di commento e il numero di righe del codice e deve essere maggiore del valore sufficiente specificato, \obb, Interno{,} \pdq 0.1.9},
    {RQO10, Lo Structural Fan-In{,} cioè il numero di procedure che chiamano una specifica procedura{,} deve essere maggiore del valore sufficiente specificato, \obb, Interno{,} \pdq{} 0.1.10},
    {RQO11, Lo Structural Fan-Out{,} cioè il numero di procedure di cui necessita una specifica procedura{,} deve essere minore del valore sufficiente specificato, \obb, Interno{,} \pdq{} 0.1.11},
}
%endTable





%%%
%%%
%%%
\newcommand*\requisitiqtable{}
\foreach \x [count=\nj] in \requisitiq
{
    \foreach \y [count=\ni] in \x
    {
        \ifnum\ni=4
            \xappto\requisitiqtable{\y}
            \gappto\requisitiqtable{\\}
            \gappto\requisitiqtable{\hline}
        \else
            \xappto\requisitiqtable{\y & }
        \fi
    }
}

%\subsection{Requisiti funzionali}

% Impostazioni della tabella
\tabulinesep = 2mm % padding
\taburowcolors [1] 2{pari .. dispari} % colori delle righe
\addcontentsline{lot}{table}{Requisiti di qualità}
\begin{longtabu} to \textwidth {| X[0.2 l m] | X[0.4 l m] |  X[0.2 l m] | X[0.2 l m] |} % larghezza delle colonne
\hline
\rowcolor{header} % colore dell'header
    
\textbf{Requisito} & \textbf{Descrizione} & \textbf{Classificazione} & \textbf{Fonte} \\
\hline
\requisitiqtable

\end{longtabu}