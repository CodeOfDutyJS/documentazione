\documentclass[a4paper]{article}

\usepackage[italian]{babel}
\usepackage[utf8]{inputenc}
\usepackage[T1]{fontenc}
\usepackage{enumitem}
\usepackage{graphicx}
\usepackage{float}
\usepackage{longtable}
\usepackage[table]{xcolor}
\usepackage{geometry}

\usepackage{lastpage}
\usepackage[bottom]{footmisc}
\usepackage{fancyhdr}
\usepackage{tabu}

\usepackage{pgffor}
\usepackage{etoolbox}

\usepackage[official]{eurosym}

% Navigazione pdf
\usepackage{hyperref}
\hypersetup{
	colorlinks=true,
	linkcolor=black,
	filecolor=magenta,      
	urlcolor=blue,
}

\usepackage{tabularx}

\newcommand{\glo}{\textsubscript{\emph{G}}}
\newcommand\hd{\emph{HD Viz}}
\newcommand\cod{\emph{Code of Duty}}
\newcommand{\myparagraph}[1]{\paragraph{#1}\mbox{}\\}
\newcommand{\mysubparagraph}[1]{\subparagraph{#1}\mbox{}\\}

\definecolor{header}{HTML}{FA8D21}
\definecolor{pari}{HTML}{DBDBDB}
\definecolor{dispari}{HTML}{F5F5F5}

\setcounter{secnumdepth}{4}

\pagestyle{fancy}
\lhead{\includegraphics[scale=0.06]{../_template/images/logo_crop.png}}

%Titolo del documento
\rhead{\titolodocumento{}}
\cfoot{Pagina \thepage\ di \pageref{LastPage}}
\renewcommand{\footrulewidth}{0.4pt}

\newcommand{\titolodocumento}{Analisi dei requisiti} % Titolo documento
\newcommand{\versione}{ } % Versione documento
\newcommand{\approvazione}{ } % Responsabile di progetto
\newcommand{\redazione}{ } % Redattori di questo documento
\newcommand{\verifica}{ } % Verificatori di questo documento
\newcommand{\stato}{In lavorazione} % Approvato
\newcommand{\uso}{Esterno} % Interno - Esterno
\newcommand{\destinazione}{ \parbox[t]{4cm}{ } } % Destinatario ( D1\\D2\\D3 )
\newcommand{\descrizionedocumento}{ } % Descrizione documento

% Mettere sempre la virgola dopo l'ultima riga, se no si rompe la tabella

\def\modifiche{
    {0.0.1,2021-01-04,Damiano Zanardo, Analista, Stesura struttura documento},
}



% Per modificare il frontespizio e diario delle modifiche andare sulla cartella source
% Per aggiungere il contenuto andare sulla cartella source/sections e creare un nuovo file.tex per ogni sezione

% Per parole che devono andare nel glossario aggiungere /glo{} dopo la parola

\begin{document}
    % Frontespizio 
    \newcommand{\titolodocumento}{Analisi dei requisiti} % Titolo documento
\newcommand{\versione}{ } % Versione documento
\newcommand{\approvazione}{ } % Responsabile di progetto
\newcommand{\redazione}{ } % Redattori di questo documento
\newcommand{\verifica}{ } % Verificatori di questo documento
\newcommand{\stato}{In lavorazione} % Approvato
\newcommand{\uso}{Esterno} % Interno - Esterno
\newcommand{\destinazione}{ \parbox[t]{4cm}{ } } % Destinatario ( D1\\D2\\D3 )
\newcommand{\descrizionedocumento}{ } % Descrizione documento

    
    % Diario delle modifiche
    \newcommand*\mytablecontents{}
\foreach \x [count=\nj] in \modifiche
{
    \foreach \y [count=\ni] in \x
    {
        \ifnum\ni=6
            \xappto\mytablecontents{\y}
            \gappto\mytablecontents{\\}
            \gappto\mytablecontents{\hline}
        \else
            \xappto\mytablecontents{\y&}
        \fi
    }
}

\section*{Diario delle modifiche}

% Impostazioni della tabella
\tabulinesep = 2mm % padding
\taburowcolors [1] 2{pari .. dispari} % colori delle righe
\begin{longtabu} to \textwidth {| X[0.3, c m] | X[0.6,c m] | X[0.7,c m] | X[0.9,c m] | X[0.7,c m] | X[1.2,l m]|} % larghezza delle colonne
\hline
\rowcolor{header} % colore dell'header

\textbf{Ver.} & \textbf{Data} & \textbf{Nominativo} & \textbf{Ruolo} & \textbf{Verificatore} & \multicolumn{1}{c|}{\textbf{Descrizione}}\\
\hline
\mytablecontents

\end{longtabu}

    \pagebreak
    % Indice
    \tableofcontents
    \pagebreak
    % Se necessari
    % \listoffigures
    % \listoftables
    % Contenuto
    \section{A}

%startTable
\def\definizioniA{
{Amazon CloudWatch,Servizio di monitoraggio che raccoglie log in una visualizzazione unificata per tutti i servizi AWS.\\ \href{https://aws.amazon.com/it/cloudwatch/}{https://aws.amazon.com/it/cloudwatch/}},
{Analisi dei dati,Processo di ispezione, pulizia, trasformazione e modellazione di dati con il fine di evidenziare informazioni.},
{Apache Kafka,Piattaforma a bassa latenza ed altà velocità per il processing di dati in tempo reale.\\ \href{https://kafka.apache.org/}{https://kafka.apache.org/}},
{API Rest,Acronico per Application Programming Interface. Sono un insieme di definizioni e protocolli per l'integrazione di software. REST è un insieme di principi architetturali per sviluppare API.},
{AWS,Azienda che fornisce vari servizi cloud.\\ \href{https://aws.amazon.com/it/}{https://aws.amazon.com/it/}},
{AWS AppSync,Servizio per lo sviluppo di API GraphQL.\\ \href{https://aws.amazon.com/it/appsync/}{https://aws.amazon.com/it/appsync/}},
{AWS CloudFormation,Servizio per la gestione di risorse AWS e di terze parti.\\ \href{https://aws.amazon.com/it/cloudformation/}{https://aws.amazon.com/it/cloudformation/}},
{AWS Cognito Identity,Servizio per la gestione di registrazioni e accessi da parte di utenti per le app Web.\\ \href{https://aws.amazon.com/it/cognito/}{https://aws.amazon.com/it/cognito/}},
{AWS GameLift,Servizio per la gestione di server per giochi multiplayer.\\ \href{https://aws.amazon.com/it/gamelift/}{https://aws.amazon.com/it/gamelift/}},
{AWS Lambda,Servizio per l'esecuzione di codice serverless.\\ \href{https://aws.amazon.com/it/lambda/}{https://aws.amazon.com/it/lambda/}},
}
%endTable

\begin{description}
\foreach \x [count=\nj] in \definizioniA
{
    \foreach \y [count=\ni] in \x
    {
        \ifnum\ni=1
            \item[\y] \hfill\\
        \else
            \y
        \fi
    }
}
\end{description}
    \pagebreak
    \section{B}

%startTable
\def\definizioniB{
{Backend,Parte di una soluzione software che gestisce, elabora e utilizza i dati forniti dal front end.},
{Blockchain,Struttura dati condivisa e immutabile. Le voci sono raggruppate a blocchi che vengono concatenati in ordine cronologico. L'immutabilità viene garantita dall'uso della crittografia.},
{Bootstrap,Toolkit per lo sviluppo di front end per siti e applicazioni web.\\ \href{https://getbootstrap.com/}{https://getbootstrap.com/}},
{Business Logic,Parte logica e di elaborazione di una soluzione software.},
}
%endTable

\begin{description}
\foreach \x [count=\nj] in \definizioniB
{
    \foreach \y [count=\ni] in \x
    {
        \ifnum\ni=1
            \item[\y] \hfill\\
        \else
            \y
        \fi
    }
}
\end{description}
    \pagebreak
    \section{C}

%startTable
\def\definizioniC{
    {Albero, è una pianta},
    {Angelo, è una pianta},
}
%endTable

\begin{description}
\foreach \x [count=\nj] in \definizioniC
{
    \foreach \y [count=\ni] in \x
    {
        \ifnum\ni=1
            \item[\y] \hfill\\
        \else
            \y
        \fi
    }
}
\end{description}
    \pagebreak
    \section{D}

%startTable
\def\definizioniD{
    {Albero, è una pianta},
    {Angelo, è una pianta},
}
%endTable

\begin{description}
\foreach \x [count=\nj] in \definizioniD
{
    \foreach \y [count=\ni] in \x
    {
        \ifnum\ni=1
            \item[\y] \hfill\\
        \else
            \y
        \fi
    }
}
\end{description}
    \pagebreak
    \section{E}

%startTable
\def\definizioniE{
{Ethereum,Piattaforma decentralizzata per la creazione e pubblicazione di smart-contracts.\\ \href{https://ethereum.org/it/}{https://ethereum.org/it/}},
}
%endTable

\begin{description}
\foreach \x [count=\nj] in \definizioniE
{
    \foreach \y [count=\ni] in \x
    {
        \ifnum\ni=1
            \item[\y] \hfill\\
        \else
            \y
        \fi
    }
}
\end{description}
    \pagebreak
    \section{F}

%startTable
\def\definizioniF{
{Feature,Campo dati, solitamente numerico, che fornisce un'informazione su una caratteristica. (es: Lunghezza, Altezza)},
{Force Field,Visualizzazione che simula forze tra particelle che si attraggono e respingono.},
{Frontend,Parte di una soluzione software che interagisce con l'utente.},
}
%endTable

\foreach \x [count=\nj] in \definizioniF
{
    \foreach \y [count=\ni] in \x
    {
        \ifnum\ni=1
            \subsection\y \hfill\\
        \else
            \y
        \fi
    }
}
    \pagebreak
    \section{G}

%startTable
\def\definizioniG{
{GitHub,Piattaforma per versionamento tramite Git e condivisione di codice. Fornisce anche strumenti di CI/CD.\\ \href{https://github.com/}{https://github.com/}},
{GitHub Actions,Servizio offerto da GitHub per integrare la CI/CD direttamente sul repository.},
{Gitlab,Piattaforma per versionamento tramite Git e condivisione di codice. Fornisce anche strumenti di CI/CD.\\ \href{https://about.gitlab.com/}{https://about.gitlab.com/}},
}
%endTable

\foreach \x [count=\nj] in \definizioniG
{
    \foreach \y [count=\ni] in \x
    {
        \ifnum\ni=1
            \subsection\y \hfill\\
        \else
            \y
        \fi
    }
}
    \pagebreak
    \section{H}

%startTable
\def\definizioniH{
{Heatmap,Visualizzazione dove i dati contenuti in una matrice sono rappresentati da colori.},
}
%endTable

\foreach \x [count=\nj] in \definizioniH
{
    \foreach \y [count=\ni] in \x
    {
        \ifnum\ni=1
            \subsection\y \hfill\\
        \else
            \y
        \fi
    }
}
    \pagebreak
    \section{I}

%startTable
\def\definizioniI{
    {Albero, è una pianta},
    {Angelo, è una pianta},
}
%endTable

\begin{description}
\foreach \x [count=\nj] in \definizioniI
{
    \foreach \y [count=\ni] in \x
    {
        \ifnum\ni=1
            \item[\y] \hfill\\
        \else
            \y
        \fi
    }
}
\end{description}
    \pagebreak
    \section{J}

%startTable
\def\definizioniJ{
{Java,Linguaggio di programmazione orientato agli oggetti e alla tipizzazione statica.},
{Javascript,Linguaggio di programmazione orientato agli oggetti e agli eventi.},
{JSON,Formato di file per conservare dati. Acronimo di JavaScript Object Notation.},
}
%endTable

\begin{description}
\foreach \x [count=\nj] in \definizioniJ
{
    \foreach \y [count=\ni] in \x
    {
        \ifnum\ni=1
            \item[\y] \hfill\\
        \else
            \y
        \fi
    }
}
\end{description}
    \pagebreak
    \section{K}

%startTable
\def\definizioniK{
    {Albero, è una pianta},
    {Angelo, è una pianta},
}
%endTable

\begin{description}
\foreach \x [count=\nj] in \definizioniK
{
    \foreach \y [count=\ni] in \x
    {
        \ifnum\ni=1
            \item[\y] \hfill\\
        \else
            \y
        \fi
    }
}
\end{description}
    \pagebreak
    \section{L}

%startTable
\def\definizioniL{
{Label,Campo dati che identifica il dato. (es: Nome, Categoria)},
{Leaflet,Libreria JavaScript per sviluppare mappe geografiche interattive.},
{Lifecycle (Serverless Framework),Ciclo di vita di applicazioni serverless. Develop, Deploy, Test, Monitor, Secure.\\ \href{https://www.serverless.com/blog/serverless-now-full-lifecycle}{https://www.serverless.com/blog/serverless-now-full-lifecycle}},
}
%endTable

\foreach \x [count=\nj] in \definizioniL
{
    \foreach \y [count=\ni] in \x
    {
        \ifnum\ni=1
            \subsection\y \hfill\\
        \else
            \y
        \fi
    }
}
    \pagebreak
    \section{M}

%startTable
\def\definizioniM{
{MAPI,Acronimo per Messaging Application Programming Interface. Architettura di messaggistica basata sulle API per Windows.},
{Microservizio,Servizio che viene associato ad altri microservizi per organizzare un'applicazione completa.},
{Modello a V,Modello di sviluppo software esteso dal modello a cascata. Pone la scrittura del test del software nelle fasi iniziali dello sviluppo e prima della codifica.},
}
%endTable

\foreach \x [count=\nj] in \definizioniM
{
    \foreach \y [count=\ni] in \x
    {
        \ifnum\ni=1
            \subsection\y \hfill\\
        \else
            \y
        \fi
    }
}
    \pagebreak
    \section{N}

%startTable
\def\definizioniN{
{NFC,Acronimo per Near Field Communication. Tecnologia di ricetrasmissione che fornisce connettività senza fili a piccola distanza (massimo 10cm).},
{Node.js,Runtime orientato agli eventi asincroni per l'esecuzione di codice JavaScript.\\ \href{https://nodejs.org/it/about/}{https://nodejs.org/it/about/}},
}
%endTable

\begin{description}
\foreach \x [count=\nj] in \definizioniN
{
    \foreach \y [count=\ni] in \x
    {
        \ifnum\ni=1
            \item[\y] \hfill\\
        \else
            \y
        \fi
    }
}
\end{description}
    \pagebreak
    \section{O}

%startTable
\def\definizioniO{
{Open source,Termine per indica un tipo di software che tramite una licenza, i detentori dei diritti favoriscono la modifica, lo studio, l'utilizzo e la redistribuzione del codice sorgente.},
{OpenShift,PAAS di RedHat.\\ \href{https://www.openshift.com/}{https://www.openshift.com/}},
}
%endTable

\foreach \x [count=\nj] in \definizioniO
{
    \foreach \y [count=\ni] in \x
    {
        \ifnum\ni=1
            \subsection\y \hfill\\
        \else
            \y
        \fi
    }
}
    \pagebreak
    \section{P}

%startTable
\def\definizioniP{
{PAAS,Platform as a service. Piattaforme cloud che mettono a disposizione insieme di software pronti per l'utilizzo.},
{Parallel Coordinates,Visualizzazione che rappresenta degli assi paralleli. Ogni linea passa da un asse all'altro in corrispondenza del valore della feature.},
{Payment provider,Servizio per accettare pagamenti elettronici.},
{PCA,Principal Components Analysis. Tecnica di riduzione dimensionale.\\ \href{https://en.wikipedia.org/wiki/Principal_component_analysis}{https://en.wikipedia.org/wiki/Principal-component-analysis}},
{PCs,Principal Components. Sono i vettori direzionali calcolati dall'algoritmo che esegue la PCA.\\ \href{https://en.wikipedia.org/wiki/Principal_component_analysis}{https://en.wikipedia.org/wiki/Principal-component-analysis}},
{Product Baseline,Descrizione concordata degli attributi di un prodotto, in un certo istante. Utile come base per la definizione di cambiamenti.},
{Proiezione Lineare Multi Asse,Visualizzazione in cui dati con molte dimensioni vengono rappresentati su un piano.},
{Proof Of Concept,Realizzazione incompleta o abbozzata di un progetto.},
{Proponente,Ente o azienda che compie l’atto di proporre il capitolato d’appalto per un progetto.},
{PWA,Applicazioni web che si comportano come app native. Solitamente dopo che si è visitato la pagina la prima volta è possibile utilizzarle anche offline.},
{Python,Linguaggio di programmazione orientato agli oggetti.},
}
%endTable

\begin{description}
\foreach \x [count=\nj] in \definizioniP
{
    \foreach \y [count=\ni] in \x
    {
        \ifnum\ni=1
            \item[\y] \hfill\\
        \else
            \y
        \fi
    }
}
\end{description}

    \pagebreak
    \section{Q}

%startTable
\def\definizioniQ{
{Qt,Libreria multipiattaforma per lo sviluppo di applicazione con GUI.\\ \href{https://www.qt.io/}{https://www.qt.io/}},
}
%endTable

\foreach \x [count=\nj] in \definizioniQ
{
    \foreach \y [count=\ni] in \x
    {
        \ifnum\ni=1
            \subsection\y \hfill\\
        \else
            \y
        \fi
    }
}
    \pagebreak
    \section{R}

%startTable
\def\definizioniR{
{Rancher,Piattaforma per la gestione di cluster Kubernetes.\\ \href{https://rancher.com/}{https://rancher.com/}},
{React,Libreria JavaScript per la creazione di interfacce utente.\\ \href{https://it.reactjs.org/}{https://it.reactjs.org/}},
{Repository,Ambiente in cui vengono conservati e gestiti i file di un progetto.},
}
%endTable

\begin{description}
\foreach \x [count=\nj] in \definizioniR
{
    \foreach \y [count=\ni] in \x
    {
        \ifnum\ni=1
            \item[\y] \hfill\\
        \else
            \y
        \fi
    }
}
\end{description}
    \pagebreak
    \section{S}

%startTable
\def\definizioniS{
    {Albero, è una pianta},
    {Angelo, è una pianta},
}
%endTable

\begin{description}
\foreach \x [count=\nj] in \definizioniS
{
    \foreach \y [count=\ni] in \x
    {
        \ifnum\ni=1
            \item[\y] \hfill\\
        \else
            \y
        \fi
    }
}
\end{description}
    \pagebreak
    \section{T}

%startTable
\def\definizioniT{
{t-SNE,t-distributed stochastic neighbor embedding. Tecnica di riduzione dimensionale non lineare\\ \href{https://en.wikipedia.org/wiki/T-distributed_stochastic_neighbor_embedding}{https://en.wikipedia.org/wiki/T-distributed-stochastic-neighbor-embedding}},
{Tag RFID,Etichette elettroniche per l'identificazione e/o memorizzazione. Acronimo per Radio-frequency identification.},
{Technology Baseline,Base tecnica su cui si svilupperà il progetto.},
{Telegram,Telegram è un servizio di messaggistica istantanea e broadcasting basato su cloud.\\ \href{https://telegram.org/}{https://telegram.org/}},
{Tomcat,Server web sviluppato da Apache.\\ \href{https://tomcat.apache.org/}{https://tomcat.apache.org/}},
{TypeScript,Linguaggio di programmazione che è super-set di JavaScript sviluppato da Microsoft per la creazione di grandi applicazioni.},
}
%endTable

\foreach \x [count=\nj] in \definizioniT
{
    \foreach \y [count=\ni] in \x
    {
        \ifnum\ni=1
            \subsection\y \hfill\\
        \else
            \y
        \fi
    }
}
    \pagebreak
    \section{U}

%startTable
\def\definizioniU{
{UMAP,Uniform manifold approximation and projection. Tecnica di riduzione dimensionale non lineare.\\ \href{https://en.wikipedia.org/wiki/Nonlinear_dimensionality_reduction#Uniform_manifold_approximation_and_projection}{https://en.wikipedia.org/wiki/Nonlinear_dimensionality_reduction#Uniform_manifold_approximation_and_projection}},
{Unity,Motore grafico multipiattaforma.\\ \href{https://unity.com/}{https://unity.com/}},
{Unreal Engine,Motore grafico multipiattaforma.\\ \href{https://www.unrealengine.com/}{https://www.unrealengine.com/}},
}
%endTable

\begin{description}
\foreach \x [count=\nj] in \definizioniU
{
    \foreach \y [count=\ni] in \x
    {
        \ifnum\ni=1
            \item[\y] \hfill\\
        \else
            \y
        \fi
    }
}
\end{description}
    
\end{document}
