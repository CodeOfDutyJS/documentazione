\section{P}

%startTable
\def\definizioniP{
{PAAS,Platform as a service. Piattaforme cloud che mettono a disposizione insieme di software pronti per l'utilizzo.},
{Parallel Coordinates,Visualizzazione che rappresenta degli assi paralleli. Ogni linea passa da un asse all'altro in corrispondenza del valore della feature.},
{Payment provider,Servizio per accettare pagamenti elettronici.},
{PCA,Principal Components Analysis. Tecnica di riduzione dimensionale.\\ \href{https://en.wikipedia.org/wiki/Principal_component_analysis}{https://en.wikipedia.org/wiki/Principal-component-analysis}},
{PCs,Principal Components. Sono i vettori direzionali calcolati dall'algoritmo che esegue la PCA.\\ \href{https://en.wikipedia.org/wiki/Principal_component_analysis}{https://en.wikipedia.org/wiki/Principal-component-analysis}},
{Product Baseline,Descrizione concordata degli attributi di un prodotto, in un certo istante. Utile come base per la definizione di cambiamenti.},
{Proiezione Lineare Multi Asse,Visualizzazione in cui dati con molte dimensioni vengono rappresentati su un piano.},
{Proof Of Concept,Realizzazione incompleta o abbozzata di un progetto.},
{Proponente,Ente o azienda che compie l’atto di proporre il capitolato d’appalto per un progetto.},
{PWA,Applicazioni web che si comportano come app native. Solitamente dopo che si è visitato la pagina la prima volta è possibile utilizzarle anche offline.},
{Python,Linguaggio di programmazione orientato agli oggetti.},
}
%endTable

\foreach \x [count=\nj] in \definizioniP
{
    \foreach \y [count=\ni] in \x
    {
        \ifnum\ni=1
            \subsection\y \hfill\\
        \else
            \y
        \fi
    }
}
