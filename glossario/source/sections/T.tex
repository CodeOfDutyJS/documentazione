\section{T}

%startTable
\def\definizioniT{
{t-SNE,t-distributed stochastic neighbor embedding. Tecnica di riduzione dimensionale non lineare\\ \href{https://en.wikipedia.org/wiki/T-distributed_stochastic_neighbor_embedding}{https://en.wikipedia.org/wiki/T-distributed-stochastic-neighbor-embedding}},
{Tag RFID,Etichette elettroniche per l'identificazione e/o memorizzazione. Acronimo per Radio-frequency identification.},
{Technology Baseline,Base tecnica su cui si svilupperà il progetto.},
{Telegram,Telegram è un servizio di messaggistica istantanea e broadcasting basato su cloud.\\ \href{https://telegram.org/}{https://telegram.org/}},
{Tomcat,Server web sviluppato da Apache.\\ \href{https://tomcat.apache.org/}{https://tomcat.apache.org/}},
{TypeScript,Linguaggio di programmazione che è super-set di JavaScript sviluppato da Microsoft per la creazione di grandi applicazioni.},
}
%endTable

\begin{description}
\foreach \x [count=\nj] in \definizioniT
{
    \foreach \y [count=\ni] in \x
    {
        \ifnum\ni=1
            \item[\y] \hfill\\
        \else
            \y
        \fi
    }
}
\end{description}