\section{A}

%startTable
\def\definizioniA{
{Amazon CloudWatch,Servizio di monitoraggio che raccoglie log in una visualizzazione unificata per tutti i servizi AWS.\\ \href{https://aws.amazon.com/it/cloudwatch/}{https://aws.amazon.com/it/cloudwatch/}},
{Apache Kafka,Piattaforma a bassa latenza ed altà velocità per il processing di dati in tempo reale.\\ \href{https://kafka.apache.org/}{https://kafka.apache.org/}},
{API Rest,Acronico per Application Programming Interface. Sono un insieme di definizioni e protocolli per l'integrazione di software. REST è un insieme di principi architetturali per sviluppare API.},
{AWS,Azienda che fornisce vari servizi cloud.\\ \href{https://aws.amazon.com/it/}{https://aws.amazon.com/it/}},
{AWS AppSync,Servizio per lo sviluppo di API GraphQL.\\ \href{https://aws.amazon.com/it/appsync/}{https://aws.amazon.com/it/appsync/}},
{AWS Cognito Identity,Servizio per la gestione di registrazioni e accessi da parte di utenti per le app Web.\\ \href{https://aws.amazon.com/it/cognito/}{https://aws.amazon.com/it/cognito/}},
{AWS GameLift,Servizio per la gestione di server per giochi multiplayer.\\ \href{https://aws.amazon.com/it/gamelift/}{https://aws.amazon.com/it/gamelift/}},
{AWS Lambda,Servizio per l'esecuzione di codice serverless.\\ \href{https://aws.amazon.com/it/lambda/}{https://aws.amazon.com/it/lambda/}},
}
%endTable

\begin{description}
\foreach \x [count=\nj] in \definizioniA
{
    \foreach \y [count=\ni] in \x
    {
        \ifnum\ni=1
            \item[\y] \hfill\\
        \else
            \y
        \fi
    }
}
\end{description}