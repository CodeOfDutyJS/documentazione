\section{Tecnologie}
    \subsection{Tecnologie per lo sviluppo}
    \subsubsection{Javascript}
        Il linguaggio principale usato per l'implementazione di HD Viz è javascript, in particolare viene usata la versione ES2018.
    \subsubsection{ESlint}
        ESlint è uno strumento per l'analisi statica del codice javscript. La configurazione può essere consultata nel file .eslintrc.js. Lo stile di codifica adottato è quello di Airbnb.
    \subsubsection{Parte server}
    In questa sezione vengono elencate le tecnologie utilizzate per la parte server.
        \myparagraph{Node.Js}
        Node.Js è un ambiente runtime di Javascript basato sul motore \href{https://v8.dev/}{V8 Javascript engine}. Node.Js permette d'implementare il cosidetto paradigma "JavaScript everywhere", difatti sia la parte client che la parte server sono scritte utilizzando questo linguaggio.
        \myparagraph{ExpressJs}
        Express.js è un framework back end che semplifica la creazione di API rest.
        \url{http://expressjs.com/}
    \subsubsection{Parte client}
        \myparagraph{D3.js}
        D3.js è una libreria che permette di manipolare il DOM con un approccio data-driven, in \textit{HD Viz} viene usata principalmente per creare le visualizzazioni.
        \url{https://d3js.org/}
        \myparagraph{React}
        React è un framework JavaScript per la creazione di interfacce utente. Molto flessibile nello sviluppo di web app a singola pagina e mobile. React si occupa del rendering di elementi sul DOM tramite un Virtual DOM che aggiornerà  solamente le componenti che hanno subito un cambiamento. Solitamente viene affiancato da altre librerie per gestire lo state dell'applicazione.
        \myparagraph{Ant Design}
        Ant Design è un framework che utilizzato assieme a React permette la creazione di interfacce utente semplici ed espressive.
    \subsubsection{Testing}
        \myparagraph{Jest}
        Jest è un framework per il testing di codice Javascript, il quale permette di ottenere importanti informazioni dall'esecuzione dei test (code coverage). Fornisce inoltre semplici strumenti per il mock di oggetti esterni.
    \subsubsection{Altre librerie impiegate}
        \myparagraph{ml.js}
        Collezione di librerie JavaScript comprendente vari strumenti utili per applicazioni di Machine Learning. In HD Viz è stata utilizzata per calcolare matrici e distanze tra punti dimensionali.
        \myparagraph{PapaParse}
        Libreria per il parsing di file CSV in oggetti JSON. Viene utilizzata per convertire un file CSV in un formato utile per l'analisi dei dati e con la stessa struttura dei dati che vengono forniti dai database attraverso il server.