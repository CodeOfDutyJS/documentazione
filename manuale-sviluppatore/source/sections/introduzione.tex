\section{Introduzione}

    \subsection{Scopo del documento}
    Lo scopo del seguente documento è quello di evidenziare l'architettura del prodotto HD Viz, insieme alle modalità e alle convenzioni da seguire per manuntenere ed estendere il progetto.
    
    \subsection{Scopo del prodotto}
    Lo scopo del capitolato \emph{HD Viz} è lo sviluppo di un'applicazione web che permetta la visualizzazione di dati con molte dimensioni a supporto della fase esplorativa dell'analisi dei dati. La parte di visualizzazione verrà affidata alla libreria JavaScript \textbf{d3.js}. Il gruppo \emph{Code of Duty} propone lo sviluppo di una web app in grado di visualizzare dati provenienti da origini differenti. L'applicazione sarà in grado di funzionare anche offline.
    
    \subsection{Glossario}
    Al fine di rimuovere eventuali significati ambigui utilizzati all'interno del documento, quest'ultimo riporta un glossario con tutti i termini che necessitano di definizione. Ogni termine presente nel glossario presenterà come pedice la lettera \textbf{G}.
    
    \subsection{Riferimenti}
	\subsubsection{Normativi}
	\begin{itemize}
        \item \textbf{Capitolato d'Appalto}: \href{https://www.math.unipd.it/~tullio/IS-1/2020/Progetto/C4.pdf}{https://www.math.unipd.it/~tullio/IS-1/2020/Progetto/C4.pdf}.
	\end{itemize}
    \subsubsection{Informativi}
	\begin{itemize}
<<<<<<< HEAD
		\item \textbf{Pattern Template JavaScript}:  \href{https://nanofaroque.medium.com/template-method-design-pattern-in-javascript-286384155823}{https://nanofaroque.medium.com/template-method-design-pattern-in-javascript-286384155823}
		\item \textbf{Diagrammi package}: \\ \url{https://www.math.unipd.it/\%7Ercardin/swea/2021/Diagrammi\%20dei\%20Package_4x4.pdf}
		\item \textbf{Diagrammi classi}: \\ \url{https://www.math.unipd.it/\%7Ercardin/swea/2021/Diagrammi\%20delle\%20Classi_4x4.pdf}
		\item \textbf{Diagrammi sequenza}: \\ \url{https://www.math.unipd.it/\%7Ercardin/swea/2021/Diagrammi\%20di\%20Sequenza_4x4.pdf}
		\item \textbf{Analisi dei requisiti 2.3.0};
=======
		\item \textbf{Pattern Template JavaScript}:  \href{https://bit.ly/32yUACL}{Template method design pattern in javascript}
		\item \textbf{Diagrammi package}: \\ \url{https://www.math.unipd.it/\%7Ercardin/swea/2021/Diagrammi\%20dei\%20Package_4x4.pdf}
		\item \textbf{Diagrammi classi}: \\ \url{https://www.math.unipd.it/\%7Ercardin/swea/2021/Diagrammi\%20delle\%20Classi_4x4.pdf}
		\item \textbf{Diagrammi sequenza}: \\ \url{https://www.math.unipd.it/\%7Ercardin/swea/2021/Diagrammi\%20di\%20Sequenza_4x4.pdf}
>>>>>>> breggion/manuale-sviluppatore
		\item \textbf{Per l'implementazione del pattern MVVM}: \url{https://mobx.js.org/defining-data-stores.html}
	\end{itemize}
