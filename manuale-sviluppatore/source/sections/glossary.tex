\section{Glossario}
    \subsection{API Rest}
    Acronimo per Application Programming Interface. Sono un insieme di definizioni e protocolli
    per l’integrazione di software. REST è un insieme di principi architetturali per sviluppare API.

    \subsection{Backend}
    Parte di una soluzione software che gestisce elabora e utilizza i dati forniti dal frontend.

    \subsection{d3.js}
    Libreria JavaScript per creare visualizzazioni sul web.

    \subsection{Feature}
    Campo dati solitamente numerico che fornisce un’informazione su una caratteristica. (es: Lunghezza Altezza)

    \subsection{Frontend}
    Parte di una soluzione software che interagisce con l’utente.

    \subsection{GitHub}
    Piattaforma per versionamento tramite Git e condivisione di codice. Fornisce anche strumenti
    di CI/CD.

    \subsection{JavaScript}
    Linguaggio di programmazione orientato agli oggetti e agli eventi.

    \subsection{Node.js}
    Runtime orientato agli eventi asincroni per l’esecuzione di codice JavaScript.

    \subsection{Open source}
    Termine per indicare un tipo di software che tramite una licenza i detentori dei diritti favoriscono
    la modifica lo studio l’utilizzo e la redistribuzione del codice sorgente.

    \subsection{React}
    Libreria JavaScript per la creazione di interfacce utente.

    \subsection{Repository}
    Ambiente in cui vengono conservati e gestiti i file di un progetto.
