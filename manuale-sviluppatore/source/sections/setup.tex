\section{Setup}
    \subsection{Requisiti hardware}
    HD Viz richiede le seguenti specifiche hardware:
    \begin{itemize}
        \item \textbf{Processore}: Dual Core;
        \item \textbf{Memoria}: 2GB.
    \end{itemize}
    \subsection{Requisiti software}
        \subsubsection{Requisiti per l'installazione}
        \begin{itemize}
            \item \href{https://yarnpkg.com/}{Yarn v1.22.0} o superiore.
        \end{itemize}
        Se si vuole eseguire il server:
        \begin{itemize}
            \item \href{https://nodejs.org/en/}{Node v14.0.0} o superiore.
        \end{itemize}
    \subsection{Installazione}
    La repository si trova al seguente link:
    \url{https://github.com/CodeOfDutyJS/hdviz}
    \\
    per clonare la repository con Git spostarsi nella cartella dove si desidera mettere il progetto ed eseguire:
    \begin{verbatim}
        git clone https://github.com/CodeOfDutyJS/hdviz
    \end{verbatim}
    alternativamente è possibile scaricare il codice direttamente da Github.
        \subsubsection{Parte Server}
        per eseguire la parte server aprire il terminale e spostarsi nella cartella server con il comando:
        \begin{verbatim}
        cd path/della/cartella/server
        \end{verbatim}
        a questo punto basta eseguire il comando:
        \begin{verbatim}
        node index.js
        \end{verbatim}
        \subsubsection{Parte Client}
        Se si desidera eseguire solo la parte client spostarsi nella cartella client con il comando:
        \begin{verbatim}
        cd path/della/cartella/client
        \end{verbatim}
        a questo punto è necessario installare le dipendenze con il comando: 
        \begin{verbatim}
        yarn
        \end{verbatim}
        o alternativamente con:
        \begin{verbatim}
        yarn install
        \end{verbatim}
        a questo punto si può eseguire un Development Server con il comando:
        \begin{verbatim}
        yarn start
        \end{verbatim}
        la Web App sarà disponibile su localhost alla porta mostrata dal terminale (usualmente la porta 3000). A questo punto basta aprire uno dei browser supportati e immettere l'indirizzo:
        \begin{verbatim}
        http://localhost:3000/
        \end{verbatim}
        nel caso la porta mostrata dal terminale sia diversa dalla porta 3000 sostituire 3000 con la nuova porta nell'indirizzo.