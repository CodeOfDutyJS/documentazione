\section{Capitolato 3 - GDP}
\subsection{Informazioni generali}
\begin{itemize}
    \item \textbf{Nome}: \emph{GDP}: Gathering Detection Platform
    \item \textbf{Proponente}: \emph{Sync Lab S.r.l.}
    \item \textbf{Committente}: \emph{Prof. Tullio Vardanega e Prof. Riccardo Cardin}
\end{itemize}
\subsection{Descrizione capitolato}
Lo scopo del progetto è lo sviluppo di una piattaforma interattiva in grado di raccogliere i dati riguardanti il flusso di persone da varie fonti e provvedere alla costruzione di un modello tramite \textbf{Machine Learning} capace di effettuare previsioni sulla affluenza nei vari luoghi in modo da mostrare tramite \emph{heatmap} la situazione di tali flussi.
\subsection{Finalità del progetto}
Il prodotto dovrà offrire la possibilità di raccogliere, analizzare e flussi di dati rigurdante lo spostamento di persone, in particolare:
\begin{itemize}
    \item Vengono raccolti i dati grezzi da sorgenti eterogenee, e processati
    \item Questi vengono poi archiviati in un database
    \item Eventualmente vengono usati per l'online learning di un modello precedentemente costruito tramite machine learning
    \item Questi vengono mostrati con bassa latenza su una heatmap, o nel caso venga richiesta una previsione, vengono forniti al modello
    \item L'utilizzatore prende una decisione basandosi sull'output fornito
\end{itemize}
\subsection{Tecnologie interessate}
\begin{itemize}
    \item \textbf{Web development} L'azienda preferisce l'utilizzo di java per lo sviluppo back-end e Angular per il front-end.
    \item \textbf{Leaflet}: Framework javascript per la gestione di mappe.
    \item \textbf{Machine Learning}: L'azienda non suggerisce particolare restrizione alle librerie da utilizzare o agli algoritmi da addottare.
    \item \textbf{Apache Kafka}: Piattaforma open-source per il processo di stream di dati in tempo reale.
\end{itemize}
\subsection{Aspetti positivi}
\begin{itemize}
    \item Viene lasciata agli sviluppatori la scelta riguardo all'implementazione degli algoritmi di machine learning
    \item La conoscenza di alcune librerie di machine learning e delle tecnologie proposte nel progetto potrebbe rivelarsi utile per un futuro nel mondo del lavoro
    \item Il capitolato suggerisce le tecnologie e i pattern da adottare
\end{itemize}
\subsection{Criticità e fattori di rischio}
\begin{itemize}
    \item Alta quantità di tecnologie nuove, che richiedono una quantità di studio non indifferente
    \item Richiesti alti standard di qualità
\end{itemize}
\subsection{Conclusioni}