\section{Capitolato 7 - SSD}
\subsection{Informazioni generali}
\begin{itemize}
    \item \textbf{Nome}: \emph{SSD}: soluzioni di sincronizzazione desktop;
    \item \textbf{Proponente}: \emph{Zextras};
    \item \textbf{Committente}: \emph{Prof. Tullio Vardanega e Prof. Riccardo Cardin}.
\end{itemize}
\subsection{Descrizione capitolato}
Il Capitolato propone una soluzione dei problemi di sincronizzazione desktop per utenti professionali, quindi di sviluppo di una piattaforma dove gli utenti possano lavorare dal loro dispositivo ma mantenere una copia del lavoro salvata su cloud.
Gli obbiettivi in particolare sono 3:
\begin{itemize}
    \item lo sviluppo di un algoritmo solido ed efficiente in grado di garantire il salvataggio in cloud ed un lavoro contemporaneo sincronizzato dai cambiamenti;
    \item lo sviluppo di un'interfaccia multipiattaforma (MacOS, Windows, Linux);
    \item l'utilizzo dell'algoritmo in sincronizzazione verso il prodotto già in commercio del proponente Zextras Drive.
\end{itemize}
\subsection{Finalità del progetto}
Il proponente richiede le seguenti funzionalità:
\begin{itemize}
    \item configurazione ed autenticazione dell'utente;
    \item gestione di cosa sincronizzare e di cosa ignorare nelle cartelle cloud;
    \item gestione di cosa sincronizzare e di cosa ignorare nelle cartelle locali;
    \item sincronizzazione costante dei cambiamenti, siano essi locali o remoti;
    \item possibilità di modifica delle preferenze a posteriori;
    \item sistema di notifica utente dei cambiamenti.
\end{itemize}
Inoltre come funzionalità avanzate propone:
\begin{itemize}
    \item Gestione delle condivisioni, integrazione protocollo MAPI e integrazione con il prodotto web.
\end{itemize}
\subsection{Tecnologie interessate}
Il proponente lascia al gruppo la decisione delle tecnologie usate, ma consiglia l'uso delle seguenti tecnologie:
\begin{itemize}
    \item \textbf{Qt}: framework basato su C++. Consigliato per lo sviluppo dell'interfaccia data la sua ottima performance e la possibilità di creare widget;
    \item \textbf{Python}: linguaggio per lo sviluppo della Business Logic e del back-end. In particolare la Standard Library di Python.
\end{itemize}
\subsection{Aspetti positivi}
\begin{itemize}
    \item Il framework consigliato Qt è già stato utilizzato da tutti i membri del gruppo, come anche l'architettura del progetto, e quindi necessita di un approfondimento minore.
\end{itemize}
\subsection{Criticità e fattori di rischio}
\begin{itemize}
    \item La tempistiche di apprendimento degli algoritmi richiesti e dello sviluppo degli stessi possono sfociare facilmente in un prolungamento eccessivo del progetto;
    \item il proponente risulta molto vago nelle specifiche delle problematiche di contesto;
    \item non esiste una classificazione concreta di "utente professionista" che dovrebbe utilizzare il prodotto.
\end{itemize}
\subsection{Conclusioni}
Lo scopo del capitolato e le tecnologie annesse non hanno suscitato interesse nella grande maggioranza dei componenti del gruppo. Inoltre le problematiche ti tempo richieste hanno spostato l'attenzione dei membri verso altri capitolati che rispecchiano l'interesse dei membri del gruppo, e che assicurano uno svolgimento più efficace e di maggior crescita per gli stessi.