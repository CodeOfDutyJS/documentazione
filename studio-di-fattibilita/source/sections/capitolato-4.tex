\section{CAPITOLATO SCELTO: Capitolato 4 - \hd}
\subsection{Informazioni generali}
\begin{itemize}
    \item \textbf{Nome}: \hd: Visualizzazione di dati con molte dimensioni;
    \item \textbf{Proponente}: \emph{Zucchetti S.p.A.};
    \item \textbf{Committente}: \emph{Prof. Tullio Vardanega e Prof. Riccardo Cardin}.
\end{itemize}
\subsection{Descrizione capitolato}
Il proponente richiede l'implementazione di un'applicazione di visualizzazione di dati con molte dimensioni. Tale applicazione verrà sviluppata tramite le tecnologie HTML/CSS/JavaScript, con supporto della libreria D3.js. È inoltre richiesta una parte server di supporto alla presentazione nel browser e per interfacciarsi a un  database, la quale può essere implementata in Java con server Tomcat oppure in JavaScript con server Node.js.
\subsection{Finalità del progetto}
Tramite la realizzazione di questo software l'azienda può visualizzare in maniera più efficiente grosse quantità di dati. Un esempio pratico può essere il controllo dei cedolini degli stipendi dei dipendenti. Il programma dovrà:
\begin{itemize}
    \item i dati da visualizzare dovranno poter avere almeno 5 dimensioni;
    \item I dati devono poter essere forniti al sistema di visualizzazione, sia con query a un database che da file in formato CSV.
\end{itemize}
Inoltre \hd\ dovrà presentare almeno le seguenti visualizzazioni:
\begin{enumerate}
	\item Scatter plot Matrix (fino a 5 dimensioni);
	\item Force Field;
	\item Heat Map;
	\item Proiezione Lineare Multi Asse.
\end{enumerate}
Riguardo al grafico Heat Map \hd\ dovrà ordinare i punti nel grafico per evidenziare i cluster presenti nei dati.
\subsection{Tecnologie interessate}
Il software richiesto dovrà utilizzare le seguenti tecnologie:
\begin{itemize}
	\item \textbf{JavaScript/HTML/CSS}: linguaggi da impiegare per lo sviluppo dell'applicazione di visualizzazione;
	\item \textbf{Libreria D3.js}: libreria che contiene tipi di visualizzazione predefiniti;
	\item \textbf{Java o JavaScript}: linguaggi da utilizzare per la presentazione nel browser e per l'interfacciamento a database;
	\item \textbf{Tomcat o Node.js}: web server per verificare il corretto funzionamento dell'applicativo.
\end{itemize}
\subsection{Aspetti positivi}
\begin{itemize}
	\item Come primo aspetto, lavorare con un'azienda di grandi dimensioni è un'esperienza importante;
	\item l'Azienda si è dimostrata molto disponibile a chiarimenti di tipo tecnico;
	\item l'utilizzo di JavaScript che interessa alla totalità del gruppo;
	\item la chiarezza con il quale il proponente ha esposto il dominio del problema nel seminario.
\end{itemize}
\subsection{Criticità e fattori di rischio}
\begin{itemize}
	\item L'apprendimento delle tecnologie richieste potrebbe risultare lento per quei membri del gruppo che non le hanno mai utilizzate.
\end{itemize}
\subsection{Conclusioni}
Inizialmente il progetto sembrava complicato per via della parte di manipolazione dei dati, tuttavia l'incontro con la proponente ha chiarito le idee. La parte di modellazione e rappresentazione dei dati rimane una sfida difficile, ma ha catturato l'interesse del gruppo.