\section{Capitolato 6 - RGP}
\subsection{Informazioni generali}
\begin{itemize}
    \item \textbf{Nome}: \emph{RGP}: Realtime Gaming Platform;
    \item \textbf{Proponente}: \emph{Zero12};
    \item \textbf{Committente}: \emph{Prof. Tullio Vardanega e Prof. Riccardo Cardin}.
\end{itemize}
\subsection{Descrizione capitolato}
Il capitolato prevede la realizzazione di un videogioco a scorrimento verticale in stile Aero Fighters per dispositivi mobile, deve implementare sia una modalità di gioco in single player e una in multi player.
Il proponente richiede che venga tenuta in considerazione la scalabilità dell'applicazione, sopratutto lato server.
Richiede che venga sviluppata l'applicazione mobile per almeno uno dei principali sistemi operativi per smartphone consigliando iOS (Swift/SwiftUI) fornendo anche un certificato per effettuare i test sui dispositivi.
Il proponente richiede di effettuare un'analisi preliminare su quale tecnologia AWS sia meglio utilizzare, schema dell'architettura cloud, una documentazione dettagliata delle API e un piano di test di unità. Il tutto dovrà poi essere motivato e consegnato al proponente.
\subsection{Finalità del progetto}
La modalità multi player del gioco deve essere sviluppata su tecnologia Amazon Web Service (AWS) e permetterà all'utente di monitorare i progressi degli altri giocatori in tempo reale, vince l'ultimo giocatore a non aver esaurito le vite. 
La modalità single player invece sarà una modalità infinita, terminerà solo con l'esaurimento delle vite.
\subsection{Tecnologie interessate}
\begin{itemize}
    \item \textbf{NodeJS};
	\item \textbf{Swift/SwiftUI};
	\item \textbf{Kotlin};
	\item \textbf{Python};
	\item \textbf{Bootstrap}: framework che richiede conoscenze di HTML5, CSS3 e JavaScript;
	\item \textbf{AWS}: in particolare bisogna scegliere tra AWS GameLift o AWS AppSync.
\end{itemize}
\subsection{Aspetti positivi}
\begin{itemize}
    \item Usufruire dei servizi AWS permettendoci di studiare e approfondire le conoscenze dei servizi AWS;
    \item sviluppo mobile iOS/Android;
    \item il proponente sembra essere molto disponibile e interessato allo sviluppo del progetto, fornendo servizi e supporto al team.
\end{itemize}
\subsection{Criticità e fattori di rischio}
\begin{itemize}
    \item Richiede lo sviluppo nativo sia iOS e/o Android, negando di fatto la possibilità di usare motori grafici come Unity o Unreal Engine che permetterebbero di ridurre l'ammontare di ore necessario allo sviluppo, in quanto permettono la compilazione per entrambi gli SO mobile.
\end{itemize}
\subsection{Conclusioni}
Il progetto ha catturato l'interesse solo della minoranza dei membri del gruppo, per questo motivo è stato rigettato.