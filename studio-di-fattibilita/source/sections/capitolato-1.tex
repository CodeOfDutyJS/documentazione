\section{Capitolato 1 - BlockCOVID}
\subsection{Informazioni generali}
\begin{itemize}
    \item \textbf{Nome}: \emph{BlockCOVID}: supporto digitale al contrasto della pandemia;
    \item \textbf{Proponente}: \emph{Imola informatica};
    \item \textbf{Committente}: \emph{Prof. Tullio Vardanega e Prof. Riccardo Cardin}.
\end{itemize}
\subsection{Descrizione capitolato}
L'obiettivo del capitolato è quello di creare una piattaforma web e un'app che segnalano a un server il tracciamento delle presenze in tempo reale nel luogo di lavoro e della pulizia delle postazioni di lavoro. Il tracciamento deve essere immutabile e certificato e quindi è necessario utilizzare un sistema di \textbf{blockchain}, in questo caso basato su \emph{Ethereum}. La piattaforma web è dedicata all'amministratore mentre l'app è dedicata agli utenti (dipendenti e addetto alle pulizie).
\subsection{Finalità del progetto}
Il proponente descrive i seguenti casi d'uso possibili della piattaforma e app:
\begin{itemize}
    \item \textbf{Amministratore}
    \begin{itemize}
        \item creare, modificare o eliminare postazioni e stanze;
        \item creare, modificare o eliminare credenziali per utenti e personale specializzato;
        \item monitorare numero di dipendenti in stanze e postazioni;
        \item ricercare e effettuare report sui dati di tracciamento.
    \end{itemize}
    \item \textbf{Utente - dipendente}
    \begin{itemize}
        \item effettuare prenotazioni di postazioni;
        \item posizionare il telefono sul tag RFID in postazioni per ricevere informazioni sullo stato della postazione e/o per segnalare la propria presenza;
        \item segnalare l'igienizzazione autonoma della postazione;
        \item ricevere il nome di una postazione all'interno di una stanza.
    \end{itemize}
    \item \textbf{Utente - addetto alle pulizie}
    \begin{itemize}
        \item ricevere elenco delle stanze (e postazioni) da igienizzare;
        \item segnalare una stanza come igienizzata.
    \end{itemize}
\end{itemize}
\subsection{Tecnologie interessate}
Il proponente lascia la possibilità di scegliere le tecnologie da utilizzare, ma consiglia:
\begin{itemize}
    \item \textbf{Java, Python o NodeJS}: linguaggi per la parte di server back-end per la creazione di API Rest;
    \item \textbf{Sistema di blockchain Ethereum}: piattaforma decentralizzata per la creazione di applicazioni basate su smart-contract;
    \item \textbf{IAAS Kubernetes o di un PAAS, OpenShift o Rancher}: strumenti per il rilascio delle componenti server del progetto, utilizzabili anche per gestire la scalabilità delle operazioni;
    \item \textbf{Tag RFID}: etichette elettroniche in grado di memorizzare informazioni, in questo caso il codice della postazione; sono leggibili da appositi dispositivi oppure da molti modelli di smartphone tramite lo standard NFC;
    \item \textbf{Java o Kotlin}: linguaggi per sviluppare l'applicazione per dispositivi Android.
\end{itemize}
\subsection{Aspetti positivi}
\begin{itemize}
    \item Il proponente ha specificato in modo dettagliato il capitolato spiegando cosa si aspetta di ottenere;
    \item la parte di codifica non sembra troppo complessa (però piuttosto lunga) fatta eccezione per l'utilizzo di blockchain.
\end{itemize}
\subsection{Criticità e fattori di rischio}
\begin{itemize}
    \item Le tecnologie da utilizzare sono nuove per quasi tutti i membri del gruppo, in particolare i sistemi di blockchain, e sembrano abbastanza complessi da implementare.
\end{itemize}
\subsection{Conclusioni}
Il capitolato ha catturato l'interesse della maggior parte dei membri del gruppo, le tecnologie coinvolte e il problema da affrontare sono state valutate stimolanti e formative.
Dopo aver partecipato ai numerosi seminari tecnologici ed esserci riuniti in più occasioni si è deciso però di rigettare il capitolato in favore del C4.
