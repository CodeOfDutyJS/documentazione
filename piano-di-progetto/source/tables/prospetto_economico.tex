\newcommand*\salarysummary{}
\foreach \x [count=\nj] in \salarycontent
{
    \foreach \y [count=\ni] in \x
    {
        \ifnum\ni=1
            \xappto\salarysummary{\noexpand\textbf{\y}&}
        \else\ifnum\ni=3
            \xappto\salarysummary{\noexpand\euro\ \y&}
        \else\ifnum\ni=4
            \xappto\salarysummary{\noexpand\euro\ \y}
            \gappto\salarysummary{\\}
            \gappto\salarysummary{\hline} 
        \else
            \xappto\salarysummary{\y&}
        \fi\fi\fi
    }
}

% Impostazioni della tabella
\tabulinesep = 2mm % padding
\taburowcolors [1] 2{pari .. dispari} % colori delle righe
\begin{longtabu} to \textwidth {| X[0.1, c m] | X[0.1, c m] | X[0.1, c m] | X[0.1, c m] |}
\hline
\rowcolor{header} % colore dell'header
\textbf{Ruolo} &
\textbf{Ore} &
\textbf{Costo unitario} & 
\textbf{Costo totale} \\
\hline
\salarysummary
\end{longtabu}
\undef\salarysummary