\section{Analisi dei rischi}
	Nel corso dello sviluppo di un progetto è naturale incorrere in problemi, tuttavia è possibile evitarne alcuni attraverso l'analisi dei richi. Questa attività è stata effettuata attraverso l'analisi dei principali fattori di rischio. Per ogni fattore di rischio rilevato è stata utilizzata la medesima procedura di identificazionione e risoluzione:
	\begin{enumerate}
		\item \textbf{Individuazione}: Individuazione dei fattori di rischio che rallenterebbero o impedirebbero il proseguimento del progetto
		\item \textbf{Analisi}: Attività di studio dei fattori di rischio, ad ognuno di essi è stata assegnata una probabilità che si verifichi e un indice di gravità, basato sull'impatto che potrebbe avere sul progetto
		\item \textbf{Pianificazione di Controllo}: Studio di un metodologia per evitare il verificarsi di un determinato fattore di rischio e un piano di risoluzione nel caso essa si verifichi
		\item \textbf{Monitoraggio}: I fattori di rischio vanno tenuti sotto constante controllo, cercando di evitare che si verifichino se possibile oppure agire tempestivamente per minimizzare i danni
	\end{enumerate}
	Ogni fattore di rischio rilevato è stato definito e raggruppato secondo varie tipologie:
	\begin{itemize}
		\item \textbf{RT}: Rischio Tecnologici
		\item \textbf{RO}: Rischi Organizzativi
		\item \textbf{RI}: Rischi Interpersonali
	\end{itemize}
	\subsection{Rischi Tecnologici}
	\def\productquality{
    		{
        		Correttezza dello scambio dei dati,
        		$D_{err} = \#\ di\ errori$, 
        		-,
        		$D_{err} = 0$,
        		un valore sufficiente sarà deciso in fasi successive del progetto
    		},
	}
\newcommand*\metricstable{}
\foreach \x [count=\nj] in \productquality
{
    \foreach \y [count=\ni] in \x
    {
        \ifnum\ni=4
            \xappto\metricstable{\y}
            \gappto\metricstable{\\}
            \gappto\metricstable{\hline}
        \else\ifnum\ni=5
            \xappto\metricstable{\noexpand\multicolumn{4}{| l |}{NOTE: \y}}
            \gappto\metricstable{\\}
            \gappto\metricstable{\hline}
        \else
            \xappto\metricstable{\y&}
        \fi\fi
    }
}

% Impostazioni della tabella
\tabulinesep = 2mm % padding
\taburowcolors [1] 2{pari .. dispari} % colori delle righe
\begin{longtabu} to \textwidth {| X[0.7,c m] | X[0.7,c m] | X[0.4,c m] | X[0.4,c m]|} % larghezza delle colonne
\hline
\rowcolor{header} % colore dell'header

\textbf{Nome} & \textbf{Formula} & \textbf{Valore sufficiente} & \textbf{Valore ottimo}\\
\hline
\metricstable

\end{longtabu}

\undef\metricstable{}
	\subsection{Rischi Organizzativi}
	\subsection{Rischi Interpersonali}
	\subsection{Rischi legati al lavoro a distanza}