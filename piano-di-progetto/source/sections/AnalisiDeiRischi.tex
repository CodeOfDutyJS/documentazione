\section{Analisi dei rischi}
	Nel corso dello sviluppo di un progetto è naturale incorrere in problemi, tuttavia è possibile evitarne alcuni attraverso l'analisi dei richi. Questa attività è stata effettuata attraverso l'analisi dei principali fattori di rischio. Per ogni fattore di rischio rilevato è stata utilizzata la medesima procedura di identificazionione e risoluzione:
	\begin{itemize}
		\item \emph{Individuazione}: Individuazione dei fattori di rischio che rallenterebbero o impedirebbero il proseguimento del progetto
		\item \emph{Analisi}: Attività di studio dei fattori di rischio, ad ognuno di essi è stata assegnata una probabilità che si verifichi e un indice di gravità, basato sull'impatto che potrebbe avere sul progetto
		\item \emph{Pianificazione di Controllo}: Studio di un metodologia per evitare il verificarsi di un determinato fattore di rischio e un piano di risoluzione nel caso essa si verifichi
		\item \emph{Monitoraggio}: I fattori di rischio vanno tenuti sotto constante controllo, cercando di evitare che si verifichino se possibile oppure agire tempestivamente per minimizzare i danni
	\end{itemize}
	Ogni fattore di rischio rilevato è stato definito e raggruppato secondo varie tipologie:
	\begin{itemize}
		\item \emph{RT}: Rischio Tecnologici
		\item \emph{RO}: Rischi Organizzativi
		\item \emph{RI}: Rischi Interpersonali
	\subsection{Rischi Tecnologici}
	\subsection{Rischi Organizzativi}
	\subsection{Rischi Interpersonali}
	\subsection{Rischi legati al lavoro a distanza}