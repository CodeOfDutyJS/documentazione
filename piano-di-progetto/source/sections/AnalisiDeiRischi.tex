\section{Analisi dei rischi}
	Nel corso dello sviluppo di un progetto è naturale incorrere in problemi, tuttavia è possibile evitarne alcuni attraverso l'analisi dei richi. Questa attività è stata effettuata attraverso l'analisi dei principali fattori di rischio. Per ogni fattore di rischio rilevato è stata utilizzata la medesima procedura di identificazionione e risoluzione:
	\begin{enumerate}
		\item \textbf{Individuazione}: Individuazione dei fattori di rischio che rallenterebbero o impedirebbero il proseguimento del progetto
		\item \textbf{Analisi}: Attività di studio dei fattori di rischio, ad ognuno di essi è stata assegnata una probabilità che si verifichi e un indice di gravità, basato sull'impatto che potrebbe avere sul progetto
		\item \textbf{Pianificazione di Controllo}: Studio di un metodologia per evitare il verificarsi di un determinato fattore di rischio e un piano di risoluzione nel caso essa si verifichi
		\item \textbf{Monitoraggio}: I fattori di rischio vanno tenuti sotto constante controllo, cercando di evitare che si verifichino se possibile oppure agire tempestivamente per minimizzare i danni
	\end{enumerate}
	Ogni fattore di rischio rilevato è stato definito e raggruppato secondo varie tipologie:
	\begin{itemize}
		\item \textbf{RT}: Rischio Tecnologici
		\item \textbf{RO}: Rischi Organizzativi
		\item \textbf{RI}: Rischi Interpersonali
	\end{itemize}
	\subsection{Rischi Tecnologici}
	\def\productquality{
    		{
        		Inesperienza
			Tecnologica
			RT1,
        		La maggior parte delle tecnologie richieste nello sviluppo del progetto sono nuove per molti componenti del team, 
        		Il responsabile dovrà rilevare conoscenze e lacune dei vari componenti del team provando ad indicare una via ottimale per risolvere la mancanza di conoscenze,
        		Occorrenza: Alta 
			Pericolosità: Alta,
        		I compiti più onerasi verranno assegnati a più persone favorendo l'assistenza reciproca
    		},
	}
	\newcommand*\metricstable{}
\foreach \x [count=\nj] in \productquality
{
    \foreach \y [count=\ni] in \x
    {
        \ifnum\ni=4
            \xappto\metricstable{\y}
            \gappto\metricstable{\\}
            \gappto\metricstable{\hline}
        \else\ifnum\ni=5
            \xappto\metricstable{\noexpand\multicolumn{4}{| l |}{NOTE: \y}}
            \gappto\metricstable{\\}
            \gappto\metricstable{\hline}
        \else
            \xappto\metricstable{\y&}
        \fi\fi
    }
}

% Impostazioni della tabella
\tabulinesep = 2mm % padding
\taburowcolors [1] 2{pari .. dispari} % colori delle righe
\begin{longtabu} to \textwidth {| X[0.7,c m] | X[0.7,c m] | X[0.4,c m] | X[0.4,c m]|} % larghezza delle colonne
\hline
\rowcolor{header} % colore dell'header

\textbf{Nome} & \textbf{Formula} & \textbf{Valore sufficiente} & \textbf{Valore ottimo}\\
\hline
\metricstable

\end{longtabu}

\undef\metricstable{}

	\subsection{Rischi Organizzativi}

	\def\productquality{
    		{
        		Calcolo
			Tempistiche e Costi
			RO1,
        		Causa RT1 le valutazioni sulle tempistiche e i costi economici potrebbero essere imprecisi, 
        		Vengono predisposte delle tabelle sulle tempistiche e sarà compito del responsabile monitorare l'andamento dello sviluppo,
        		Occorrenza: Alta 
			Pericolosità: Alta,
        		AAAAAAAAAAAAAAAAAAAAAAAAAAAAAAAAAAAAAAAAAAAAAAAAAAAAAAAAAAAAAAAAAAAAAAAAAAAAAAAAAAAAAAAAAAAAAAAAAAAAAAAAAAAAAAAAAAAAA
    		},
	}
	\newcommand*\metricstable{}
\foreach \x [count=\nj] in \productquality
{
    \foreach \y [count=\ni] in \x
    {
        \ifnum\ni=4
            \xappto\metricstable{\y}
            \gappto\metricstable{\\}
            \gappto\metricstable{\hline}
        \else\ifnum\ni=5
            \xappto\metricstable{\noexpand\multicolumn{4}{| l |}{NOTE: \y}}
            \gappto\metricstable{\\}
            \gappto\metricstable{\hline}
        \else
            \xappto\metricstable{\y&}
        \fi\fi
    }
}

% Impostazioni della tabella
\tabulinesep = 2mm % padding
\taburowcolors [1] 2{pari .. dispari} % colori delle righe
\begin{longtabu} to \textwidth {| X[0.7,c m] | X[0.7,c m] | X[0.4,c m] | X[0.4,c m]|} % larghezza delle colonne
\hline
\rowcolor{header} % colore dell'header

\textbf{Nome} & \textbf{Formula} & \textbf{Valore sufficiente} & \textbf{Valore ottimo}\\
\hline
\metricstable

\end{longtabu}

\undef\metricstable{}

	\def\productquality{
    		{
        		Calcolo
			Tempistiche e Costi
			RO1,
        		Causa RT1 le valutazioni sulle tempistiche e i costi economici potrebbero essere imprecisi, 
        		Vengono predisposte delle tabelle sulle tempistiche e sarà compito del responsabile monitorare l'andamento dello sviluppo,
        		Occorrenza: Alta 
			Pericolosità: Alta,
        		Eventuali scostamenti dalle tempistiche prenostichate verranno comunicate al committente
    		},
	}
	\newcommand*\metricstable{}
\foreach \x [count=\nj] in \productquality
{
    \foreach \y [count=\ni] in \x
    {
        \ifnum\ni=4
            \xappto\metricstable{\y}
            \gappto\metricstable{\\}
            \gappto\metricstable{\hline}
        \else\ifnum\ni=5
            \xappto\metricstable{\noexpand\multicolumn{4}{| l |}{NOTE: \y}}
            \gappto\metricstable{\\}
            \gappto\metricstable{\hline}
        \else
            \xappto\metricstable{\y&}
        \fi\fi
    }
}

% Impostazioni della tabella
\tabulinesep = 2mm % padding
\taburowcolors [1] 2{pari .. dispari} % colori delle righe
\begin{longtabu} to \textwidth {| X[0.7,c m] | X[0.7,c m] | X[0.4,c m] | X[0.4,c m]|} % larghezza delle colonne
\hline
\rowcolor{header} % colore dell'header

\textbf{Nome} & \textbf{Formula} & \textbf{Valore sufficiente} & \textbf{Valore ottimo}\\
\hline
\metricstable

\end{longtabu}

\undef\metricstable{}

		%{
        	%	Impegni
		%	Accademici
		%	RO2,
        	%	Il periodo di sviluppo del progetto inizia a ridosso della sessione d'esami universitaria a cui tutti i membri del gruppo si vedono impegnati in diverse occasioni, 
        	%	In sede di riunione il team ha deciso di condividere i periodi di tempo in cui il loro contributo al progetto potrebbe calare o venir meno,
        	%	Occorrenza: Media 
		%	Pericolosità: Media,
        	%	La suddivisione del lavoro e le scadenze verranno stabilite nel rispetto dei diversi impegni accademici dei membri del gruppo
    		%},
		%{
        	%	Impegni
		%	Personali
		%	RO3,
        	%	è possibile il verificarsi di imprevisti personali che potrebbero influire nel corretto sviluppo del progetto, 
        	%	è compito del di ogni membro del gruppo segnalare un eventuale imprevisto al responsabile del gruppo in modo da permettergli di riorganizzare l'agenda,
        	%	Occorrenza: Bassa 
		%	Pericolosità: Bassa,
        	%	è compito del responsabile del gruppo riallocare risorse al fine di evitare un estensione nelle tempistiche del progetto
    		%},
	%}
	%\newcommand*\metricstable{}
\foreach \x [count=\nj] in \productquality
{
    \foreach \y [count=\ni] in \x
    {
        \ifnum\ni=4
            \xappto\metricstable{\y}
            \gappto\metricstable{\\}
            \gappto\metricstable{\hline}
        \else\ifnum\ni=5
            \xappto\metricstable{\noexpand\multicolumn{4}{| l |}{NOTE: \y}}
            \gappto\metricstable{\\}
            \gappto\metricstable{\hline}
        \else
            \xappto\metricstable{\y&}
        \fi\fi
    }
}

% Impostazioni della tabella
\tabulinesep = 2mm % padding
\taburowcolors [1] 2{pari .. dispari} % colori delle righe
\begin{longtabu} to \textwidth {| X[0.7,c m] | X[0.7,c m] | X[0.4,c m] | X[0.4,c m]|} % larghezza delle colonne
\hline
\rowcolor{header} % colore dell'header

\textbf{Nome} & \textbf{Formula} & \textbf{Valore sufficiente} & \textbf{Valore ottimo}\\
\hline
\metricstable

\end{longtabu}

\undef\metricstable{}
	\subsection{Rischi Interpersonali}
		%\def\productquality{
		%{
        	%	Irreperibilita
		%	momentanea,
        	%	Potrebbero verificarsi momenti in cui uno o più membri del team siano irreperibili, 
        	%	è responsabilità di ogni membro del gruppo comunicare eventuali imprevisti e organizzarsi in modo da non ostacolare il calendario delle consegne,
        	%	Occorrenza: Bassa 
		%	Pericolosità: Media,
        	%	Abbiamo a disposizione diversi mezzi di comunicazione dove comunicare eventuali problematiche
    		%},
		%{
        	%	Contrasti 
		%	interni,
        	%	Potrebbero verificarsi divergenze tra i membri del gruppo, 
        	%	Ciascun dei membri del team si impegna ad agire al fine di non ostacolare il naturale svolgimento del progetto e discutere di eventuali problemi solo in seduta di riunione,
        	%	Occorrenza: Bassa 
		%	Pericolosità: Media,
        	%	Il responsabile avrà il compito di fare da mediatore in caso ci fossero contrasti tra i membri del gruppo
    		%},
	%}
	%\newcommand*\metricstable{}
\foreach \x [count=\nj] in \productquality
{
    \foreach \y [count=\ni] in \x
    {
        \ifnum\ni=4
            \xappto\metricstable{\y}
            \gappto\metricstable{\\}
            \gappto\metricstable{\hline}
        \else\ifnum\ni=5
            \xappto\metricstable{\noexpand\multicolumn{4}{| l |}{NOTE: \y}}
            \gappto\metricstable{\\}
            \gappto\metricstable{\hline}
        \else
            \xappto\metricstable{\y&}
        \fi\fi
    }
}

% Impostazioni della tabella
\tabulinesep = 2mm % padding
\taburowcolors [1] 2{pari .. dispari} % colori delle righe
\begin{longtabu} to \textwidth {| X[0.7,c m] | X[0.7,c m] | X[0.4,c m] | X[0.4,c m]|} % larghezza delle colonne
\hline
\rowcolor{header} % colore dell'header

\textbf{Nome} & \textbf{Formula} & \textbf{Valore sufficiente} & \textbf{Valore ottimo}\\
\hline
\metricstable

\end{longtabu}

\undef\metricstable{}
	\subsection{Rischi legati al lavoro a distanza}