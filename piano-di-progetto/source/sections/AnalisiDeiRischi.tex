\section{Analisi dei rischi}
	Nel corso dello sviluppo di un progetto è naturale incorrere in problemi, tuttavia è possibile evitarne alcuni attraverso l'analisi dei rischi. \\ Questa attività è stata effettuata attraverso l'analisi dei principali fattori di rischio, per ogni fattore rilevato è stata utilizzata la medesima procedura di identificazionione e risoluzione:
	\begin{enumerate}
		\item \textbf{Individuazione}: Individuazione dei fattori di rischio che rallenterebbero o impedirebbero il proseguimento del progetto;
		\item \textbf{Analisi}: Attività di studio dei fattori di rischio, ad ognuno di essi è stata assegnata una probabilità che si verifichi e un indice di gravità, basato sull'impatto che potrebbe avere sul progetto;
		\item \textbf{Pianificazione di Controllo}: Studio di un metodologia per evitare il verificarsi di un determinato fattore di rischio e un piano di risoluzione nel caso essa si verifichi;
		\item \textbf{Monitoraggio}: I fattori di rischio vanno tenuti sotto constante controllo, cercando di evitare che si verifichino se possibile oppure agire tempestivamente per minimizzarne i danni;
	\end{enumerate}
	Ogni fattore di rischio rilevato è stato definito e raggruppato secondo varie tipologie:
	\begin{itemize}
		\item \textbf{RT}: Rischio Tecnologici;
		\item \textbf{RO}: Rischi Organizzativi;
		\item \textbf{RI}: Rischi Interpersonali;
	\end{itemize}
	\subsection{Rischi Tecnologici}
	\def\productquality{
    		{
        		Inesperienza
			Tecnologica
			RT1,
        		La maggior parte delle tecnologie richieste nello sviluppo del progetto sono nuove per molti componenti del team, 
        		Il responsabile dovrà rilevare conoscenze e lacune dei vari componenti del team provando ad indicare una via ottimale per risolvere la mancanza di conoscenze,
        		Occorrenza: Alta 
			Pericolosità: Alta
    		},
	}
	\newcommand*\metricstable{}
\foreach \x [count=\nj] in \productquality
{
    \foreach \y [count=\ni] in \x
    {
        \ifnum\ni=4
            \xappto\metricstable{\y}
            \gappto\metricstable{\\}
            \gappto\metricstable{\hline}
        \else\ifnum\ni=5
            \xappto\metricstable{\noexpand\multicolumn{4}{| l |}{NOTE: \y}}
            \gappto\metricstable{\\}
            \gappto\metricstable{\hline}
        \else
            \xappto\metricstable{\y&}
        \fi\fi
    }
}

% Impostazioni della tabella
\tabulinesep = 2mm % padding
\taburowcolors [1] 2{pari .. dispari} % colori delle righe
\begin{longtabu} to \textwidth {| X[0.7,c m] | X[0.7,c m] | X[0.4,c m] | X[0.4,c m]|} % larghezza delle colonne
\hline
\rowcolor{header} % colore dell'header

\textbf{Nome} & \textbf{Formula} & \textbf{Valore sufficiente} & \textbf{Valore ottimo}\\
\hline
\metricstable

\end{longtabu}

\undef\metricstable{}
	\textbf{Piano di contingenza} : I compiti più onerosi verrano assegnati a più persone, questo favorirà l'assistenza reciproca e velocizzerà l'esecuzione delle task.
	\pagebreak
	\subsection{Rischi Organizzativi}

		\def\productquality{
    			{
        			Calcolo
				Tempistiche e Costi
				RO1,
        			Causa RT1 le valutazioni sulle tempistiche e i costi economici potrebbero essere imprecisi, 
        			Vengono predisposte delle tabelle sulle tempistiche e sarà compito del responsabile monitorare l'andamento dello sviluppo,
        			Occorrenza: Alta 
				Pericolosità: Alta
    			},
		}
		\newcommand*\metricstable{}
\foreach \x [count=\nj] in \productquality
{
    \foreach \y [count=\ni] in \x
    {
        \ifnum\ni=4
            \xappto\metricstable{\y}
            \gappto\metricstable{\\}
            \gappto\metricstable{\hline}
        \else\ifnum\ni=5
            \xappto\metricstable{\noexpand\multicolumn{4}{| l |}{NOTE: \y}}
            \gappto\metricstable{\\}
            \gappto\metricstable{\hline}
        \else
            \xappto\metricstable{\y&}
        \fi\fi
    }
}

% Impostazioni della tabella
\tabulinesep = 2mm % padding
\taburowcolors [1] 2{pari .. dispari} % colori delle righe
\begin{longtabu} to \textwidth {| X[0.7,c m] | X[0.7,c m] | X[0.4,c m] | X[0.4,c m]|} % larghezza delle colonne
\hline
\rowcolor{header} % colore dell'header

\textbf{Nome} & \textbf{Formula} & \textbf{Valore sufficiente} & \textbf{Valore ottimo}\\
\hline
\metricstable

\end{longtabu}

\undef\metricstable{}
		\textbf{Piano di contingenza}: Il responsabile in accordo con il task owner provvederà all'assegnazione di maggiori risorse per evitare un prolungarsi delle tempistiche.
		\def\productquality{
			{
        			Impegni
				Accademici
				RO2,
        			Il periodo di sviluppo del progetto inizia a ridosso della sessione d'esami universitaria a cui tutti i membri del gruppo si vedono impegnati in diverse occasioni, 
        			In sede di riunione il team ha deciso di condividere i periodi di tempo in cui il loro contributo al progetto potrebbe calare o venir meno,
        			Occorrenza: Media 
				Pericolosità: Media
    			},
		}
		\newcommand*\metricstable{}
\foreach \x [count=\nj] in \productquality
{
    \foreach \y [count=\ni] in \x
    {
        \ifnum\ni=4
            \xappto\metricstable{\y}
            \gappto\metricstable{\\}
            \gappto\metricstable{\hline}
        \else\ifnum\ni=5
            \xappto\metricstable{\noexpand\multicolumn{4}{| l |}{NOTE: \y}}
            \gappto\metricstable{\\}
            \gappto\metricstable{\hline}
        \else
            \xappto\metricstable{\y&}
        \fi\fi
    }
}

% Impostazioni della tabella
\tabulinesep = 2mm % padding
\taburowcolors [1] 2{pari .. dispari} % colori delle righe
\begin{longtabu} to \textwidth {| X[0.7,c m] | X[0.7,c m] | X[0.4,c m] | X[0.4,c m]|} % larghezza delle colonne
\hline
\rowcolor{header} % colore dell'header

\textbf{Nome} & \textbf{Formula} & \textbf{Valore sufficiente} & \textbf{Valore ottimo}\\
\hline
\metricstable

\end{longtabu}

\undef\metricstable{}
		\textbf{Piano di contingenza}: Le task verranno assegnate nel rispetto degli altri impegni accademici.
		\pagebreak
		\def\productquality{
			{
        			Impegni
				Personali
				RO3,
        			è possibile il verificarsi di imprevisti personali che potrebbero influire nel corretto sviluppo del progetto, 
        			è compito di ogni membro del gruppo segnalare un eventuale imprevisto al responsabile del gruppo in modo da permettergli di riorganizzare l'agenda,
        			Occorrenza: Bassa 
				Pericolosità: Bassa
    			},
		}
		\newcommand*\metricstable{}
\foreach \x [count=\nj] in \productquality
{
    \foreach \y [count=\ni] in \x
    {
        \ifnum\ni=4
            \xappto\metricstable{\y}
            \gappto\metricstable{\\}
            \gappto\metricstable{\hline}
        \else\ifnum\ni=5
            \xappto\metricstable{\noexpand\multicolumn{4}{| l |}{NOTE: \y}}
            \gappto\metricstable{\\}
            \gappto\metricstable{\hline}
        \else
            \xappto\metricstable{\y&}
        \fi\fi
    }
}

% Impostazioni della tabella
\tabulinesep = 2mm % padding
\taburowcolors [1] 2{pari .. dispari} % colori delle righe
\begin{longtabu} to \textwidth {| X[0.7,c m] | X[0.7,c m] | X[0.4,c m] | X[0.4,c m]|} % larghezza delle colonne
\hline
\rowcolor{header} % colore dell'header

\textbf{Nome} & \textbf{Formula} & \textbf{Valore sufficiente} & \textbf{Valore ottimo}\\
\hline
\metricstable

\end{longtabu}

\undef\metricstable{}
		\textbf{Piano di contingenza}: Il responsabile provvederà a riallocare delle risorse cercando di limitare i danni di una possibile temporanea mancanza del task owner.
		
		\def\productquality{
			{
        			Rischi legati al lavoro a distanza
				RO4,
        			La pandemia covid ci impedisce il lavoro in presenza esponendoci ai rischi del lavoro a distanza, 
        			Malfunzionamenti hardware o instabilità della connessione ad internet non sono fattori di rischio controllabili o pronosticabili,
        			Occorrenza: Bassa 
				Pericolosità: Media
    			},
		}
		\newcommand*\metricstable{}
\foreach \x [count=\nj] in \productquality
{
    \foreach \y [count=\ni] in \x
    {
        \ifnum\ni=4
            \xappto\metricstable{\y}
            \gappto\metricstable{\\}
            \gappto\metricstable{\hline}
        \else\ifnum\ni=5
            \xappto\metricstable{\noexpand\multicolumn{4}{| l |}{NOTE: \y}}
            \gappto\metricstable{\\}
            \gappto\metricstable{\hline}
        \else
            \xappto\metricstable{\y&}
        \fi\fi
    }
}

% Impostazioni della tabella
\tabulinesep = 2mm % padding
\taburowcolors [1] 2{pari .. dispari} % colori delle righe
\begin{longtabu} to \textwidth {| X[0.7,c m] | X[0.7,c m] | X[0.4,c m] | X[0.4,c m]|} % larghezza delle colonne
\hline
\rowcolor{header} % colore dell'header

\textbf{Nome} & \textbf{Formula} & \textbf{Valore sufficiente} & \textbf{Valore ottimo}\\
\hline
\metricstable

\end{longtabu}

\undef\metricstable{}
		\textbf{Piano di contingenza}: Ogni membro del team proverà a sopperire per quanto possibile ai diversi rischi che possono incorrere nel lavoro a distanza



	\subsection{Rischi Interpersonali}
		\def\productquality{
			{
        			Irreperibilita
				momentanea
				RI1,
        			Potrebbero verificarsi momenti in cui uno o più membri del team siano irreperibili, 
        			è responsabilità di ogni membro del gruppo comunicare eventuali imprevisti e organizzarsi in modo da non ostacolare il calendario delle consegne,
        			Occorrenza: Bassa 
				Pericolosità: Media
    			},
		}
		\newcommand*\metricstable{}
\foreach \x [count=\nj] in \productquality
{
    \foreach \y [count=\ni] in \x
    {
        \ifnum\ni=4
            \xappto\metricstable{\y}
            \gappto\metricstable{\\}
            \gappto\metricstable{\hline}
        \else\ifnum\ni=5
            \xappto\metricstable{\noexpand\multicolumn{4}{| l |}{NOTE: \y}}
            \gappto\metricstable{\\}
            \gappto\metricstable{\hline}
        \else
            \xappto\metricstable{\y&}
        \fi\fi
    }
}

% Impostazioni della tabella
\tabulinesep = 2mm % padding
\taburowcolors [1] 2{pari .. dispari} % colori delle righe
\begin{longtabu} to \textwidth {| X[0.7,c m] | X[0.7,c m] | X[0.4,c m] | X[0.4,c m]|} % larghezza delle colonne
\hline
\rowcolor{header} % colore dell'header

\textbf{Nome} & \textbf{Formula} & \textbf{Valore sufficiente} & \textbf{Valore ottimo}\\
\hline
\metricstable

\end{longtabu}

\undef\metricstable{}
		\textbf{Piano di contingenza}: Il gruppo si è organizzato in modo da avere molteplici vie di comunicazione, rendendo l'irreperibilità momentanea quasi impossibile.
		\def\productquality{
			{
        			Contrasti 
				interni
				RI2,
        			Potrebbero verificarsi divergenze tra i membri del gruppo, 
        			Ciascuno dei membri del team si impegna ad agire al fine di non ostacolare il naturale svolgimento del progetto e discutere di eventuali problemi solo in seduta di riunione,
        			Occorrenza: Bassa 
				Pericolosità: Media
    			},
		}
		\newcommand*\metricstable{}
\foreach \x [count=\nj] in \productquality
{
    \foreach \y [count=\ni] in \x
    {
        \ifnum\ni=4
            \xappto\metricstable{\y}
            \gappto\metricstable{\\}
            \gappto\metricstable{\hline}
        \else\ifnum\ni=5
            \xappto\metricstable{\noexpand\multicolumn{4}{| l |}{NOTE: \y}}
            \gappto\metricstable{\\}
            \gappto\metricstable{\hline}
        \else
            \xappto\metricstable{\y&}
        \fi\fi
    }
}

% Impostazioni della tabella
\tabulinesep = 2mm % padding
\taburowcolors [1] 2{pari .. dispari} % colori delle righe
\begin{longtabu} to \textwidth {| X[0.7,c m] | X[0.7,c m] | X[0.4,c m] | X[0.4,c m]|} % larghezza delle colonne
\hline
\rowcolor{header} % colore dell'header

\textbf{Nome} & \textbf{Formula} & \textbf{Valore sufficiente} & \textbf{Valore ottimo}\\
\hline
\metricstable

\end{longtabu}

\undef\metricstable{}
		\textbf{Piano di contingenza}: Il responsabile avrà il compito di mediare nel caso si verifichino contrasti interni tra i membri del gruppo.
