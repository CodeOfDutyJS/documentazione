\section{Introduzione}
\subsection{Scopo del documento}
	In questo documento vengono illustrate le modilità con cui il gruppo \emph{Code of Duty} affronterà lo sviluppo del progetto \hd, quali:
	\begin{itemize}
    		\item Analisi dei rischi;
    		\item Presentazione del modello di sviluppo adottato;
    		\item Pianificazione delle attività e divisione dei ruoli;
    		\item Stima dei costi economici e risorse necessarie al compimento del progetto.
	\end{itemize}
\subsection{Scopo del prodotto}
	L'obbiettivo del progetto è quello di creare un'applicazione di visualizzazione di dati con molte dimensioni. L'applicazione dovra permettere diverse visualizzazioni dei dati tramite browser ed è richiesta anche una parte server di supporto che permetta il prelevamento dei dati e la loro elaborazione.
\subsection{Glossario}
	In questo documento vengono definiti e descritti tutti i termini e tecnologie al fine di evitare ambiguità relative al linguaggio utilizzato nei documenti formali. Per facilitare la lettura i termini saranno contrassegnati da una 'G' a pedice.  
\subsection{Riferimenti}
	\subsubsection{Normativi}
		\begin{itemize}
			\item \textbf{Norme di Progetto}: Norme di progetto
			\item \textbf{Regolamento organigramma e specifica tecnico-economica} : \href{https://www.math.unipd.it/~tullio/IS-1/2020/Progetto/RO.html}{RO}
		\end{itemize}
	\subsubsection{Informativi}
		\begin{itemize}
			\item \textbf{Capitolato 4 : HD Viz}: \href{https://www.math.unipd.it/~tullio/IS-1/2020/Progetto/C4.pdf}{HD Viz - Visualizzazione di dati con molte dimensioni}
			\item \textbf{Il ciclo di vita del Software}: \href{https://www.math.unipd.it/~tullio/IS-1/2020/Dispense/L05.pdf}{L05}
			\item \textbf{Gestione di progetto}: \href{https://www.math.unipd.it/~tullio/IS-1/2020/Dispense/L06.pdf}{L06}
		\end{itemize}
\subsection{Scadenze}
	Il gruppo CODEofDuty ha deciso di impegnarsi a rispettare le seguenti scadenze per lo sviluppo del progetto \emph{HDViz}
	\begin{itemize}
		\item \textbf{Revisione dei Requisiti}: 11-Gennaio-2021;
		\item \textbf{Revisione di Progettazione}: 1-Marzo-2021;
		\item \textbf{Revisione di Qualifica}: 2-Aprile-2021;
		\item \textbf{Revisione di Accettazione}: 3-Maggio-2021;
	\end{itemize}
