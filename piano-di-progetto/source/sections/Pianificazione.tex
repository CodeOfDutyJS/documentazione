\section{Pianificazione}
	Alla luce delle scadenze riportate nella sezione 1.5 e alle scadenze interne si è deciso di dividere il progetto in cinque fasi:
	\begin{itemize}
		\item \textbf{Analisi};
		\item \textbf{Consolidamento dei requisiti};
		\item \textbf{Progettazione architetturale};
		\item \textbf{Progettazione di dettaglio e codifica};
		\item \textbf{Validazione e collaudo}.
	\end{itemize}
	Ognuna di queste cinque fasi sarà formata da diverse sottoattività mostrate nel corrispettivo diagramma di Gantt.

	\subsection{Analisi}
	\textbf{Periodo}: dal 26-11-2021 al 10-01-2021 \\
	La fase di Analisi è composta da sei attività che corrispondono alla produzione dei relativi documenti:
	\begin{itemize}
		\item \textbf{Studio di fattibilità}: Ogni capitolato proposto viene analizzato e discusso con i membri del gruppo. Ogni capitolato viene poi classificato in base alle preferenze e ai punti di iteresse riscontrati. Il capitolato che avrà riscosso più preferenze viene scelto come capitolato effetivo;
		\item \textbf{Norme di progetto}: Vengono definite le regole che il team dovrà rispettare durante lo sviluppo del progetto;
		\item \textbf{Piano di Progetto}: Si analizzato le attività, i compiti e le risorse che verranno distribuite tra i membri del team, presenta il calcolo del preventivo per la realizzazione del progetto;
		\item \textbf{Analisi dei requisiti}: Vengono studiati e analizzati i requisiti del capitolato scelto ( \textbf{C4} ) nello studio di fattibilità;
		\item \textbf{Piano di qualifica}: Durante questa attività si individuano le metodologie da usare per garantire la qualità del prodotto.
		\item \textbf{Glossario}: Questo documento viene redatto per definire la terminologia usata al fine di evitare ambiguità.
	\end{itemize}