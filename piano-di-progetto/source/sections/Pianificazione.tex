\section{Pianificazione}
	Alla luce delle scadenze riportate nella sezione 1.5 e alle scadenze interne si è deciso di dividere il progetto in cinque fasi:
	\begin{itemize}
		\item \textbf{Analisi};
		\item \textbf{Consolidamento dei requisiti};
		\item \textbf{Progettazione architetturale};
		\item \textbf{Progettazione di dettaglio e codifica};
		\item \textbf{Validazione e collaudo}.
	\end{itemize}
	Ognuna di queste cinque fasi sarà formata da diverse sottoattività mostrate nel corrispettivo diagramma di Gantt.

	\subsection{Analisi}
	\textbf{Periodo}: dal 26-11-2020 al 10-01-2021 \\
	La fase di Analisi è composta da sei attività che corrispondono alla produzione dei relativi documenti:
	\begin{itemize}
		\item \textbf{Studio di fattibilità}: Ogni capitolato proposto viene analizzato e discusso con i membri del gruppo. Ogni capitolato viene poi classificato in base alle preferenze e ai punti di iteresse riscontrati. Il capitolato che avrà riscosso più preferenze viene scelto come capitolato effetivo;
		\item \textbf{Norme di progetto}: Vengono definite le regole che il team dovrà rispettare durante lo sviluppo del progetto;
		\item \textbf{Piano di Progetto}: Si analizzato le attività, i compiti e le risorse che verranno distribuite tra i membri del team, presenta il calcolo del preventivo per la realizzazione del progetto;
		\item \textbf{Analisi dei requisiti}: Vengono studiati e analizzati i requisiti del capitolato scelto ( \textbf{C4} ) nello studio di fattibilità;
		\item \textbf{Piano di qualifica}: Durante questa attività si individuano le metodologie da usare per garantire la qualità del prodotto;
		\item \textbf{Glossario}: Questo documento viene redatto per definire la terminologia usata al fine di evitare ambiguità;
	\end{itemize}
	\subsubsection{Diagramma}
		\begin{figure}[H]
        		\centering
        		\includegraphics[width=0.8\textwidth]{source/img/analisiattivita.png}
        		\caption{Diagramma di Gantt dell\textquotesingle attività di analisi}
    		\end{figure}

	\subsection{Consolidamento dei requisiti}
	\textbf{Periodo}: dal 12-01-2021 al 18-01-2021 \\
	La fase di consolidamento dei requisiti è intermedia tra la fine della'attività di Analisi e il giorno della presentazione della Revisione dei Requisiti. \\
	Durante la seguente fase si svolgeranno attività di consolidamento dei requisiti osservati nella fase precedente e verrà preparata la presentazione per la Revisione dei Requisiti.
	
	\subsubsection{Diagramma}
		\begin{figure}[H]
        		\centering
        		\includegraphics[width=0.8\textwidth]{source/img/Consolidamento_Requisiti.png}
        		\caption{Diagramma di Gantt dell\textquotesingle attività di consolidamento requisiti}
    		\end{figure}
	
	\subsection{Progettazione architetturale}
	\textbf{Periodo}: dal 19-01-2021 al 01-03-2021 \\
	Successivamente alla presentazione della Revisione dei Requisiti inizia la fase di progettazione architetturare che terminera con la consegna della Revisione di Progettazione, lo scopo di questo periodo è l'individuazione di una soluzione architetturale che soddisfi i seguenti requisiti:
	\begin{itemize}
		\item \textbf{Technology Baseline}: verrà stilato l'allegato tecnico nel quale verranno individuati i design pattern che verranno adottati nello sviluppo del progetto. Inoltre verrà redatto il Proof Of Concept da consegnare al committente;
		\item \textbf{Incremento e verifica}: Se necessarrio vengono migliorati i documenti consegnati durante la fase iniziale.
	\end{itemize}
	
	\subsubsection{Diagramma}
		\begin{figure}[H]
        		\centering
        		\includegraphics[width=0.8\textwidth]{source/img/Progettazione_architetturale.png}
        		\caption{Diagramma di Gantt dell\textquotesingle attività di progettazione architetturale}
    		\end{figure}
	\subsubsection{Tabella Incrementi}
		\def\tabellaincrementi{
    			{
        			Incremento 1,
        			dal 19-01-2021 al 26-01-2021,
        			Ricevuto il feedback del committente i prodotti consegnati nella fase precedente verranno aggiornati e corretti per migliorare la qualità del prodotto
    			},
		}
		\newcommand*\risktable{}
\foreach \x [count=\nj] in \tabellaincrementi
{
    \foreach \y [count=\ni] in \x
    {
        \ifnum\ni=3
            \xappto\metricstable{\y}
            \gappto\metricstable{\\}
            \gappto\metricstable{\hline}
        \else
            \xappto\metricstable{\y & }
        \fi
    }
}
% Impostazioni della tabella
\tabulinesep = 2mm % padding
\taburowcolors [1] 2{pari .. dispari} % colori delle righe
\begin{longtabu} to \textwidth {| X[0.7,c m] | X[0.7,c m] | X[0.4,c m]|} % larghezza delle colonne
\hline
\rowcolor{header} % colore dell'header
\textbf{Nome Incremento} & \textbf{Periodo} & \textbf{Descrizione} \\
\hline
\tabellaincrementi
\end{longtabu}
\undef\risktable{}
	
	\subsection{Progettazione di dettaglio e codifica}
	\textbf{Periodo}: dal 02-03-2021 al 02-04-2021 \\
	Col termine della Revisione di Progettazione, inizia la fase di progettazione di dettaglio e codifica e terminerà con la Revisione di Qualifica, la seguente fase è composta da nuovi incrementi e attività:
	\begin{itemize}
		\item \textbf{Incremento e verifica}: Se necessario verranno migliorati i documenti consegnati durante la fase precedente;
		\item \textbf{Product Baseline}: Vengono studiati i design pattern, classi e attività necessarie alla codifica;
		\item \textbf{Specifica Tecnica}: Realizzazione di un documento contenente le specifiche del prodotto motivandone la scelta;
		\item \textbf{Codifica}
		\item \textbf{Manuale Utente}: Redazione del manuale utente del prodotto.
	\end{itemize}
	
	\subsubsection{Diagramma}
		\begin{figure}[H]
        		\centering
        		\includegraphics[width=0.8\textwidth]{source/img/Progettazionedettaglio_codifica.png}
        		\caption{Diagramma di Gantt dell\textquotesingle attività di progettazione di dettaglio e codifica}
    		\end{figure}

	\subsection{Validazione e Collaudo}
	\textbf{Periodo}: dal 03-03-2021 al 03-04-2021 \\
	Ultima fase del progetto che terminerà con la Revisione di Accettazione, la seguente fase è composta da incrementi e nuove attività da svolgere:
	\begin{itemize}
		\item \textbf{Incremento e verifica}: Se necessario verranno migliorati i documenti consegnati durante la fase precedente;
		\item \textbf{Validazione e Collaudo}: Si eseguono i test finali di collaudo dell'applicazione nella sua interezza.
	\end{itemize}
	
	\subsubsection{Diagramma}
		\begin{figure}[H]
        		\centering
        		\includegraphics[width=0.8\textwidth]{source/img/Validazione_collaudo.png}
        		\caption{Diagramma di Gantt dell\textquotesingle attività di validazione e collaudo}
    		\end{figure}