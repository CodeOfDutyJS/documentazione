\section{Attualizzazione dei rischi}
\label{section:rischi_rilevati}
In questo appendice sono raccolti i problemi verificati durante l'attività di progetto ed è riportata la soluzione attuata o la situazione del problema al momento della redazione di questo documento.
Ogni rischio rilevato ha associato un codice: \\
\textit{identificativoRischio-numeroIncrementale} e.g. RT1-0. \\

\subsection{Rischi tecnologici verificati}
%startTable
\def\problems{
    { RT1-0 },
    { \noexpand\textbf{Fase di progettazione architetturale} },
    {   
        Alcuni membri del gruppo non conoscono o non sono pratici dello strumento GitHub. Tali membri hanno dedicato allo strumento un breve periodo di autoformazione prima d'iniziare l'esperienza sul campo. I membri più esperti del team sono stati e resteranno sempre disponibili per chiarimenti.
    },
}
%endTable
\newcommand*\improvementeval{}
\foreach \x [count=\nj] in \problems
{
    \foreach \y [count=\ni] in \x
    {
        \ifnum\ni=1
            \xappto\improvementeval{\noexpand\textbf{\y}&}
        \else\ifnum\ni=4
            \xappto\improvementeval{\y}
            \gappto\improvementeval{\\}
            \gappto\improvementeval{\hline}
        \else
            \xappto\improvementeval{\y&}
        \fi\fi
    }
}

% Impostazioni della tabella
\tabulinesep = 2mm % padding
\taburowcolors [1] 2{pari .. dispari} % colori delle righe
\begin{longtabu} to \textwidth {| X[0.1,c m] | X[0.2,c m] | X[0.3,l m] | X[0.3,l m] |} % larghezza delle colonne
\hline
\rowcolor{header} % colore dell'header

\textbf{Codice rischio} & \textbf{Problema} & \textbf{Descrizione} & \textbf{Soluzione} \\
\hline
\improvementeval

\end{longtabu}

\undef\improvementeval{}

%startTable
\def\problems{
    { RT1-1 },
    { \noexpand\textbf{Fase di analisi} },
    {   
        L'esperienza con LaTeX da parte del team (nella sua interezza) è poca o nulla. La redazione dei documenti si è inizialmente rivelato un processo lento e frustrante{,} tuttavia è stato considerato un male necessario. Ancora adesso si riscontrano difficoltà rispetto alla creazione di tabelle e posizionamento di immagini.
    },
    { \noexpand\textbf{Fase di progettazione di dettaglio e codifica} },
    { 
        I documenti sono ben avviati è l'attività di manutenzione di questi è prevalentemente ripetitiva e richiede conoscenze già maturate in precedenza. 
    },
}
%endTable
\newcommand*\improvementeval{}
\foreach \x [count=\nj] in \problems
{
    \foreach \y [count=\ni] in \x
    {
        \ifnum\ni=1
            \xappto\improvementeval{\noexpand\textbf{\y}&}
        \else\ifnum\ni=4
            \xappto\improvementeval{\y}
            \gappto\improvementeval{\\}
            \gappto\improvementeval{\hline}
        \else
            \xappto\improvementeval{\y&}
        \fi\fi
    }
}

% Impostazioni della tabella
\tabulinesep = 2mm % padding
\taburowcolors [1] 2{pari .. dispari} % colori delle righe
\begin{longtabu} to \textwidth {| X[0.1,c m] | X[0.2,c m] | X[0.3,l m] | X[0.3,l m] |} % larghezza delle colonne
\hline
\rowcolor{header} % colore dell'header

\textbf{Codice rischio} & \textbf{Problema} & \textbf{Descrizione} & \textbf{Soluzione} \\
\hline
\improvementeval

\end{longtabu}

\undef\improvementeval{}

\def\problems {
    { RT1-2 },
    { \noexpand\textbf{Fase di progettazione architetturale } },
    {
        Nessun membro del gruppo è pratico o ha conoscenze approfondite delle tecnologie coinvolte per la realizzazione del POC. È stato dedicato un periodo di 2 settimane all'autoapprendimento delle tecnologie coinvolte{,} in particolare è stato deciso che ogni membro del gruppo deve avere per ognuna una conoscenza almeno superficiale.
    },
    { \noexpand\textbf{Fase di progettazione di dettaglio e codifica } },
    {
        Ogni membro del gruppo è specializzato in una particolare tecnologia{,} tuttavia queste non sono ancora completamente padroneggiate. Sebbene contenuto il rischio risulta parzialmente superato.
    },
    { \noexpand\textbf{Fase di validazione e collaudo} },
    { 
        Il rischio si è ripresentato per il testing{,} in particolare per il framework jest. Tuttavia in questa fase{,} a differenza delle precedenti progettazione e programmazione sono pressoché terminate per cui l'intero team si è specializzato nel framework e la curva di apprendimento è stata piuttosto veloce.
    },
}
%endTable
\newcommand*\improvementeval{}
\foreach \x [count=\nj] in \problems
{
    \foreach \y [count=\ni] in \x
    {
        \ifnum\ni=1
            \xappto\improvementeval{\noexpand\textbf{\y}&}
        \else\ifnum\ni=4
            \xappto\improvementeval{\y}
            \gappto\improvementeval{\\}
            \gappto\improvementeval{\hline}
        \else
            \xappto\improvementeval{\y&}
        \fi\fi
    }
}

% Impostazioni della tabella
\tabulinesep = 2mm % padding
\taburowcolors [1] 2{pari .. dispari} % colori delle righe
\begin{longtabu} to \textwidth {| X[0.1,c m] | X[0.2,c m] | X[0.3,l m] | X[0.3,l m] |} % larghezza delle colonne
\hline
\rowcolor{header} % colore dell'header

\textbf{Codice rischio} & \textbf{Problema} & \textbf{Descrizione} & \textbf{Soluzione} \\
\hline
\improvementeval

\end{longtabu}

\undef\improvementeval{}

\subsection{Rischi organizzativi verificati}
%startTable
\def\problems{
    { RO1-0 },
    { \noexpand\textbf{Fase di analisi} },
    {  
        Risulta complicato trovare un giorno e anche un orario per svolgere gli incontri. Grazie al supporto dei canali di comunicazione individuati dalle \noexpand\NdP\ e all'attività di verbalizzazione non è necessario che sia presente l'intero team ogni incontro. Per gli incontri più importanti{,} dove è necessario che sia presente tutto il gruppo{,} è necessario mettersi d'accordo con largo anticipo.
    },
    { \noexpand\textbf{Fase di progettazione architetturale} },
    { Il rischio non è stato tracciato durante questa fase. },
    { \noexpand\textbf{Fase di progettazione di dettaglio e codifica} },
    {
        Il team è più affiatato{,} gli incontri sono molto più frequenti anche se si sono trasformati in lavoro collaborativo{,} la maggior parte delle decisioni viene presa in modalità asincrona. Il rischio risulta superato.
    },
}
%endTable
\newcommand*\improvementeval{}
\foreach \x [count=\nj] in \problems
{
    \foreach \y [count=\ni] in \x
    {
        \ifnum\ni=1
            \xappto\improvementeval{\noexpand\textbf{\y}&}
        \else\ifnum\ni=4
            \xappto\improvementeval{\y}
            \gappto\improvementeval{\\}
            \gappto\improvementeval{\hline}
        \else
            \xappto\improvementeval{\y&}
        \fi\fi
    }
}

% Impostazioni della tabella
\tabulinesep = 2mm % padding
\taburowcolors [1] 2{pari .. dispari} % colori delle righe
\begin{longtabu} to \textwidth {| X[0.1,c m] | X[0.2,c m] | X[0.3,l m] | X[0.3,l m] |} % larghezza delle colonne
\hline
\rowcolor{header} % colore dell'header

\textbf{Codice rischio} & \textbf{Problema} & \textbf{Descrizione} & \textbf{Soluzione} \\
\hline
\improvementeval

\end{longtabu}

\undef\improvementeval{}

%startTable
\def\problems{
    { RO1-1 },
    { \noexpand\textbf{Fase di analisi} },
    {    
        Il ruolo di amministratore ha richiesto più ore rispetto a quanto preventivato{,} come riportato nel \noexpand\PdP {,} è stato dedicato molto tempo a impostare la repository e le GitHub Actions{,} con l'obiettivo di automatizzare il più possibile alcune operazioni. Non sono state considerate azioni di mitigazione.
        L'aver dedicato molte ore nell'impostare l'ambiente di sviluppo è sicuramente un investimento per il futuro{,} sia per quanto riguarda il prodotto del processo di documentazione che per il prodotto software{,} infatti impostare la repository dedicata al codice sarà sicuramente un processo più agevole.
    },
    { \noexpand\textbf{Fase di progettazione architetturale} },
    {
        La compilazione dei documenti a partire dalla sorgente \noexpand\LaTeX è automatizzata{,} l'aggiunta a pedice della \noexpand\textbf{G} è a sua volta automatizzata. L'aver dedicato più tempo all'impostazione della repository ha ripagato nel lungo termine. 
    },
}
%endTable
\newcommand*\improvementeval{}
\foreach \x [count=\nj] in \problems
{
    \foreach \y [count=\ni] in \x
    {
        \ifnum\ni=1
            \xappto\improvementeval{\noexpand\textbf{\y}&}
        \else\ifnum\ni=4
            \xappto\improvementeval{\y}
            \gappto\improvementeval{\\}
            \gappto\improvementeval{\hline}
        \else
            \xappto\improvementeval{\y&}
        \fi\fi
    }
}

% Impostazioni della tabella
\tabulinesep = 2mm % padding
\taburowcolors [1] 2{pari .. dispari} % colori delle righe
\begin{longtabu} to \textwidth {| X[0.1,c m] | X[0.2,c m] | X[0.3,l m] | X[0.3,l m] |} % larghezza delle colonne
\hline
\rowcolor{header} % colore dell'header

\textbf{Codice rischio} & \textbf{Problema} & \textbf{Descrizione} & \textbf{Soluzione} \\
\hline
\improvementeval

\end{longtabu}

\undef\improvementeval{}

%startTable
\def\problems{
    { RO1-2 },
    { \noexpand\textbf{Fase di progettazione architetturale} }, 
    {
        Causato da RT1-2{,} i programmatori hanno avuto bisogno di più tempo di quanto preventivato per utilizzare efficacemente le tecnologie richieste per la realizzazione del POC. È stato tolto del tempo alla progettazione per concentrarsi maggiormente alla programmazione.
    },
    { \noexpand\textbf{Fase di progettazione di dettaglio e codifica} }, 
    { 
        L'aver dedicato più tempo alla formazione e utilizzo delle tecnologie individuate ha permesso al team di procedere più speditamente durante questa fase{,} tuttavia ha costretto il team a ripetere la PB.  
    },
}
%endTable
\newcommand*\improvementeval{}
\foreach \x [count=\nj] in \problems
{
    \foreach \y [count=\ni] in \x
    {
        \ifnum\ni=1
            \xappto\improvementeval{\noexpand\textbf{\y}&}
        \else\ifnum\ni=4
            \xappto\improvementeval{\y}
            \gappto\improvementeval{\\}
            \gappto\improvementeval{\hline}
        \else
            \xappto\improvementeval{\y&}
        \fi\fi
    }
}

% Impostazioni della tabella
\tabulinesep = 2mm % padding
\taburowcolors [1] 2{pari .. dispari} % colori delle righe
\begin{longtabu} to \textwidth {| X[0.1,c m] | X[0.2,c m] | X[0.3,l m] | X[0.3,l m] |} % larghezza delle colonne
\hline
\rowcolor{header} % colore dell'header

\textbf{Codice rischio} & \textbf{Problema} & \textbf{Descrizione} & \textbf{Soluzione} \\
\hline
\improvementeval

\end{longtabu}

\undef\improvementeval{}

%startTable
\def\problems{
    { RO2-0 },
    { \noexpand\textbf{Fase di progettazione di architetturale} },
    {
        La sessione invernale è stata particolarmente impegnativa per la maggior parte del gruppo. È stata utilizzata la revisione a modalità \noexpand\textit{Sportello}: il team ha deciso di ritardare la consegna della Technology Baseline{,} nonché la consegna dei documenti per la RP rispettivamente al 2021-03-09 e al 2021-03-10.,
    },
    { \noexpand\textbf{Fase di progettazione di dettaglio e codifica} },
    {
        Tre membri del team hanno deciso di sostenere l'esame teorico di Ingegneria del Software{,} ha richiesto un periodo di preparazione. Il lavoro assegnato ai membri temporaneamente poco produttivi è stato ridistribuito{,} inoltre lo slittamento dell'inizio dei colloqui per l'approvazione della Product Baseline ha permesso un certo \noexpand\textit{slack}.
    },
    { \noexpand\textbf{Fase di validazione e collaudo} },
    {
        4 membri del team hanno sostenuto l'esame teorico di Ingegneria del Software{,} la preparazione richiesta per sostenere l'esame ha reso questi membri meno produttivi. L'applicazione della stessa misura di mitigazione usata nella fase precedente ha permesso al team di procedere nell'attività senza ritardi.
    },
}
%endTable
\newcommand*\improvementeval{}
\foreach \x [count=\nj] in \problems
{
    \foreach \y [count=\ni] in \x
    {
        \ifnum\ni=1
            \xappto\improvementeval{\noexpand\textbf{\y}&}
        \else\ifnum\ni=4
            \xappto\improvementeval{\y}
            \gappto\improvementeval{\\}
            \gappto\improvementeval{\hline}
        \else
            \xappto\improvementeval{\y&}
        \fi\fi
    }
}

% Impostazioni della tabella
\tabulinesep = 2mm % padding
\taburowcolors [1] 2{pari .. dispari} % colori delle righe
\begin{longtabu} to \textwidth {| X[0.1,c m] | X[0.2,c m] | X[0.3,l m] | X[0.3,l m] |} % larghezza delle colonne
\hline
\rowcolor{header} % colore dell'header

\textbf{Codice rischio} & \textbf{Problema} & \textbf{Descrizione} & \textbf{Soluzione} \\
\hline
\improvementeval

\end{longtabu}

\undef\improvementeval{}

%startTable
\def\problems{
    { RO3-0 },
    { \noexpand\textbf{Fase di progettazione di dettaglio e codifica}},
    {
        Un membro del team si è assentato per  una settimana.,
        Non sono state effettuate azioni di mitigazione: le attività in gestione al membro erano buono stato di avanzamento.
    },
}
%endTable
\newcommand*\improvementeval{}
\foreach \x [count=\nj] in \problems
{
    \foreach \y [count=\ni] in \x
    {
        \ifnum\ni=1
            \xappto\improvementeval{\noexpand\textbf{\y}&}
        \else\ifnum\ni=4
            \xappto\improvementeval{\y}
            \gappto\improvementeval{\\}
            \gappto\improvementeval{\hline}
        \else
            \xappto\improvementeval{\y&}
        \fi\fi
    }
}

% Impostazioni della tabella
\tabulinesep = 2mm % padding
\taburowcolors [1] 2{pari .. dispari} % colori delle righe
\begin{longtabu} to \textwidth {| X[0.1,c m] | X[0.2,c m] | X[0.3,l m] | X[0.3,l m] |} % larghezza delle colonne
\hline
\rowcolor{header} % colore dell'header

\textbf{Codice rischio} & \textbf{Problema} & \textbf{Descrizione} & \textbf{Soluzione} \\
\hline
\improvementeval

\end{longtabu}

\undef\improvementeval{}

