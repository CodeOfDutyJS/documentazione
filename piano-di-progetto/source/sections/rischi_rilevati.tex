\subsection{Rischi rilevati}
In questa sottosezione sono raccolti i problemi verificatosi durante l'attività di progetto ed è riportata la soluzione attuata o la situazione del problema al momento della redazione di questo documento.
\subsubsection{Rischi tecnologici verificati}
%startTable
\def\problems{
    {   
        RT1,
        Github,
        Alcuni membri del gruppo non conoscono o non sono pratici dello strumento Github,
        Tali membri hanno dedicato allo strumento un breve periodo di autoformazione prima d'iniziare l'esperienza sul campo. I membri più esperti del team sono stati e resteranno sempre disponibili per chiarimenti.
    },
    {
        RT1,
        Latex,
        L'esperienza con Latex da parte del team (nella sua interezza) è poca o nulla,
        La redazione dei documenti si è inizialmente rivelato un processo lento e frustrante{,} tuttavia è stato considerato un male necessario. Ancora adesso si riscontrano difficoltà rispetto alla creazione di tabelle e posizionamento di immagini.
    },
}
%endTable
\newcommand*\improvementeval{}
\foreach \x [count=\nj] in \problems
{
    \foreach \y [count=\ni] in \x
    {
        \ifnum\ni=1
            \xappto\improvementeval{\noexpand\textbf{\y}&}
        \else\ifnum\ni=4
            \xappto\improvementeval{\y}
            \gappto\improvementeval{\\}
            \gappto\improvementeval{\hline}
        \else
            \xappto\improvementeval{\y&}
        \fi\fi
    }
}

% Impostazioni della tabella
\tabulinesep = 2mm % padding
\taburowcolors [1] 2{pari .. dispari} % colori delle righe
\begin{longtabu} to \textwidth {| X[0.1,c m] | X[0.2,c m] | X[0.3,l m] | X[0.3,l m] |} % larghezza delle colonne
\hline
\rowcolor{header} % colore dell'header

\textbf{Codice rischio} & \textbf{Problema} & \textbf{Descrizione} & \textbf{Soluzione} \\
\hline
\improvementeval

\end{longtabu}

\undef\improvementeval{}

\subsubsection{Rischi organizzativi verificati}
%startTable
\def\problems{
    {
        RO1,
        Organizzazione incontri,
        Risulta complicato trovare un giorno e anche un orario per svolgere gli incontri.,
        Grazie al supporto dei canali di comunicazione individuati dalle \noexpand\textit{Norme di Progetto 1.0.0} e all'attività di verbalizzazione non è necessario che sia presente l'intero team ogni incontro. Per gli incontri più importanti{,} dove è necessario che sia presente tutto il gruppo{,} è necessario mettersi d'accordo con largo anticipo.
    },
    {
        RO1,
        Amministratore e ambiente di sviluppo,
        Il ruolo di amministratore ha richiesto più ore rispetto a quanto preventivato{,} come riportato nel \noexpand\textit{Piano di Progetto 1.0.0} {,} è stato dedicato molto tempo a impostare la repository e le Github Actions{,} con l'obiettivo di automatizzare il più possibile alcune operazioni.,
        L'aver dedicato molte ore nell'impostare l'ambiente di sviluppo è sicuramente un investimento per il futuro{,} sia per quanto riguarda il prodotto del processo di documentazione che per il prodotto software{,} infatti impostare la repository dedicata al codice sarà sicuramente un processo più agevole.
    },
}
%endTable
\newcommand*\improvementeval{}
\foreach \x [count=\nj] in \problems
{
    \foreach \y [count=\ni] in \x
    {
        \ifnum\ni=1
            \xappto\improvementeval{\noexpand\textbf{\y}&}
        \else\ifnum\ni=4
            \xappto\improvementeval{\y}
            \gappto\improvementeval{\\}
            \gappto\improvementeval{\hline}
        \else
            \xappto\improvementeval{\y&}
        \fi\fi
    }
}

% Impostazioni della tabella
\tabulinesep = 2mm % padding
\taburowcolors [1] 2{pari .. dispari} % colori delle righe
\begin{longtabu} to \textwidth {| X[0.1,c m] | X[0.2,c m] | X[0.3,l m] | X[0.3,l m] |} % larghezza delle colonne
\hline
\rowcolor{header} % colore dell'header

\textbf{Codice rischio} & \textbf{Problema} & \textbf{Descrizione} & \textbf{Soluzione} \\
\hline
\improvementeval

\end{longtabu}

\undef\improvementeval{}

