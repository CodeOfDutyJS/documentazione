\section{Processi Organizzativi}
	\subsection{Gestione Organizzativa}
		
		
		\subsubsection{Scopo}
		Lo scopo di questo processo è quello di:
		\begin{itemize}
			\item creare un modello organizzativo tramite il quale vengono specificati i rischi che si possono verificare;
			\item definire un modello di sviluppo da seguire;
			\item pianificare il lavoro seguendo le scadenze;
			\item ottenere un prospetto economico suddiviso per ruoli;
			\item effettuare un bilancio finale sulle spese.
		\end{itemize}
		Tali attività sopraelencate sono a carico del responsabile di progetto e devono essere raccolte nel \textit{Piano di Progetto}.
		
		\subsubsection{Aspettative}
		Gli obiettivi di questo processo sono i seguenti:
		\begin{itemize}
			\item produrre una pianificazione delle attività da seguire;
			\item coordinare i membri del gruppo assegnando loro ruoli e compiti, facilitando la comunicazione;
			\item utilizzare processi per regolare le attività, rendendole il meno dispendiose possibile;
			\item garantire un controllo sul progetto in maniera efficace e non invasiva, monitorando il gruppo, i processi e i prodotti.
		\end{itemize}
	
		\subsubsection{Descrizione}
		Le attività di gestione sono:
		\begin{itemize}
			\item inizio e definizione dello scopo;
			\item istanziazione dei processi;
			\item pianificazione e stima di risorse, tempi e costi;
			\item assegnazione di ruoli e compiti;
			\item esecuzione e controllo;
			\item revisione e valutazione periodica delle attività.
		\end{itemize}
		
		\subsubsection{Ruoli di progetto}
		Ciascun membro del gruppo deve ricoprire il ruolo che gli viene assegnato, e che corrisponde all'omonima figura aziendale. I ruoli verranno distribuiti a rotazione, in modo tale che qualunque membro del gruppo possa "toccare con mano" ogni singolo ruolo. Le attività assegnate agli specifici ruoli vengono organizzate e pianificate nel \textit{Piano di Progetto}. I ruoli che ogni componente del gruppo deve svolgere sono descritti di seguito.
		
			\myparagraph{Responsabile di progetto}
			Su tale figura ricadono importanti responsabilità tra cui: pianificazione, gestione, controllo e coordinamento. Altro compito del responsabile di progetto è quello di interfacciare il gruppo con il mondo esterno. Sarà perciò compito del responsabile di progetto comunicare con committente e proponente. I compiti di tale ruolo possono essere così riassunti:
			\begin{itemize}
				\item gestione, controllo e coordinazione di risorse e attività del gruppo;
				\item gestione, controllo e coordinazione dei componenti del gruppo;
				\item analisi e gestione delle criticità;
				\item approvazione dei documenti.
			\end{itemize}
		
			\myparagraph{Amministratore di progetto}
			L'amministratore di progetto è la figura che fornisce supporto e controllo all'ambiente di lavoro. Tale ruolo dovrà dunque:
			\begin{itemize}
				\item dirigere le infrastrutture di supporto;
				\item risolvere problemi legati alla gestione dei processi;
				\item gestire la documentazione;
				\item controllare versioni e configurazioni.
			\end{itemize}
		
			\myparagraph{Analista}
			Tale figura si occupa di fornire un'analisi sui problemi e sul dominio applicativo. Perciò tale figura non sarà sempre presente per tutta la durata del progetto.
			I compiti di tale figura possono essere così riassunti:
			\begin{itemize}
				\item studiare il dominio del problema;
				\item definire la complessità e i requisiti del problema;
				\item redigere i documenti: \textit{Analisi dei Requisiti e Studio di Fattibilità}.
			\end{itemize}
		
			\myparagraph{Progettista}
			Il progettista si occupa di gestire gli aspetti tecnologici e tecnici del progetto.
			Egli deve:
			\begin{itemize}
				\item prendere scelte efficienti ed efficaci su aspetti tecnici del progetto;
				\item sviluppare un'architettura che sfrutti tecnologie note ed ottimizzate su cui basare un prodotto stabile e mantenibile.
			\end{itemize}
		
			\myparagraph{Programmatore}
			Il programmatore è responsabile della codifica del progetto e delle componenti di supporto per la verifica e validazione del prodotto. Egli deve inoltre: 
			\begin{itemize}
				\item implementare in maniera efficiente le decisioni del progettista;
				\item creare e gestire gli strumenti di supporto volti alla verifica e validazione del codice.
			\end{itemize}
		
			\myparagraph{Verificatore}
			Tale ruolo si occupa di controllare e verificare il prodotto (codice/documentazione) del lavoro svolto dal gruppo. Per adempiere a tale ruolo si affida agli standard definiti nelle \textit{Norme di Progetto} uniti all'esperienza dell'ultimo. Il verificatore deve:
			\begin{itemize}
				\item controllare e ispezionare i prodotti in fase di revisione, utilizzando le tecniche e gli strumenti definiti nelle \textit{Norme di Progetto};
				\item riscontrare eventuali difetti ed errori del prodotto considerato;
				\item segnalare gli eventuali errori trovati al responsabile della componente presa in considerazione.
			\end{itemize}
		
		
		\subsubsection{Procedure}
		Vengono di seguito riportare le procedure che il collettivo adotterà durante la realizzazione del progetto. Le comunicazioni potranno essere interne(tra i membri del collettivo) oppure esterne (oltre ai membri del collettivo vi sono anche proponente e committente).
		
			\myparagraph{Gestione delle comunicazioni}
			
				\mysubparagraph{Comunicazioni interne}
				Le comunicazioni interne del gruppo vengono svolte mediante l'utilizzo di:
				\begin{itemize}
					\item \textbf{Telegram}: strumento di messaggistica con il quale i membri del gruppo scambiano informazioni;
					\item \textbf{Discord}: Strumento per la collaborazione, tramite questo i membri del gruppo possono tenere le riunioni all'interno di canali privati. Offre possibilità di creare più "channels" nei quali discutere e riflettere su argomenti di altro carattere.
				\end{itemize}  
				
				\mysubparagraph{Comunicazioni esterne}
				Le comunicazioni esterne (con soggetti esterni al gruppo) sono a carico del responsabile. Lo strumento predefinito utilizzato per le comunicazioni esterne è l'indirizzo di posta elettronica: \textbf{info@codeofduty.it}.
				
			
			\myparagraph{Gestione degli incontri}
			
				\mysubparagraph{Incontri interni}
				Gli incontri interni sono organizzati dal responsabile in accordo con tutti gli altri membri del gruppo. 
				
				\mysubparagraph{Verbali di riunioni interne}
				Ad ogni riunione interna verrà prodotto un verbale, il quale riassumerà l'esito dell'incontro. Tale verbale verrà redatto da un segretario nominato direttamente dal responsabile. Ogni decisione presente nel verbale interno verrà identificata dalla seguente dicitura \textbf{VIYYY}:
				\begin{itemize}
					\item \textbf{V}: indica un verbale;
					\item\textbf{I}: indica che si tratta di un verbale interno;
					\item\textbf{YYY}: individua una decisione all'interno del verbale(numerazione progressiva).
				\end{itemize}
			
				\mysubparagraph{Incontri esterni del team}
				Gli incontri esterni vengono organizzati dal responsabile. Nel caso in cui un membro del gruppo o il proponente/committente ritengono necessario un incontro, il responsabile dovrà occuparsi di pianificare una data (possibilmente in accordo tra le due parti) e di comunicarla tramite gli strumenti di comunicazione sopra citati.
				
				\mysubparagraph{Verbali di riunioni esterne}
				Ad ogni riunione esterna verrà prodotto un verbale, il quale riassumerà l'esito dell'incontro. Il verbale verrà redatto da un segretario nominato direttamente dal responsabile. Ogni decisione presente nel verbale esterno verrà identificata dalla seguente dicitura \textbf{VEYYY}:
				\begin{itemize}
					\item\textbf{V}: indica un verbale;
					\item\textbf{E}: indica che si tratta di un verbale esterno;
					\item\textbf{YYY}: individua una decisione all'interno del verbale esterno (numerazione progressiva).
				\end{itemize}
			
			
			\myparagraph{Gestione degli strumenti per la coordinazione}
			
				\mysubparagraph{Sistema di Ticketing}
				Il ticketing è un sistema che permette al gruppo di avere, in ogni momento, una situazione chiara su tutte le attività in corso. Tramite tale strumento il responsabile di progetto è in grado di assegnare i compiti ai membri del gruppo, e di verificare lo stato di avanzamento delle attività. Lo strumento di ticketing utilizzato dal gruppo sono le Board di GitHub. Una o più board possono essere associate ad un progetto; esse permettono di dividere i compiti in 4 categorie:
					\begin{itemize}
						\item to do (da fare);
						\item doing (in lavorazione);
						\item verifying (in verifica);
						\item done (completato).
					\end{itemize}
				Il gruppo ha deciso di utilizzare le board di GitHub in quanto offerte già dal sistema di controllo del versionamento. Ciò permette al progetto di essere gestito in maniera compatta tramite un solo strumento (GitHub).
			
			
			\myparagraph{Gestione dei rischi}
			Il responsabile di progetto ha il dovere di rilevare i rischi e renderli noti al gruppo, redigendo tale attività nel \textit{Piano di Progetto}. In particolare, egli dovrà attenersi a questo iter:
			\begin{itemize}
				\item individuazione di nuovi rischi e monitorazione di quelli già noti;
				\item documentare qualsiasi riscontro previsto dei rischi all'interno del \textit{Piano di Progetto};
				\item includere all'interno del \textit{Piano di Progetto} i nuovi rischi individuati;
				\item rivedere e, se necessario, ridefinire le strategie di gestione dei rischi.
			\end{itemize}
			
				\mysubparagraph{Codifica dei rischi}
				I rischi possono essere divisi in categorie, in base alla loro origine.
				\begin{itemize}
					\item \textbf{RO}: Rischi organizzativi;
					\item \textbf{RT}: Rischi tecnologici;
					\item \textbf{RI}: Rischi interpersonali.	
				\end{itemize}
			
		\subsubsection{Strumenti}
		Di seguito vengono riportati gli strumenti utilizzati dal gruppo:
		\begin{itemize}
			\item \textbf{Telegram}: applicazione di messaggistica utilizzata dal gruppo per comunicare;
			\item \textbf{Discord}: Utilizzato per tenere riunioni interne al gruppo;
			\item \textbf{Git}: Sistema per il controllo del versionamento;
			\item \textbf{GitHub}: utilizzato per il versionamento ed il salvataggio in remoto di tutti i file che riguardano il progetto. Utilizzato inoltre come strumento di Ticketing tramite le board messe a disposizione dal servizio;
			\item \textbf{Sistemi operativi}: Linux, Windows e Mac OS.
		\end{itemize}
		
		
		\subsection{Formazione}
			\subsubsection{Scopo}
			Il processo di formazione ha come scopo principale quello di istruire e mantenere istruito il personale. L'acquisizione, la fornitura, lo sviluppo e la manutenzione del software sono fortemente dipendenti dalle conoscenze del personale del gruppo.
			
			\subsubsection{Aspettative}
			Al termine del processo di formazione il personale dovrebbe aver acquisito:
			\begin{itemize}
				\item Competenze riguardo agli argomenti trattati;
				\item Comprensione dei campi applicativi; 
				\item Capacità di condivisione delle conoscenze.
			\end{itemize}
			
			\subsubsection{Descrizione}
			I componenti del gruppo dovranno provvedere in maniera autonoma alla propria formazione. Ciò può essere ottenuto andando a studiare le tecnologie usate e andando a colmare eventuali carenze. In tale modo il gruppo potrà garantire una migliore qualità del lavoro in base alle aspettative. Per la formazione i componenti del gruppo dovranno fare riferimento alla documentazione presente nella sezione \hyperref[sec:Riferimenti]{Riferimenti} di questo processo , nonché alla documentazione reperita autonomamente.
			È inoltre possibile che venga richiesta autoformazione su argomenti non ancora ben chiari e che verranno trattati con lo sviluppo del progetto.	
			
			\subsubsection{Riferimenti} \label{sec:Riferimenti}
				\begin{itemize}
					\item \textbf{Latex}: \url{https://devdocs.io/javascript/};
					\item \textbf{JavaScript}: \url{https://devdocs.io/javascript/};
					\item \textbf{D3.js}: \url{https://github.com/d3/d3/wiki};
					\item \textbf{React}: \url{https://reactjs.org/docs/getting-started.html};
					\item \textbf{Node.js}: \url{https://nodejs.org/en/docs/};
					\item \textbf{Express.js}: \url{http://expressjs.com/en/5x/api.html};
					\item \textbf{Jest.js}: \url{https://jestjs.io/docs/en/getting-started};
					\item \textbf{Ant}: \url{https://ant.design/docs/react/introduce};
					\item \textbf{Mobx}: \url{https://mobx.js.org/README.html};
					\item \textbf{Yarn}: \url{https://classic.yarnpkg.com/en/docs/};
					\item \textbf{Git}: \url{https://git-scm.com/};
					\item \textbf{GitHub}: \url{https://help.github.com/}.
				\end{itemize}
		
