\section{Processi Organizzativi}
	\subsection{Gestione Organizzativa}
		
		
		\subsubsection{Scopo}
		Lo scopo di questo processo è quello di:
		\begin{itemize}
			\item creare un modello organizzativo tramite il quale vengono specificati i rischi che si possono verificare;
			\item definire un modello di sviluppo da seguire;
			\item pianificare il lavoro seguendo le scadenze;
			\item ottenere un prospetto economico suddiviso per ruoli;
			\item effettuare un bilancio finale sulle spese;
		\end{itemize}
		Tali attività sopraelencate sono a carico del responsabile di progetto e devono essere raccolte nel \textit{Piano di Progetto}.
		
		\subsubsection{Aspettative}
		Gli obiettivi di questo processo sono i seguenti:
		\begin{itemize}
			\item produrre una pianificazione delle attività da seguire;
			\item coordinare i membri del gruppo assegnando loro ruoli e compiti, facilitando la comunicazione;
			\item utilizzare processi per regolare le attività, rendendole il meno dispendiose possibile;
			\item garantire un controllo sul progetto in maniera efficace e non invasiva, monitorando il gruppo, i processi e i prodotti;
		\end{itemize}
	
		\subsubsection{Descrizione}
		
