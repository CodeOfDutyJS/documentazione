\section{Processi Organizzativi}
	\subsection{Gestione Organizzativa}
		
		
		\subsubsection{Scopo}
		Lo scopo di questo processo è quello di:
		\begin{itemize}
			\item creare un modello organizzativo tramite il quale vengono specificati i rischi che si possono verificare;
			\item definire un modello di sviluppo da seguire;
			\item pianificare il lavoro seguendo le scadenze;
			\item ottenere un prospetto economico suddiviso per ruoli;
			\item effettuare un bilancio finale sulle spese;
		\end{itemize}
		Tali attività sopraelencate sono a carico del responsabile di progetto e devono essere raccolte nel \textit{Piano di Progetto}.
		
		\subsubsection{Aspettative}
		Gli obiettivi di questo processo sono i seguenti:
		\begin{itemize}
			\item produrre una pianificazione delle attività da seguire;
			\item coordinare i membri del gruppo assegnando loro ruoli e compiti, facilitando la comunicazione;
			\item utilizzare processi per regolare le attività, rendendole il meno dispendiose possibile;
			\item garantire un controllo sul progetto in maniera efficace e non invasiva, monitorando il gruppo, i processi e i prodotti;
		\end{itemize}
	
		\subsubsection{Descrizione}
		Le attività di gestione sono:
		\begin{itemize}
			\item inizio e definizione dello scopo;
			\item istanziazione dei processi;
			\item pianificazione e stima di risorse, tempi e costi;
			\item assegnazione di ruoli e compiti;
			\item esecuzione e controllo;
			\item revisione e valutazione periodica delle attività;
		\end{itemize}
		
		\subsubsection{Ruoli di progetto}
		Ciascun membro del gruppo deve ricoprire il ruolo che gli viene assegnato, e che corrisponde all'omonima figura aziendale. I ruoli verranno distribuiti a rotazione, in modo tale che qualunque membro del gruppo possa "toccare con mano" ogni singolo ruolo. Le attività assegnate agli specifici ruoli vengono organizzate e pianificate nel \textit{Piano di Progetto}. I ruoli che ogni componente del gruppo deve svolgere sono descritti di seguito.
		
			\paragraph{Responsabile di progetto}
			Su tale figura ricadono importanti responsabilità tra cui: pianificazione, gestione, controllo e coordinamento. Altro compito del responsabile di progetto è quello di interfacciare il gruppo con il mondo esterno. Sarà perciò compito del responsabile di progetto comunicare con committente e proponente. I compiti di tale ruolo possono essere così riassunti:
			\begin{itemize}
				\item gestione, controllo e coordinazione di risorse e attività del gruppo;
				\item gestione, controllo e coordinazione dei componenti del gruppo;
				\item analisi e gestione delle criticità;
				\item approvazione dei documenti;
			\end{itemize}
		
			\paragraph{Amministratore di progetto}
			L'amministratore di progetto è la figura che fornisce supporto e controllo all'ambiente di lavoro. Tale ruolo dovrà dunque:
			\begin{itemize}
				\item dirigere le infrastrutture di supporto;
				\item risolvere problemi legati alla gestione dei processi;
				\item gestire la documentazione;
				\item controllare versioni e configurazioni;
			\end{itemize}
		
			\paragraph{Analista}
			Tale figura si occupa di fornire un'analisi sui problemi e sul dominio applicativo. Perciò tale figura non sarà sempre presente per tutta la durata del progetto.
			I compiti di tale figura possono essere così riassunti:
			\begin{itemize}
				\item studiare il dominio del problema;
				\item definire la complessità e i requisiti del problema;
				\item redigere i documenti: \textit{Analisi dei Requisiti e Studio di Fattibilità}
			\end{itemize}
		
			\paragraph{Progettista}
			Il progettista si occupa di gestire gli aspetti tecnologici e tecnici del progetto.
			Egli deve:
			\begin{itemize}
				\item prendere scelte efficienti ed efficaci su aspetti tecnici del progetto;
				\item sviluppare un'architettura che sfrutti tecnologie note ed ottimizzate su cui basare un prodotto stabile e mantenibile
			\end{itemize}
		
			\paragraph{Programmatore}
			Il programmatore è responsabile della codifica del progetto e delle componenti di supporto per la verifica e validazione del prodotto. Egli deve inoltre: 
			\begin{itemize}
				\item implementare in maniera efficiente le decisioni del progettista.
				\item creare e gestire gli strumenti di supporto volti alla verifica e validazione del codice.
			\end{itemize}
		
			\paragraph{Verificatore}
			Tale ruolo si occupa di controllare e verificare il prodotto(codice/documentazione) del lavoro svolto dal collettivo. Per adempiere a tale ruolo si affida agli standard definiti nelle \textit{Norme di Progetto} uniti all'esperienza dell'ultimo. Il verificatore deve:
			\begin{itemize}
				\item controllare e ispezionare i prodotti in fase di revisione, utilizzando le tecniche e gli strumenti definiti nelle \textit{Norme di Progetto};
				\item riscontrare eventuali difetti ed errori del prodotto considerato;
				\item segnalare gli eventuali errori trovati al responsabile della componente presa in considerazione.
			\end{itemize}
		
		
		\subsubsection{Procedure}
		Vengono di seguito riportare le procedure che il collettivo adotterà durante la realizzazione del progetto. Le comunicazioni potranno essere interne(tra i membri del collettivo) oppure esterne (oltre ai membri del collettivo vi sono anche proponente e committente).
		
			\paragraph{Gestione delle comunicazioni}
			
				\subparagraph{Comunicazioni interne}
				Le comunicazioni interne del gruppo vengono svolte mediante l'utilizzo di Slack o Discord. Entrambi sono software di collaborazione aziendale adatti al lavoro collettivo. Mediante l'utilizzo di tali software è possibile dividere il workspace in più sezioni(nelle quali si può discutere), ognuna di queste è indipendente dalle altre. Inoltre entrambi i software mettono a disposizione una serie di plugin utili. Tali argomentazioni hanno portato alla scelta di questi due software.
				
				\subparagraph{Comunicazioni esterne}
				
			
		
