\section{Processi Primari}
	\subsection{Fornitura}
		\subsubsection{Scopo}
		Il processo di fornitura si occupa di determinare le procedure e le risorse necessarie allo svolgimento del progetto, incluso lo sviluppo del \textit{Piano di Progetto} e l'esecuzione dello stesso. Il processo inizia una volta presa la decisione di rispondere alla proposta dell'acquirente e dopo aver compreso le sue richieste stilando uno \textit{Studio di fattibilità}. Il processo si compone delle seguenti attività:
		\begin{itemize}
		    \item avvio;
		    \item preparazione delle risposte;
		    \item stipulazione del contratto;
		    \item pianificazione;
		    \item esecuzione e controllo;
		    \item revisione e valutazione;
		    \item consegna e completamento.
		\end{itemize}
		
		\subsubsection{Aspettative}
		Per permettere di avere un lavoro concorde con quanto il proponente richiede, il gruppo si impegna a partecipare attivamente ad un dialogo protratto nel tempo (sincrono e asincrono). Ciò permette di avere:
		\begin{itemize}
			\item una migliore visione per quanto riguarda gli aspetti cruciali del progetto;
			\item un riscontro diretto sulle preferenze del proponente;
			\item instaurare un rapporto sano e costruttivo con il proponente;
		\end{itemize}
		
		\subsubsection{Descrizione}
		Questa sezione norma le fasi di progettazione, sviluppo e consegna del prodotto \textit{HD viz}. 
		\subsubsection{Attività}
		    \myparagraph{Studio di Fattibilità}
		    Gli analisti redigono uno \textit{Studio di Fattibilità} per ogni capitolato, indicando:
		    \begin{itemize}
		        \item \textbf{Informazioni generali}: informazioni riguardanti il nome del progetto, il proponente e il committente;
		        \item \textbf{Descrizione capitolato}: descrizione suntuaria del capitolato e delle aspettative e richieste sul prodotto finale;
		        \item \textbf{Finalità del progetto}: descrizione del prodotto finito;
		        \item \textbf{Aspetti positivi}: caratteristiche che rendono il gruppo più disponibile a proporre un'offerta verso il proponente;
		        \item \textbf{Criticità e fattori di rischio}: aspetti negativi con potenziali ripercussioni sullo svolgimento del progetto;
		        \item \textbf{Conclusioni}: breve spiegazione sul perchè il capitolato è stato scartato o accettato dal gruppo.
		    \end{itemize}
			\myparagraph{Piano di Progetto}
			Il Responsabile di Progetto con gli amministratori redige un \textit{Piano di progetto} volto a fornire un preventivo ed una pianificazione dettagliata sullo svolgimento del progetto al proponente contiene:
			\begin{itemize}
			    \item \textbf{Analisi dei Rischi}: vengono elencati i rischi che potrebbero presentarsi durante lo svolgimento del progetto, insieme ad una indicazione probabilistica del loro effettivo avvenimento e le modalità con le quali si intende mitigare questi rischi;
			    \item \textbf{Modello di sviluppo}: viene fornito un modello di sviluppo da seguire durante il progetto;
			    \item \textbf{Pianificazione}: vengono pianificate le attività e le scadenze temporali;
			    \item \textbf{Preventivo}: qui viene fatta la stima dello sforzo previsto in termini di ore di lavoro, e dei costi associati con l'esecuzione di processo, riportando in dettaglio organigramma e il programma degli orari richiesti per una consegna puntuale del progetto. Viene quindi fornito un preventivo iniziale basato su queste stime;
			    \item \textbf{Consuntivo}: il documento viene aggiornato periodicamente con un consuntivo dei costi effettivi dello sviluppo, corredato di una spiegazione su un eventuale differenza con il preventivo.
			\end{itemize}
			\myparagraph{Piano di Qualifica}
			I verificatori devono redigere un \textit{Piano di Qualifica}, contenente tutte le informazioni riguardanti il controllo di qualità  per i processi e il prodotto, basato su quantificazioni misurabili. Il \textit{Piano di Qualifica} contiene:
			\begin{itemize}
			    \item \textbf{Qualità di processo}: vengono cercati negli standard i processi da attuare, individuati degli obiettivi di qualità e le metriche corrispondenti, vengono stilati metodi per perseguire gli obiettivi posti, il documento di riferimento è la ISO-15504;
			    \item \textbf{Qualità di prodotto}: vengono identificate delle metriche corrispondenti agli attributi del prodotto e definiti degli obiettivi di qualità, il documento di riferimento è la ISO-9126;
			    \item \textbf{Specifiche dei test}: vengono definiti i test che garantiscono la qualità del prodotto;
			    \item \textbf{Resoconti di verifica ed esiti delle revisioni}: qui sono riportati gli esiti delle attività di verifica e delle revisioni.
			\end{itemize}
		\subsubsection{Metriche}
        Di seguito vengono elencate le metriche inerenti alla qualità del processo di \textbf{Fornitura}. I valori sono normati nel piano di qualifica. La modalità di rilevazione non è indicata per tutte le metriche: tale dato sarà 
        inserito in fasi successive del progetto.
        
        \myparagraph{Percentuale di requisiti soddisfatti}
        La percentuale in centesimi dei requisiti saddisfatti.
        \subparagraph{Formula}
        \begin{displaymath}
         PROS = Rs / Rt *100
        \end{displaymath}
        Dove:
        \begin{itemize}
            \item[] $Rs =$ requisiti soddisfatti
            \item[] $Rt =$ requisiti totali
        \end{itemize}
        
        \myparagraph{Budget at completion}
        Budget totale allocato per il progetto.
        \mysubparagraph{Misurazione} Numero intero.
        
        \myparagraph{Earned value}
        Utilizzato anche nel calcolo di $SV$ e $CV$. Abbreviato $EV$, indica il valore del lavoro compiuto fino al momento del calcolo.
        \subparagraph{Formula}
        \begin{displaymath}
          EV = BAC * \% \textrm{di lavoro completato}
        \end{displaymath}
        Dove:
        \begin{itemize}
            \item[] $EV =$ Earned value
            \item[] $BAC =$ Budget at copletion
        \end{itemize}
        
        \myparagraph{Planned value}
        Abbreviato in $PV$, rappresenta il valore del lavoro pianificato al momento del calcolo. Utilizzato anche nel calcolo di $SV$ e $CV$.
        \subparagraph{Formula}
        \begin{displaymath}
          PV = BAC * \% \textrm{di lavoro pianificato}
        \end{displaymath}
        Dove:
        \begin{itemize}
            \item[] $PV =$ Planned value
            \item[] $BAC =$ Budget at completion
        \end{itemize}
        
        \myparagraph{Actual Cost}
        Per il calcolo in altre metriche l'abbreviazione è $AC$.
        Rappresenta il denaro speso fino al momento del calcolo.
        \mysubparagraph{Misurazione} La misurazione avviene con il rilevamento dell'intero che rappresenta l'$AC$ pertanto non ha formule.
        
        \myparagraph{Schedule variance}
        Rappresenta l'anticipo o il ritardo nello svolgimento del progetto rispetto al valore pianificato. Abbreviato in $SV$.
        \subparagraph{Formula}
        \begin{displaymath}
          SV = EV - PV
        \end{displaymath}
        Dove:
        \begin{itemize}
            \item[] $SV =$ Schedule variance
            \item[] $EV =$ Earned value
            \item[] $PV =$ Planned value
        \end{itemize}
        
        \myparagraph{Cost variance}
        Rappresenta la differenza tra il lavoro completato e quello pianificato, ed è abbreviato in $CV$.
        \subparagraph{Formula}
        \begin{displaymath}
          CV = EV - AC
        \end{displaymath}
        Dove:
        \begin{itemize}
            \item[] $CV =$ Cost variance
            \item[] $EV =$ Earned value
            \item[] $AC =$ Actual cost
        \end{itemize}
        
		\subsubsection{Strumenti}
	    Qui vengono riportati gli strumenti utilizzati nel processo di fornitura:
		\mysubparagraph{Google Sheets}
	    Applicazione web per la creazione di fogli elettronici che permette a più persone di modificare lo stesso foglio.
	    \mysubparagraph{Diagrams.net}
	    Applicazione web per la produzione di diagrammi, viene utilizzata per la produzione dei diagrammi UML.\\
	    \centerline{\url{https://app.diagrams.net/}}
			
		\subsection{Sviluppo}
		\subsubsection{Scopo}
		Il processo di sviluppo contiene le attività da svolgere al fine di ottenere il prodotto finale.
		
		\subsubsection{Aspettative}
		\begin{itemize}
			\item dare obbiettivi di sviluppo;
			\item fissare vincoli tecnologici e di design;
			\item realizzare un prodotto che sia in grado di superare i test e che soddisfi i requisiti prefissati.
		\end{itemize}
		
		\subsubsection{Descrizione}
		Le attività che caratterizzano il processo di sviluppo sono:
		\begin{itemize}
			\item \textit{Analisi dei requisiti};
			\item \textit{Codifica del software};
			\item \textit{Progettazione architetturale}.
		\end{itemize}
		
		\subsubsection{Attività}
		\myparagraph{Analisi dei requisiti}
		Gli analisti redigono un'\textit{Analisi dei Requisiti} volta  individuare ed elencare casi d'uso e requisiti, al fine di esemplificare il problema da trattare.
		\mysubparagraph{Scopo}
		L'analisi dei requisiti deve definire lo scopo del lavoro e mettere in evidenza funzionalità e requisiti in modo da fornire riferimenti tracciabili sia ai programmatori che ai verificatori.
		
		\mysubparagraph{Aspettative}
		Dall'attività di Analisi dei requisiti ci si aspetta la produzione di documentazione la quale contenga tutti i requisiti e le specifiche dettate dal proponente.
	
		\mysubparagraph{Descrizione}
		Descrizione delle finalità del prodotto, degli attori in causa, dei vincoli imposti dal proponente, e delle assunzioni fatte dagli analisti.
		
		
			\mysubparagraph{Contenuti}
		L'\textit{Analisi dei requisiti} contiene le seguenti sezioni:
		\begin{itemize}
			\item Descrizione
			\item Casi d'uso
			\item Requisiti
		\end{itemize}
		Di seguito verranno definite le sezioni nello specifico.
		
		
		\mysubparagraph{Casi d'uso}
		I casi d'uso rappresentano una lista di azioni necessarie alla realizzazione di un obbiettivo desiderabile all'interazione fra un attore ed il sistema.
		La struttura dei casi d'uso è così definita:
		\begin{itemize}
		    \item attore;
		    \item descrizione;
		    \item precondizione;
		    \item postcondizione;
		    \item scenario principale;
		    \item estensioni (opzionale);
		    \item specializzazioni (opzionale);
		    \item generalizzazioni (opzionale).
		\end{itemize}
		Ogni caso d'uso adotta un codice identificativo in questa forma:\\
	    \centerline{\textbf{UC[codiceCaso].[codiceSpecializzazione].[ulterioreCodiceSpecializzazione]}}
        dove:
	    \begin{itemize}
	        \item \textbf{codiceCaso}: è il codice identificativo del caso;
	        \item \textbf{codiceSpecializzazione}: è il codice della specializzazione.
	        \item \textbf{ulterioreCodiceSpecializzazione}: è un ulteriore codice della specializzazione.
	    \end{itemize}
	    \mysubparagraph{Requisiti}
	    I requisiti sono identificati dal seguente codice:\\
	    \centerline{\textbf{R[tipo][classe][identificativo]}}
	    dove:
	    \begin{itemize}
	        \item \textbf{tipo}: può essere:
	        \begin{itemize}
	            \item \textbf{F}: requisito funzionale;
	            \item \textbf{V}: vincolo;
	            \item \textbf{Q}: requisito di qualità;
	            \item \textbf{P}: requisito prestazionale. 
	        \end{itemize}
	        \item \textbf{classe}: può essere:
	        \begin{itemize}
	            \item \textbf{O}: obbligatorio;
	            \item \textbf{F}: facoltativo;
	            \item \textbf{D}: desiderabile;
	        \end{itemize}
	        \item \textbf{identificativo}: identificativo progressivo del requisito.
	    \end{itemize}
	    Tutti i requisiti sono contenuti in tabelle composte da quattro (4) colonne, i cui attributi sono:
	    \begin{itemize}
	        \item \textbf{Requisito}: codice del requisito;
	        \item \textbf{Descrizione}: descrizione del requisito;
	        \item \textbf{Classificazione}: la classe del requisito;
	        \item \textbf{Fonte}: fonte da cui il requisito è stato estrapolato.
	    \end{itemize}
	    Le tabelle sono divise per tipo di requisito. Ad ogni requisito corrisponde una riga di una ed una sola tabella.
		\myparagraph{Progettazione}
		
		\mysubparagraph{Scopo}
		Tale attività permette di definire la migliore soluzione che soddisfi i requisiti per gli stakeholders, tramite un individuazione delle caratteristiche che il prodotto software deve soddisfare.
		
		\mysubparagraph{Aspettative}
		Prima della realizzazione dell'architettura:
		\begin{itemize}
			\item devono essere definite le tecnologie da utilizzare;
			\item devono essere approfonditi gli aspetti positivi della realizzazione e le criticità emerse;
			\item si deve produrre un prototipo, detto \textit{Proof of Concept}, il quale dimostri l'approfondimento svolto.
		\end{itemize}
		\mysubparagraph{Descrizione}
		Messi in mostra i requisiti stilati nell'\textit{Analisi dei requisiti} gli analisti ricercano una possibile soluzione al problema compreso, identificando un'architettura di alto livello che comprenda una lista di sottosistemi hardware e software, specificandone le funzionalità ed assicurandosi che comprendano un'implementazione di tutti i requisiti tracciati.
		
		\mysubparagraph{Prodotti}
		Vengono prodotte due parti, la \textbf{Technology baseline} e la \textbf{Product baseline}.
		\mysubparagraph{Technology baseline}
		Redatta dal progettista, contiene le specifiche di progettazione architetturale di alto livello, l'insieme di diagrammi UML che la definiscono ed i test d'integrazione.\newline
		include quindi:
		\begin{itemize}
		    \item \textbf{Tecnologie}: devono essere descritte e giustificate le tecnologie adottate, elencandone svantaggi e vantaggi, ed esponendone il ruolo nel progetto;
		    \item \textbf{Design Pattern}: descrizione dei design pattern adottati, corredati di descrizione, motivazione della scelta, diagrammi UML che li descrivano;
		    \item \textbf{Diagrammi UML}:
		    \begin{itemize}
		        \item Diagrammi delle classi;
		        \item Diagrammi di package;
		        
		        \item Diagrammi di attività;
		        \item Diagrammi di sequenza.
		    \end{itemize}
		    \item \textbf{Tracciamento}: tracciamento dei componenti ai relativi requisiti che soddisfano.
		\end{itemize}
		\myparagraph{Product Baseline}
		Redatta dai progettisti dovrà contenere:
		\begin{itemize}
		    \item \textbf{Definizione delle classi}: descrizione delle classi utilizzate;
		    \item \textbf{Tracciamento delle classi}: collegamento delle classi ai relativi requisiti;
		    \item \textbf{Test di unità}: definizione dei test di unità al fine di stabilire il corretto funzionamento delle classi.
		\end{itemize}
		\paragraph{Codifica}
		\mysubparagraph{Scopo}
	    Lo scopo della sezione è normare lo stile di codifica da utilizzare nella programmazione, intesa come effettiva implementazione software del prodotto. Lo stile di codifica è da considerarsi come parte integrante della qualità del prodotto stesso.
	    \mysubparagraph{Aspettative}
	    Le aspettative per l'attività di codifica sono:
	    \begin{itemize}
	    	\item generazione di codice leggibile e facilmente manutenibile; 
	    	\item facilitare le attività di verifica e validazione;
	    	\item più in generale, migliorare la qualità del prodotto.
	    \end{itemize}
	    \mysubparagraph{Descrizione}
	    La stesura del codice deve rispettare quanto riportato nella documentazione di prodotto. Dovranno dunque essere perseguiti gli obiettivi posti all'interno del \textit{Piano di Progetto 2.0.0}, i quali garantiranno la stesura di codice di qualità. Il codice viene steso adottando le norme dello stile di codifica Airbnb, alcune norme notabili vengono riportate di seguito.
	    
	    \mysubparagraph{Indentazione}
	    L'indentazione viene effettuata con due (2) spazi, e non con il carattere "tab". Quando si parla di "tabulare" o di "tabulazione" si deve intendere, appunto, l'inserire due (2) spazi.
	    
	    \mysubparagraph{Blocchi innestati}
	    Ciascun blocco innestato viene tabulato una volta.
	    
	    \mysubparagraph{Parentesizzazione}
	    Le parentesi di delimitazione di un blocco di codice vanno inserite nel seguente modo:
	    \begin{itemize}
	        \item la prima và inserita alla fine della prima riga, preceduta da uno spazio;
	        \item la seconda su una nuova riga, senza caratteri di spazio antecedenti;
	    \end{itemize}

        \mysubparagraph{Linter}
        Al fine di ottenere uno stile di codifica uniforme tra tutti i membri del gruppo si è scelto di inserire nel build lifecycle Eslint, le cui impostazioni sono specificate nel file eslint.rc.

	    \subparagraph{Casing}
	    \begin{itemize}
	        \item \textbf{Classi}: i nomi delle classi iniziano con la lettera maiuscola;
	        \item \textbf{Costanti}: vanno scritte tutte in maiuscolo;
	        \item \textbf{Variabili}: camel case, iniziano con la lettera minuscola;
	        \item \textbf{Metodi}: camel case, iniziano con la lettera minuscola.
	    \end{itemize}
	    \mysubparagraph{Metodi}
	    Il codice deve essere il più leggibile possibile. A tal fine, e anche al fine di esemplificare i test, è desiderabile che i metodi siano composti da poche righe di codice, con nomi quanto più corti possibile ma al contempo descrittivi.
	    \mysubparagraph{Lingua}
	    Il codice insieme ad i commenti viene scritto in inglese.
	    \mysubparagraph{Ricorsione}
	    Evitare il più possibile la ricorsione.
\paragraph{Strumenti}
        \subsubsection{Metriche}
        Di seguito vengono elencate le metriche inerenti alla qualità del processo di \textbf{Sviluppo}. I valori sono normati nel piano di qualifica. La modalità di rilevazione non è indicata per tutte le metriche: tale dato sarà 
        inserito in fasi successive del progetto.
        \myparagraph{Percentuale di requisiti soddisfatti}
        La percentuale in centesimi dei requisiti soddisfatti.
        \subparagraph{Formula}
        \begin{displaymath}
         PROS = Rs / Rt *100
        \end{displaymath}
        Dove:
        \begin{itemize}
            \item[] $Rs =$ requisiti soddisfatti
            \item[] $Rt =$ requisiti totali
        \end{itemize}
        
        \subsubsection{Strumenti}
	    \mysubparagraph{UML}
	    Linguaggio di modellizzazione atto a visualizzare un sistema software. Viene utillizzata la versione 2.0 del linguaggio.
	    \mysubparagraph{Diagrams.net}
	    Applicazione web per la produzione di diagrammi, viene utilizzata per la produzione dei diagrammi UML.\\
	    \centerline{\url{https://app.diagrams.net/}}