\section{Processi Primari}
	\subsection{Fornitura}
	\subsection{Sviluppo}
		\subsubsection{Scopo}
		Il processo di fornitura si occupa di determinare le procedure e le risorse necessarie allo svolgimento del progetto, incluso lo sviluppo del \textit{Piano di Progetto} e l'esecuzione dello stesso. Il processo inizia una volta presa la decisione di rispondere alla proposta dell'acquirente e dopo aver compreso le sue richieste stilando uno \textit{Studio di fattibilità} il processo si compone delle seguenti attività:
		\begin{itemize}
		    \item avvio;
		    \item preparazione delle risposte;
		    \item stipulazione del contratto;
		    \item pianificazione;
		    \item esecuzione e controllo;
		    \item revisione e valutazione;
		    \item consegna e completamento.
		\end{itemize}
		\subsubsection{Descrizione}
		Questa sezione norma le fasi di progettazione, sviluppo e consegna del prodotto \textit{HD viz}. 
		\subsubsection{Attività}
		    \paragraph{Studio di Fattibilità}
		    Gli analisti redigono uno \textit{Studio di Fattibilità} per ogni capitolato, indicando:
		    \begin{itemize}
		        \item \textbf{Informazioni genereali}: informazioni riguardanti il nome del progetto, il proponente e il committente;
		        \item \textbf{Descrizione capitolato}: descrizione suntuaria del capitolato e delle aspettative e richieste sul prodotto finale;
		        \item \textbf{Finalità del progetto}: descrizione del prodotto finito;
		        \item \textbf{Aspetti positivi}: caratteristiche che rendono il gruppo più disponibile a proporre un'offerta verso il proponente;
		        \item \textbf{Criticità e fattori di rischio}: aspetti negativi con potenziali ripercussioni sullo svolgimento del progetto;
		        \item \textbf{Conclusioni}: breve spiegazione sul perchè il capitolato è stato scartato o accettato dal gruppo.
		    \end{itemize}
			\paragraph{Piano di Progetto}
			Il Responsabile di Progetto con gli amministratori redige un \textit{Piano di progetto} volto a fornire un preventivo ed una pianificazione dettagliata sullo svolgimento del progetto al proponente contiene:
			\begin{itemize}
			    \item \textbf{Analisi dei Rischi}: vengono elencati i rischi che potrebbero presentarsi durante lo svolgimento del progetto, insieme ad una indicazione probabilistica del loro effettivo avvenimento e le modalità con le quali si intende mitigare questi rischi;
			    \item \textbf{Modello di sviluppo}: viene fornito un modello di sviluppo da seguire durante il progetto;
			    \item \textbf{Pianificazione}: vengono pianificate le attività e le scadenze temporali;
			    \item \textbf{Preventivo}: qui viene fatta la stima dello sforzo previsto in termini di ore di lavoro, e dei costi associati con l'esecuzione di processo, riportando in dettaglio organigramma e il programma degli orari richiesti per una consegna puntuale del progetto. Viene quindi fornito un preventivo iniziale basato su queste stime;
			    \item \textbf{Consuntivo}: il documento viene aggiornato periodicamente con un consuntivo dei costi effettivi dello sviluppo, corredato di una spiegazione su un eventuale differenza con il preventivo.
			\end{itemize}
			\paragraph{Piano di Qualifica}
			I verificatori devono redigere un \textit{Piano di Qualifica}, contenente tutte le informazioni riguardanti il controllo di qualità  per i processi e il prodotto, basato su quantificazioni misurabili. Il \textit{Piano di Qualifica} contiene:
			\begin{itemize}
			    \item \textbf{Qualità di processo}: vengono cercati negli standard i processi da attuare, individuati degli obiettivi di qualità e le metriche corrispondenti, vengono stilati metodi per perseguire gli obiettivi posti;
			    \item \textbf{Qualità di prodotto}: vengono identificate delle metriche corrispondenti agli attributi del prodotto e definiti degli obiettivi di qualità, il documento di riferimento è la ISO-9126;
			    \item \textbf{Specifiche dei test}: vengono definiti i test di accettazione, sistema, integrazione, di modulo e unita volti ad assicurare la qualità del prodotto;
			    \item \textbf{Resoconti di verifica ed esiti delle revisioni}: qui sono riportati gli esiti delle attività di verifica e delle revisioni.
			\end{itemize}
			\paragraph{Strumenti}
			Qui vengono riportati gli strumenti utilizzati nel processo di fornitura:
			
		\subsubsection{Sviluppo}
		
		\subsubsection{Strumenti}
