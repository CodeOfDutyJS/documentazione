\section{Processi Primari}
	\subsection{Fornitura}
		\subsubsection{Scopo}
		Il processo di fornitura si occupa di determinare le procedure e le risorse necessarie allo svolgimento del progetto, incluso lo sviluppo del \textit{Piano di Progetto} e l'esecuzione dello stesso. Il processo inizia una volta presa la decisione di rispondere alla proposta dell'acquirente e dopo aver compreso le sue richieste stilando uno \textit{Studio di fattibilità} il processo si compone delle seguenti attività:
		\begin{itemize}
		    \item avvio;
		    \item preparazione delle risposte;
		    \item stipulazione del contratto;
		    \item pianificazione;
		    \item esecuzione e controllo;
		    \item revisione e valutazione;
		    \item consegna e completamento.
		\end{itemize}
		\subsubsection{Descrizione}
		Questa sezione norma le fasi di progettazione, sviluppo e consegna del prodotto \textit{HD viz}. 
		\subsubsection{Attività}
		    \myparagraph{Studio di Fattibilità}
		    Gli analisti redigono uno \textit{Studio di Fattibilità} per ogni capitolato, indicando:
		    \begin{itemize}
		        \item \textbf{Informazioni generali}: informazioni riguardanti il nome del progetto, il proponente e il committente;
		        \item \textbf{Descrizione capitolato}: descrizione suntuaria del capitolato e delle aspettative e richieste sul prodotto finale;
		        \item \textbf{Finalità del progetto}: descrizione del prodotto finito;
		        \item \textbf{Aspetti positivi}: caratteristiche che rendono il gruppo più disponibile a proporre un'offerta verso il proponente;
		        \item \textbf{Criticità e fattori di rischio}: aspetti negativi con potenziali ripercussioni sullo svolgimento del progetto;
		        \item \textbf{Conclusioni}: breve spiegazione sul perchè il capitolato è stato scartato o accettato dal gruppo.
		    \end{itemize}
			\myparagraph{Piano di Progetto}
			Il Responsabile di Progetto con gli amministratori redige un \textit{Piano di progetto} volto a fornire un preventivo ed una pianificazione dettagliata sullo svolgimento del progetto al proponente contiene:
			\begin{itemize}
			    \item \textbf{Analisi dei Rischi}: vengono elencati i rischi che potrebbero presentarsi durante lo svolgimento del progetto, insieme ad una indicazione probabilistica del loro effettivo avvenimento e le modalità con le quali si intende mitigare questi rischi;
			    \item \textbf{Modello di sviluppo}: viene fornito un modello di sviluppo da seguire durante il progetto;
			    \item \textbf{Pianificazione}: vengono pianificate le attività e le scadenze temporali;
			    \item \textbf{Preventivo}: qui viene fatta la stima dello sforzo previsto in termini di ore di lavoro, e dei costi associati con l'esecuzione di processo, riportando in dettaglio organigramma e il programma degli orari richiesti per una consegna puntuale del progetto. Viene quindi fornito un preventivo iniziale basato su queste stime;
			    \item \textbf{Consuntivo}: il documento viene aggiornato periodicamente con un consuntivo dei costi effettivi dello sviluppo, corredato di una spiegazione su un eventuale differenza con il preventivo.
			\end{itemize}
			\myparagraph{Piano di Qualifica}
			I verificatori devono redigere un \textit{Piano di Qualifica}, contenente tutte le informazioni riguardanti il controllo di qualità  per i processi e il prodotto, basato su quantificazioni misurabili. Il \textit{Piano di Qualifica} contiene:
			\begin{itemize}
			    \item \textbf{Qualità di processo}: vengono cercati negli standard i processi da attuare, individuati degli obiettivi di qualità e le metriche corrispondenti, vengono stilati metodi per perseguire gli obiettivi posti, il documento di riferimento è la ISO-15504;
			    \item \textbf{Qualità di prodotto}: vengono identificate delle metriche corrispondenti agli attributi del prodotto e definiti degli obiettivi di qualità, il documento di riferimento è la ISO-9126;
			    \item \textbf{Specifiche dei test}: vengono definiti i test che garantiscono la qualità del prodotto;
			    \item \textbf{Resoconti di verifica ed esiti delle revisioni}: qui sono riportati gli esiti delle attività di verifica e delle revisioni.
			\end{itemize}
			\subsection{Metriche}
			Di seguito vengono elencate le metriche i quali valori vanno normati nel piano di qualifica. 
La modalità di rilevazione non è indicata per tutte le metriche: tale dato sarà 
inserito in fasi successive del progetto.
\subsubsection{Metriche di processo}
Queste sono le metriche inerenti alla qualità di processo.
\myparagraph{Percentuale di requisiti soddisfatti}
La percentuale in centesimi dei requisiti saddisfatti
\par{Formula}
La formula che individua la metrica è:
\begin{displaymath}
 PROS = Rs / Rt *100
\end{displaymath}
Rs = requisiti soddisfatti
Rt = requisiti totali


\myparagraph{Budget at completion}
Budget totale allocato per il progetto.
Musirazione: numero intero


\myparagraph{Earned value}
Utilizzato anche nel calcolo di \textit{SV} e \textit{CV}. Abbreviato EV, indica il valore del lavoro compiuto fino al momento del calcolo
\par{Formula}
La formula che individua la metrica è:
\begin{displaymath}
  BAC * \% di lavoro cmpletato
\end{displaymath}
BAC = Budget at copletion


\myparagraph{Planned value}
Abbreviato in PV, rappresenta il valore del lavoro pianificato al momento del calcolo. Utilizzato anche nel calcolo di \textit{SV} e \textit{CV}
\par{Formula}
La formula che individua la metrica è:
\begin{displaymath}
  BAC * \% di lavoro pianif icato
\end{displaymath}
BAC = Budget at copletion


\myparagraph{Actual Cost}
per il calcolo in altre metriche l'abbreviazione è AC.
rappresenta il denaro speso fino al momento del calcolo.
La misurazione avviene con il rilevamento dell'intero che rappresenta l'AC pertanto non ha Formule


\myparagraph{Schedule variance}
Rappresenta l'anticipo o il ritardo nello svolgimento del progetto rispetto al valore pianificato. Abbreviato in \textit{SV}.
\par{Formula}
La formula che individua la metrica è:
\begin{displaymath}
  \textit{SV} = \textit{EV} - \textit{PV}
\end{displaymath}
\textit{SV} = Schedule variance
\textit{EV} = Earned value
\textit{PV} = Planned value


\myparagraph{Cost variance}
Rappresenta la differenza tra il lavoro completato e quello pianificato, ed è abbreviato in \textit{CV}
\par{Formula}
La formula che individua la metrica è:
\begin{displaymath}
  \textit{CV} = \textit{EV} - \textit{AC}
\end{displaymath}
\textit{CV} = Cost variance
\textit{EV} = Earned value
\textit{AC} = actual cost


\myparagraph{Gunning's fog index}
L'indice di Gunning fog equivale all'indice della facilità di lettura di un testo. In particolare Stima gli anni di educazione scolastica per complrendere il testo.
\par{Formula}
La formula che individua la metrica è:
\begin{displaymath}
  \textit{IGF} = 0.4·(\frac{ \textit{Parole}}{\textit{Frasi}} + 100· \frac{\textit{Complesse}}{ \textit{Frasi}}
\end{displaymath}
\textit{IGF} = Gunning's fog index
\textit{Parole} = numero complessivo di parole
\textit{Frasi} = numero complessivo di frasi
\textit{Complesse} = numero complessivo di parole ritenute complesse
\textit{Frasi} = numero complessivo di frasi


\myparagraph{Gulpease index}
Rappresenta la leggibilità di un testo valutando la lunghezza delle parole e delle frasi al numero totale di lettere
\par{Formula}
La formula che individua la metrica è:
\begin{displaymath}
\textit{IG} = 89 + \frac{300 * \textit{Frasi} - 10* \textit{Lettere}}{\textit{Parole}}
\end{displaymath}
\textit{IG} = Gulpease index
\textit{Frasi} = numero complessivo di frasi
\textit{Lettere} = numero complessivo di lettere
\textit{Parole} = numero complessivo di parole


\myparagraph{Correttezza ortografica}
Rappresenta il numero totale di errori ortografici presenti. Essendo un valore intero che non è composto da altri sottoelemento non ha una formula
\par{Modalità di rilevazione}
Viene calcolato durante la verifica di un documento.


\myparagraph{Percentuale di metriche soddisfatte}
Indica la percentuale in centesimi del numero di metriche con valore soddisfacente. Una basso valore percentuale è indice quindi di una bassa qualità.
\par{Formula}
La formula che individua la metrica è:
\begin{displaymath}
  \textit{PMS} = \frac{\textit{Soddisfatte}}{\textit{Totali}}
\end{displaymath}
  \textit{PMS} = Percentuale di metriche soddisfatte
  \textit{Soddisfatte} = numero totale di metriche con valore soddisfaciente
  \textit{Totali} = numero totale di metriche calcolate



\myparagraph{Code coverage}
Indica il valore percentuale in centesimi del numero di righe di codice testate.
\par{Formula}
La formula che individua la metrica è:
\begin{displaymath}
  \textit{CC} = \frac{\textit{Testate}}{\textit{Totali}}
\end{displaymath}
 \textit{CC} = Code coverage
 \textit{Testate} = numero di righe testate
 \textit{Totali} = numero di righe totali


\myparagraph{Numero di test superati}
Indica il nuemero totale intero di test superati. Questa metrica non ha formula in quanto il valore è dato dal calcolo del programmatore in fase di test del codice.
\subsubsection{Metriche di prodotto}
Queste sono le Metriche inerenti alla qualità di prodotto.
\myparagraph{Correttezza dello scambio dei dati}
Il prodotto software ha bisogno di dati provenienti da fonti esterne. Questa 
metrica misura che i dati ricevuti da tali fonti non siano modificati o errati 
in qualsiasi modo. Per dato si intende l'informazione più atomica possibile
\textit{ESEMPIO: in una banca dati anagrafica un dato può essere la data di 
nascita, oppure il cognome.}
\par{Formula}
La formula che individua la metrica è:
\begin{displaymath}
  D_{err} = \#\ di\ errori
\end{displaymath}
Ossia $N_{err}$ è il numero di errori totali rilevati sui dati importati 
nell'applicativo.

\myparagraph{Densità di errori}
Indica il numero di errori rilevati sul codice tramite i test di unità, è una 
misura del rapporto tra il totale dei test di unità errati e il totale dei test 
di unità eseguiti.
\par{Formula}
\begin{displaymath}
  E_{density} = A_{err}/B_{tests}
\end{displaymath}
Dove:
\begin{itemize}
  \item[] $E_{density} =$ densità di errori;
  \item[] $A_{err} =$ totale test di unità errati;  
  \item[] $B_{tests} =$ totale test di unità eseguiti.
\end{itemize}

\myparagraph{Qualità della messaggistica}
È una metrica che indica la chiarezza dei messaggi di errore o di avviso 
all'utente. Per essere chiaro un messaggio deve avere le seguenti 
caratteristiche:
\begin{itemize}
  \item deve indicare qual'è l'errore, o il motivo dell'avviso;
  \item deve indicare una o più azioni correttive.
\end{itemize}
Messaggi che non rispettano le suddette caratteristiche non saranno considerati 
chiari.
\par{Formula}
\begin{displaymath}
  Q_{mex} = A_{clear}/B_{tot}
\end{displaymath}
Dove:
\begin{itemize}
  \item[] $Q_{mex} =$ qualità della messaggistica;
  \item[] $A_{clear} =$ totale messaggi considerati chiari; 
  \item[] $B_{tot} =$ totale messaggi all'utente.
\end{itemize}
\par{Modalità di rilevazione}
Gli avvisi e gli errori segnalati all'utente sono contati manualmente.

\myparagraph{Numero di click}
Conta il numero di click necessari per completare uno specifico task. Il task 
in analisi è il seguente: 
\begin{itemize}
  \item click necessari per importare un set di dati nell'applicativo.
\end{itemize}
\textit{Sarà fornita una descrizione più precisa del task in fasi successive 
dell'attività di progetto.}
\par{Formula}
\begin{displaymath}
  C_{click}(t) = \#\ di\ click
\end{displaymath}
Dove:
\begin{itemize}
  \item[] $C_{click} =$ indica il numero totale di click per completare il 
task; 
  \item[] $t =$ è il task da misurare.
\end{itemize}
\par{Modalità di rilevazione}
Per ottenere valori ai fini della metrica saranno eseguiti dei test 
sull'applicativo in funzione.

\myparagraph{Site depth}
È la profondità dell'albero che rappresenta la struttura dell'applicativo.
\par{Formula}
\begin{displaymath}
  S_{depth} = td(G)
\end{displaymath}
Dove:
\begin{itemize}
  \item[] $S_{depth} =$ è la profondità della struttura del prodotto;
  \item[] $td(G) =$ è la funzione che calcola la massima profondità (tree 
depth).
\end{itemize}

\myparagraph{Response time}
Metrica di tempo che misura la durata di uno specifico task in secondi. Il 
task misurato è:
\begin{itemize}
  \item caricamento di un set di dati all'interno del progetto.
\end{itemize}
\textit{In fasi successive del progetto saranno fornite ulteriori specifiche 
riguardanti al task (ad esempio quale set di dati viene utilizzato per la 
misurazione).}
\par{Formula}
\begin{displaymath}
  T_{response} = B_{end}-A_{start}
\end{displaymath}
Dove:
\begin{itemize}
  \item[] $T_{response} =$ durata totale del task;
  \item[] $B_{end} =$ termine del task;
  \item[] $A_{start} =$ inizio del task.
\end{itemize}

\myparagraph{Complessità ciclomatica}
La complessità ciclomatica è la misura della quantità dei possibili percorsi di 
branching di una porzione di codice. Il codice è rappresentato tramite un 
grafo orientato con un nodo di start e finish dove gli altri nodi rappresentano 
istruzioni decisionali, la complessità ciclomatica è quindi il numero di 
cammini lineari indipendenti dal nodo start al nodo finish.
\par{Formula}
\begin{displaymath}
  v(G) = e - n + 2p
\end{displaymath}
Dove:
\begin{itemize}
  \item[] $v(G) =$ complessità ciclomatica del grafo G;
  \item[] $e =$ numero di archi del grafo;
  \item[] $n =$ numero di nodi del grafo;
  \item[] $p =$ numero di componenti connesse.
\end{itemize}

\myparagraph{Indipendenza dei test}
Questa metrica misura l'indipendenza dei test, ossia:
\begin{itemize}
  \item un test non deve fare affidamento su altri test o elementi esterni;
  \item un test deve essere ripetibile e il suo risultato deve rimanere 
invariato.
\end{itemize}
\par{Formula}
\begin{displaymath}
  I_{test} = A_{ind}/B_{tests}
\end{displaymath}
Dove:
\begin{itemize}
  \item[] $I_{test} =$ è l'indice di indipendenza dei test;
  \item[] $A_{ind} =$ numero di test considerati indipendenti;
  \item[] $B_{tests}$ numero totale di test.
\end{itemize}

\myparagraph{Facilità di comprensione}
È un indice che misura la facilità di comprensione del codice. Un codice 
sorgente facilmente comprensibile avrà bisogno di meno commenti, viceversa 
codice complicato avrà più commenti.
\par{Formula}
\begin{displaymath}
  F_{compr} = A_{comm}/SLOC
\end{displaymath}
Dove:
\begin{itemize}
  \item[] $F_{compr} =$ indice della facilità di comprensione;
  \item[] $A_{comm} =$ numero totale di righe di commento;
  \item[] $SLOC =$ numero di righe del codice sorgente.
\end{itemize}

\myparagraph{Sfin}
Structural Fan-In, indica il numero di procedure che chiamano una specifica 
procedura. L'oggetto della misura sono le unità architetturali.
\par{Formula}
\begin{displaymath}
  sfin(u) =  \sum u_{caller}
\end{displaymath}
Dove:
\begin{itemize}
  \item[] $sfin(u) =$ calcolo dello sfin dell'unità $u$;
  \item[] $\sum u_{caller} =$ somma delle unità che utilizzano $u$.
\end{itemize}

\myparagraph{Sfout}
Structural Fan-Out, indica il numero di procedure di cui necessita una 
specifica procedura. L'oggetto della misura sono le unità architetturali. 
\par{Formula}
\begin{displaymath}
  sfout(u) = \sum u_{callee}
\end{displaymath}
Dove:
\begin{itemize}
  \item[] $sfout(u) =$ calcolo dello sfout dell'unità $u$;
  \item[] $\sum u_{callee} =$ somma delle unità utilizzate da $u$.
\end{itemize}
			\subsubsection{Strumenti}
			Qui vengono riportati gli strumenti utilizzati nel processo di fornitura:
			
		\subsection{Sviluppo}
		\subsubsection{Scopo}
		Il processo di sviluppo contiene le attività da svolgere al fine di ottenere il prodotto finale.
		\subsubsection{Attività}
		\myparagraph{Analisi dei requisiti}
		Gli analisti redigono un'\textit{Analisi dei Requisiti} volta  individuare ed elencare casi d'uso e requisiti, al fine di esemplificare il problema da trattare.
		\mysubparagraph{Scopo}
		L'analisi dei requisiti deve definire lo scopo del lavoro e mettere in evidenza funzionalità e requisiti in modo da fornire riferimenti tracciabili sia ai programmatori che ai verificatori.
		\mysubparagraph{Contenuti}
		\mysubparagraph{Descrizione}
		Descrizione delle finalità del prodotto, degli attori in causa, dei vincoli imposti dal proponente, e delle assunzioni fatte dagli analisti.
		\mysubparagraph{Casi d'uso}
		I casi d'uso rappresentano una lista di azioni necessarie alla realizzazione di un obbiettivo desiderabile all'interazione fra un attore ed il sistema.
		La struttura dei casi d'uso è così definita:
		\begin{itemize}
		    \item attore
		    \item descrizione
		    \item precondizione
		    \item postcondizione
		    \item scenario principale
		    \item estensioni
		    \item generalizzazioni
		    \item specializzazioni
		\end{itemize}
		Ogni caso d'uso adotta un codice identificativo in questa forma:\\
		
	    \centerline{\textbf{UC[codiceCaso].[codiceSpecializzazione]}}
	    dove:
	    \begin{itemize}
	        \item \textbf{codiceCaso}: è il codice identificativo del caso;
	        \item \textbf{codiceSpecializzazione}: è il codice della specializzazione.
	    \end{itemize}
	    \mysubparagraph{Requisiti}
	    I requisiti sono identificati dal seguente codice:\\
	    \centerline{\textbf{R[tipo][classe][identificativo]}}
	    dove:
	    \begin{itemize}
	        \item \textbf{tipo}: può essere:
	        \begin{itemize}
	            \item \textbf{F}: requisito funzionale;
	            \item \textbf{V}: vincolo;
	            \item \textbf{Q}: requisito di qualità;
	            \item \textbf{P}: requisito prestazionale. 
	        \end{itemize}
	        \item \textbf{classe}: può essere:
	        \begin{itemize}
	            \item \textbf{O}: obbligatorio;
	            \item \textbf{F}: facoltativo;
	            \item \textbf{D}: desiderabile;
	        \end{itemize}
	        \item \textbf{identificativo}: identificativo progressivo del requisito.
	    \end{itemize}
	    tutti i requisiti sono contenuti in tabelle composte da quattro (4) colonne, i cui attributi sono:
	    \begin{itemize}
	        \item \textbf{Requisito}: codice del requisito;
	        \item \textbf{Descrizione}: descrizione del requisito;
	        \item \textbf{Classificazione}: la classe del requisito;
	        \item \textbf{Fonte}: fonte da cui il requisito è stato estrapolato.
	    \end{itemize}
	    le tabelle sono divise per tipo di requisito, e ad ogni requisito corrisponde una riga di una ed una sola tabella.
		\myparagraph{Progettazione}
		Messi in mostra i requisiti stilati nell'\textit{Analisi dei requisiti} gli analisti ricercano una possibile soluzione al problema compreso, identificando un'architettura di alto livello che comprenda una lista di sottosistemi hardware e software, specificandone le funzionalità ed assicurandosi che comprendano un'implementazione di tutti i requisiti tracciati.
		\mysubparagraph{Prodotti}
		Vengono prodotte due parti, la \textbf{Technology baseline} e la \textbf{Product baseline}.
		\mysubparagraph{Technology baseline}
		Redatta dal progettista, contiene le specifiche di progettazione architetturale di alto livello, l'insieme di diagrammi UML che la definiscono ed i test d'integrazione.\newline
		include quindi:
		\begin{itemize}
		    \item \textbf{Tecnologie}: devono essere descritte e giustificate le tecnologie adottate, elencandone svantaggi e vantaggi, ed esponendone il ruolo nel progetto;
		    \item \textbf{Design Pattern}: descrizione dei design pattern adottati, corredati di descrizione, motivazione della scelta, diagrammi UML che li descrivano;
		    \item \textbf{Diagrammi UML}:
		    \begin{itemize}
		        \item Diagrammi delle classi;
		        \item Diagrammi di package;
		        
		        \item Diagrammi di attività;
		        \item Diagrammi di sequenza.
		    \end{itemize}
		    \item \textbf{Tracciamento}: tracciamento dei componenti ai relativi requisiti che soddisfano.
		\end{itemize}
		\myparagraph{Product Baseline}
		Redatta dai progettisti dovrà contenere:
		\begin{itemize}
		    \item \textbf{Definizione delle classi}: descrizione delle classi utilizzate;
		    \item \textbf{Tracciamento delle classi}: collegamento delle classi ai relativi requisiti;
		    \item \textbf{Test di unità}: definizione dei test di unità al fine di stabilire il corretto funzionamento delle classi.
		\end{itemize}
		\myparagraph{Codifica}
		\mysubparagraph{Scopo}
	    Scopo della sezione è normare lo stile di codifica da utilizzare nella programmazione, intesa come effettiva implementazione software del prodotto. Lo stile di codifica è da considerarsi come parte integrante della qualità del prodotto stesso.
	    \mysubparagraph{Indentazione}
	    L'indentazione viene effettuata con tre spazi, e non con il carattere tab. Quando si parla di "tabulare" o di "tabulazione" si deve intendere, appunto, l'inserire tre spazi.
	    \mysubparagraph{Blocchi innestati}
	    Ciascun blocco innestato viene tabulato una volta.
	    \mysubparagraph{Parentesizzazione}
	    le parentesi di delimitazione di un blocco di codice vanno inserite nel seguente modo:
	    \begin{itemize}
	        \item la prima và inserita alla fine della prima riga, preceduta da uno spazio;
	        \item la seconda su una nuova riga, senza caratteri di spazio antecedenti;
	    \end{itemize}
	    
	    \mysubparagraph{Casing}
	    \begin{itemize}
	        \item \textbf{Classi}: i nomi delle classi iniziano con la lettera maiuscola;
	        \item \textbf{Costanti}: vanno scritte tutte in maiuscolo;
	        \item \textbf{Variabili}: camel case, iniziano con la lettera minuscola;
	        \item \textbf{Metodi}: camel case, iniziano con la lettera minuscola.
	    \end{itemize}
	    \mysubparagraph{Metodi}
	    Il codice deve essere il più leggibile possibile. A tal fine, e anche al fine di esemplificare i test, è desiderabile che i metodi siano composti da poche righe di codice, con nomi quanto più corti possibile ma al contempo descrittivi.
	    \mysubparagraph{Lingua}
	    Il codice insieme ad i commenti viene scritto in inglese.
	    \mysubparagraph{Ricorsione}
	    Evitare il più possibile la ricorsione.
