\section{Processi di Supporto}
	\subsection{Documentazione}
	    
	    \subsubsection{Scopo}
	    Lo scopo della documentazione è fornire riferimenti precisi ed universali su ogni attività e processo inerenti al progetto. Questa sezione norma come stilare tutti i documenti prodotti durante il ciclo di vita del software. I documenti sono reperibili al seguente repository\glo{}:
	    \url{https://github.com/CodeOfDutyJS/documentazione}
	    
	    \subsubsection{Ciclo di vita del Documento}
	    \begin{itemize}
	        \item \textbf{Creazione} Il documento viene creato da fonti accettabili e conformamente alle norme, in particolare sono fonti accettabili push sul repository, viene usato un template LaTeX fornito nello stesso;
	        \item \textbf{Implementazione della struttura} Sempre secondo le norme, viene creata la struttura del documento, che deve sempre contenere un registro delle modifiche e un indice che tiene traccia delle voci.
	        \item \textbf{Redazione} il documento viene scritto in forma incrementale, con ogni voce creata interamente;
	        \item \textbf{Revisione} L'implementazione e le modifiche del documento deve seguire gli standard documentativi forniti, e devono essere approvate da un membro del gruppo in base al loro formato, adeguatezza, contenuto tecnico e stile di presentazione;
	        \item \textbf{Approvazione della versione} Una volta che il documento contiene tutte le voci descritte nella struttura, viene approvato da un membro del gruppo che non deve aver contribuito precedentemente al documento stesso;
	        \item \textbf{Manuntenzione} Una volta che il documento viene aggiornato come da normativa, viene di nuovo sottoposto a \textbf{Revisione} e \textbf{Approvazione}.
	    \end{itemize}
	    
	    \subsubsection{Template LaTex e automazione}
	    Allo scopo di uniformare lo stile e velocizzare la produzione dei documenti viene fornito un template LaTeX.
	    \paragraph{Struttura}
	    Un file "main.tex" raccoglie le sezioni del documento, ed in testa raccoglie in input un file "package.tex", contenente tutti i package necessari alla compilazione, ed un file "config.tex", contenente i comandi di configurazione.
	    \paragraph{Automazione}
	    vengono usate le Github Actions per automatizzare la compilazione del file LaTeX, in modo da creare un artefatto consultabile da tutto il gruppo ad ogni cambiamento dello stesso, ed in modo  da assicurarsi circa la compilazione stessa del file LaTeX. Inoltre uno script Python automaticamente appone l'apposito pedice alle voci da inserire nel glossario.
	    \subsubsection{Pagine}
	    Di seguito una descrizione di pagine sempre presenti nei documenti prodotti dal gruppo.
	    \paragraph{Frontespizio}
	    Il frontespizio di tutti il documenti del gruppo è descritto nel template LaTeX e contiene dall'alto verso il basso, centrati:
	    \begin{itemize}
	        \item il logo del gruppo;
	        \item il nome del gruppo, seguito da un trattino orrizzontale e il titolo del progetto, entrambi in grassetto;
	        \item il nome del documento in grassetto;
	        \item una tabella a due colonne recante le seguenti voci:
	        \begin{enumerate}
	            \item \textbf{Versione}: la versione del documento;
	            \item \textbf{Approvazione}: lo stato di approvazione, seguito dai nomi dei componenti del gruppo incaricati di tale attività;
	            \item \textbf{Redazione}: i nomi dei componenti del gruppo incaricati di redarre il documento
	            \item \textbf{Verifica}: i nomi dei componenti del gruppo incaricati della verifica del documento
	            \item \textbf{Stato}: lo stadio del ciclo di vita nel quale si trova il documento;
	            \item \textbf{Uso}: può essere interno o esterno;
	            \item \textbf{Destinato a}: i destinatari del documento (lasciare vuoto se ad uso interno)
	        \end{enumerate}
	        \item il recapito mail del gruppo.
	    \end{itemize}
	    \paragraph{Diario delle modifiche}
	    La seconda pagina contiene sempre il diario delle modifiche, una tabella atta ad elencare ed a descrivere sinteticamente le modifiche, in ordine cronmologico, apposte al documento.
	    Il diario delle modifiche contiene le seguenti voci:
	    \begin{enumerate}
	        \item \textbf{Versione}: la versione del documento;
	        \item \textbf{Data}: la data della modifica, della revisione o approvazione;
	        \item \textbf{Nominativo}: chi ha apportato la modifica, revisione o approvazione;
	        \item \textbf{Ruolo}: chi ha apportato la modifica, revisione o approvazione;
	        \item \textbf{Descrizione}: descrizione suntuaria delle attività effettuate.
	    \end{enumerate}
	    \paragraph{Indice}
	    Il documento contiene poi sempre un indice ordinato e facilmente consultabile contenente le voci all'interno del documento, in modo da informare sulla struttura dello stesso e di dare modo di trovare velocemente le sezioni ricercate.
	    \paragraph{Intestazione del contenuto}
	    Tutte le pagine di contenuto hanno separata da una linea orizzontale un'intestazione contenente:
	    \begin{itemize}
	        \item a sinistra il logo del gruppo, in versione apposita per intestazione;
	        \item a destra il nome del documento.
	    \end{itemize}
	    \paragraph{Piè di pagina}
	    il piè di pagina contiene, centrato, i numeri di pagina corrente e pagine totali del documento.
	    \paragraph{Note}
	    In caso di note queste vanno numerate per pagina ed essere riportate con la loro numerazione a piè di pagina e descritte il più brevemente possibile.
	    \subsubsection{Norme di stile}
	    \paragraph{Immagini}
	    Le immagini vanno inserite centrate e fornite di apposita didascalia.
	    \paragraph{Date}
	    Le date vanno inserite usando il formato YYYY-MM-DD (anno per esteso, mese a due cifre e giorno a due cifre) in conformità all' ISO-8601
	    \paragraph{Nomi di file}
	    Quando ci si riferisce ad un particolare file, e più in generale tutti i file e le cartelle devono avere nomi chiari, descrittivi del contenuto, ma per quanto possibile sintetici.
	    Per quanto riguarda la convenzione da usare per i nomi, si usa lo Snake Case e si devono seguire le seguenti norme:
	    \begin{itemize}
	        \item tutte le parole da cui è composto il nome devono iniziare con la lettera minuscola;
	        \item nel caso siano presenti date queste sono scritte alla fine del nome seguendo le convenzioni date
	    \end{itemize}
	    \paragraph{Glossario}
	    Ogni termine riportato nel glossario ha una lettera \textbf{G} maiuscola ed in grassetto apposta sotto il nome. Questa viene apposta automaticamente nel LaTeX del documento da uno script Python in base alle voci presenti nel file glossary.txt presente nella repository, che deve essere aggiornato con tutte le voci presenti nel Glossario.
	    \paragraph{Stile del testo}
	    \begin{itemize}
	        \item \textbf{Maiuscolo}: vengono scritti per intero in maiuscolo solo gli acronimi;
	        \item \textbf{Corsivo}: vengono scritti in corsivo il nome del gruppo, del proponente, del progetto e dei documenti;
	        \item \textbf{Grassetto}: vengono scritti in grassetto i termini su cui si vuole far ricadere l'attenzione del lettore.
	    \end{itemize}
	    \paragraph{Riferimenti a documenti}
	    \begin{itemize}
	        \item nel caso il riferimento al documento sia in un titolo a come voce di un elenco non si usa il corsivo ma il grassetto
	        \item ogni qualvolta si fa riferimento al documento in un testo o ad un suo contenuto il nome viene accompagnato, separato da uno spazio, dalla versione, anch'essa in corsivo
	        \item il nome del documento viene scritto per intero, e ogni parola di cui è composto deve iniziare con la lettera maiuscola
	    \end{itemize}
	    \paragraph{Elenchi}
	    ogni voce di un elenco può seguire due convenzioni stilistiche:
	    \begin{itemize}
	        \item nel caso la voce non abbia un titolo comincia con la lettera minuscola;
	        \item \textbf{Titolo}: nel caso la voce abbia un titolo questo viene scritto in grassetto, comincia con la lettera maiuscola ed è seguito dai due punti ed da una descrizione.
	    \end{itemize}
	    a discapito della convenzione seguita ogni voce termina con il punto e virgola \textbf{";"} tranne l'ultima che temina con un punto \textbf{"."}, ogni sottoelenco segue le medesime regole.
	    \paragraph{Sigle}
	    Nella stesura dei documenti vengono adottate le segunti sigle:
	    \begin{itemize}
	        \item \textbf{Glossario}: \textbf{G}
	        \item \textbf{Revisione dei Requisiti}: \textbf{RR}
	        \item \textbf{Revisione di Progettazione}: \textbf{RP}
	        \item \textbf{Revisione di Qualifica}: \textbf{RQ}
	        \item \textbf{Revisione di Accettazione}: \textbf{RA}
	        \item \textbf{Responsabile di progetto}: \textbf{Re}
	        \item \textbf{Amministratore}: \textbf{Am}
	        \item \textbf{Analista}: \textbf{An}
	        \item \textbf{Progettista}: \textbf{Pr}
	        \item \textbf{Programmatore}: \textbf{Pg}
	        \item \textbf{Verificatore}: \textbf{Ve}
	    \end{itemize}
	    \subsubsection{Elementi grafici}
	    Di seguito si trovano le norme per gli elementi grafici. Si distinguono due tipi di elementi grafici:
	    \begin{itemize}
	        \item \textbf{Figure}: sono figure le tabelle, i grafici ed i diagrammi UML;
	        \item \textbf{Immagini}: tutti gli altri elementi grafici. 
	    \end{itemize}
	    \paragraph{Figure}
	    Ogni figura deve essere contrassegnata da una didascalia descrittiva posta al di sopra di essa, segurita fa una numerazione progressiva assegnata per sezione. È esente da questa convenzione il diario delle modifiche.
	    la numerazione delle figure è composta da tre cifre \textbf{X.Y.Z}:
	    \begin{itemize}
	        \item \textbf{X.Y}: riferimento alla sezione;
	        \item \textbf{Z}: riferimento progressivo all'interno della sezione.
	    \end{itemize}
	    \paragraph{Immagini}
	    Le immagini vanno inserite centrate e corredate di didascalia descrittiva sottostante all'immagine stessa.
	    \subsubsection{Strumenti per la stesura}
	    \paragraph{LaTeX}
	    Viene usato LaTeX come linguaggio markdown per facilizzare ed uniformare la scrittura collaborativa dei documenti.
	    \paragraph{TexLive}
	    Per la stesura dei documenti e la loro compilazione.
	    \paragraph{Overleaf}
	    Per la stesura dei documenti.
	\subsection{Verifica}
	\subsection{Validazione}
	\subsection{Strumenti}
