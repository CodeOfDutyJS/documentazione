\section{Processi di Supporto}
	\subsection{Documentazione}
	    
	    \subsubsection{Scopo}
	    Lo scopo della documentazione è fornire riferimenti precisi ed universali su ogni attività e processo inerenti al progetto. Questa sezione norma come stilare tutti i documenti prodotti durante il ciclo di vita del software. I documenti sono reperibili al seguente repository\glo{}:
	    \url{https://github.com/CodeOfDutyJS/documentazione}
	    
	    \subsubsection{Ciclo di vita del Documento}
	    \begin{itemize}
	        \item \textbf{Creazione} Il documento viene creato da fonti accettabili e conformamente alle norme, in particolare sono fonti accettabili push sul repository, viene usato un template LaTeX fornito nello stesso;
	        \item \textbf{Implementazione della struttura} Sempre secondo le norme, viene creata la struttura del documento, che deve sempre contenere un registro delle modifiche e un indice che tiene traccia delle voci.
	        \item \textbf{Redazione} il documento viene scritto in forma incrementale, con ogni voce creata interamente;
	        \item \textbf{Revisione} L'implementazione e le modifiche del documento deve seguire gli standard documentativi forniti, e devono essere approvate da un membro del gruppo in base al loro formato, adeguatezza, contenuto tecnico e stile di presentazione;
	        \item \textbf{Approvazione della versione} Una volta che il documento contiene tutte le voci descritte nella struttura, viene approvato da un membro del gruppo che non deve aver contribuito precedentemente al documento stesso;
	        \item \textbf{Manuntenzione} Una volta che il documento viene aggiornato come da normativa, viene di nuovo sottoposto a \textbf{Revisione} e \textbf{Approvazione}.
	    \end{itemize}
	    
	    \subsubsection{Template}
	    Allo scopo di uniformare lo stile e velocizzare la produzione dei documenti viene fornito un template LaTeX.
	    \subsubsection{Struttura}
	    Un file "main.tex" raccoglie le sezioni del documento, ed in testa raccoglie in input un file "package.tex", contenente tutti i package necessari alla compilazione, ed un file "config.tex", contenente i comandi di configurazione.
	    \subsubsection{Automazione}
	    vengono usate le Github Actions per automatizzare la compilazione del file LaTeX, in modo da creare un artefatto consultabile da tutto il gruppo ad ogni cambiamento dello stesso, ed in modo  da assicurarsi circa la compilazione stessa del file LaTeX.
	\subsection{Verifica}
	\subsection{Validazione}
	\subsection{Strumenti}
	\subsection{Documentazione}
	\subsection{Verifica}
	\subsection{Validazione}
	\subsection{Strumenti}
