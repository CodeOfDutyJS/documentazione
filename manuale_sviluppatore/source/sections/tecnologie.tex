\section{Tecnologie}
    \subsection{Tecnologie per lo sviluppo}
    \subsubsection{Javascript}
        Il linguaggio principale usato per l'implementazione di HD Viz è javascript, in particolare viene usata la versione ES2018.
    \subsubsection{ESlint}
        Il linter usato nella stesura del progetto è ESlint, la cui configurazione può essere consultata nel file .eslintrc.js. Lo stile di codifica adottato è Airbnb.
    \subsubsection{Parte server}
    In questa sezione vengono elencate le tecnologie utilizzate per la parte server.
        \myparagraph{Node.Js}
        Node.Js è un ambiente runtime di Javascript basato sul motore \href{https://v8.dev/}{V8 Javascript engine}. Node.Js permette d'implementare il cosidetto paradigma "Javascript everywhere", difatti sia la parte client che la parte server sono scritte utilizzando questo """linguaggio""".
        \myparagraph{ExpressJs}
        Express.js è un framework back end che semplifica la creazione di API rest.
        \url{http://expressjs.com/}
    \subsubsection{Parte client}
        \myparagraph{D3.js}
        D3.js è una libreria che permette di manipolare il DOM con un approccio data-driven, in \textit{HD Viz} viene usata principalmente per creare le visualizzazioni.
        \url{https://d3js.org/}
        \myparagraph{React}
        React è un framework
        \myparagraph{Mobx}
        \myparagraph{Antd.js}
    \subsubsection{Testing}
        \myparagraph{Jest}
    \subsubsection{Altre librerie impiegate}
        \myparagraph{ml.js}
        \myparagraph{PapaParse}